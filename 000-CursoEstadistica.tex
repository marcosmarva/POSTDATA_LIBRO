% !Mode:: "Tex:UTF-8"
\documentclass[10pt,a4paper]{book}
\usepackage{makeidx}
\newcommand{\idioma}{spanish}
\newcommand{\opcionesIdioma}{,es-nodecimaldot}
%%%%%%%%%%%%%%%%%%%%%Carga de Packages
%%poner \newcommand{\idioma}{spanish} o \newcommand{\idioma}{english} en el documento
\usepackage{pdfsync}
\usepackage{srcltx}
\usepackage[\idioma\opcionesIdioma]{babel}
\usepackage[utf8x]{inputenc}
\usepackage{graphicx}
\graphicspath{{/users/fernando/figuras/}{./}{./figuras/}{/fernando/figuras/}{/fernando/figuras/jpg/}}
\usepackage{epsfig}
%\usepackage{listings}
%\usepackage{algorithm}
\usepackage{amsmath}
\usepackage{amsfonts}
\usepackage{amssymb}
\usepackage{amsthm}
\usepackage{fancybox}
\usepackage{array}
\input{xy}
\xyoption{all}
\usepackage[dvipsnames,usenames]{color}
\usepackage{comment}
\excludecomment{spanish}
\excludecomment{english}
\includecomment{\idioma}

%\usepackage{noweb}
%\usepackage{clrscode}
\usepackage{eurosym}
\usepackage{wasysym}
\usepackage{multirow}
%\usepackage{margins}
\usepackage{lscape}
\usepackage[normalem]{ulem}
\usepackage{xr-hyper}



\excludecomment{ocultar}


% Matriz (par‚ntesis)
\def\matr#1#2{\left(\begin{array}{#1}#2\end{array}\right)}
% Determinante (barras)
\def\deter#1#2{\left|\begin{array}{#1}#2\end{array}\right|}
% Sistema de ecuaciones. (llave a la izda.)
\def\seq#1#2{\left\{\begin{array}{#1}#2\end{array}\right.}
% Ecuaci\'on de varias lineas (sin llave a la izda.)
\def\evl#1#2{\begin{array}{#1}#2\end{array}}

%%%%%%%%%%%%%%%%%%%%%%%%%%%%%%%%%%%%%%%%%%%%%%
%%%%%%%%%%%%%%%%%%%%%%%%%%%%%%%%%%%%%%%%%%%%%%
%%%%%%%%%%%%%%%%% M\'{a}rgenes %%%%%%%%%%%%%%%%
%
%
%\parindent=0mm
%
%\textwidth=160mm
%\textheight=220mm
%\hoffset=-20mm
%\voffset=-15mm
%\parskip=0mm
\marginparsep=3mm
\marginparwidth=25mm
%
%%%%%%%%%%%%%%%%%%%%%%%%%%%% Contadores para listas de problemas
%\newcommand{\adc}{\addtocounter{enumi}{1}}
\newcommand{\adc}{\stepcounter{enumi}}
\newcommand{\adci}{\stepcounter{enumii}}
\newcommand{\xadc}{\addtocounter{xcounter}{1}}
\newcommand{\be}{\begin{enumerate}}
\newcommand{\ee}{\end{enumerate}}
\newcommand{\bi}{\begin{itemize}}
\newcommand{\ei}{\end{itemize}}
\newcounter{xcounter}


\newcommand{\nin}{{\noindent}}

%\newcounter{prob}{}
%\def\pr{\addtocounter{prob}{1}(\theprob)\ }
%\def\pr2{\addtocounter{prob}{2}(\theprob)\ }

%%%%%%%%%%%%%%%%%%%%%%%%%%%Fin de demostraciones, ejemplos, etc.
\newcommand{\fin}{$\square$}
%%%%%%%%%%%%%%%%%%%%%%%%%%Notaci\'{o}n matem\'{a}ticas generales
%\newcommand{\suc}[1]{\{#1_n\}}
%\newcommand{\sucn}[1]{\{#1_n\}_{n\in\mathbb{N}}}
%\newcommand{\ser}[1]{\sum #1_n}
%\newcommand{\sern}[1]{\sum_{n\geq 1} #1_n}
%\newcommand{\limn}{\lim_{n\rightarrow\infty}}
%\newcommand{\limnd}{\displaystyle\lim_{n\rightarrow\infty}}
%\newcommand{\mf}[1]{\mathbf{#1}}
%\newcommand{\mb}[1]{\mathbb{#1}}
%\newcommand{\D}[1]{\Dv_{\mf{#1}}}
%\newcommand{\bsigma}{\pmb{\sigma}}
%\newcommand{\bPhi}{\pmb{\Phi}}
%\newcommand{\vol}{\operatorname{vol}}
%\newcommand{\ldbr}{[\hspace{-1.5pt}[}
%\newcommand{\rdbr}{]\hspace{-1.5pt}]}
%\newcommand{\fpws}[2]{{#1}\ldbr{#2}\rdbr}
%\newcommand{\leftPui}{<\hspace{-3pt}<}
%\newcommand{\rightPui}{>\hspace{-3pt}>}
%\newcommand{\Pui}[2]{{#1}\hspace{-6pt}\leftPui{#2}\rightPui}

%%%%%%%%%%Conjuntos de n\'{u}meros
\newcommand{\N}{\mathbb{N}} %conjunto de n\'{u}meros naturales
\newcommand{\Z}{\mathbb{Z}} %conjunto de n\'{u}meros enteros
\newcommand{\R}{\mathbb{R}} %conjunto de n\'{u}meros reales
\newcommand{\C}{\mathbb{C}} %conjunto de n\'{u}meros complejos
\newcommand{\Q}{\mathbb{Q}} %conjunto de n\'{u}meros racionales
\newcommand{\EP}{\mathbb{P}} %espacios proyectivos
\newcommand{\K}{\mathbb{K}} %cuerpo gen\'{e}rico
\newcommand{\A}{\mathbb{A}} %espacios afines
%%%%%%%%%%Funciones
\def\arcsen{\operatorname{arcsen}}
\def\arctg{\operatorname{arctg}}
\def\argCosh{\operatorname{argCosh}}
\def\argSenh{\operatorname{argSenh}}
\def\argTgh{\operatorname{argTgh}}
\def\cosec{\operatorname{cosec}}
\def\Cosh{\operatorname{Cosh}}
\def\cotg{\operatorname{cotg}}
\def\Dv{\operatorname{D}}
\def\discrim{\operatorname{discrim}}
\def\dive{\operatorname{div}}
\def\dom{\operatorname{dom}}
\def\Ext{\operatorname{Ext}}
\def\Fr{\operatorname{Fr}}
\def\dder#1#2{\dfrac{d #1}{d #2} } %derivada en estilo display
\def\gr{\operatorname{gr}}
\def\grad{\operatorname{grad}}
\def\Imag{\operatorname{Im}}
\def\mcm{\operatorname{mcm}}
\def\rang{\operatorname{rang}}
\def\rot{\operatorname{rot}}
\def\sen{\operatorname{sen}}
\def\Senh{\operatorname{Senh}}
\def\sgn{\operatorname{sgn}}
\def\sig{\operatorname{sig}}
\def\tg{\operatorname{tg}}
\def\Tgh{\operatorname{Tgh}}
\def\E{\operatorname{E}}
\def\VAR{\operatorname{VAR}}
\newcommand{\margWeb}[2]{\noindent{#2}\marginpar[\hspace{-18mm}\link{#1}{WEB}]{\hspace*{-18mm}\link{#1}{WEB}}}

%%%%%%%%%%%%%%%%%%%%%%\'{A}lgebra conmutativa.
\def\multideg{\operatorname{multideg}} %multidegree of a polynomial
\def\LT{\operatorname{lt}} %leading term of a polynomial
\def\LC{\operatorname{lc}} %leading coefficient of a polynomial
\def\LM{\operatorname{lm}} %leading monomial of a polynomial
\def\Mexp{\mathbb{Z}^n_{\geq 0}} %set of multiexponents of monomials
\def\set#1{\left\{{#1}\right\}}
\newcommand{\vlist}[2]{\mbox{${#1}_{1},\ldots,{#1}_{#2}$}}
\def\deg{\operatorname{deg}} %grado de un polinomio
\def\cp{\operatorname{cp}} %coeficiente principal de un polinomio
\def\CP{\operatorname{cp}} %coeficiente principal de un polinomio
\def\set#1{\left\{{#1}\right\}} %llaves de conjunto
\newcommand{\V}{{\bf V}} %variedad de un conjunto de polinomios
\newcommand{\I}{{\bf I}} %ideal de un conjunto
\newcommand{\MCD}{\operatorname{mcd}} %m\'{a}ximo com\'{u}n divisor
\newcommand{\MCM}{\operatorname{mcm}} %m\'{\i}nimo com\'{u}n m\'{u}ltiplo
\newcommand{\LCM}{\operatorname{lcm}} %least common multiple
\newcommand{\GCD}{\operatorname{gcd}} %greatest common divisor
\newcommand{\Ker}{\operatorname{Ker}} %N\'{u}cleo
\newcommand{\IM}{\operatorname{IM}} %Imagen
\newcommand{\Rad}{\operatorname{Rad}} %radical de un ideal
\newcommand{\Jac}{\operatorname{Jac}} %radical de Jacobson de un anillo
\newcommand{\Ann}{\operatorname{Ann}} %anulador de un ideal
\newcommand{\Res}{\operatorname{Res}} %resultante de polinomios
\newcommand{\Mult}{\operatorname{mult}} %multiplicidad
\newcommand{\Gen}{\operatorname{Gen}} %g\'{e}nero
\newcommand{\Card}{\operatorname{Card}} %cardinal
\newcommand{\ord}{\operatorname{ord}} %orden
\newcommand{\prim}{\operatorname{prim}} %parte primitiva
\newcommand{\NP}{\operatorname{NP}} %NP idea
\newcommand{\cont}{\operatorname{cont}} %parte primitva
\newcommand{\pp}{\operatorname{pp}} %parte primitva
\newcommand{\PP}{\mathop{\mathrm{PP}}\nolimits}
\newcommand{\Int}{\operatorname{Int}}
\newcommand{\Ind}{\operatorname{index}}
\newcommand{\Lcoeff}{\operatorname{lc}} %leading coefficient of a polynomial
\newcommand{\Sqf}{\operatorname{Sqf}} %square free part of a polynomial

\def\pd#1#2{\frac{\partial #1}{\partial #2}} %derivada parcial
\def\mult{\text{mult}} %multiplicity
\def\Sing{\text{Sing}} %multiplicity
\def\Cl#1{\overline{#1}} %cierre topol\'{o}gico
\def\fobox#1{\begin{center}\fbox{$\displaystyle #1 $}\end{center}}

%\newcommand{\Ext}{\operatorname{Ext}}


%%%%%%%%%%%%%%%%%%%%%%%%S\'{\i}mbolos rodeados de un c\'{\i}rculo
\def\circled#1{\xymatrix{*+[o][F]{#1}}}

%%%%%%%%%%%%%%%%%%%Geometr\'{\i}a
\newcommand{\CH}{{\cal CH}} %%cierre convexo

%%%%%%%%%%%%%%%%%%%%Tipos de letra especiales
%%Caligr\'{a}ficas
\newcommand{\cA}{{\cal A}}
\newcommand{\cB}{{\cal B}}
\newcommand{\cC}{{\cal C}}
\newcommand{\cD}{{\cal D}}
\newcommand{\cE}{{\cal E}}
\newcommand{\cF}{{\cal F}}
\newcommand{\cG}{{\cal G}}
\newcommand{\cH}{{\cal H}}
\newcommand{\cI}{{\cal I}}
\newcommand{\cJ}{{\cal J}}
\newcommand{\cK}{{\cal K}}
\newcommand{\cL}{{\cal L}}
\newcommand{\cM}{{\cal M}}
\newcommand{\cN}{{\cal N}}
\newcommand{\cO}{{\cal O}}
\newcommand{\cP}{{\cal P}}
\newcommand{\cQ}{{\cal Q}}
\newcommand{\cR}{{\cal R}}
\newcommand{\cS}{{\cal S}}
\newcommand{\cT}{{\cal T}}
\newcommand{\cU}{{\cal U}}
\newcommand{\cV}{{\cal V}}
\newcommand{\cW}{{\cal W}}
\newcommand{\cX}{{\cal X}}
\newcommand{\cY}{{\cal Y}}
\newcommand{\cZ}{{\cal Z}}

%%%%%%%%%%%%%%%%%%%%%%%%%%Notaci\'{o}n matem\'{a}ticas generales
\newcommand{\sucn}[1]{\{#1_n\}_{n\in\mathbb{N}}}
\newcommand{\ser}[1]{\sum #1_n}
\newcommand{\sern}[1]{\sum_{n\geq 1} #1_n}
\newcommand{\limn}{\lim_{n\rightarrow\infty}}
\newcommand{\mf}[1]{\mathbf{#1}}
\newcommand{\mb}[1]{\mathbb{#1}}
\newcommand{\D}[1]{\Dv_{\mf{#1}}}
\newcommand{\bsigma}{\pmb{\sigma}}
\newcommand{\bPhi}{\pmb{\Phi}}
\newcommand{\vol}{\operatorname{vol}}
\newcommand{\ldbr}{[\hspace{-1.5pt}[}
\newcommand{\rdbr}{]\hspace{-1.5pt}]}
\newcommand{\fpws}[2]{{#1}\ldbr{#2}\rdbr}
\newcommand{\leftPui}{<\hspace{-3pt}<}
\newcommand{\rightPui}{>\hspace{-3pt}>}
\newcommand{\Pui}[2]{{#1}\hspace{-6pt}\leftPui{#2}\rightPui}
\newcommand{\pdd}[2]{\dfrac{\partial{#1}}{\partial{#2}}}




\newcommand{\pendiente}{\textcolor{red}{PENDIENTE: }}
%\newcommand{\link}[2]{\textcolor{blue}{{\href{#1}{#2}}}}
\newcommand{\link}[2]{\textcolor{blue}{{\href{#1}{#2}}}}

%%%%%%%%%%%%%%%%%%COLORES

\DefineNamedColor{named}{Brown}{cmyk}{0,0.81,1,0.60}
\definecolor{Gris050}{gray}{0.50}
\definecolor{Gris025}{gray}{0.75}


%%%%%%%%%%%%%%%%%%%%%Package Algorithms
%\begin{spanish}
%\renewcommand{\algorithmicrequire}{{precondici\'{o}n:}}
%\renewcommand{\algorithmicensure}{{postcondici\'{o}n:}}
%\renewcommand{\algorithmicend}{{fin}}
%\renewcommand{\algorithmicif}{{si}}
% \renewcommand{\algorithmicthen}{{entonces}}
% \renewcommand{\algorithmicelse}{{si no}}
% \renewcommand{\algorithmicelsif}{\algorithmicelse\ \algorithmicif}
% \renewcommand{\algorithmicendif}{\algorithmicend\ \algorithmicif}
% \renewcommand{\algorithmicfor}{{para}}
% \renewcommand{\algorithmicforall}{{para todo}}
% \renewcommand{\algorithmicdo}{{hacer}}
% \renewcommand{\algorithmicendfor}{\algorithmicend\ \algorithmicfor}
% \renewcommand{\algorithmicwhile}{{mientras}}
% \renewcommand{\algorithmicendwhile}{\algorithmicend\ \algorithmicwhile}
% \renewcommand{\algorithmicrepeat}{{repetir}}
% \renewcommand{\algorithmicuntil}{{hasta}}
% \end{spanish}

%%%%%%%%%%%%%%%%%%%%%%%%%%%%%%%%%%Package Amsthm
\begin{spanish}
%\theoremstyle{definition}% default
\theoremstyle{plain}
\newtheorem{thm}{Teorema}[section]
\newtheorem{teo}{Teorema}[section]
\newtheorem{teorema}{Teorema}[section]
\newtheorem{lem}[thm]{Lema}
\newtheorem{lema}[thm]{Lema}
\newtheorem{prop}[thm]{Proposici\'{o}n}
\newtheorem{proposicion}[thm]{Proposici\'{o}n}
\newtheorem{cor}[thm]{Corolario}
\newtheorem{corolario}[thm]{Corolario}
\newtheorem*{KL}{Klein's Lemma}
%\theoremstyle{definition}
\newtheorem{defn}[thm]{Definici\'{o}n}
\newtheorem{definicion}[thm]{Definici\'{o}n}
\newtheorem{conj}[thm]{Conjetura}
\newtheorem{conjetura}[thm]{Conjetura}
\newtheorem{definicionInformal}[thm]{Definición Informal}
\newtheorem{exmp}[thm]{Ejemplo}
\newtheorem{ejemplo}[thm]{Ejemplo}
\newtheorem{Ejemplo}[thm]{Ejemplo}
\newtheorem{ejem}[thm]{Ejemplo}
\newtheorem{ejercicio}[thm]{Ejemplo}
%\theoremstyle{remark}
\newtheorem*{rem}{Observaci\'{o}n}
\newtheorem{observacion}[thm]{Observaci\'{o}n}
\newtheorem*{note}{Nota}
\newtheorem{nota}[thm]{Nota}
\newtheorem{case}[thm]{Caso}
\newtheorem{caso}[thm]{Caso}
\newtheorem{regla}[thm]{Regla}
\end{spanish}

\begin{english}
\theoremstyle{plain}% default
%\theoremstyle{definition}
\newtheorem{thm}{Theorem}[section]
\newtheorem{lem}[thm]{Lemma}
\newtheorem{prop}[thm]{Proposition}
\newtheorem{cor}[thm]{Corollary}
\newtheorem*{KL}{Klein's Lemma}
\newtheorem{defn}[thm]{Definition}
\newtheorem{conj}[thm]{Conjecture}
\newtheorem{exmp}[thm]{Example}
\theoremstyle{remark}
\newtheorem*{rem}{Remark}
\newtheorem*{note}{Note}
\newtheorem{case}{Case}
\end{english}

%%%%%%%%%%%%%%%Package Listings
%\lstset{showstringspaces=false}
%\newcommand{\PAS}[1]{\lstinline@#1@}
%\newcommand{\CPP}[1]{\lstinline@#1@}


%%%%%%%%%%%%Estilo para bibliograf\'{\i}a

\bibliographystyle{plain}

%%%%%%%%%%%%Mis anotaciones
\newcommand{\Pendiente}{\textcolor{blue}{Pendiente: }}

%%%%%%%%%%%%%%%% Enlace al indice
%\renewcommand{\chaptermark}[1]{\markboth{\chaptername\ \thechapter.#1 \ref{index}}{}}

%%%%%%%%%%%%%%%%%%Traducci\'{o}n de clrscode
%\renewcommand{\For}{\textbf{Para} }
%\renewcommand{\To}{\textbf{hasta} }
%\renewcommand{\By}{\textbf{incremento} }
%\renewcommand{\Downto}{\textbf{downto} }
%\renewcommand{\While}{\textbf{mientras} }
%\renewcommand{\Repeat}{\textbf{repetir}\>\>\addtocounter{indent}{1}}
%\renewcommand{\Until}{\kill\addtocounter{indent}{-1}\liprint\>\>\textbf{hasta que}\hspace*{-0.7em}\'}
%\renewcommand{\If}{\textbf{si} }
%\renewcommand{\Then}{\>\textbf{entonces}\hspace{13mm}\>\addtocounter{indent}{1}}
%\renewcommand{\Else}{\kill\addtocounter{indent}{-1}\liprint\>\textbf{sino}\>\addtocounter{indent}{1}}
%\renewcommand{\End}{\addtocounter{indent}{-1}}
%\renewcommand{\ElseIf}{\kill\addtocounter{indent}{-1}\liprint\textbf{sino si} }
%\renewcommand{\ElseNoIf}{\kill\addtocounter{indent}{-1}\liprint\textbf{si no}\addtocounter{indent}{1}}
%\renewcommand{\Do}{\>\>\textbf{hacer}\hspace*{-0.7em}\'\addtocounter{indent}{1}}
%\renewcommand{\Return}{\textbf{devolver} }
%\renewcommand{\Comment}{$\hspace*{-0.075em}\rhd$ }
%\renewcommand{\RComment}{\`\Comment}
%\renewcommand{\Goto}{\textbf{Ir a} }
%\renewcommand{\Error}{\textbf{error} }


%%%%%%%%%%%%%%%%%%%%%%%%%%%%%%%%%%%%%%%%%%%%%%%%%%%%%%%%%%%%%%%
%Cabecera para ejercicios
%\documentclass[11pt]{article}
%\newcommand{\idioma}{spanish}
%\input definiciones
%
%\textwidth=160mm \textheight=240mm \hoffset=-20mm \voffset=-30mm
%%\parskip=0mm
%%\marginparsep=-25mm \evensidemargin=82pt\evensidemargin=44pt
%
%
%\includecomment{solucion}
%%\excludecomment{solucion}

%%Compatibilidad con documentos antiguos
\newcounter{prob}{}
\def\pr{\noindent\addtocounter{prob}{1}(\theprob)\ }
\def\bepro{ \setcounter{prob}{0}}

%%Compatibilidad con documentos antiguos
% \def\ojo#1{
% \noindent$\btr$#1
% \marginpar[
% {GeoGebra}]
% {GeoGebra}}

% \def\atencion#1{\noindent #1
% \marginpar[
% {\includegraphics*[scale=1,width=1.2cm,keepaspectratio=true]{hipoizda}}]
% {\includegraphics*[scale=1,width=1.2cm,keepaspectratio=true]{hipodcha}}}


\def\Rlogo#1{\noindent #1
\marginpar[
{\includegraphics*[scale=1,width=1.2cm,keepaspectratio=true]{Rlogo.jpg}}]
{\includegraphics*[scale=1,width=1.2cm,keepaspectratio=true]{Rlogo.jpg}}}



\def\atencion{
\marginpar[
{\includegraphics*[scale=1,width=2cm,keepaspectratio=true]{hipoizda}}]
{\includegraphics*[scale=1,width=2cm,keepaspectratio=true]{hipodcha}}}


\def\ojo#1{
\noindent #1
\marginpar[
{\includegraphics*[scale=1,width=1.5cm,keepaspectratio=true]{hipoojoi}}]
{\includegraphics*[scale=1,width=1.5cm,keepaspectratio=true]{hipoojod}}}

\def\ojo2{
\marginpar[
{\includegraphics*[scale=1,width=1.5cm,keepaspectratio=true]{hipoojoi}}]
{\includegraphics*[scale=1,width=1.5cm,keepaspectratio=true]{hipoojod}}}


\def\lio#1{
\noindent$\btr$#1
\marginpar{\includegraphics*[scale=1,width=1.1cm,keepaspectratio=true]{hipolio}}}

\def\cuentas{
\marginpar{\includegraphics*[scale=1,width=1.3cm,keepaspectratio=true]{hipocuen}}}

\def\pensar{
\marginpar{\includegraphics*[scale=1,width=1.5cm,keepaspectratio=true]{hipopens}}}

\def\facil{
\marginpar{\includegraphics*[scale=1,width=2cm,keepaspectratio=true]{hipofcil}}}



\newcommand{\WikipediaLogo}{\marginpar{\includegraphics*[scale=1,width=1.2cm,keepaspectratio=true]{LogoWikipedia}}}
\newcommand{\MoodleLogo}{\marginpar{\includegraphics*[scale=1,width=1.2cm,keepaspectratio=true]{MoodleLogo}}}
\newcommand{\WirisGeoGebraLogo}{\marginpar{\includegraphics*[scale=1,width=1.2cm,keepaspectratio=true]{WirisGeoGebraLogo}}}
\newcommand{\WirisLogo}{\marginpar{\includegraphics*[scale=1,width=1.2cm,keepaspectratio=true]{WirisLogo}}}
\newcommand{\GeoGebraLogo}{\marginpar{\includegraphics*[scale=1,width=1.2cm,keepaspectratio=true]{GeoGebra-Logo}}}


\newcommand{\enObras}[1]{\includegraphics*[scale=1,width=0.5cm,keepaspectratio=true]{obras.png}\textcolor{blue}{#1}}



\newcommand{\GeoGebra}[2]{\noindent #1
\marginpar[{\link{#2}{\small Moodle}\\\includegraphics*[scale=1,width=1.2cm,keepaspectratio=true]{MoodleLogo}}]{\link{#2}{\small Moodle}\\\includegraphics*[scale=1,width=1.2cm,keepaspectratio=true]{MoodleLogo}}}

\newcommand{\Moodle}[2]{\noindent #1
\marginpar[{\link{#2}{\small Moodle}\\\includegraphics*[scale=1,width=1.2cm,keepaspectratio=true]{MoodleLogo}}]{\link{#2}{\small Moodle}\\\includegraphics*[scale=1,width=1.2cm,keepaspectratio=true]{MoodleLogo}}}

\newcommand{\Wikipedia}[2]{\noindent #1
\marginpar[{\link{#2}{\small Wikipedia}\\\includegraphics*[scale=1,width=1.2cm,keepaspectratio=true]{LogoWikipedia}}]{\link{#2}{\small Wikipedia}\\\includegraphics*[scale=1,width=1.2cm,keepaspectratio=true]{LogoWikipedia}}}


\newcommand{\pder}[2]{\frac{\partial #1}{\partial #2}}

%%%%%%%%%%%%%%%%%%%%%%%%%%%%%%%%%%%%%%%%%%%%%%
%%%%%%%%%%%%%%%%%%%%%%%%%%%%%%%%%%%%%%%%%%%%%%%
%%%%%%%%%%%%%%%%%% M\'{a}rgenes %%%%%%%%%%%%%%%%
%%
%%
%%\parindent=0mm
%%
%\textwidth=160mm \textheight=220mm \hoffset=-20mm \voffset=-15mm
%\parskip=0mm
%\marginparsep=-25mm
%%
%%%%%%%%%%%%%%%%%%%%%%%%%%%%% Contadores para listas de problemas
%%\newcommand{\adc}{\addtocounter{enumi}{1}}
%\newcommand{\adc}{\stepcounter{enumi}}
%\newcommand{\adci}{\stepcounter{enumii}}
%\newcommand{\xadc}{\addtocounter{xcounter}{1}}
%\newcommand{\be}{\begin{enumerate}}
%\newcommand{\ee}{\end{enumerate}}
%\newcommand{\bi}{\begin{itemize}}
%\newcommand{\ei}{\end{itemize}}
%\newcounter{xcounter}
%\newcounter{probl}
%\setcounter{probl}{0}
%\newcommand{\pro}{\addtocounter{probl}{1}}
%\newcommand{\pr}{{\pro}{(\theprobl.)}}
%%%%%%%%%%%%%%%%%%%%%%%%%%%%Fin de demostraciones, ejemplos, etc.
%\newcommand{\fin}{$\square$}
%%%%%%%%%%%%%%%%%%%%%%%%%%%Notaci\'{o}n matem\'{a}ticas generales
%\newcommand{\suc}[1]{\{#1_n\}}
%\newcommand{\sucn}[1]{\{#1_n\}_{n\in\mathbb{N}}}
%\newcommand{\ser}[1]{\sum #1_n}
%\newcommand{\sern}[1]{\sum_{n\geq 1} #1_n}
%\newcommand{\limn}{\lim_{n\rightarrow\infty}}
%\newcommand{\mf}[1]{\mathbf{#1}}
%\newcommand{\mb}[1]{\mathbb{#1}}
%\newcommand{\D}[1]{\Dv_{\mf{#1}}}
%\newcommand{\bsigma}{\pmb{\sigma}}
%\newcommand{\bPhi}{\pmb{\Phi}}
%\newcommand{\vol}{\operatorname{vol}}
%\newcommand{\ldbr}{[\hspace{-1.5pt}[}
%\newcommand{\rdbr}{]\hspace{-1.5pt}]}
%\newcommand{\fpws}[2]{{#1}\ldbr{#2}\rdbr}
%\newcommand{\leftPui}{<\hspace{-3pt}<}
%\newcommand{\rightPui}{>\hspace{-3pt}>}
%\newcommand{\Pui}[2]{{#1}\hspace{-6pt}\leftPui{#2}\rightPui}
%\newcommand{\pdd}[2]{\dfrac{\partial{#1}}{\partial{#2}}}
%%%%%%%%%%%Conjuntos de n\'{u}meros
%\newcommand{\N}{\mathbb{N}} %conjunto de n\'{u}meros naturales
%\newcommand{\Z}{\mathbb{Z}} %conjunto de n\'{u}meros enteros
%\newcommand{\R}{\mathbb{R}} %conjunto de n\'{u}meros reales
%\newcommand{\C}{\mathbb{C}} %conjunto de n\'{u}meros complejos
%\newcommand{\Q}{\mathbb{Q}} %conjunto de n\'{u}meros racionales
%\newcommand{\EP}{\mathbb{P}} %espacios proyectivos
%\newcommand{\K}{\mathbb{K}} %cuerpo gen\'{e}rico
%\newcommand{\A}{\mathbb{A}} %espacios afines
%%%%%%%%%%%Funciones
%\def\arcsen{\operatorname{arcsen}}
%\def\arctg{\operatorname{arctg}}
%\def\argCosh{\operatorname{argCosh}}
%\def\argSenh{\operatorname{argSenh}}
%\def\argTgh{\operatorname{argTgh}}
%\def\cosec{\operatorname{cosec}}
%\def\Cosh{\operatorname{Cosh}}
%\def\cotg{\operatorname{cotg}}
%\def\Dv{\operatorname{D}}
%\def\discrim{\operatorname{discrim}}
%\def\dive{\operatorname{div}}
%\def\dom{\operatorname{dom}}
%\def\Ext{\operatorname{Ext}}
%\def\Fr{\operatorname{Fr}}
%\def\gr{\operatorname{gr}}
%\def\grad{\operatorname{grad}}
%\def\Imag{\operatorname{Im}}
%\def\mcm{\operatorname{mcm}}
%\def\rang{\operatorname{rang}}
%\def\rot{\operatorname{rot}}
%\def\sen{\operatorname{sen}}
%\def\Senh{\operatorname{Senh}}
%\def\sgn{\operatorname{sgn}}
%\def\sig{\operatorname{sig}}
%\def\tg{\operatorname{tg}}
%\def\Tgh{\operatorname{Tgh}}
%\def\E{\operatorname{E}}
%\def\VAR{\operatorname{VAR}}
%
%%%%%%%%%%%%%%%%%%%%%%%\'{A}lgebra conmutativa.
%\def\multideg{\operatorname{multideg}} %multidegree of a polynomial
%\def\LT{\operatorname{lt}} %leading term of a polynomial
%\def\LC{\operatorname{lc}} %leading coefficient of a polynomial
%\def\LM{\operatorname{lm}} %leading monomial of a polynomial
%\def\Mexp{\mathbb{Z}^n_{\geq 0}} %set of multiexponents of monomials
%\def\set#1{\left\{{#1}\right\}}
%\newcommand{\vlist}[2]{\mbox{${#1}_{1},\ldots,{#1}_{#2}$}}
%\def\deg{\operatorname{deg}} %grado de un polinomio
%\def\cp{\operatorname{cp}} %coeficiente principal de un polinomio
%\def\CP{\operatorname{cp}} %coeficiente principal de un polinomio
%\def\set#1{\left\{{#1}\right\}} %llaves de conjunto
%\newcommand{\V}{{\bf V}} %variedad de un conjunto de polinomios
%\newcommand{\I}{{\bf I}} %ideal de un conjunto
%\newcommand{\MCD}{\operatorname{mcd}} %m\'{a}ximo com\'{u}n divisor
%\newcommand{\MCM}{\operatorname{mcm}} %m\'{\i}nimo com\'{u}n m\'{u}ltiplo
%\newcommand{\LCM}{\operatorname{lcm}} %least common multiple
%\newcommand{\GCD}{\operatorname{gcd}} %greatest common divisor
%\newcommand{\Ker}{\operatorname{Ker}} %N\'{u}cleo
%\newcommand{\IM}{\operatorname{IM}} %Imagen
%\newcommand{\Rad}{\operatorname{Rad}} %radical de un ideal
%\newcommand{\Jac}{\operatorname{Jac}} %radical de Jacobson de un anillo
%\newcommand{\Ann}{\operatorname{Ann}} %anulador de un ideal
%\newcommand{\Res}{\operatorname{Res}} %resultante de polinomios
%\newcommand{\Mult}{\operatorname{mult}} %multiplicidad
%\newcommand{\Gen}{\operatorname{Gen}} %g\'{e}nero
%\newcommand{\Card}{\operatorname{Card}} %cardinal
%\newcommand{\ord}{\operatorname{ord}} %orden
%\newcommand{\prim}{\operatorname{prim}} %parte primitiva
%\newcommand{\NP}{\operatorname{NP}} %NP idea
%\newcommand{\cont}{\operatorname{cont}} %parte primitva
%\newcommand{\pp}{\operatorname{pp}} %parte primitva
%\newcommand{\PP}{\mathop{\mathrm{PP}}\nolimits}
%\newcommand{\Int}{\operatorname{Int}}
%\newcommand{\Ind}{\operatorname{index}}
%\newcommand{\Lcoeff}{\operatorname{lc}} %leading coefficient of a polynomial
%\newcommand{\Sqf}{\operatorname{Sqf}} %square free part of a polynomial
%
%\def\pd#1#2{\frac{\partial #1}{\partial #2}} %derivada parcial
%\def\mult{\text{mult}} %multiplicity
%\def\Sing{\text{Sing}} %multiplicity
%\def\Cl#1{\overline{#1}} %cierre topol\'{o}gico
%
%%\newcommand{\Ext}{\operatorname{Ext}}
%
%%%%%%%%%%%%%%%%%%%%%%%%%S\'{\i}mbolos rodeados de un c\'{\i}rculo
%\def\circled#1{\xymatrix{*+[o][F]{#1}}}
%
%%%%%%%%%%%%%%%%%%%%Geometr\'{\i}a
%\newcommand{\CH}{{\cal CH}} %%cierre convexo
%
%%%%%%%%%%%%%%%%%%%%%Tipos de letra especiales
%%%Caligr\'{a}ficas
%\newcommand{\cA}{{\cal A}}
%\newcommand{\cB}{{\cal B}}
%\newcommand{\cC}{{\cal C}}
%\newcommand{\cD}{{\cal D}}
%\newcommand{\cE}{{\cal E}}
%\newcommand{\cF}{{\cal F}}
%\newcommand{\cG}{{\cal G}}
%\newcommand{\cH}{{\cal H}}
%\newcommand{\cI}{{\cal I}}
%\newcommand{\cJ}{{\cal J}}
%\newcommand{\cK}{{\cal K}}
%\newcommand{\cL}{{\cal L}}
%\newcommand{\cM}{{\cal M}}
%\newcommand{\cN}{{\cal N}}
%\newcommand{\cO}{{\cal O}}
%\newcommand{\cP}{{\cal P}}
%\newcommand{\cQ}{{\cal Q}}
%\newcommand{\cR}{{\cal R}}
%\newcommand{\cS}{{\cal S}}
%\newcommand{\cT}{{\cal T}}
%\newcommand{\cU}{{\cal U}}
%\newcommand{\cV}{{\cal V}}
%\newcommand{\cW}{{\cal W}}
%\newcommand{\cX}{{\cal X}}
%\newcommand{\cY}{{\cal Y}}
%\newcommand{\cZ}{{\cal Z}}
%
%
%%%%%%%%%%%%%%%%%%%COLORES
%
%\DefineNamedColor{named}{Brown}{cmyk}{0,0.81,1,0.60}
%\definecolor{Gris050}{gray}{0.50}
%\definecolor{Gris025}{gray}{0.50}
%
%
%%\theoremstyle{plain}
%%\newtheorem{thm}{Teorema}[section]
%%%\newtheorem{teo}{Teorema}[section]
%%\newtheorem{lem}[thm]{Lema}
%%\newtheorem{prop}[thm]{Proposici\'{o}n}
%%\newtheorem{cor}[thm]{Corolario}
%%\newtheorem*{KL}{Klein's Lemma}
%%%\theoremstyle{definition}
%%\newtheorem{defn}[thm]{Definici\'{o}n}
%%\newtheorem{conj}[thm]{Conjetura}
%%\newtheorem{exmp}[thm]{Ejemplo}
%%\newtheorem{ejem}[thm]{Ejemplo}
%%\theoremstyle{remark}
%%\newtheorem*{rem}{Observaci\'{o}n}
%%\newtheorem*{note}{Nota}
%%\newtheorem{case}{Caso}
%%\newtheorem{regla}[thm]{Regla}
%
%\theoremstyle{plain}
%\newtheorem{thm}{Teorema}%[subsection]
%%\newtheorem{teo}{Teorema}[section]
%%\newtheorem{teorema}{Teorema}[section]
%\newtheorem{lem}[thm]{Lema}
%\newtheorem{lema}[thm]{Lema}
%\newtheorem{prop}[thm]{Proposici\'{o}n}
%\newtheorem{proposicion}[thm]{Proposici\'{o}n}
%\newtheorem{cor}[thm]{Corolario}
%\newtheorem{corolario}[thm]{Corolario}
%\newtheorem*{KL}{Klein's Lemma}
%%\theoremstyle{definition}
%\newtheorem{defn}[thm]{Definici\'{o}n}
%\newtheorem{definicion}[thm]{Definici\'{o}n}
%\newtheorem{conj}[thm]{Conjetura}
%\newtheorem{conjetura}[thm]{Conjetura}
%\newtheorem{exmp}[thm]{Ejemplo}
%\newtheorem{ejemplo}[thm]{Ejemplo}
%\newtheorem{ejem}[thm]{Ejemplo}
%\newtheorem{ejercicio}[thm]{Ejemplo}
%\theoremstyle{remark}
%\newtheorem*{rem}{Observaci\'{o}n}
%\newtheorem*{observacion}{Observaci\'{o}n}
%\newtheorem*{note}{Nota}
%\newtheorem*{nota}{Nota}
%\newtheorem{case}{Caso}
%\newtheorem{caso}{Caso}
%\newtheorem{regla}[thm]{Regla}
%
%%%%%%%%%%%%%Estilo para bibliograf\'{\i}a
%
%\bibliographystyle{plain}
%
%%%%%%%%%%%%%Mis anotaciones
%\newcommand{\Pendiente}{\textcolor{blue}{Pendiente: }}


\usepackage[pageanchor=true]{hyperref}
\makeindex

%\input{sahp}
\includecomment{com}
%\excludecomment{com}
%\usepackage[dvips]{hyperref}
%\usepackage{pstricks}
\usepackage{attachfile}

\textwidth=150mm \textheight=260mm
\hoffset=-1cm
\voffset=-25mm
%\textwidth=160mm \textheight=240mm \hoffset=-20mm \voffset=-20mm \parskip=0mm \marginparsep=-25mm
\setlength{\parindent}{0pt}


\begin{document}
\author{Fernando San Segundo}
\title{\textcolor{blue}{POSTDATA}\\
{\small (Un curso de estadística para principiantes)}}
\date{\today}

\pagestyle{plain}
\maketitle

\setcounter{tocdepth}{1}
\tableofcontents


\part{Estadística descriptiva}
%\input{001-EstadisticaDescriptiva}

    \chapter{Introducción a la estadística descriptiva}%\addcontentsline{toc}{chapter}{Introducción a la Estadística}
    % !Mode:: "Tex:UTF-8"

%\setcounter{section}{0}
%\section*{\fbox{\colorbox{Gris025}{{Sesión 2 (21/09/2011). Estadística descriptiva}}}}

%\noindent{\bf Atención: este fichero pdf lleva adjuntos los ficheros de datos necesarios para la clase de hoy, que se abren usando los enlaces que contiene.}

%\subsection*{\fbox{1. Ejemplos preliminares }}
%\setcounter{tocdepth}{1}
%\tableofcontents
\setcounter{section}{1}

\section{Población y muestra.}

\begin{itemize}
    \item La estadística se divide habitualmente en varios temas relacionados. Esa división resulta evidente mirando el programa de esta asignatura, o la división por capítulos de cualquiera de los manuales que aparecen en la bibliografía. Algunos de los temas habituales en cualquier curso de introduccion a la Estadística son
        \begin{center}
        $
        \mbox{Estadística }=\begin{cases}
        \mbox{Estadística Descriptiva}\\
        \mbox{Distribuciones de Probabilidad}\\
        \mbox{Inferencia Estadística}\\
        \mbox{Diseño de Experimentos}\\
        \mbox{Regresión}\\
        \mbox{...}
        \end{cases}
        $
        \end{center}

    \item Cuando tenemos un conjunto de datos y queremos describirlos, representarlos o  visualizarlos, utilizamos técnicas que corresponden a la {\sf Estadística Descriptiva}. Así que la Estadística Descriptiva se encarga del trabajo con los {\em datos que de hecho tenemos}, y con los que podemos hacer operaciones. Es el tema con el que vamos a empezar el curso.

    \item En muchas ocasiones esos datos corresponden a una {\sf muestra}, es decir, a un subconjunto (más o menos pequeño), de una {\sf población} (más o menos grande), que nos gustaría estudiar. El problema es que estudiar toda la población puede ser demasiado difícil o directamente imposible. En ese caso surge la pregunta ¿hasta qué punto los datos de la muestra son representativos de la población? Es decir, ¿podemos usar los datos de la muestra para {\em inferir} los datos de la población completa? La {\sf Inferencia Estadística} se encarga de dar sentido y formalizar estas preguntas.

    \item La razón por la que la inferencia, que es la esencia de la Estadística, funciona es porque muchos casos (ya discutiremos cuales), cualquier muestra {\em bien elegida} (ya veremos lo que significa esto) es bastante representativa de la población. Dicho de otra manera, si pensamos en el conjunto de todas las posibles muestras bien elegidas que podríamos tomar, la inmensa mayoría de ellas serán representativas de la población. Así que si tomamos una al azar, casi con seguridad habremos tomado una muestra representativa. Puesto que hemos mencionado el azar, parece evidente que la manera de hacer que estas frases un poco imprecisas se conviertan en afirmaciones rigurosas y medibles es utilizar el lenguaje de la {\sf Probabilidad}. Por esa razón necesitamos hablar en ese lenguaje para poder hacer Estadística rigurosa.

    \item Las técnicas de {\sf muestreo} y las de {\sf diseño de experimentos} también forman parte de la Estadística, para garantizar el buen funcionamiento de las técnicas que hemos descrito.


\end{itemize}

\section{Estadística descriptiva. Tipos de Variables.}

\subsection*{Contenido:}
\begin{itemize}
 \item Variables cualitativas y cuantitativas.
 \item Variables cuantitativas discretas y continuas.
 \item Notación para las variables. Frecuencia.
\end{itemize}


\subsection{Variables cualitativas y cuantitativas}

\begin{itemize}
    \item A veces se dice que las variables cuantitativas son las variables numéricas, y las cualitativas las no numéricas. La diferencia es, en realidad, un poco más sutil. Una variable es {\sf cualitativa nominal} cuando {\sf sólo} se utiliza para establecer categorías, y {\em no para hacer operaciones con ella}. Es decir, para poner nombres, crear clases o especies dentro de los individuos que estamos estudiando. Por ejemplo, cuando clasificamos a los seres vivos en especies, no estamos {\em midiendo nada}. Podemos {\em representar} esas especies mediante números, naturalmente, pero en este caso la utilidad de ese número se acaba en la propia representación, y en la clasificación que los números permiten. Pero no utilizamos las propiedades de los números (las operaciones aritméticas, suma, resta, etc.). Volviendo al ejemplo de las especies, no tiene sentido sumar especies de seres vivos.

    \item Una {\sf variable cuantitativa}, por el contrario, tiene un valor numérico, y las operaciones matemáticas que se pueden hacer con ese número son importantes para nosotros. Por ejemplo, podemos medir la presión arterial de un animal y utilizar fórmulas de la mecánica de fluidos para estudiar el flujo sanguíneo.

    \item En la frontera, entre las variables cuantitativas y las cualitativas, se incluyen las {\sf cualitativas ordenadas}. En este caso existe una ordenación dentro de los valores de la variable. Por ejemplo, la gravedad del pronóstico de un enfermo ingresado en un hospital. Pero no tiene sentido hacer otras operaciones con esos valores: no podemos sumar grave con leve. Como ya hemos dicho, se pueden codificar mediante números de manera que el orden se corresponda con el de los códigos numéricos\footnote{¿Dónde encaja el {\em pronóstico reservado} en esta escala?}:
        \begin{center}
        \begin{tabular}{|c|c|}
        \hline
        {\em Pronóstico}&{\em Código}\\
        \hline
        Leve&1\\
        \hline
        Moderado&2\\
        \hline
        Grave&3\\
        \hline
        \end{tabular}
        \end{center}
        En este caso es especialmente importante no usar esos números para operaciones estadísticas que pueden no tener significado (por ejemplo, calcular la media).
\end{itemize}

\subsection{Variables cuantitativas discretas y continuas.}

\begin{itemize}
    \item A su vez, las variables cuantitativas se dividen en {\sf discretas} y {\sf continuas}. Puesto que se trata de números, y queremos hacer operaciones con ellos, la clasificación depende de las operaciones matemáticas que vamos a realizar.
    \item Cuando utilizamos los {\sf números enteros} ($\mathbb Z$) o un subconjunto de ellos como modelo, la variable es discreta. Y entonces con esos números podemos sumar, restar, multiplicar --pero no siempre dividir.
    \item Por el contrario, si usamos {\sf los números reales} ($\mathbb R$), entonces la variable aleatoria es continua. La diferencia entre un tipo de datos y el otro se corresponde en general con la diferencia entre digital y analógico, o con la diferencia entre ecuaciones en diferencias y ecuaciones diferenciales.
    \item Es importante entender que la diferencia entre discreto y continuo es, en general, una diferencia que establecemos nosotros al crear un {\sf modelo} con el que estudiar un fenómeno, y que la elección correcta determina la utilidad del modelo.
\end{itemize}

\subsection{Notación para las variables. Tablas de frecuencia. Datos agrupados}

\begin{itemize}

    \item En cualquier caso, vamos a tener siempre una serie de valores (observaciones, medidas) de una variable, que representaremos con símbolos como
    \[x_1,x_2,\ldots,x_n\]
    El {\sf número $n$} se utiliza habitualmente en Estadística para referirse al {\sf número total de valores} de los que se dispone).
    Por ejemplo, en el fichero \textattachfile{GonickSmith-p009-GeneroPesoEdad.csv}{\textcolor{blue}{GonickSmith-p009-GeneroPesoEdad.csv}}
    (\textattachfile{GonickSmith-p009-GeneroPesoEdad.ods}{\textcolor{blue}{Calc}} y versión \textattachfile{GonickSmith-p009-GeneroPesoEdad.xls}{\textcolor{blue}{Excel}}) tenemos los pesos (en libras) de los $n=92$ alumnos de una clase de Universidad de los Estados Unidos\footnote{Tomado de 'La Estadística en Comic', capítulo 2.} (los datos llegan a la fila 93, pero la primera fila es el nombre de las variables).

    Si utilizamos $p_1,p_2,\ldots,p_{92}$ para referirnos a estos datos de peso, entonces $p_1$ es el dato en la segunda fila, $p_2$ el dato en la tercera, y $p_{35}$ el dato de la fila $36$. En estos casos puede ser una buena idea introducir una columna adicional con el índice $i$ que corresponde a $p_i$ ($i$ es el número de la observación). Porque, como ya veremos, puede ser cómodo conservar los nombres de las variables en la primera fila de la hoja de cálculo.

    \item Un mismo valor de la variable puede aparecer repetido varias veces en la serie de observaciones. En el fichero de los alumnos de nuestra clase, la variable género sólo toma dos valores, Hombre o Mujer. Pero cada uno de esos valores aparece repetido bastantes veces. En concreto en la clase hay 35 alumnas y 57 alumnos. Este {\sf número de repeticiones de un valor} es lo que llamamos la {\sf frecuencia} de ese valor.

    \item El número de repeticiones de un valor, del que hemos hablado en el anterior párrafo, se llama {\sf frecuencia absoluta}, para distinguirlo de la {\sf frecuencia relativa}, que se obtiene dividiendo la frecuencia absoluta por $n$ (el total de observaciones).

        También se pueden utilizar {\sf porcentajes} en lugar de frecuencias relativas (la frecuencia relativa es el tanto por uno, y el porcentaje el tanto por ciento).

    \item Cuando tratamos con variables cualitativas o con variables discretas, muchas veces, en lugar del valor de cada observación la información que tenemos es la de las frecuencias de cada uno de los posibles valores distintos de esas variables. Esto es lo que se conoce como una {\sf tabla de frecuencias}. Vamos a ver como se obtiene (usando la función {\tt FRECUENCIA()}) la tabla de frecuencias de la variable género en el ejemplo de la clase que acabamos de ver (el resultado está en estos ficheros \textattachfile{GonickSmith-p009-frecuenciasGenero.ods}{\textcolor{blue}{Calc}} y \textattachfile{GonickSmith-p009-frecuenciasGenero.xls}{\textcolor{blue}{Excel}}).

    \item ¿Qué sucede en este ejemplo con la variable peso? ¿Podemos calcular una tabla de frecuencias? Sí, en principio, podemos. Pero hay demasiados valores distintos, y la información presentada así no es útil. De hecho, como el peso {\em es una variable continua}, si nos dieran los pesos de los alumnos con, por ejemplo, dos cifras decimales, {\em es muy posible que no haya dos valores iguales de la variable} peso. Por otra parte, si los pesos de dos alumnos se diferencian en unas décimas, seguramente preferiremos representarlos todos por un valor común. En el caso de variables continuas, lo habitual es {\em dividir el rango de posibles valores de esa variable continua en intervalos}. La {\sf tabla de frecuencia por intervalos} mide, para estas variables, cuantos de los valores observados caen dentro de cada uno de los intervalos.

    En el ejemplo que estamos utilizando, podemos dividir arbitrariamente los valores del peso en intervalos de 10 libras, desde 85 hasta 215, y obtenemos esta tabla de frecuencias:
    \begin{center}
    \begin{tabular}{|c|c|c|}
    \hline
    {\bf Intervalo}&{\bf Significado}&{\bf Frecuencia}\\ \hline
    (75,85]&Peso$<$85&0\\ \hline
    (85,95]&85$<$Peso$\leq$95&1\\ \hline
    (95,105]&95$<$Peso$\leq$105&1\\ \hline
    (105,115]&105$<$Peso$\leq$115&7\\ \hline
    (115,125]&115$<$Peso$\leq$125&15\\ \hline
    (125,135]&125$<$Peso$\leq$135&10\\ \hline
    (135,145]&135$<$Peso$\leq$145&13\\ \hline
    (145,155]&145$<$Peso$\leq$155&22\\ \hline
    (155,165]&155$<$Peso$\leq$165&7\\ \hline
    (165,175]&165$<$Peso$\leq$175&6\\ \hline
    (175,185]&175$<$Peso$\leq$185&4\\ \hline
    (185,195]&185$<$Peso$\leq$195&5\\ \hline
    (195,205]&195$<$Peso$\leq$205&0\\ \hline
    (205,215]&205$<$Peso$\leq$215&1\\ \hline
    (215,225]&210$<$Peso&0\\ \hline
    \end{tabular}
    \end{center}
    Y aquí están los correspondientes ficheros \textattachfile{GonickSmith-p009-frecuenciasPesos.ods}{\textcolor{blue}{Calc}} y \textattachfile{GonickSmith-p009-frecuenciasPesos.xls}{\textcolor{blue}{Excel}}.

    Los intervalos, insistimos, se han elegido de manera arbitraria en este ejemplo. ¿Piensas que eso afecta de alguna manera a la información de la tabla de frecuencias?

    \item En cualquier caso, conviene recordar que una tabla de frecuencias es una forma de resumir la información, y que al pasar del conjunto de datos inicial a las tablas de frecuencias de Peso y Género se pierde información.

    \item Cuando los valores de una variable continua se presentan en forma de tabla de frecuencias por intervalos hablaremos de {\sf datos agrupados}.

\end{itemize}

\section{Tablas y visualización gráfica de datos.}

\subsection*{Contenido:}

\begin{enumerate}
 \item Diagramas de líneas, sectores y barras
 \item Histogramas
\end{enumerate}

\subsection{Diagramas de líneas, sectores y barras}

\begin{itemize}
     \item Los {\sf diagramas de líneas} se utilizan para mostrar tendencias temporales. Como el que se muestra en este fichero \textattachfile{HeadFirst-p003.ods}{\textcolor{blue}{Calc}} (versión \textattachfile{HeadFirst-p003.xls}{\textcolor{blue}{Excel}})

     \item Los diagramas de sectores y barras se utilizan cuando queremos mostrar frecuencias (o porcentajes, recuentos, etcétera). Se pueden utilizar para ilustrar las frecuencias de variables tanto cualitativas como cuantitativas.

     \item Los diagramas de {\sf sectores circulares} son útiles para mostrar proporciones cuando los valores son bastante distintos entre sí. Pero en muchas ocasiones pueden resultar confusos o poco precisos. Hay un ejemplo en este fichero \textattachfile{HeadFirst-p008.ods}{\textcolor{blue}{Calc}} (versión \textattachfile{HeadFirst-p008.xls}{\textcolor{blue}{Excel}}).
         \begin{center}
         \includegraphics[height=6cm]{2011_09_21_Figura01.PNG}
         \end{center}
    \item Los {\sf diagramas de barras} tienen, en general más precisión que los de sectores. Además se pueden utilizar para mostrar varios conjuntos de datos simultáneamente. Ver un ejemplo en este fichero \textattachfile{HeadFirst-p010.ods}{\textcolor{blue}{Calc}} (versión \textattachfile{HeadFirst-p010.xls}{\textcolor{blue}{Excel}}).
        \begin{center}
        \includegraphics[height=7cm]{2011_09_21_Figura02.PNG}
        \end{center}


\end{itemize}

\subsection{Histogramas}

\begin{itemize}

    \item Un {\sf histograma} es un tipo especial de diagrama de barras que se utiliza para variables cualitativas agrupadas en intervalos. Las dos propiedades que definen el histograma son:
        \begin{enumerate}
            \item Las {\em bases de cada una de las barras se corresponden con los intervalos} en los que hemos dividido el rango de valores de la variable continua.
            \item El {\em área de cada barra es proporcional a la frecuencia correspondiente a ese intervalo.}
        \end{enumerate}
         \begin{center}
         \includegraphics[height=7cm]{2011_09_21_Figura03.PNG}
         \end{center}

    \item Dos observaciones adicionales: en primer lugar, puesto que los intervalos deben cubrir todo el recorrido de la variable, en un histograma no hay espacio entre las barras. Y, como práctica recomendable, para que la visualización sea efectiva, no es conveniente utilizar un histograma con más de 10 o 12 intervalos como máximo.

    \item En el caso de {\sf variables cuantitativas discretas}, normalmente los intervalos se extienden a valores intermedios (que la variable no puede alcanzar) para que no quede espacio entre las barras del histograma.

    \item Los pasos para obtener el histograma son estos:
        \begin{enumerate}
            \item Si no nos lo dan hecho, debemos empezar determinar los intervalos. Para ello podemos localizar el valor máximo y el mínimo de los valores, restarlos y obtenemos el {\em rango}.
            \item Dividimos el rango entre el número de intervalos deseados, para obtener la longitud de cada uno de los intervalos. Construimos los intervalos y la tabla de frecuencias correspondiente.
            \item Calculamos la altura de cada barra, teniendo en cuenta que área=base $\cdot$ altura, y que el área (¡no la altura!) es proporcional a la frecuencia. Por lo tanto podemos usar:
                \[\fbox{$\mbox{altura}=\dfrac{\mbox{frecuencia}}{\mbox{base}}=\dfrac{\mbox{frecuencia del intervalo}}{\mbox{anchura del intervalo}.}$}\]
                para calcular la altura de cada una de las barras.
        \end{enumerate}

    \item Pronto aprenderemos a obtener muchos de los resultados de la clase de hoy usando R. Como aperitivo, aquí tenemos un \textattachfile{Sesion001.R}{\textcolor{blue}{fichero}} de instrucciones para el ejemplo que hemos usado hoy.

\end{itemize}

%\section*{Tareas asignadas para esta sesión.}
%
%\begin{enumerate}
% \item Si todavía no lo has hecho, inscríbete en Moodle y completa la tarea de ayer.
% \item Descarga e instala R y R-commander en tu ordenador. Si tienes problemas, pregunta en el foro de Moodle.
% \item En clase no hemos discutido el concepto de frecuencias acumuladas. Búsca su significado en alguna de las fuentes bibliográficas, y trata de obtener tablas de frecuencias acumuladas para algunos de los ejemplos que hemos visto en clase.
% \item En Moodle tienes un enlace para descargarte un fichero aleatorio personalizado (uno distinto para cada uno de vosotros), con datos de los alumnos de una clase ficiticia. Abre el fichero con una hoja de cálculo, calcula el peso medio y la edad media de los alumnos, y dibuja un diagrama de barras de la variable edad. (¿Qué opinas de ese diagrama?) Los resultados deben aparecer en la propia hoja de cálculo (tendrás que grabarla con otra extensión; si tienes dudas, ya sabes: al foro de Moodle). Después usa el enlace que aparece en las tareas de Moodle para hoy, para subir el fichero modificado con tus resultados (no cambies el nombre del fichero).
%\end{enumerate}



    \chapter{Valores centrales y dispersión}
    % !Mode:: "Tex:UTF-8"

%\setcounter{section}{0}
%\section*{\fbox{\colorbox{Gris025}{{Sesión 3. Estadística descriptiva.}}}}
%
%\subsection*{\fbox{\colorbox{Gris025}{{Valores centrales: la media aritmética .}}}}
%
%\subsection*{Fecha: Viernes, 23/09/2011, 14h.}
%
%\noindent{\bf Atención:
%\begin{enumerate}
%\item \textcolor{red}{En la última sesión llegamos hasta el final de la Sección 2 (Estadística descriptiva. Tipos de Variables.) La sección 3, así como las tareas 3 y 4 de aquella sesión, corresponden a esta sesión del viernes.}
%\item Este fichero pdf lleva adjuntos los ficheros de datos necesarios para la clase de hoy, que se abren usando los enlaces que contiene.
%\end{enumerate}
%}
%
%%\subsection*{\fbox{1. Ejemplos preliminares }}
%\setcounter{tocdepth}{1}
%%\tableofcontents

\section{La media aritmética (y otros conceptos de media).}

\subsection*{Contenido:}
\begin{itemize}
 \item Definición de la media aritmética.
 \item La media aritmética a partir de una tabla de frecuencias.
\end{itemize}

\subsection{Definición de la media aritmética.}
\begin{itemize}
    \item La idea de media aritmética apenas necesita presentación. Dados $n$ valores de una {\sf variable cuantitativa}, sean $x_1,x_2,\ldots,x_n$, su {\sf media aritmética} es:
        \[\fbox{$\bar x=\dfrac{x_1+\cdots+x_n}{n}=\dfrac{\displaystyle\sum_{i=1}^nx_i}{n}.$}\]

        Algunos comentarios sobre la notación. El símbolo $\bar x$ refleja la notación establecida en Estadística: la media de una variable se representa con una barra sobre el nombre de esa variable. Y el símbolo $\displaystyle\sum_{i=1}^n$, que espero que ya conozcáis, es un {\sf sumatorio}, y representa en forma abreviada, la frase ``suma todos estos valores $x_i$ donde $i$ es un número que va desde 1 hasta $n$''.
    \item Insistimos en esto: la {\bf media aritmética sólo tiene sentido para variables cuantitativas} (discretas o continuas). Aunque una variable cualitativa se represente numéricamente, la media
        aritmética de esos números seguramente sea una cantidad sin ningún significado estadístico.
    \item Las Hojas de cálculo incluyen normalmente una función {\tt PROMEDIO()} que calcula la media aritmética de un rango de celdas de la hoja. Además suelen incluir otras funciones promedio, que suponen pequeñas modificaciones sobre esta función, y que es bueno conocer.

        Para practicar esto, en el \textattachfile{Grullas01.ods}{\textcolor{blue}{fichero adjunto}} se han recogido unos datos (ficticios) sobre los tamaños de los grupos de Grulla Común que se han visto abandonando el dormidero de la Laguna de Gallocanta por la mañana\footnote{Si vives en España y no sabes de que va esto de las grullas de Gallocanta, creeme que te estás perdiendo algo}. ¿Cuál es el tamaño medio del grupo?
\end{itemize}


\subsection{La media aritmética a partir de una tabla de frecuencias.}

\begin{itemize}
    \item Supongamos que queremos calcular la {\sf media a partir de la tabla de frecuencias de una variable cuantitativa}, como esta:
   \begin{center}
        \begin{tabular}{|c|c|}
        \hline
        \rule{0cm}{4mm}{\bf Intervalo}&{\bf Frecuencia}\\ \hline
        \rule{0cm}{4mm}$x_1$&$f_1$\\[2mm] \hline
        \rule{0cm}{4mm}$x_2$&$f_2$\\[2mm] \hline
        $\vdots$&$\vdots$\\[2mm] \hline
        \rule{0cm}{4mm}$x_k$&$f_k$\\[2mm] \hline
        \end{tabular}
        \end{center}
        Aquí los valores {\em distintos} de la variable\footnote{Acuérdate de que tenemos $n$ observaciones de la variable, pero puede haber valores repetidos. Aquí estamos usando el número de valores distintos, sin repeticiones, y ese número es $k$.} son $x_1,\ldots,x_k$ y sus frecuencias absolutas respectivas son $f_1,f_2,\ldots,f_k$. Está claro entonces que:
        \[f_1+f_2+\cdots+f_k=(\mbox{nro. de observ. de }x_k)+\cdots+(\mbox{nro. de observ. del valor }x_k)= \]
        \[=(\mbox{suma del número de observaciones de todos los valores distintos})=n\]
        Recordemos que para calcular la media tenemos que sumar el valor de todas (las $n$ observaciones). Y como el valor $x_i$ se ha observado $f_i$ veces, su contribución a la suma es
        \[x_i\cdot f_i=x_i+x_i+\cdots+x_i\quad (\mbox{sumamos $f_i$ veces})\]
        Teniendo en cuenta entonces la contribución cada uno de los $k$ valores distintos, vemos que
        para calcular la media debemos hacer:
        \[\fbox{$
        \bar x=\dfrac{x_1\cdot f_1+x_2\cdot f_2+\cdots+x_k\cdot f_k}{f_1+f_2+\cdots+f_k}=
        \dfrac{\displaystyle\sum_{i=1}^k x_i\cdot f_i}{\displaystyle\sum_{i=1}^k f_i}$}
        \]
        {\sf Ejemplo:} en una instalación deportiva el precio de la entrada para adultos es de \EUR{10} y de \EUR{4} para menores. Hoy han visitado esa instalación $230$ adultos y $45$ menores. ¿Cuál es el ingreso medio por visitante que recibe esa instalación?\\
        Tenemos dos posibles valores de la variable $x=${\em precio de la entrada}, que son $x_1=10$ y $x_2=4$. Además sabemos las frecuencias correspondientes: $f_1=230$ y $f_2=45$. Por lo tanto:
        \[\bar x=\dfrac{x_1\cdot f_1+x_2\cdot f_2}{f_1+f_2}=\dfrac{10\cdot 230+4\cdot 45}{230+45}=9.02\]
        El ingreso medio es de \EUR{9.02} por visitante.$\Box$

    \item Si lo que queremos es calcular la {\sf media aritmética a partir de la tabla de frecuencias agrupadas por intervalos de una variable cuantitativa}, las cosas son --sólo un poco-- más complicadas. En este caso vamos a tener una tabla de frecuencias por intervalos\footnote{los intervalos a veces se llaman también {\sf clases}.} como esta:
        \begin{center}
        \begin{tabular}{|c|c|}
        \hline
        \rule{0cm}{4mm}{\bf Intervalo}&{\bf Frecuencia}\\ \hline
        \rule{0cm}{4mm}$[a_1,b_1)$&$f_1$\\[2mm] \hline
        \rule{0cm}{4mm}$[a_2,b_2)$&$f_2$\\[2mm] \hline
        $\vdots$&$\vdots$\\[2mm] \hline
        \rule{0cm}{4mm}$[a_k,b_k)$&$f_k$\\[2mm] \hline
        \end{tabular}
        \end{center}
        Comparando esta tabla con el caso anterior está claro que lo que nos falta son los valores $x_1,\ldots,x_k$ y, en su lugar, tenemos los intervalos $[a_1,b_1),\ldots,[a_k,b_k)$. Lo que hacemos en estos casos es {\em fabricar} unos valores $x_i$ a partir de los intervalos. Se toma como valor $x_i$ el punto medio del intervalo $[a_i,b_i)$; es decir:
        \[\fbox{$x_i=\dfrac{a_i+b_i}{2}$,\quad para $i=1,\ldots,n.$}\]
        Estos valores $x_i$ se denominan {\sf marcas de clase} (o marcas de intervalo). Una vez calculadas las marcas de clase, podemos usar la misma fórmula que en el caso anterior.

        Para practicar, aquí tienes un \textattachfile{Spiegel-p073.ods}{\textcolor{blue}{fichero Calc}}  en el que aparecen datos sobre la altura (en pulgadas) de los estudiantes de un instituto americano.  El fichero contiene la tabla de frecuencias agrupadas por intervalo de la variable altura, y a partir de ella debemos calcular las marcas de clase y la media. Aquí tienes \textattachfile{Spiegel-p073-Solucion.ods}{\textcolor{blue}{la solución}}.
\end{itemize}


%\section*{Tareas asignadas para esta sesión.}
%
%\begin{enumerate}
% \item \textcolor{red}{Las tareas que aparecían como 3 y 4 en la anterior sesión se han desplazado a esta sesión.}
% \item Completa el cuestionario sobre medias aritméticas que encontrarás en Moodle, en el apartado correspondiente a esta sesión.
% \item Ya está disponible la primera hoja de Ejercicios, y la semana que viene empezamos las prácticas. Hay una probabilidad muy alta (para eso estamos en Estadística) de que sea conveniente ir a la clase de prácticas habiendo pensado cómo se hacen los ejercicios de esta hoja.
%\end{enumerate}


%\section*{\fbox{\colorbox{Gris025}{{Sesión 4. Estadística descriptiva.}}}}
%
%\subsection*{\fbox{\colorbox{Gris025}{{Más valores centrales. Medidas de dispersión, primera parte.}}}}
%\subsection*{Fecha: Martes, 27/09/2011, 14h.}
%
%\noindent{\bf Atención:
%\begin{enumerate}
%\item \textcolor{red}{En la última sesión llegamos hasta el final del resumen, pero nos falta practicar algunos ejemplos de cálculo de medias aritméticas para datos agrupados. Empezaremos con esto.}
%\item Este fichero pdf lleva adjuntos los ficheros de datos necesarios.
%\end{enumerate}
%}
%
%%\subsection*{\fbox{1. Ejemplos preliminares }}
%\setcounter{tocdepth}{1}
%%\tableofcontents
%\section*{Lectura recomendada\footnote{Además de la parte restante de la anterior sesión.}}

%Al menos uno de los siguientes:
%    \begin{itemize}
%    \item Capítulo 2 de "La estadística en Comic" (hasta el final).
%    \item Capítulo 2 de Head First Statistics (pág. 53 al final) y Capítulo 3.
%    \item Tema 2 de Bioestadística: Métodos y Aplicaciones, Univ. de Málaga (cubre aspectos que nosotros apenas vamos a tratar).
%    \item Apuntes de la segunda sesión del Curso 2010-2011, apartado 4 (desde la pág. 10 hasta el final) y tercera sesión hasta la página 20.
%
%    \end{itemize}

\section{Otros valores centrales. Mediana, percentiles, moda.}

\subsection*{Contenido:}
\begin{itemize}
 \item Mediana.
 \item Moda
 %\item Opcional. Otras medias: media geométrica, armónica, etcétera.
\end{itemize}


\subsection{Mediana. }

\begin{itemize}
    \item Como en el caso de la media aritmética, vamos a suponer que tenemos $n$ observaciones de una variable cuantitativa:
        \[x_1,x_2,\ldots,x_n.\]
        Antes de seguir: la variable ha de ser cuantitativa, y estamos suponiendo que los datos no están agrupados en una tabla de frecuencia. Más abajo veremos el caso de datos agrupados.

        Como los $x_i$ son números, vamos a suponer que los hemos ordenado de menor a mayor:
        \[x_1\leq x_2\leq\cdots\leq x_{n-1}\leq x_n.\]
        Entonces, la {\sf mediana} (inglés: median) de ese conjunto de datos es el {\em valor central} de esa serie ordenada. Es decir:
        \begin{itemize}
            \item[]{\bf Caso impar:} si tenemos una cantidad impar de datos, sólo hay un valor central, y ese es la mediana. Por ejemplo, para siete datos:
                \[
                \begin{array}{rcl}
                \underbrace{x_1\leq x_2\leq x_3}_{\mbox{mitad izda.}}\leq &\hspace{-7mm}\textcolor{red}{x_4}&\hspace{-7mm}\leq
                \underbrace{x_5\leq x_6\leq x_7}_{\mbox{mitad dcha.}}\\[-5mm]
                &\hspace{-7mm}\textcolor{red}{\uparrow}&\\
                &\hspace{-7mm}\textcolor{red}{\mbox{\small mediana}}&
                \end{array}
                \]
            \item[]{\bf Caso par:} Por contra, si el número de datos es par, entonces tomamos el valor máximo de la mitad izquierda, y el valor mínimo de la mitad derecha y hacemos la media. Por ejemplo, para seis datos:
                \[
                \begin{array}{rcl}
                \underbrace{x_1\leq x_2\leq x_3}_{\mbox{mitad izda.}}\leq &\hspace{0mm}\textcolor{red}{\dfrac{x_3+x_4}{2}}&\hspace{0mm}\leq
                \underbrace{x_4\leq x_5\leq x_6}_{\mbox{mitad dcha.}}\\[-3mm]
                &\hspace{0mm}\textcolor{red}{\uparrow}&\\
                &\hspace{0mm}\textcolor{red}{\mbox{\small mediana}}&
                \end{array}
                \]
        \end{itemize}
        En el caso de un número impar de datos la mediana siempre coincide con uno de los datos originales. Pero en el caso de un número par de datos la mediana pueden darse los dos casos. Por ejemplo, si tenemos estos seis datos ordenados:
        \[
        2\leq 5\leq 6\leq 7\leq 11\leq 15,
        \]
        Entonces la mediana es $6.5$
        \[
        2\leq 5\leq 6\leq\textcolor{red}{\mbox{\Large\bf 6.5}}\leq 7\leq 11\leq 15,
        \]
        que no aparecía en el conjunto original (fíjate en particular en que, como pasaba con la media aritmética, aunque todos los datos originales sean enteros, la mediana puede no serlo). Mientras que si tenemos estos seis datos, con los dos datos centrales iguales:
        \[
        2\leq 5\leq 6\leq 6\leq 11\leq 15,
        \]
        Entonces la mediana es $6$, que ya estaba (repetido) entre los datos originales.
        \[
        2\leq 5\leq 6\leq\textcolor{red}{\mbox{\Large\bf 6}}\leq 8\leq 11\leq 15,
        \]

        \item ¿Qué {\sf ventajas} aporta la mediana frente a la media aritmética? Fundamentalmente, la mediana se comporta mejor cuando el conjunto de datos contiene {\sf datos atípicos} (inglés: outliers). Es decir, datos cuyo valor se aleja {\em mucho} de la media. Todavía no podemos precisar esto porque para hacerlo necesitamos la noción de medidas de dispersión que vamos a ver en la próxima sección. Pero la idea intuitiva es que si tenemos un conjunto de datos, e introducimos un dato adicional que se aleja mucho de la media aritmética inicial, entonces en el nuevo conjunto de datos podemos tener una media aritmética bastante distinta de la inicial. En cambio la mediana sufre modificaciones mucho menores frente a esos datos atípicos.

            Para observar este comportamiento podéis abrir el \textattachfile{MediaMedianaOutliers.html}{\textcolor{blue}{fichero html adjunto}} (se abre en el navegador, y requiere Java), y experimentar con él.

            \item En el caso de que queramos {\sf calcular la mediana a partir de la tabla de frecuencias para los distintos valores de una variable cuantitativa discreta}, construimos la tabla de frecuencias relativas acumuladas (f.r.a.):
                \begin{center}
                    \begin{tabular}{|c|c|c|}
                    \hline
                    \rule{0cm}{4mm}{\bf Intervalo}&{\bf Frecuencia}&\textcolor{red}{\bf F.r.a.}\\ \hline
                    \rule{0cm}{4mm}$x_1$&$f_1$&$\textcolor{red}{g_1}$\\[2mm] \hline
                    \rule{0cm}{4mm}$x_2$&$f_2$&$\textcolor{red}{g_2}$\\[2mm] \hline
                    $\vdots$&$\vdots$&$\textcolor{red}{\vdots}$\\[2mm] \hline
                    \rule{0cm}{4mm}$x_k$&$f_k$&$\textcolor{red}{g_k=1}$\\[2mm] \hline
                    \end{tabular}
                \end{center}
            ¿Qué que es eso de las frecuencias relativas acumuladas? En definitiva, se trata de los {\em tantos por uno acumulados}: es decir, que para cada uno de los valores $x_1,\ldots, x_n$ vamos sumando las frecuencias de ese valor y {\sf de todos los que le preceden}. En fórmulas:
            \[g_1=\dfrac{f_1}{n},\quad g_2=\dfrac{f_1+f_2}{n},\quad g_3=\dfrac{f_1+f_2+f_3}{n},\quad\mbox{etc.}\]
            En el \textattachfile{Mediana-TablaFrecRelAcum.ods}{\textcolor{blue}{fichero adjunto}} puedes ver cómo se hace se cálculo para ua tabla de frecuencias (de valores  aleatorios). Cada vez que se abre el fichero genera un conjunto de datos distinto, así que puedes recargarlo varias veces para ver ejemplos variados.

        \item ¿Y si lo que necesitamos es {\sf calcular la mediana a partir de la tabla de frecuencias de una variable cuantitativa}, pero agrupada en intervalos? En este caso las cosas se complican un poco. Si hemos entendido la idea de histograma, esta forma de verlo nos puede ayudar: la mediana es el valor de la variable (por lo tanto es el punto del eje horizontal) que divide el histograma en dos mitades con el mismo área. Existen fórmulas para calcular la mediana en estos casos (usando interpolación). Pero aquí no nos vamos a entretener.

\end{itemize}

\subsection{Moda.}

\begin{itemize}
    \item La media aritmética y la mediana se utilizan para variables cuantitativas. La moda en cambio puede utilizarse además con variables de tipo cualitativo (y es, de los que vamos a ver, el único tipo de valor promedio que puede usarse con variables cualitativas). {\sf La moda de una serie de valores agrupados en una tabla de frecuencias es el valor con la frecuencia más alta.}

    \item Puesto que puede haber dos o más valores que tengan la misma frecuencia, hay conjuntos de datos que tienen más de una moda.

    \item El cálculo de la moda (o modas) es inmediato a partir de las tablas de frecuencias. Aquí tienes \textattachfile{titanic.csv}{\textcolor{blue}{adjunto un fichero}} con datos de viajeros del Titanic (tomado de los ejemplos que acompañan al libro {\em Estadística Básica con R y R–Commander} de la Univ. de Cadiz). Para practicar podéis calcular la moda de todas las variables que aparecen en él (aunque algunas modas son evidentes por simple inspección de los datos: la mayoría de los viajeros eran adultos, por ejemplo).
\end{itemize}


%\subsection{Opcional. Otras medias: media geométrica, armónica, etcétera.}
%\begin{itemize}
%    \item La {\sf media geométrica} de los números $x_1,x_2,\ldots,x_n$ es la raíz n-ésima del producto de esos números:
%    \[\sqrt[n]{x_1\cdot x_2\cdot\cdots\cdot x_n}=\sqrt[n]{\prod_{i=1}^n{x_i}}\]
%    (El símbolo $\prod$ dentro de la segunda raíz es un {\sf productorio}, el análogo del sumatorio para el producto.)
%    El logaritmo de la media geométrica es igual a la media aritmética de los logaritmos de los valores de la variable.
%
%    Ventajas: considera todos los valores de la distribución y es menos sensible que la media aritmética a los valores extremos.
%
%    Desventajas: es de significado estadístico menos intuitivo que la media aritmética, su cálculo es más difícil y
%        en ocasiones no queda determinada; por ejemplo, si un valor es nulo, entonces la media geométrica se anula. Solo es relevante la media geométrica si todos los números son positivos.
%
%\end{itemize}

\section{Medidas de dispersión. Primera parte}

\subsection*{Introducción}

Hasta ahora hemos estado calculando valores que nos sirvieran como representantes de una colección de datos. Sin embargo, es fácil entender que un mismo valor de la media aritmética o de la mediana, etcétera, puede corresponder a muchas colecciones de datos distintas. {\sf No sólo necesitamos un valor representativo, además necesitamos una forma de medir la calidad de ese representante.} ¿Cómo podemos hacer esto? La idea que vamos a utilizar es la de {\sf dispersión}. Una colección de números es poco dispersa cuando los datos están muy concentrados alrededor de la media. Pero tenemos que concretar más ¿cómo podemos pedir eso? En esta sección vamos a introducir varios métodos de medir la dispersión de una colección de datos.

\subsection{Rango, Cuartiles y Percentiles.}

\begin{itemize}

    \item La idea más elemental de dispersión es el {\sf rango}, que ya hemos encontrado al pensar en las representaciones gráficas. El rango es simplemente la diferencia entre el máximo y el mínimo de los valores. Es una manera rápida, pero excesivamente simple de analizar la dispersión de los datos, porque depende exclusivamente de dos valores (el máximo y el mínimo), que pueden ser casos muy excepcionales.

    \item Hemos visto que la mediana es el valor que deja a la mitad de los datos a cada lado. Esta idea se puede generalizar fácilmente: el valor que deja al primer cuarto a su izquierda es el {\sf primer cuartil} del conjunto de datos. Dicho de otra forma: la mediana divide a los datos en dos mitades, la mitad izquierda y la mitad derecha. Pues entonces el primer cuartil es la mediana de la mitad izquierda. Y de la misma forma el {\sf tercer cuartil} es la mediana de la mitad derecha. Y por tanto es el valor que deja a su derecha al último cuarto de los datos\footnote{ Por si te lo estás preguntando, sí, la mediana es el segundo cuartil, pero nadie la llama así, claro.}.

    \item La mediana y los cuartiles son los valores que señalan la posición del $25\%$, el $50\%$ y el $75\%$ de los datos. Se pueden utilizar los cuartiles  para medir la dispersión de los datos, calculando el {\sf rango intercuartílico} (en inglés interquartile range, IQR), que es la diferencia entre el tercer y el primer cuartil.

    \item El {\sf cálculo de los cuartiles} se basa en los mismos principio que el de la mediana (porque como hemos visto se trata de medianas).

    \item Los datos que son mucho menores que el primer cuartil o mucho mayores que el tercer cuartil se consideran atípicos. ¿Cómo de lejos tienen que estar de los cuartiles para considerarlos {\em raros o excepcionales}? La forma habitual de proceder es considerar que {\sf un valor mayor que el tercer cuartil, y cuya diferencia con ese cuartil es mayor que $1.5$ veces el rango intercuartílico es un valor atípico} (en inglés, outlier).  De la misma forma, también es un valor atípico aquel valor menor que el tercer cuartil, cuya diferencia con ese cuartil es mayor que $1.5\cdot$IQR.

    \item La mediana, los cuartiles y el rango intercuartílico se utilizan para dibujar los diagramas llamados de {\sf caja y bigotes} (boxplot en inglés), como el que se muestra más abajo. En estos diagramas se dibuja una caja cuyos extremos son el primer y tercer cuartiles. Dentro de esa caja se dibuja el valor de la mediana. Los valores atípicos se suelen mostrar como puntos individuales (fuera de la caja, claro), y finalmente se dibujan segmentos que unen la caja con los datos más alejados que no son atípicos.
         \begin{center}
         \includegraphics[height=8cm]{2011_09_27_Figura01_BoxPlot.png}
         \end{center}

    \item La idea de los cuartiles se puede generalizar fácilmente. Como hemos dicho, el primer cuartil deja a su izquierda el $25\%$ de los datos. Si pensamos en el valor que deja a su izquierda el $10\%$ de los datos, estamos pensando en un {\sf percentil.} Los percentiles se suelen dar en porcentajes, pero también en tantos por uno, es decir en números comprendidos entre 0 y 1.

    \item Las hojas de cálculo incluyen funciones para calcular la mediana, los cuartiles y los percentiles de un conjunto de datos. En este \textattachfile{CuartilesPercentiles.ods}{\textcolor{blue}{fichero Calc}} tienes una colección de datos aleatorios (serán distintos en cada apertura del fichero), y funciones para calcular su mediana, cuartiles y percentiles. Y, para irnos iniciando en el uso de R, aquí tienes un \textattachfile{DatosParaBoxPlot.csv}{\textcolor{blue}{fichero de datos}} y un \textattachfile{Sesion002.R}{\textcolor{blue}{fichero de instrucciones R}} con los que practicar el cálculo de medianas, cuartiles, percentiles y los diagramas de cajas.
\end{itemize}


%\subsection{Varianza.}
%
%\begin{itemize}
%    \item
%\end{itemize}
%
%\subsection{Desviación típica.}
%
%\begin{itemize}
%    \item
%\end{itemize}




%\section*{Tareas asignadas para esta sesión.}
%
%\begin{enumerate}
% \item Completa el cuestionario sobre medias, medianas, etcétera que encontrarás en Moodle, en la sesión de hoy.
% \item Con el mismo fichero de datos que usaste para la Tarea 2 del viernes 23/09, usa R-Commander para dibujar un diagrama de cajas de la variable peso. Guarda ese diagrama como un fichero con el mismo nombre, pero con gráfico (con extension png o jpg; si no te aclaras, ya sabes, al foro a por ayuda).
%\end{enumerate}
%
%
%
%
%\section*{\fbox{\colorbox{Gris025}{{Sesión 4. Estadística descriptiva.}}}}
%
%\subsection*{\fbox{\colorbox{Gris025}{{Medidas de dispersión: varianza y desviación típica.}}}}
%\subsection*{Fecha: Viernes, 30/09/2011, 14h.}
%
%\noindent{\bf Atención:
%\begin{enumerate}
%\item \textcolor{red}{Hemos llegado hasta el final del punto 1.1 (la mediana), pero no hemos visto la Moda. Ni los cuartiles, percentiles, etc. de la sección 2.}
%\item \textcolor{red}{Aunque, en un momento de ofuscación, el profesor lo haya dicho en clase,\\ \underline{{\em recordad que es falso que la media de las medias sea la media}.}\\ Os pido disculpas por la metedura de pata.
%Si tenemos una colección de $n_1$ valores con media $\bar x_1$, y otra colección de $n_2$ valores (de la misma variable, claro) con media $\bar{x}_2$, la media conjunta es:
%\[\dfrac{n_1\bar x_1+n_2\bar x_2}{n_1+n_2}.\]
%}
%\item Este fichero pdf lleva adjuntos los ficheros de datos necesarios.
%\end{enumerate}
%}
%
%%\subsection*{\fbox{1. Ejemplos preliminares }}
%\setcounter{tocdepth}{1}
%%\tableofcontents
%\section*{Lectura recomendada}
%
%La misma de la anterior sesión. Es decir, al menos uno de los siguientes:
%    \begin{itemize}
%    \item Capítulo 2 de "La estadística en Comic" (hasta el final).
%    \item Capítulo 2 de Head First Statistics (pág. 53 al final) y Capítulo 3.
%    \item Tema 2 de Bioestadística: Métodos y Aplicaciones, Univ. de Málaga (cubre aspectos que nosotros apenas vamos a tratar).
%    \item Apuntes de la segunda sesión del Curso 2010-2011, apartado 4 (desde la pág. 10 hasta el final) y tercera sesión hasta la página 20.
%
%    \end{itemize}


\section{Varianza y desviación típica.}


\subsection{Varianza}

\begin{itemize}

    \item Las medidas de dispersión que hemos visto (el rango y el rango intercuartílico) se expresan en términos de cuartiles (o percentiles), y  por lo tanto tienen más que ver con la mediana que con la media aritmética. Sin embargo, uno de los objetivos más importantes -si no el más importante- de la Estadística es hacer inferencias desde una muestra a la población, como discutimos en la segunda sesión del curso. Y cuando se trata de hacer inferencias, vamos a utilizar de modo preferente la media aritmética como valor central o representativo de los datos. Por eso estas medidas de dispersión relacionadas con la mediana, y no con la media, no son las mejores para hacer inferencia. {\sf Necesitamos una medida de dispersión relacionada con la media aritmética.}

    \item Tenemos, como siempre, un conjunto de $n$ datos,
        \[x_1,x_2,\ldots,x_n\]
        que corresponden a $n$ valores de una {\sf variable cuantitativa.}
        La primera idea que se nos puede ocurrir es medir la diferencia entre cada uno de esos valores y la media (la {\em desviación individual} de cada uno de los valores):
        \[x_1-\bar x, x_2-\bar x,\ldots, x_n-\bar x,\]
        Y para tener en cuenta la contribución de todos los valores podríamos pensar en hacer la media de estas desviaciones individuales:
        \[\dfrac{(x_1-\bar x)+(x_2-\bar x)+\cdots+(x_n-\bar x)}{n}.\]
        El problema es que esta suma siempre vale cero. Vamos a fijarnos en el numerador (y recuerda la definición de media aritmética):
        \[(x_1-\bar x)+(x_2-\bar x)+\cdots+(x_n-\bar x)=(x_1+x_2+\cdots+x_n)-n\cdot\bar x=0.\]
        Está claro que tenemos que hacer algo más complicado, para evitar que el signo de unas desviaciones se compense con el de otras. A partir de aquí se nos abren dos posibilidades, usando dos operaciones matemáticas que eliminan el efecto de los signos. Podemos usar el valor absoluto de las desviaciones individuales:
        \[\dfrac{|x_1-\bar x|+|x_2-\bar x|+\cdots+|x_n-\bar x|}{n},\]
        o podemos elevarlas al cuadrado:
        \[\dfrac{(x_1-\bar x)^2+(x_2-\bar x)^2+\cdots+(x_n-\bar x)^2}{n}.\]
        Las razones para elegir entre una u otra alternativa son técnicas: vamos a usar la que mejor se comporte para hacer inferencias. Y esa resulta ser la opción que utiliza los cuadrados.

    \item La {\sf varianza (o desviación cuadrática media)} del conjunto de datos $x_1,x_2,\ldots,x_n$ es:
        \begin{equation}
        \fbox{$v=\mbox{Var($x$)}=\dfrac{(x_1-\bar x)^2+(x_2-\bar x)^2+\cdots+(x_n-\bar x)^2}{n}=\dfrac{\displaystyle\sum_{i=1}^n(x_i-\bar x)^2}{n}.$}
        \end{equation}

    \item En muchos libros veréis que se presenta una cantidad llamada {\sf varianza muestral} mediante la fórmula
            \[\fbox{$s^2=\dfrac{\displaystyle\sum_{i=1}^n(x_i-\bar x)^2}{\textcolor{red}{\bf n-1}}.$}\]
            Este concepto será importante cuando hablemos de inferencia, y entonces entenderemos el papel que juega la varianza muestral, y su relación con la varianza tal como la hemos definido. Lo que sí es {\sf\large muy importante}, usando software o calculadoras, es que sepamos si el número que se obtiene es la varianza o la varianza muestral.

     \item {\sf Método abreviado de cálculo:} desarrollando los cuadrados con la fórmula del binomio, y usando un poco de álgebra se puede obtener esta otra fórmula para la varianza:
           \[v=\mbox{Var($x$)}=\dfrac{\displaystyle\sum_{i=1}^n x_i^2}{n}-(\bar x)^2\]
           Esta fórmula es un poco más eficiente, y es mucho mejor cuando las cuentas se hacen con calculadoras no avanzadas (que no incluyen directamente el cálculo de la varianza de una colección de datos, pero sí saben calcular la suma de los cuadrados, por ejemplo). \\

           De la misma forma, para la varianza muestral se obtiene una fórmula abreviada un poco más complicada:
           \[s^2=\dfrac{\displaystyle\left(\sum_{i=1}^n x_i^2\right)-n(\bar x)^2}{n-1}.\]

     \item ¿Tenemos que recordar estas fórmulas abreviadas? En realidad, no. Basta con recordar que existen, y saber donde buscarlas si llegaran a ser necesarias. En la mayor parte de los casos usaremos una hoja de cálculo (o software específico, como R) para esto. Lo más importante es, insistimos:
            \begin{enumerate}
            \item recordar la definición que aparece en la ecuación (1), y la diferencia entre esta y la varianza muestral
            \item saber, cuando usamos una herramienta para calcular, si el valor que obtenemos es la varianza o la varianza muestral. Siempre se puede hacer un experimento con una pequeña colección de datos para salir de dudas.
            \end{enumerate}
           Para aclarar todo esto, vamos a usar \textattachfile{VarianzasConHojaCalculo-Plantilla.ods}{\textcolor{blue}{este fichero (plantilla en Calc)}} con 20 datos aleatorios para calcular su varianza usando la hoja de cálculo, directamente y usando funciones. Y en \textattachfile{VarianzasConHojaCalculo.ods}{\textcolor{blue}{este otro}}  un fichero con los cálculos ya incorporados (cuidado, son otros datos salvo que los copiemos sin cerrar el original).

    \item Cuando lo que tenemos son datos descritos mediante una tabla de frecuencias, debemos proceder así:
            \begin{enumerate}
            \item la fórmula (1) se sustituye por:
                 \[\fbox{$v=\mbox{Var($x$)}=\dfrac{\displaystyle\sum_{i=1}^k\textcolor{red}{\bf f_i\cdot}(x_i-\bar x)^2}{\textcolor{red}{\bf \displaystyle\sum_{i=1}^k f_i}}.$}\]
                 donde, ahora, $x_1,\ldots,x_k$ son los valores {\em distintos} de la variable, y $f_1,\ldots,f_k$ son las correspondientes frecuencias.
            \item en las fórmulas abreviadas, la suma de valores de la variable se sustituye por $\sum_{i=1}^kx_i\cdot f_i$, la suma de cuadrados por $\sum_{i=1}^kx_i^2\cdot f_i$ (cuidado: no hay que elevar la frecuencia al cuadrado), y debemos recordar que $n=\sum_{i=1}^k f_i$.
            \item en el caso de datos agrupados por intervalos, los valores $x_i$ que utilizaremos serán las marcas de clase.\\
            Podemos practicar esto con esta  \textattachfile{EjemploCalculoVarianza.csv}{\textcolor{blue}{tabla de datos (formato csv)}}. Y aquí tenemos una \textattachfile{EjemploCalculoVarianza-Plantilla.ods}{\textcolor{blue}{plantilla (Calc)}} para calcular la desviación típica paso a paso. En este otro fichero está la  \textattachfile{EjemploCalculoVarianza-Solucion.ods}{\textcolor{blue}{solución}}.
            \end{enumerate}



\end{itemize}


\subsection{Desviación típica.}

\begin{itemize}
    \item La varianza, como medida de dispersión, tiene un grave inconveniente: puesto que hemos elevado al cuadrado, las unidades en las que se expresa son el cuadrado de las unidades originales en las que se medía la variable $x$. Y nos gustaría que una medida de dispersión nos diera una idea de, por ejemplo, cuantos metros se alejan de la media los valores de una variable medida en metros. Dar la dispersión en metros cuadrados es, cuando menos, extraño. Por esa razón, entre otras, vamos a necesitar una nueva definición.

    \item La {\sf desviación típica} es la raíz cuadrada de la varianza:
            \begin{equation}
        \fbox{$\mbox{DT($x$)}=\sqrt{\mbox{Var($x$)}} =\sqrt{\dfrac{\displaystyle\sum_{i=1}^n(x_i-\bar x)^2}{n}}.$}
        \end{equation}
        También existe una {\sf desviación típica muestral}, calculada a partir de la varianza muestral.

    \item El cálculo de la desviación típica tiene las mismas características que el de la varianza. Y, de nuevo, es {\sf\large muy importante}, usando software o calculadoras, es que sepamos si el número que se obtiene es la desviación típica o la desviación típica muestral.


\end{itemize}


%\section*{Lectura recomendada para este capítulo}
%
%Al menos uno de los siguientes:
%    \begin{itemize}
%    \item Capítulo 2 de "La estadística en Comic" (desde la pág 13 a la 18).
%    \item Capítulo 2 de Head First Statistics.
%    \item Tema 2 de Bioestadística: Métodos y Aplicaciones, Univ. de Málaga; es mejor esperar hasta el final de la próxima sesión, para poder leerlo sin saltos.
%    \item Apuntes de la segunda sesión del Curso 2010-2011, apartado 4 (hasta la pág. 14).
%    \item Capítulo 2 de "La estadística en Comic" (hasta el final).
%    \item Capítulo 2 de Head First Statistics (pág. 53 al final) y Capítulo 3.
%    \item Tema 2 de Bioestadística: Métodos y Aplicaciones, Univ. de Málaga (cubre aspectos que nosotros apenas vamos a tratar).
%    \item Apuntes de la segunda sesión del Curso 2010-2011, apartado 4 (desde la pág. 10 hasta el final) y tercera sesión hasta la página 20.
%    \end{itemize}




%
%\section*{Tareas asignadas para esta sesión.}
%
%\begin{enumerate}
%%    \item Usa este \textattachfile{2011-09-30-Tarea1.ods}{\textcolor{blue}{este fichero (Calc)}} (y aquí la \textattachfile{2011-09-30-Tarea1.xls}{\textcolor{blue}{versión Excel}}) para obtener una lista de 50 números. Usando este fichero de instrucciones como plantilla, haz las modificaciones necesarias y usa R-Commander para dibujar un diagrama de cajas de esos datos. (\Pendiente{más detalles en Moodle}). Guarda ese diagrama como un fichero con el mismo nombre, pero con gráfico (con extension png o jpg; si no te aclaras, ya sabes, al foro a por ayuda).
%     \item Completa el cuestionario sobre cuartiles, varianzas y desviaciones típicas que encontrarás en Moodle, en la sesión de hoy (hasta el 11/10/2011).
%     \item Puede haber una tarea adicional para la sesión de hoy, compruébalo en Moodle.
%%     \item IDEA 1: que hagan un ejercicio de la primera hoja, y que lo entreguen rápido.
%%     \item IDEA 2: otro con ficheros y php de por medio :)
%\end{enumerate}
%
%


\part{Probabilidad y variables aleatorias}
%   \input{002-ProbabilidadVariablesAleatorias}

    \chapter{Probabilidad}
    % !Mode:: "Tex:UTF-8"

%\setcounter{section}{0}
%\section*{\fbox{\colorbox{Gris025}{{Sesión 7. Probabilidad.}}}}
%
%\subsection*{\fbox{\colorbox{Gris025}{{Nociones básicas de probabilidad.}}}}
%\subsection*{Fecha: Viernes, 07/10/2011, 14h.}
%
%\noindent{\bf Atención:
%\begin{enumerate}
%\item Este fichero pdf lleva adjuntos los ficheros de datos necesarios.
%\end{enumerate}
%}
%
%%\subsection*{\fbox{1. Ejemplos preliminares }}
%\setcounter{tocdepth}{1}
%%\tableofcontents

\section{El lenguaje de la Probabilidad.}


\subsection{El papel de la Probabilidad en la Estadística.}

\begin{itemize}

    \item Hemos venido diciendo desde el principio del curso que el objetivo más importante de la Estadística es realizar inferencias. Recordemos en que consiste esa idea: estamos interesados en estudiar un fenómeno que ocurre en una determinada {\sf población}. En este contexto, población no se refiere sólo a seres vivos. Si queremos estudiar la antigüedad del parque móvil de España, la población la forman todos los vehículos a motor del país (cada vehículo es un individuo). Si queremos estudiar la dotación tecnológica de los centros de secundaria, la población la forman todos los institutos, y cada instituto es un individuo de esa población. En general, resulta imposible, indeseable o inviable estudiar uno por uno todos los individuos de la población. Por esa razón, lo que hacemos es obtener información sobre una {\sf muestra} de la población. Es decir, un subconjunto de individuos de la población original, de los que obtenemos información sobre el fenómeno que nos interesa.

    \item Tenemos que distinguir por lo tanto, en todo lo que hagamos a partir de ahora, qué afirmaciones se refieren a la población (la colección completa) y cuáles se refieren a la muestra. Los únicos datos a los que realmente tendremos acceso son los de la muestra (o muestras) que hayamos obtenido. La muestra nos proporcionará datos sobre alguna variable (o variables) relacionadas con el fenómeno que estamos estudiando. Es decir, que podemos empezar pensando que en la muestra tenemos, como en todo lo que hemos hecho hasta ahora un conjunto de $n$ datos,
        \[x_1,x_2,\ldots,x_n.\]

    \item En el ejemplo del parque móvil, podríamos haber obtenido las fichas técnicas de 1000 vehículos (la población completa consta de cerca de 28 millones de vehículos\footnote{En concreto, 27963880, según datos del informe de \link{http://www.anfac.com/}{ANFAC 2010}. La web de Anfac contiene una sección de Estadísticas que podéis encontrar interesante.}). Y una variable que nos puede interesar para estudiar la antigüedad del parque móvil es el año de matriculación. Así que tendríamos 1000 valores $x_1,\ldots,x_{1000}$, donde cada uno de esos valores representa el año de matriculación de un vehículo. Con esos 1000 valores podemos calcular una media, que llamaremos la {\sf media muestral}:
        \[\bar x=\dfrac{x_1+x_2+\cdots+x_{1000}}{1000}\]
        Naturalmente, si accediéramos a {\em todos} los datos, nos encontraríamos con una lista {\em mucho} más larga, de alrededor de 28 millones de números:
        \[m_1,m_2,m_3,\ldots,m_{27963880}.\]
        Y podríamos hacer la media de todos estos datos, que llamaremos la {\sf media poblacional}:
        \[\mu=\dfrac{m_1+m_2+m_3+\ldots+m_{27963880}}{27963880}.\]
        Los símbolos que hemos elegido no son casuales. Vamos a utilizar siempre $\bar x$ para referirnos a la media muestral y $\mu$ (la letra griega mu) para referirnos a la media poblacional. Este es un convenio firmemente asentado en los usuarios de la Estadística.

    \item Naturalmente, hacer esta media poblacional es mucho más difícil, complicado, caro, etcétera. Y ahí es donde aparece la idea de inferencia, que se puede formular aproximadamente así, en un sentido intuitivo:
    \begin{center}
        \fbox{\colorbox{Gris025}{
        \begin{minipage}{13cm}
        {\sf si hemos seleccionado esos 1000 coches al \textcolor{red}{\bf azar} de entre los 28 millones posibles, entonces es muy \textcolor{red}{\bf probable} que la media muestral $\bar x$  se parezca mucho a la media poblacional $\mu$.}
        \end{minipage}}
        }
    \end{center}
    \quad\\
    Hemos destacado en esta frase las palabras azar y probable, porque son la explicación de lo que vamos a estar haciendo en los próximos capítulos. Para poder usar la Estadística con rigor científico, tenemos que entender qué quiere decir exactamente {\em seleccionar al azar}, y cómo se puede {\em medir la probabilidad} de algo. Para esto necesitamos el lenguaje matemático de la Teoría de la Probabilidad.

\end{itemize}

\subsection{Primeras nociones sobre Probabilidad.}

\begin{itemize}

    \item El estudio de la Probabilidad nació, como disciplina científica, en el siglo XVII y en relación con los juegos de azar y las apuestas. Y es en el contexto de lanzar monedas, dados, cartas y ruletas, donde todavía se siguen encontrando la mayoría de los ejemplos con los que se presenta la teoría a los que se inician en su estudio. Nosotros vamos a hacer lo mismo.
        \begin{enumerate}
            \item Lanzamiento de dados: cuando se lanzan unos dados (sin trucar), el resultado de cada lanzamiento individual es imposible de predecir. Se observa, tras realizar un número muy grande de lanzamientos, que cada uno de los seis posibles resultados aparece aproximadamente una sexta parte de las veces.
            \item Lanzamiento de monedas: del mismo modo, cuando se lanzn una moneda (sin trucar), se observa, al cabo de muchos lanzamientos, que cada uno de los dos posibles resultados aparece aproximadamente la mitad de las veces.
            \item Las loterías, con la extracción de bolas numeradas de una urna o un bombo giratorio; o la ruleta, en la que la bola que se lanza puede acabar en cualquiera de las 36 (o 37) casillas.  Estos juegos y otros similares, ofrecían ejemplos adicionales con elementos comunes a los anteriores.
        \end{enumerate}
        Las apuestas, basadas en esos juegos de azar, y los casinos son, desde antiguo, un entretenimiento muy apreciado. Para hacer más interesante el juego, la humanidad fue construyendo otros juegos combinados más complicados. Por ejemplo, apostamos cada uno un euro y lanzamos dos dados: si la suma de los resultados es par, yo me llevo los dos euros. Si es impar, los ganas tú. La pregunta es evidente ¿Es este un juego {\em justo} para ambos jugadores? En concreto, lo que queremos saber es: si jugamos muchas, muchas veces ¿cuántos euros perderé o ganaré yo en promedio por cada euro invertido? ¿Y cuántos ganarás o perderás tú?

        Está claro que para que un jugador esté dispuesto a participar, y a arriesgar su fortuna, y desde luego para que alguien considere rentable el casino como negocio, o la lotería como forma de recaudar dinero, es preciso ofrecerle {\sf información precisa sobre cuáles son las ganancias esperadas del juego}.

    \item Otra cosa que la humanidad constató rápidamente al tratar con los juegos de azar es que nuestra intuición, en este terreno, es especialmente débil. Las personas en general, tendemos a subestimar o sobrevalorar mucho las probabilidades de muchos fenómenos. Y así consideramos como milagros algunos fenómenos perfectamente normales y predecibles, o viceversa. Como ejemplo de lo engañosa que puede ser nuestra intuición cuando se trata de probabilidades, en esta clase vamos a ver un vídeo de un fragmento (Episodio 13, 1º temporada) de la serie de televisión \link{http://en.wikipedia.org/wiki/Numb3rs}{Numb3rs} en el que se describe el conocido como \link{http://es.wikipedia.org/wiki/Problema_de_Monty_Hall}{problema de Monty Hall}. En Moodle tenéis un enlace al vídeo por si queréis volver a verlo.\\[3mm]
        Otro ejemplo, que se describe en detalle en el capítulo 3 de ``La Estadística en Comic'' como problema del \link{http://es.wikipedia.org/wiki/Antoine_Gombaud}{Caballero de Mere}: ¿qué es más probable, (a) obtener al menos un seis en cuatro tiradas de un dado, o (b) obtener al menos un seis doble en 24 tiradas de dos dados? Los jugadores que se planteaban esta pregunta respondían inicialmente así:
        \begin{itemize}
            \item[(a)] La probabilidad de obtener un seis en cada tirada es $\dfrac{1}{6}$. Por lo tanto, en cuatro tiradas es \[\dfrac{1}{6}+\dfrac{1}{6}+\dfrac{1}{6}+\dfrac{1}{6}=\dfrac{2}{3}.\]
            \item[(b)] La probabilidad de obtener un doble seis en cada tirada de dos dados es $\dfrac{1}{36}$, porque hay 36 resultados distintos, y todos aparecen con la misma frecuencia. Por lo tanto, en veinticuatro tiradas será \[\dfrac{1}{36}+\cdots+\dfrac{1}{36}=\dfrac{24}{36}=\dfrac{2}{3}.\]
        \end{itemize}
        Así que en principio ambas apuestas son iguales, y las cuentas parecen indicar que recuperaríamos dos de cada tres euros invertidos (el 66\%). Sin embargo, no es así, como algunos de esos jugadores debieron experimentar dolorosamente en sus patrimonios. Aquí tenéis enlazadas dos hojas de cálculo, llamadas \textattachfile{DeMere1.ods}{\textcolor{blue}{DeMere1.ods}} (para la apuesta (a)) y \textattachfile{DeMere2.ods}{\textcolor{blue}{DeMere2.ods}} (para la apuesta (b)), en las que hemos simulado esas dos apuestas, y hemos jugado 1000 veces cada una de ellas. Repetimos que la ganancia esperada es de un 66\% de lo invertido. Y lo que se observa es que la proporción de apuestas perdidas frente a apuestas ganadas no es, ni la que esperábamos, ni siquiera es igual en ambos casos. De hecho, vamos a aprender a calcular los valores correctos, y veremos que para la apuesta (a) ese valor es aproximadamente $0.52$, mientras que para la apuesta (b) es aproximadamente $0.49$ (los valores de la hoja de Cálculo se
parecen bastante a estos, ¿no es así?).

\end{itemize}

\subsection{Regla de Laplace}

\begin{itemize}

    \item Lo que tienen en común todas las situaciones que hemos descrito ligadas a juegos de azar es que:
        \begin{enumerate}
            \item Hay una lista de resultados individuales posibles: los seis números que aparecen en las caras de un dado, las dos caras de la moneda, las 36 casillas de la ruleta francesa, etc. Estos resultados se llaman {\sf resultados elementales}.
            \item Si repetimos el experimento muchas veces (muchos millones de veces si es necesario), y observamos los resultados, comprobamos que la {\em frecuencia relativa} de aparición de cada uno de los resultados elementales es la misma para todos ellos: $1/6$ para cada número en el dado, $1/2$ para cada cara de la moneda, $1/36$ para cada casilla de la ruleta. En ese caso decimos que los sucesos elementales son {\sf equiprobables}\footnote{Y sin embargo, las cosas no son tan sencillas. Quizá os resulte interesante leer este página Web sobre \link{http://www.microsiervos.com/archivo/azar/la-fabulosa-historia-de-los-pelayos.html}{la ruleta y la familia Pelayo}}.
        \end{enumerate}

        \item En este contexto, \link{http://es.wikipedia.org/wiki/Pierre_Simon_Laplace}{Laplace}, uno de los mayores genios matemáticos de la Ilustración francesa desarrolló la que seguramente es la primera contribución verdaderamente científica al análisis de la Probabilidad, y que se conoce como {\bf Regla de Laplace.} Vamos a fijar el lenguaje necesario para formular esa regla.
            \begin{itemize}
                \item[(a)]  Estamos interesados en un {\sf fenómeno o experimento aleatorio}. Es decir, al azar; como lanzar una moneda, un dado o un par de dados, etc. Y suponemos que ese experimento tiene $n$ {\sf resultados elementales} diferentes:
                    \[\{a_1,a_2,\ldots,a_n,\}\]
                    y que esos resultados elementales son {\sf equiprobables}, en el sentido de la igualdad de las frecuencias relativas que hemos descrito, cuando el experimento se repite muchas veces.
                \item[(b)]  Además, definimos un {\sf suceso aleatorio}, llamémoslo $A$, que es un resultado más complejo que se puede definir en términos de los resultados elementales del experimento en (a). Por ejemplo, si lanzamos un dado, $A$ puede ser: obtener un número par. O si lanzamos dos dados, $A$ puede ser: que la suma de los números sea divisible por $5$. En cualquier caso, en algunos de los resultados elementales ocurre $A$ y en otros no. Eso permite pensar en $A$ como un {\sf subconjunto del conjunto de resultados elementales}. Y aquellos resultados elementales en los que se observa $A$ se dice que son {\sf resultados favorables} al suceso $A$. Por ejemplo, si lanzamos un dado, los resultados favorables al suceso $A=$ {\em(obtener un número par)} son $\{2,4,6\}$. Y podemos decir, sin riesgo de confusión que $A=\{2,4,6\}$.
        \end{itemize}

    \item Con estas premisas, la formulación de la Regla de Laplace es esta:
        \begin{center}
            \fbox{\colorbox{Gris025}{
            \begin{minipage}{13cm}
            {\bf Regla de Laplace.\\[3mm]
            La probabilidad del suceso A es el cociente:
            \[p(A)=\dfrac{\mbox{número de sucesos elementales favorables a $A$}}{\mbox{número total de sucesos elementales}}\]
            }
            \end{minipage}}
            }
        \end{center}
        La Regla de Laplace supuso un impulso definitivo para la teoría de la Probabilidad, porque hizo posible comenzar a calcular probabilidades, y obligó a los matemáticos, a la luz de esos cálculos a pensar en las propiedades de la probabilidad. Además, esa regla se basa en el recuento de los casos favorables al suceso $A$ de entre todos los posibles. Y eso obliga a desarrollar técnicas de recuento a veces extremadamente sofisticadas (contar es algo muy difícil, aunque parezca paradójico), con lo que la Combinatoria se vio también favorecida por esta Regla de Laplace.

    \item La enorme complejidad de algunas operaciones en la Combinatoria es la mayor dificultad técnica asociada al uso de la Regla de Laplace. En este curso no nos vamos a entretener en ese tema más allá de lo imprescindible. Pero, como muestra y anticipo, podemos dar una respuesta en términos de combinatoria al problema del caballero De Mere.  Para ello tenemos que pensar en:
        \begin{itemize}
            \item[] {\sf El conjunto de todos los resultados elementales posibles del experimento "lanzar cuatro veces un dado".}
        \end{itemize}
        Esto puede resultar complicado. Como estrategia, es más fácil empezar por pensar en el caso de lanzar dos veces el dado, y nos preguntamos por la probabilidad del suceso $A=${\em obtener al menos un seis en las dos tiradas}. Como principio metodológico, esta técnica de entender primero bien una versión {\em a escala reducida} del problema es un buen recurso, al que conviene acostumbrarse.

         Para aplicar la Regla de Laplace al experimento de lanzar dos veces seguidas un dado, debemos empezar por dejar claro cuáles son los sucesos elementales equiprobables de este experimento. Los resumimos en esta tabla:
        \[
            \begin{array}{cccccc}
            (1,1)&(1,2)&(1,3)&(1,4)&(1,5)&\textcolor{red}{(1,6)}\\
            (2,1)&(2,2)&(2,3)&(2,4)&(2,5)&\textcolor{red}{(2,6)}\\
            (3,1)&(3,2)&(3,3)&(3,4)&(3,5)&\textcolor{red}{(3,6)}\\
            (4,1)&(4,2)&(4,3)&(4,4)&(4,5)&\textcolor{red}{(4,6)}\\
            (5,1)&(5,2)&(5,3)&(5,4)&(5,5)&\textcolor{red}{(5,6)}\\
            \textcolor{red}{(6,1)}&\textcolor{red}{(6,2)}&\textcolor{red}{(6,3)}&\textcolor{red}{(6,4)}&\textcolor{red}{(6,5)}&\textcolor{red}{(6,6)}
            \end{array}
        \]
        Observa que:
        \begin{itemize}
            \item El primer número del paréntesis es el resultado del primer lanzamiento, y el segundo número es el resultado del segundo lanzamiento.
            \item Hay, por tanto, $6\cdot 6=36$ sucesos elementales equiprobables.
            \item El suceso $(1,2)$ y el $(2,1)$ (por ejemplo), son distintos (y equiprobables).
            \item Hemos señalado en la tabla los sucesos elementales que son favorables al suceso $A=${\em obtener al menos un seis en las dos tiradas}. Y hay 11 de estos.
        \end{itemize}
        Así pues, la Regla de Laplace predice en este caso un valor de $\frac{11}{36}$, frente a los $\frac{12}{36}$ de la probabilidad ingenua (como la que hemos aplicado antes). Para comprobar experimentalmente nuestras ideas, tenemos una hoja de cálculo llamada \textattachfile{DeMere1a.ods}{\textcolor{blue}{DeMere1a.ods}} en la que se ha simulado ese lanzamiento. Puedes observar, recargando los valores unas cuantas veces (en Calc, prueba con {\tt Ctrl+May.+F9}), que la Regla de Laplace es mucho mejor que la probabilidad ingenua a la hora de predecir el resultado.


         Con la Regla de Laplace se pueden analizar también, usando bastante más maquinaria combinatoria, los dos experimentos (a) y (b) del apartado 1.2. Aquí dejamos simplemente un \textattachfile{DeMere2a.html}{\textcolor{blue}{documento}}  (se abre en el navegador y requiere Java) con los valores combinatorios necesarios para ese cálculo.

        Cerramos este apartado con un ejemplo-pregunta, que deberías responder antes de seguir adelante
        \begin{Ejemplo}\label{Sesion08:ejem:CualEsProbabilidadSumaDosDadosIgualASiete}
        ¿Cual es la probabilidad de que la suma de los resultados al lanzar dos dados sea igual a siete? Sugerimos usar la tabla de 36 resultados posibles que acabamos de escribir en esta sección.
        \end{Ejemplo}

\end{itemize}


%\section*{Tareas asignadas para esta sesión.}
%
%\begin{enumerate}
%   \item En la clase de hoy hemos descrito un juego de apuestas basado en el lanzamiento de dos dados, y nos hemos preguntado si era justo. En concreto, nos preguntamos cuáles son las ganancias esperadas para cada uno de los jugadores. En Moodle tienes una tarea esperándote, en la que debes responder a esa pregunta.
%\end{enumerate}




%\setcounter{section}{0}
%\section*{\fbox{\colorbox{Gris025}{{Sesiones 8, 9, 10. Probabilidad.}}}}
%
%\subsection*{\fbox{\colorbox{Gris025}{{Calculando probabilidades.}}}}
%\subsection*{Fecha: Jueves, 13/10/2011, 16h. También viernes 14/10 y martes 18/10.}
%
%\noindent{\bf Este fichero pdf lleva adjuntos los ficheros de datos necesarios.}
%
%%\subsection*{\fbox{1. Ejemplos preliminares }}
%\setcounter{tocdepth}{1}
%%\tableofcontents
%\section*{Lectura recomendada}
%
%Las mismas de la sesión anterior.


\section{Probabilidad más allá de la Regla de Laplace.}


\subsection{Definición (casi) rigurosa de probabilidad.}

\begin{itemize}

    \item La Regla de Laplace puede servir, con más o menos complicaciones combinatorias, para calcular probabilidades en casos como los de los dados, la ruleta, las monedas, etcétera. En la base de esa Regla de Laplace está la idea de {\em sucesos equiprobables}. Pero, incluso sin salir del casino, ¿qué sucede cuando los dados están cargados o las monedas trucadas? Y en el mundo real es fácil encontrar ejemplos en los que la noción de sucesos equiprobables no es de gran ayuda a la hora de calcular probabilidades: el mundo está lleno de ``dados cargados'' en favor de uno u otro resultado. Además, esa definición resulta claramente insuficiente para afrontar algunas situaciones.
        \begin{Ejemplo}\label{Sesion08:ejem:LanzamientoMonedaHastPrimeraCara}
        Por ejemplo, siguiendo en el terreno de los juegos de azar: dos jugadores A y B, juegan a lanzar una moneda. El primero que saque cara, gana, y empieza lanzando $A$. ¿Cuál es la probabilidad de que gane A?
        Si tratamos de aplicar la Regla de Laplace a este problema nos tropezamos con una dificultad; no hay límite al número de lanzamientos necesarios en el juego. Al tratar de hacer la lista de ``casos posibles'' nos tenemos que plantear la posibilidad  encontrarnos con secuencias de cruces cada vez más largas.
        \[
        \smiley, \dag\smiley, \dag \dag\smiley, \dag \dag \dag\smiley, \dag \dag \dag \dag\smiley,\ldots
        \]
        Así que si queremos asignar probabilidades a los resultados de este juego, la Regla de Laplace no parece de gran ayuda.\qed
        \end{Ejemplo}


    \item Otro problema con el que se enfrentaba la teoría de la probabilidad al aplicar la Regla de Laplace era el caso de la asignación de probabilidades a experimentos que involucran variables continuas. Por ejemplo, si en el intervalo $[0,1]$ de la recta real elegimos un número $x$ al azar, ¿cuál es la probabilidad de que sea $0\leq x\leq 1/3$? ¿Cuántos casos posibles (valores de $x$) hay? Y un ejemplo similar: en una circunferencia de radio 1 se eligen dos puntos al azar. ¿Cuál es la longitud media de la cuerda de circunferencia que definen? Aquí tenéis un \textattachfile{ProbabilidadGeometrica01.html}{\textcolor{blue}{documento html}} (se abre en el navegador, requiere Java) en el que se ilustra este problema, que es un ejemplo del tipo de problemas que genéricamente se llaman de {\em Probabilidad Geométrica}. Si queréis ver otro ejemplo famoso, aquí en la \link{http://docentes.educacion.navarra.es/msadaall/geogebra/figuras/azar\_buffon.htm}{página web de Manuel Sada}, podéis ver una ilustración del
problema de la aguja de Buffon.\\

    \item Veamos otro ejemplo de probabilidad geométrica.
     \begin{Ejemplo}\label{Sesion08:ejem:ProbabilidadGeometricaMontecarlo}
     supongamos que tenemos un cuadrado de lado cuatro, y en su interior dibujamos cierta figura $A$. Para fijar ideas, $A$ puede ser un un círculo de radio 1, centrado en el cuadrado, como en la figura.
        \begin{center}
         \includegraphics[height=4cm]{2011_10_13_Figura01_Montecarlo.png}
         \end{center}

    Si tomamos un punto al azar dentro del cuadrado ¿cuál es la probabilidad de que ese punto caiga dentro del círculo? En este otro \textattachfile{MonteCarloAreaCirculo01.html}{\textcolor{blue}{documento html}} hemos realizado ese experimento, para ayudar a aclarar la situación. Debería quedar claro, al pensar detenidamente sobre estos ejemplos, que la noción de probabilidad y la noción de área de una figura plana tienen muchas propiedades en común. El problema, claro está, es que la propia noción de área es igual de complicada de definir que la Probabilidad (como ejemplo, podéis considerar el área de la figura que se obtiene como límite siguiendo el proceso que ilustra este \textattachfile{Sierpinski.html}{\textcolor{blue}{documento html}}).\qed
    \end{Ejemplo}

    \item  A causa de estos, y otros problemas similares, los matemáticos construyeron (en el siglo XX) una teoría axiomática de la probabilidad. Aquí no podemos entrar en todos los detalles técnicos pero podemos decir que, esencialmente, se trata de lo siguiente:
        \begin{itemize}
            \item[(A)] Inicialmente tenemos un {\sf espacio muestral} $\Omega$, que representa el conjunto de todos los posibles resultados de un experimento.
            \item[(B)] Un {\sf suceso aleatorio} es un subconjunto del espacio muestral. Esta es la parte en la que vamos a ser menos rigurosos. En realidad, no todos los subconjuntos sirven, por la misma razón que hemos visto al observar que no es fácil asignar un área a todos los subconjuntos posibles. Pero para entender qué subconjuntos son sucesos y cuáles no, tendríamos que definir el concepto de $\sigma$-álgebra, y eso nos llevaría demasiado tiempo. Nos vamos a conformar con decir que hay un {\em tipo especial de subconjuntos}, los sucesos aleatorios, a los que sabemos asignarles una probabilidad.
            \item[(C)] La {\sf Función Probabilidad}, que representaremos con una letra $P$,  asigna por tanto un cierto número $P(A)$ a cada suceso aleatorio $A$ del espacio muestral $\Omega$. Y esa función probabilidad debe cumplir estas tres\\[3mm]
                \fbox{\colorbox{Gris025}{\begin{minipage}{12cm}
                \begin{center}
                \vspace{2mm}
                {\bf Propiedades fundamentales de la Función Probabilidad:}
                \end{center}
                \begin{enumerate}
                    \item Sea cual sea el suceso aleatorio $A$, siempre se cumple que $0\leq P(A)\leq 1$.
                    \item Si $A_1$ y $A_2$ son sucesos aleatorios disjuntos, es decir si $A_1\cap A_2=\emptyset$, es decir, si es imposible que $A_1$ y $A_2$ ocurran a la vez, entonces
                    \[P(A_1\cup A_2)=P(A_1)+P(A_2).\]
                    \item La probabilidad del espacio muestral completo es $1$. Es decir, $P(\Omega)=1$.
                \end{enumerate}
                \end{minipage}}}\\[3mm]
                La forma en la que se asignan las probabilidades define el {\sf modelo probabilístico} que utilizamos.\\
                Una aclaración sobre la tercera propiedad  de la probabilidad: el {\sf suceso unión} $A_1\cup A_2$ significa que suceden $A_1$ o $A_2$ (o ambos a la vez). Y, como ya hemos indicado, el {\sf suceso intersección} $A_1\cap A_2$ significa que $A_1$ y $A_2$ ocurren ambos simultáneamente.
        \end{itemize}

    \item Vamos a ver como se aplican estas ideas al ejemplo del del lanzamiento de una moneda hasta la primera cara que vimos antes.
        \begin{Ejemplo}[\bf Continuación del Ejemplo \ref{Sesion08:ejem:LanzamientoMonedaHastPrimeraCara} (pág. \pageref{Sesion08:ejem:LanzamientoMonedaHastPrimeraCara})]
        En este caso, podemos definir un modelo de probabilidad así. El espacio muestral $\Omega$ es el conjunto de todas las listas de la forma
        \[a_1=\smiley, a_2=\dag\smiley, a_3=\dag \dag\smiley,\ldots,a_k=\hspace{-7pt}\overbrace{\dag \dag \dag \cdots\dag \dag \dag }^{(k-1)\mbox{ cruces }}\hspace{-7pt}\smiley,\ldots\]
        es decir, $k-1$ cruces hasta la primera cara. Todos los subconjuntos se consideran sucesos aleatorios, y para definir la probabilidad decimos que:
        \begin{enumerate}
            \item $P(a_k)=P(\overbrace{\dag \dag \dag \cdots\dag \dag \dag }^{k-1\mbox{ cruces }}\hspace{-5pt}\smiley)=\dfrac{1}{2^{k}}$,
            \item Si $A=\{a_i\}$ es un suceso aleatorio, es decir conjunto de listas de cruces y caras, entonces $P(A)=\sum P(a_i)$. La probabilidad de un conjunto de listas es igual a la suma de las probabilidades de las listas que lo forman\footnote{No vamos a entreternos en comprobar que, con esta definición, se cumplen las tres propiedades fundamentales, pero le garantizamos al lector que, en efecto, así es.}. Es decir, que si
                \[A=\{a_1,a_3,a_6\}=\{\smiley,\,\dag \dag\smiley\, ,\, \dag \dag \dag \dag \dag\smiley\,\},\]
                entonces
                \[P(A)=P(a_1)+P(a_3)+P(a_6)=\dfrac{1}{2}+\dfrac{1}{2^3}+\dfrac{1}{2^6}.\]
        \end{enumerate}
         Ahora podemos calcular la probabilidad de que gane la persona que empieza lanzando. Ese suceso es:
         \[A=\{a_1,a_3,a_5,a_7,\ldots\}=\mbox{el primer jugador gana en la $k$-ésima jugada},\]
         y por lo tanto su probabilidad es:
         \[p(A)=\underbrace{P(a_1)+P(a_3)+P(a_5)+P(a_7)+\cdots}_{\mbox{listas de longitud impar}}=
         \dfrac{1}{2}+\dfrac{1}{2^3}+\dfrac{1}{2^5}+\dfrac{1}{2^7}+\cdots=\dfrac{2}{3}.\qed\]
         \end{Ejemplo}

    \item En los problemas-ejemplo de probabilidad geométrica que hemos discutido, esencialmente la definición de Función Probabilidad está relacionada con el área. En el Ejemplo \ref{Sesion08:ejem:ProbabilidadGeometricaMontecarlo} la probabilidad de un suceso (subconjunto del cuadrado grande) es igual al área de ese suceso dividida por 16 (el área del cuadrado grande). Un punto o una recta son sucesos de proabilidad cero (porque no tienen área).

    \item Una aclaración: la probabilidad definida mediante la Regla de Laplace cumple, desde luego, las tres propiedades fundamentales que hemos enunciado. Lo que hemos hecho ha sido {\em generalizar} la noción de probabilidad a otros contextos en los que la idea de favorables/posibles no se aplica. Pero los ejemplos que se basan en la Regla de Laplace son a menudo un buen ``laboratorio mental'' para poner a prueba nuestras ideas y nuestra comprensión de las propiedades de las probabilidades.

\end{itemize}

\subsection{Más propiedades de la Función Probabilidad.}

\begin{itemize}

    \item Las tres propiedades básicas de la Función Probabilidad tienen una serie de consecuencias que vamos a explorar en el resto de este capítulo. Las primeras y más sencillas aparecen resumidas en este cuadro:\\[3mm]
                \fbox{\colorbox{Gris025}{\begin{minipage}{12cm}
                \begin{center}
                \vspace{2mm}
                {\bf Propiedades adicionales de la Función Probabilidad:}
                \end{center}
                \begin{enumerate}
                    \item Sea cual sea el suceso aleatorio $A$, si $A^c$ es el suceso complementario (es decir ``no ocurre $A$'') siempre se cumple que
                    \[P(A^c)=1-P(A).\]
                    \item La probabilidad del suceso vacío $\emptyset$ es $0$; es decir \[P(\emptyset)=0.\]
                    \item Si $A\subset B$, (es decir si $A$ es un subconjunto de $B$, es decir si siempre que ocurre $A$ ocurre $B$), entonces
                    \[P(A)\leq P(B)\mbox{, y además }P(B)=P(A)+P(B\cap A^c).\]
                                        \item Si $A_1$ y $A_2$ son sucesos aleatorios cualesquiera,
                    \[P(A_1\cup A_2)=P(A_1)+P(A_2)-P(A_1\cap A_2).\]
                \end{enumerate}
                \end{minipage}}}\\[3mm]
    \item La última de estas propiedades se puede generalizar a $n$ sucesos aleatorios. Veamos como queda para tres, y dejamos al lector que imagine el resultado general {\em (ojo a los signos)}:
    \[P(A_1\cup A_2\cup A_3)=\]\[=\underbrace{\left(P(A_1)+P(A_2)+P(A_3)\right)}_{\mbox{tomados de 1 en 1}}\textcolor{red}{\mathbf -}
    \underbrace{\left(P(A_1\cap A_2)+P(A_1\cap A_3)+P(A_2\cap A_3)\right)}_{\mbox{tomados de 2 en 2}}\textcolor{red}{\mathbf +}\underbrace{\left(P(A_1\cap A_2\cap A_3)\right)}_{\mbox{tomados de 3 en 3}}.
    \]



\end{itemize}

\section{Probabilidad condicionada. Sucesos Independientes.}


\subsection{Probabilidad condicionada.}

\begin{itemize}

    \item El concepto de probabilidad condicionada trata de reflejar los cambios en el valor de la Función Probabilidad que se producen cuando tenemos {\em información parcial} sobre el resultado de un experimento aleatorio. Para entenderlo, vamos a usar, como ejemplo, uno de esos casos en los que la Regla de Laplace es suficiente para calcular probabilidades. Vamos a pensar que, al lanzar dos dados, nos dicen que la suma de los dados ha sido mayor que 3. Pero imagina que no sabemos el resultado; puede ser $(1,3), (2,5)$, etc., pero no, por ejemplo, $(1,1)$, o $(1,2)$. Con esa información en nuestras manos, nos piden que calculemos la probabilidad de que la suma de los dos dados haya sido un $7$. Nuestro cálculo debe ser distinto, ahora que sabemos que el resultado es mayor que 3, porque el número de resultados posibles (el denominador en la fórmula de Laplace), ha cambiado. Los resultados como $(1,1)$ o $(2,1)$ no pueden estar en la lista de resultados posibles, {\sf si sabemos que la suma es mayor que $3$}.
 {\bf La información que tenemos sobre el resultado cambia nuestro cálculo de probabilidades}. ¿Recuerdas el vídeo de Numb3rs sobre el problema de Monty Hall?

    \item Usando como ``laboratorio de ideas'' la Regla de Laplace, estamos tratando de definir la {\em probabilidad del suceso $A$ sabiendo que ha ocurrido el suceso $B$}. Esto es lo que vamos a llamar la {\sf probabilidad de $A$ condicionada por $B$, y lo representamos por $P(A|B)$}. Pensemos en cuáles son los cambios en la aplicación de la Regla de Laplace (favorables/posibles), cuando sabemos que el suceso $B$ ha ocurrido. Antes que nada recordemos que, si el total de resultados elementales posibles es $n$ entonces
        \[P(A)=\dfrac{\mbox{núm. de casos favorables a $A$}}{n},\]
        y también se cumple
        \[P(B)=\dfrac{\mbox{núm. de casos favorables a $B$}}{n}.\]
        Veamos ahora como deberíamos definir $P(A|B)$. Puesto que sabemos que $B$ ha ocurrido, los casos posibles ya no son todos los $n$ casos posibles originales: ahora los únicos casos posibles son los que corresponden al suceso $B$.  ¿Y cuáles son los casos favorables del suceso $A$, una vez que sabemos que $B$ ha ocurrido? Pues aquellos casos en los que $A$ y $B$ ocurren simultáneamente (o sea, el suceso $A\cap B$). En una fracción:
        \[P(A|B)=\dfrac{\mbox{número de casos favorables a $A\cap B$}}{\mbox{número de casos favorables a $B$}}.\]
        Si sólo estuviéramos interesados en la Regla de Laplace esto sería tal vez suficiente. Pero, para poder generalizar la fórmula a casos como la Probabilidad Geométrica, hay una manera mejor de escribirlo. Dividimos el numerador y el denominador por $n$ y tenemos:
        \[P(A|B)=\dfrac{\quad\left(\dfrac{\mbox{número de casos favorables a $A\cap B$}}{n}\right)\quad}{\left(\dfrac{\mbox{número de casos favorables a $B$}}{n}\right)}=\dfrac{P(A\cap B)}{P(B)}.\]
        ¿Qué tiene de bueno esto? Pues que la expresión que hemos obtenido ya no hace ninguna referencia a casos favorables o posibles, nos hemos librado de la Regla de Laplace, y hemos obtenido una expresión general que sólo usa la Función de Probabilidad (e, insistimos, hacemos esto porque así podremos usarla, por ejemplo, en problemas de Probabilidad Geométrica). Ya tenemos la definición:
        \begin{center}
        \fbox{\colorbox{Gris025}{\begin{minipage}{14cm}
        \begin{center}
        \vspace{2mm}
        {\bf Probabilidad Condicionada:}
        \end{center}
        La probabilidad del suceso $A$ condicionada por el suceso $B$ se define así:
            \[P(A|B)=\dfrac{P(A\cap B)}{P(B)}.\]
        donde se supone que $P(B)\neq 0$.
        \end{minipage}}}
        \end{center}

    \item Vamos a ver un ejemplo de como calcular estas probabilidades condicionadas, usando de nuevo el lanzamiento de dos dados.
    \begin{Ejemplo}\label{ejem:probabilidadCondicionadaLanzamientoDosDados}
    Se lanzan dos dados. ¿Cuál es la probabilidad de que la diferencia --en valor absoluto-- entre los valores de ambos dados (mayor-menor) sea menor que 4, sabiendo que la suma de los dados es 7?\\
    Vamos a considerar los sucesos:
    \begin{itemize}
        \item[D:] La suma de los dados es 7.
        \item[F:] La diferencia en valor absoluto de los dados es menor que 4.
    \end{itemize}
    En este caso es muy fácil calcular $P(F|D)$. Si sabemos que la suma es $7$, los resultados sólo pueden ser $(1,6),(2,5),(3,4),(4,3),(5,2),(6,1)$. Y de estos, sólo $(1,6)$ y $(6,1)$ no cumplen la condición de la diferencia. Así que $P(F|D)=4/6$. Vamos a ver si coincide con lo que predice la fórmula. El suceso $D\cap F$ ocurre cuando ocurren {\sf a la vez} $D$ y $F$. Es decir la suma es 7 {\sf y a la vez} la diferencia es menor que $4$. Es fácil ver que, de los 36 resultados posible, eso sucede en estos cuatro casos: \[(2,5),(3,4),(4,3),(5,2)\]
    La probabilidad de la intersección es $P(D\cap F)=\frac{4}{36}$. Y, por otro lado, la probabilidad del suceso $D$ es $P(D)=\frac{6}{36}$ (ver el Ejemplo \ref{Sesion08:ejem:CualEsProbabilidadSumaDosDadosIgualASiete} de la pág. \pageref{Sesion08:ejem:CualEsProbabilidadSumaDosDadosIgualASiete}; de hecho, hemos descrito los sucesos favorables a $D$ un poco más arriba). Así pues,
    \[P(F|D)=\dfrac{P(F\cap D)}{P(D)}=\dfrac{4/36}{6/36}=\dfrac{4}{6}=\dfrac{2}{3}\approx 0.666\ldots,\]
    como esperábamos. En esta \textattachfile{2011-10-13-Lanzamientos2Dados-ProbabilidadCondicionada.ods}{\textcolor{blue}{hoja de cálculo (Calc)}} se ha realizado una simulación para comprobar estos resultados.\\
    Una extensión natural de este ejemplo es tratar de calcular $P(D|F)$. ¿Puedes modificar la hoja de cálculo para simular este otro caso?\qed
    \end{Ejemplo}
\item  Para responder a la última frase del anterior ejemplo, es bueno observar que siempre se cumple
        \[\fbox{\colorbox{Gris025}{$P(A|B)P(B)=P(B|A)P(A),$}}\]
        de manera que, si se conocen las probabilidades de $A$ y $B$, se pueden relacionar fácilmente ambas probabilidades condicionadas.
\end{itemize}

\subsection{Sucesos independientes.}

\begin{itemize}

    \item ¿Qué significado debería tener la frase {\em ``el suceso $A$ es independiente del suceso $B$''}\,? Parece evidente que, si los sucesos son independientes, el hecho de saber que el suceso $B$ ha ocurrido no debería afectar para nada nuestro cálculo de la probabilidad de que ocurra $A$. Esta idea tiene una traducción inmediata en el lenguaje de la probabilidad condicionada, que es de hecho la definición de sucesos independientes:
        \vspace{-3mm}
        \begin{center}
        \fbox{\colorbox{Gris025}{\begin{minipage}{14cm}
        \begin{center}
        \vspace{2mm}
        {\bf Sucesos independientes:}
        \end{center}
        Los sucesos $A$ y $B$ son independientes si
            \[P(A|B)=P(A).\]
        Esto es equivalente a decir que
        \[P(A\cap B)=P(A)P(B).\]
        En particular, {\bf cuando los sucesos $A$ y $B$ son independientes}, se cumple:
        \[P(A\cup B)=P(A)+P(B)-P(A)P(B).\]
        \end{minipage}}}
        \end{center}

    \item En general los sucesos $A_1,\ldots,A_n$ son independientes cuando para {\em cualquier colección} que tomemos de ellos, la probabilidad de la intersección es el producto de las probabilidades.
    
    \item A menudo, al principio, hay cierta confusión entre la noción de sucesos independientes y la de sucesos disjuntos. Recordemos que dos sucesos son disjuntos si no pueden ocurrir a la vez. POr ejemplo, si $A$ es el suceso {\em ``Hoy es lunes''} y $B$ es el suceso {\em ``Hoy es vierenes''}, está claro que $A$ y $B$ no pueden ocurrir a la vez. Por otra parte, los sucesos son independientes cuando uno de ellos no aporta ninguna información sobre el otro. Y volviendo al ejemplo, en cuanto sabemos que hoy es lunes (ha ocurrido $A$), ya estamos seguros de que no es viernes (no ha ocurrido $B$). Así que la información sobre el suceso $A$ nos permite decir algo sobre el suceso $B$, y eso significa que no hay independencia. {\sf Dos sucesos disjuntos nunca son independientes}.

\end{itemize}


\section{Probabilidades totales y Teorema de Bayes.}\label{cap04:sec:ProbabilidadesTotalesReglaBayes}


\subsection{La regla de las probabilidades totales. Problemas de urnas.}\label{Sesion08:subsec:ProbabilidadesTotales}

\begin{itemize}

    \item El resultado que vamos a ver utiliza la noción de probabilidad condicionada para calcular la probabilidad de un suceso $A$ mediante la estrategia de {\em divide y vencerás}.  Se trata de descomponer el espacio muestral completo en una serie de sucesos $B_1,\ldots,B_k$ de manera que:
        \begin{enumerate}
            \item[(1)] $\Omega=B_1\cup B_2\cup\cdots\cup B_k.$
            \item[(2)] $B_i\cap B_j=\emptyset$, para cualquier pareja $i\neq j$
            \item[(3)] $P(B_i)\neq 0$ para $i=1,\ldots,k$.
        \end{enumerate}
        Entonces
        \begin{center}
        \fbox{\colorbox{Gris025}{\begin{minipage}{14cm}
        \begin{center}
        \vspace{2mm}
        {\bf Regla de las probabilidades totales:}
        \end{center}
        Si los sucesos $B_1,\ldots,B_K$ cumplen las condiciones (1), (2) y (3) entonces:
            \[P(A)=P(B_1)P(A|B_1)+P(B_2)P(A|B_2)+\cdots+P(B_k)P(A|B_k).\]
        \end{minipage}}}
        \end{center}
        Esta espresión permite calcular la probabilidad de $A$ cuando conocemos de antemano las probabilidades de los sucesos $B_1,\ldots,B_k$ y es fácil calcular las probabilidades condicionadas $P(A|B_i)$. Si los sucesos $B_i$ se han elegido bien, la información de que el suceso $B_i$ ha ocurrido puede en ocasiones simplificar mucho el cálculo de $P(A|B_i)$.

        \item El método de las probabilidades totales se usa sobre todo cuando conocemos varias vías o mecanismos por los que el suceso $A$ puede llegar a producirse. El ejemplo clásico son los {\sf problemas de urnas}, que sirven de prototipo para muchas otras situaciones.
            \begin{Ejemplo}\label{Sesion08:ejem:ProbailidadTotalEjemploUrnas}
                Supongamos que tenemos dos urnas, la primera con 3 bolas blancas y dos negras, y la segunda con 4 bolas blancas y 1 negra. Para extraer una bola lanzamos un dado. Si el resultado es $1$ o $2$ usamos la primera urna; si es cualquier otro número usamos la segunda urna. ¿cuál es la probabilidad de obtener una bola blanca?\\
                Llamemos $B_1$ al suceso {\em ``se ha usado la primera urna''}, y $B_2$ al suceso {\em ``se ha usado la segunda urna''}. Entonces es muy fácil obtener $P(B_1)=\frac{1}{3}$, $P(B_2)=\frac{2}{3}$. Y ahora, cuando sabemos que $B_1$ ha ocurrido (es decir, que estamos usando la primera urna), es muy fácil calcular $P(A|B_1)$.  Se trata de la probabilidad de extraer una bola blanca de la primera urna: $P(A|B_1)=\dfrac{3}{5}.$ De la misma forma $P(A|B_2)=\frac{4}{5}$. Con todos estos datos, el Teorema de las Probabilidades Totales da como resultado:
                \[P(A)=P(B_1)P(A|B_1)+P(B_2)P(A|B_2)=\dfrac{1}{3}\cdot\dfrac{3}{5}+\dfrac{2}{3}\cdot\dfrac{4}{5}=\dfrac{11}{15}.\]
                En esta \textattachfile{2011-10-13-ProbabilidadesTotales-Urnas.ods}{\textcolor{blue}{hoja de Calc}} puedes comprobar experimentalmente el resultado. \qed
            \end{Ejemplo}
            Este ejemplo, con dados y bolas, puede parecer engañosamente artificioso. Pero piensa en esta situación: si tenemos una fábrica que produce la misma pieza con dos máquinas distintas, y sabemos la proporción de piezas defectuosas que produce cada una de las máquinas, podemos identificar máquinas con urnas y piezas con bolas, y vemos que el método de las probabilidades totales nos permite saber cuál es la probabilidad de que una pieza producida en esa fábrica sea defectuosa. De la misma forma, si sabemos la probabilidad de desarrollar cáncer de pulmón, en  fumadores y no fumadores, y sabemos la proporción de fumadores y no fumadores que hay en la población total, podemos identificar cada uno de esos tipos de individuos (fumadores y no fumadores) con una urna, y el hecho de desarrollar o no cáncer con bola blanca o bola negra. Como puede verse, el rango de aplicaciones de este resultado es bastante mayor de lo que parecía a primera vista. Veremos más ejemplos en los ejercicios.


\end{itemize}

\subsection{Teorema de Bayes. La probabilidad de las causas.}

\begin{itemize}

    \item La regla de las probabilidades totales puede describirse así: si conocemos varios mecanismos posibles (los sucesos $B_1,\ldots,B_k$) que conducen al suceso $A$, y las probabilidades asociadas con esos mecanismos, ¿cuál es la probabilidad de ocurra el suceso $A$? El Teorema de Bayes le da la vuelta a la situación. Ahora suponemos que el suceso $A$ {\em de hecho ha ocurrido}.  Y, puesto que puede haber ocurrido a través de distintos mecanismos, nos podemos preguntar ¿cómo de probable es que el suceso $A$ haya ocurrido a través de, por ejemplo, el primer mecanismo $B_1$? Insistimos, no vamos a preguntarnos por la probabilidad del suceso $A$, puesto que suponemos que ha ocurrido. Nos preguntamos por la probabilidad de cada una de los mecanismos o causas que conducen al resultado $A$. Por eso a veces el Teorema de Bayes se describe como un resultado sobre la probabilidad de las causas.

    \item ¿Cómo podemos conseguir esto? La pregunta se puede formular así: sabiendo que el suceso $A$ ha ocurrido, ¿cuál es la probabilidad de que haya ocurrido a través del mecanismo $B_i$? De otra manera: sabiendo que el suceso $A$ ha ocurrido, ¿cuál es la probabilidad de que eso se deba a que $B_i$ ha ocurrido? Es decir, queremos averiguar el valor de
        \[P(B_i|A)\mbox{\quad para i=1,\ldots,k}.\]
        Quizá lo más importante es entender que, para calcular este valor, {\em la información de la que disponemos es exactamente la misma que en el caso de las probabilidades totales}. Es decir, conocemos los valores $P(B_1),\ldots,P(B_k)$ y las probabilidades condicionadas $P(A|B_1),\ldots,P(A|B_k)$, ¡qué son justo al revés de lo que ahora queremos!.

        Y la forma de conseguir el resultado es esta. Usando que:
        \[P(A|B_k)P(B_k)=P(A\cap B_k)=\textcolor{red}{P(B_k|A)}P(A),\]
        despejamos de aquí lo que queremos, y usamos el teorema de las probabilidades totales de una forma astuta, obteniendo:
        \begin{center}
        \fbox{\colorbox{Gris025}{\begin{minipage}{14cm}
        \begin{center}
        \vspace{2mm}
        {\bf Teorema de Bayes:}
        \end{center}
        Si los sucesos $B_1,\ldots,B_K$ cumplen las condiciones (1), (2) y (3) (ver la sección \ref{Sesion08:subsec:ProbabilidadesTotales}) entonces:
            \[P(B_k|A)=\dfrac{P(B_k)P(A|B_k)}{P(B_1)P(A|B_1)+P(B_2)P(A|B_2)+\cdots+P(B_k)P(A|B_k)}.\]
        \end{minipage}}}
        \end{center}
        Obsérvese que:
        \begin{enumerate}
        \item Los valores que necesitamos para calcular esta fracción aparecen en la fórmula de las probabilidades totales.
        \item El numerador es uno de los sumandos del denominador.
        \item Las probabilidades condicionadas de la fracción son justo al revés que las del miembro izquierdo.
        \end{enumerate}
        Con estas observaciones, la fórmula de Bayes es bastante fácil de recordar.

        \item Veamos un ejemplo de uso de la fórmula de Bayes.
        \begin{Ejemplo}{\em  (adaptado del Ejemplo 3.7.3 de  {\em Estadística para Biología y Ciencias de la Salud, 3a.ed.}, de J.S.Milton. Ed. MacGraw-Hill)}\\[2mm]
            Se sabe que la distribución de grupos sanguíneos en la población es:\\[3mm]
            \begin{tabular}{|c|c|c|c|c|}
            \hline
            Grupo&A&B&AB&O\\
            \hline
            Porcentaje&41\%&9\%&4\%&46\%\\
            \hline
            \end{tabular}\\[2mm]
            Además, el $4\%$ de las personas del grupo O se clasifican por error como personas del tipo A. Igualmente, el $4\%$ del tipo B, y el 10\% del tipo AB se clasifican por error como tipo A. El 88\% del tipo A se clasifica correctamente. Un herido ingresa en el hospital y se le clasifica como tipo A. ¿Cuál es la probabilidad de que ese tipo sea el suyo?\\
            Consideramos los siguientes sucesos:
            \begin{itemize}
                \item $A$: se clasifica como tipo A.
                \item $B_1$: es del tipo A.
                \item $B_2$: es del tipo B.
                \item $B_3$: es del tipo AB.
                \item $B_4$: es del tipo O.
            \end{itemize}
            Con esta notación, lo que queremos es calcular $P(B_1|A)$: ¿cuál es la probabilidad de que sea del tipo A (suceso $B_1$), condicionada a que se le ha clasificado como A (suceso A)? La fórmula de Bayes es, en este caso,
            \[P(B_1|A)=\dfrac{P(A|B_1)P(B_1)}{P(A|B_1)P(B_1)+P(A|B_2)P(B_2)+P(A|B_3)P(B_3)+P(A|B_4)P(B_4)}\]
            Y los datos que nos han dado se resumen en esta tabla:
            \[\begin{array}{cc}
            P(B_1)=0.41,&P(A|B_1)=0.88\\
            P(B_2)=0.09,&P(A|B_2)=0.04\\
            P(B_3)=0.04,&P(A|B_3)=0.10\\
            P(B_4)=0.46,&P(A|B_4)=0.04
            \end{array}\]
            Sustituyendo en la fórmula de Bayes, se obtiene $P(B_1|A)\approx 0.93$. Es conveniente observar que, en este ejemplo, las ``urnas'' de las que estamos extrayendo los resultados son los grupos sanguíneos. Al extraer una bola de la urna (clasificar a un paciente por su grupo sanguíneo), podemos obtener una bola blanca (clasificarlo como grupo A) o negra (clasificarlo de otra manera). Y sabemos la probabilidad de elegir una urna concreta (porque tenemos la proporción de los grupos sanguíneos) y sabemos la composición de cada una de las urnas (el porcentaje de cada grupo que acaban clasificados como grupo A). Por tanto tenemos todos los ingredientes para aplicar la fórmula de Bayes.\qed
        \end{Ejemplo}
\end{itemize}


%
%\section*{Tareas asignadas para esta sesión.}
%
%\begin{enumerate}
%   \item Ya está publicada la tercera hoja de Ejercicios. La semana que viene los usaremos en las clases prácticas. Debes intentar resolverlos por escrito, en un papel. Pero una vez resueltos, muchos de ellos admiten una ``comprobación experimental'' de los resultados usando el generador de números (pseudo)aleatorios de la hoja de cálculo. Intenta obtener un par de estas simulaciones, que también revisaremos en las clases prácticas.
%\end{enumerate}


%\section*{Recomendaciones.}
%
%\begin{enumerate}
%   \item El \link{http://www.ine.es/}{INE (Instituto Nacional de Estadística)} es el organismo oficial encargado, entre otras cosas del censo electoral, la elaboración del IPC (índice de precios de consumo), la EPA (encuesta de población activa), el PIB (producto interior bruto) etc. El instituto ofrece una enorme colección de datos estadísticos accesibles para cualquiera a través de la red (sistema INEbase). Además, tiene alojado en su página web un \link{http://www.ine.es/explica/explica.htm}{portal de divulgación estadística} en el que se pueden ver vídeos sobre estos y otros temas, que tal vez os interesen.
%\end{enumerate}
%


%\section*{\fbox{\colorbox{Gris025}{{Sesión 10. Probabilidad.}}}}
%
%\subsection*{\fbox{\colorbox{Gris025}{{Combinatoria.}}}}
%\subsection*{Fecha: Martes, 18/10/2011, 14h.}
%
%\noindent{\bf Atención:
%En esta sesión vamos a ver también la regla de la probabilidad total y el teorema de Bayes, que aparecen en el resumen de la sesión del Jueves 13/10.
%}
%
%%\subsection*{\fbox{1. Ejemplos preliminares }}
%\setcounter{tocdepth}{1}
%%\tableofcontents
%\section*{Lectura recomendada}
%
%Las mismas de la sesión anterior.

\section{Combinatoria}


\subsection{Combinaciones}

\begin{itemize}

    \item La Combinatoria es una parte de las matemáticas que estudia técnicas de recuento. En particular, estudia las posibles formas de seleccionar listas o subconjuntos de elementos de un conjunto dado siguiendo ciertos criterios (ordenados o no, con repetición o no, etcétera). Por esa razón es de mucha utilidad para el cálculo de probabilidades, sobre todo cuando se combina con la Regla de Laplace. La Combinatoria, no obstante, puede ser muy complicada, y en este curso vamos a concentrarnos en los resultados que necesitamos.

    \item En particular estamos interesados en este problema. Dado un conjunto de $n$ elementos
    \[A=\{x_1,x_2,\ldots,x_n\}\]
    y un número $k$ con $0\leq k\leq n$, ¿cuántos {\sf subconjuntos distintos} de $k$ elementos podemos formar con los elementos de $A$? Es muy importante entender que, al usar la palabra {\sf subconjunto}, estamos diciendo que:
    \begin{enumerate}
        \item el {\sf orden de los elementos es irrelevante}. El subconjunto $\{x_1,x_2,x_3\}$ es el mismo que el subconjunto $\{x_3,x_1,x_2\}$.
        \item los elementos del subconjunto {\sf no se repiten}. El subconjunto $\{x_1,x_2,x_2\}$ es, de hecho, el subconjunto $\{x_1,x_2\}$ (y nunca lo escribiríamos de la primera manera, si estamos hablando de subconjuntos).
    \end{enumerate}

    Vamos a ponerle un nombre a lo queremos calcular: el número de subconjuntos posibles es el número de {\sf combinaciones de $n$ elementos, tomados de $k$ en $k$} (cada uno de los subconjuntos es una combinación).
    \begin{Ejemplo}\label{ejem:combinacionesCuatroDosEnDos}
    Por ejemplo, en un conjunto de 4 elementos $A=\{a,b,c,d\}$, hay seis combinaciones distintas de elementos, tomados de dos en dos :
    \[\{a,b\},\{a,c\},\{a,d\},\{b,c\},\{b,d\},\{c,d\}\]
    \qed
    \end{Ejemplo}
\end{itemize}

\subsection{Números combinatorios}

\begin{itemize}

    \item Para calcular probabilidades de forma eficaz muchas veces necesitamos una manera de calcular el número de combinaciones posibles. El {\sf número combinatorio} $\binom{n}{k}$ es el número de combinaciones de $n$ elementos tomados de $k$ en $k$.

    \item Para calcular ese número necesitamos el concepto de {\sf factorial}. El factorial del número $n$ es el producto
        \[\fbox{\colorbox{Gris025}{$n!=n\cdot(n-1)\cdot(n-2)\cdot\,\ldots\,\cdot3\cdot 2\cdot 1.$}}\]
        Es decir, el producto de todos los números entre $1$ y $n$. Además definimos \fbox{$0!=1$}. Por ejemplo, se tiene:
        \[6!=6\cdot 5\cdot 4\cdot 3\cdot 2\cdot 1=720.\]
        La propiedad más llamativa del factorial es su crecimiento extremadamente rápido. Por ejemplo, $100!$ es del orden de $10^{57}$.

    \item La fórmula de los números combinatorios es esta:
    \\[3mm]
        \fbox{\colorbox{Gris025}{\begin{minipage}{14cm}
        \begin{center}
        \vspace{2mm}
        {\bf Números combinatorios:}
        \end{center}
        \[\binom{n}{k}=\dfrac{n!}{k!(n-k)!}\mbox{, \quad para $0\leq k\leq n$, y $n=0,1,2\ldots$ cualquier número natural}\]
        \quad
        \end{minipage}}}\\[3mm]
        Por ejemplo, el número de combinaciones del Ejemplo \ref{ejem:combinacionesCuatroDosEnDos} es
        \[\binom{4}{2}=\dfrac{4!}{2!(4-2)!}=\dfrac{24}{2\cdot 2}=6,\]
        como era de esperar.

    \item Hay dos observaciones que facilitan bastante el trabajo con estos números combinatorios.
    \begin{enumerate}
        \item Los números combinatorios se pueden representar en esta tabla de forma triangular, llamada el {\sf Triángulo de Pascal}:
        \[
        \begin{array}{l|llcccccccccccccc}
        n=0&&&&&&&&&1\\
        n=1&&&&&&&&1&&1\\
        n=2&&&&&&&1&&2&&1\\
        n=3&&&&&&1&&3&&3&&1\\
        n=4&&&&&1&&4&&6&&4&&1\\
        n=5&&&&1&&5&&10&&10&&5&&1\\
        n=6&&&1&&6&&15&&20&&15&&6&&1\\
        \vdots&&& &&\vdots&& &&\vdots&& &&\vdots&&
        \end{array}
        \]
        El número $\binom{n}{k}$ ocupa la fila $n$ posición $k$ (se cuenta desde $0$). Por ejemplo en la $4$ fila, posición $2$ está nuestro viejo conocido $\binom{4}{2}=6$. ¿Cuánto vale $\binom{5}{3}$?

        Los puntos suspensivos de la parte inferior están ahí para indicarnos qué podríamos seguir, y a la vez para servir de desafío. ¿Qué viene a continuación? ¿Qué hay en la línea $n=15$? Pues parece claro que empezará y acabará con un $1$. También parece claro que el segundo y el penúltimo  número valen $7$. ¿Pero y el resto? Lo que hace especial a esta tabla es que {\sf cada número que aparece en el interior de la tabla es la suma de los dos situados a su izquierda y derecha en la fila inmediatamente superior.} Por ejemplo, el $10$ que aparece en tercer lugar en la fila de $n=5$  es la suma del $4$ y el $6$ situados sobre él en la  segunda y tercera posiciones de la fila para $n=4$. Con esta información, podemos         obtener la séptima fila de la tabla, a partir de la sexta sumado según indican las flechas en este esquema:
        \[
        \begin{array}{l|llcccccccccccccccc}
        n=6&&&&1&&6&&15&&20&&15&&6&&1\\
           &&&&\swarrow\searrow&&\swarrow\searrow&&\swarrow\searrow&&\swarrow\searrow&&\swarrow\searrow&&
           \swarrow\searrow&&\swarrow\searrow\\
        n=7&&&1&&7&&21&&35&&35&&21&&7&&1
        \end{array}
        \]
        \item La segunda observación importante sobre los números quedará más clara con un ejemplo:
        \[
        \binom{12}{7}=\dfrac{12!}{7!(12-7)!}=\dfrac{12!}{7!5!}.
        \]
        Ahora observamos que $12!=(12\cdot 11\cdot\cdots\cdot 6)\cdot(5\cdot\cdots\cdot 2\cdot 1)$, y los paréntesis muestran que esto
        es igual a $(12\cdot 11\cdot\cdots\cdot 6)\cdot 5!$. Este factorial de $5$ se cancela con el del denominador y tenemos
        \[
        \binom{12}{7}=\dfrac{\overbrace{12\cdot 11\cdot 10\cdot 9\cdot 8\cdot 7\cdot 6}^{6\mbox{ factores}}}{7!}=792.
        \]
        Generalizando esta observación sobre la cancelación de factoriales, la forma en la que vamos a expresar los coeficientes binomiales será finalmente esta:
        \begin{equation}\label{cap03:ecu:expresionPseudoFactorialCoeficientesBinomiales}
        \dbinom{n}{k}=\frac{\overbrace{n\left( n-1\right) \left( n-2\right) \cdots \left( n-k+1\right) }^{k\mbox{ factores}}}{k!}
        \end{equation}
        Y, como hemos indicado, lo que caracteriza este esta expresión es que tanto el numerador como el denominador tienen $k$ factores.
    \end{enumerate}

    \item Los números combinatorios son importantes en muchos problemas de probabilidad. Veamos un par de ejemplos:
        \begin{Ejemplo}
        Tenemos una caja de 10 bombillas y sabemos que tres están fundidas. Si sacamos al azar tres bombillas de la caja\footnote{``al azar'' aquí significa que todos los subconjuntos de tres bombillas son equiprobables.}, ¿Cuál es la probabilidad de que hayamos sacado las tres que están fundidas?\\[2mm]
        En este caso, al tratar de aplicar la Regla de Laplace, usamos los números combinatorios para establecer el número de casos posibles. ¿Cuántas formas distintas hay de seleccionar tres bombillas de un conjunto de 10? Evidentemente hay $\binom{10}{3}$ formas posibles. Este número es:
        \[\binom{10}{3}=\dfrac{10\cdot 9\cdot 8}{3\cdot 2\cdot 1}=120.\]
        Estos son los casos posibles. Está claro además que sólo hay un caso favorable, cuando elegimos las tres bombillas defectuosas. Así pues, la probabilidad pedida es:
        \[\dfrac{1}{120}.\]
        \qed
        \end{Ejemplo}
        El siguiente ejemplo es \underline{extremadamente importante} para el resto del curso, porque nos abre la puerta que nos conducirá a la distribución binomial y a algunos de los resultados más profundos de la Estadística.
        \begin{Ejemplo}\label{ejem:probabilidadLanzamientoMonedas}
        Lanzamos una moneda al aire cuatro veces, y contamos el número de caras obtenidas en esos lanzamientos. ¿Cuál es la probabilidad de obtener exactamente dos caras en total?\\
        Vamos a pensar en cuál es el espacio muestral. Se trata de listas de cuatro símbolos: cara o cruz. Por ejemplo,
        \[\smiley\smiley\dagger\smiley\]
        es un resultado posible, con tres caras y una cruz. ¿Cuántas de estas listas de cara y cruz con cuatro símbolos hay? Enseguida se ve que hay $2^4$, así que ese es el número de casos posibles. ¿Y cuál es el número de casos favorables? Aquí es donde los números combinatorios acuden en nuestra ayuda. Podemos pensar así en los sucesos favorables: tenemos cuatro fichas, dos caras y dos cruces $\smiley,\smiley,\dagger,\dagger,$ y un casillero con cuatro casillas
        \begin{center}
        \begin{tabular}{|c|c|c|c|}
        \hline
         \rule{0cm}{0.5cm}\rule{1cm}{0cm}&\rule{1cm}{0cm}&\rule{1cm}{0cm} &\rule{1cm}{0cm}\\
         \hline
         \end{tabular}
         \end{center}
         en las que tenemos que colocar esas cuatro fichas. Cada manera de colocarlas corresponde a un suceso favorable. Y entonces está claro que lo que tenemos que hacer es elegir, de entre esas cuatro casillas, cuáles dos llevarán una cara (las restantes dos llevarán una cruz). Es decir, hay que elegir dos de entre cuatro. Y ya sabemos que la respuesta es $\binom{4}{2}=6$. Por lo tanto la probabilidad pedida es:
         \[p(2 \mbox{ caras} )=\dfrac{\binom{4}{2}}{2^4}=\binom{4}{2}\left(\dfrac{1}{2}\right)^4=\dfrac{6}{16}.\]
         Supongamos ahora que lanzamos la moneda $n$ veces y queremos saber cuál es la probabilidad de obtener $k$ veces cara. Un razonamiento similar produce la fórmula:
        \[p(k \mbox{ caras})=\binom{n}{k}\left(\dfrac{1}{2}\right)^k.\]
        \qed
        \end{Ejemplo}

        \item No podemos dejar de mencionar que los números combinatorios son también importantes en relación con el Teorema del Binomio, y que por eso se los conoce también como {\sf coeficientes binomiales}. En concreto, se tiene, para $a,b\in\R$, y $n\in\N$ esta {\sf Fórmula del Binomio}:
             \[
             (a+b)^n=\binom{n}{0}a^n+\binom{n}{1}a^{n-1}b+\binom{n}{2}a^{n-2}b^2+\cdots+\binom{n}{n-1}ab^{n-1}+\binom{n}{n}b^n
              \]



%                volvemos al problema del Caballero De Mere, pero con muchas más herramientas a nuestra disposición.
%        \begin{Ejemplo}\label{ejem:probabilidadLanzamientoMonedas}
%        Lanzamos un dado al aire cuatro veces, y contamos el número de seises obtenidas en esos lanzamientos. ¿Cuál es la probabilidad de obtener exactamente cuatro seises en total?\\
%
%        \qed
%        \end{Ejemplo}
%
\end{itemize}

%\section{Variables aleatorias}
%
%
%\subsection{¿Qué son las variables aleatorias?}
%
%\begin{itemize}
%
%    \item Hemos visto que cada suceso $A$ del espacio muestral $\Omega$ tiene asociado un valor $P(A)$ de la función probabilidad. Y sabemos que los valores de la función probabilidad son valores positivos, comprendidos entre $0$ y $1$. La idea de variable aleatoria es similar, pero generaliza este concepto, porque a menudo querremos asociar otros valores numéricos con los resultados de un experimento aleatorio.
%        \begin{Ejemplo}\label{ejem:VariableAleatoria:SumaDosDados}
%        Quizá uno de los ejemplos más sencillos sea lo que ocurre cuando lanzamos dos dados, y nos fijamos en la suma de los valores obtenidos. Esa suma es siempre un número del 2 al 12, y es perfectamente legítimo hacer preguntas como ¿cuál es la probabilidad de que la suma valga $7$? Para responder a esa pregunta, iríamos al espacio muestral (formado por 36 resultados posibles), veríamos el valor de la suma en cada uno de ellos, para localizar aquellos en que la suma vale $7$. Así obtendríamos un suceso aleatorio $A=\{(1,6),(2,5),(3,4),(4,3),(5,2),(6,1)\}$, cuya probabilidad es $6/32$. De hecho podemos repetir lo mismo para cada uno de los posibles valores de la suma. Se obtiene esta tabla:\\[3mm]
%        \begin{tabular}[t]{|c|c|c|c|c|c|c|c|c|c|c|c|}
%        \hline
%        \rule{0cm}{0.5cm}{\em Valor de la suma:}&2&3&4&5&6&7&8&9&10&11&12\\
%        \hline
%        \rule{0cm}{0.7cm}{\em Probabilidad de ese valor:}&$\dfrac{1}{36}$&$\dfrac{2}{36}$&$\dfrac{3}{36}$&$\dfrac{4}{36}$&$\dfrac{5}{36}$&$\dfrac{6}{36}$&$\dfrac{5}{36}$&$\dfrac{4}{36}$&$\dfrac{3}{36}$&$\dfrac{2}{36}$&$\dfrac{1}{36}$\\
%        &&&&&&&&&&&\\
%        \hline
%        \end{tabular}\\[3mm]
%        Y de hecho, esta tabla es lo que en este caso caracteriza a la variable aleatoria suma.\qed
%        \end{Ejemplo}
%        Vamos ahora a ver otro ejemplo inspirado en los problemas de probabilidad geométrica.
%        \begin{Ejemplo}
%        Consideremos un círculo $C$ centrado en el origen y de radio 1. El espacio muestral $\Omega$ está formado por todos los subconjuntos\footnote{No excesivamente ``raros'', en el sentido que ya hemos discutido.} de puntos de $C$. Y la Función de Probabilidad se define así:
%        \[P(A)=\mbox{área de }A.\]
%        Consideremos ahora la variable que a cada punto del círculo le asocia su coordenada $x$. En este caso la coordenada $x$ toma cualquier valor real entre $-1$ y $1$. Y si preguntamos {``¿cuál es la probabilidad de que tome por ejemplo el valor $1/2$?''}, la respuesta es $0$. Porque los puntos del círculo donde toma ese valor forman un segmento (una cuerda del círculo), y el segmento tiene área $0$. Las cosas cambian si preguntamos {``¿cuál es la probabilidad de que la coordenada $x$ esté entre $0$ y $1/2$?''} En este caso, como muestra la figura
%        \begin{center}
%        \includegraphics[height=7cm]{2011_10_14_Figura01-VariableAleatoriaContinua.png}
%        \end{center}
%        el conjunto de puntos del círculo cuyas coordenadas $x$ están entre $0$ y $1/2$ tiene un área bien definida y no nula. ¿Cuánto vale ese área? Aproximadamente $0.48$, y esa es la probabilidad que buscábamos. El cálculo del área se puede hacer de distintas maneras, pero el lector debe darse cuenta de que en ejemplos como este se necesita a veces recurrir al cálculo de integrales.\\
%        Naturalmente, se pueden hacer preguntas más complicadas. Por ejemplo, dado un punto $(x,y)$ del círculo $C$ podemos calcular el valor de $f(x,y)=\frac{x^2}+ 4y^2$. Y entonces nos preguntamos ¿cuál es la probabilidad de que, tomando un punto al azar en $C$, el valor de $f$ esté entre 0 y 1? La respuesta es, de nuevo, un área, pero más complicada: es el área que se muestra en esta figura:
%        \begin{center}
%        \includegraphics[height=7cm]{2011_10_14_Figura02-VariableAleatoriaContinua.png}
%        \end{center}
%        Lo que tienen en común ambos casos es que hay una función (o fórmula), que es $x$ en el primero y $f(x,y)$ en el segundo, y que nos preguntamos por la probabilidad de que los valores de esa fórmula caigan dentro de un cierto intervalo.
%        \qed
%        \end{Ejemplo}
%        Los dos ejemplos que hemos visto contienen los ingredientes básicos de la noción de variable aleatoria. En el primer caso teníamos un conjunto finito de valores posibles, y a cada uno le asignábamos una probabilidad. En el segundo caso teníamos un rango continuo de valores posibles, y podíamos asignar probabilidades a intervalos. Lo que vamos a ver a continuación no se puede considerar de ninguna manera una definición rigurosa de variable aleatoria\footnote{La situación es similar a lo que ocurría al definir los sucesos aleatorios. Un suceso aleatorio $A$ es un subconjunto que tiene bien definida la probabilidad $P(A)$. Pero ya hemos dicho hay conjuntos tan {\em raros} que no es fácil asignarles un valor de la probabilidad (igual que a veces cuesta asignar un área). De la misma forma hay funciones tan raras que no se pueden considerar variables aleatorias. Se necesitan definiciones más rigurosas, pero que aquí sólo nos complicarían.}, pero servirá a nuestros propósitos.\\[3mm]
%        \fbox{\colorbox{Gris025}{\begin{minipage}{14cm}
%        \begin{center}
%        \vspace{2mm}
%        {\bf Variables aleatorias:}
%        \end{center}
%        Una variable aleatoria $X$ es una función (o fórmula) que le asigna, a cada elemento $p$ del espacio muestral $\Omega$, un número real $X(p)$. Distinguimos dos tipos de valores aleatorias:
%        \begin{enumerate}
%            \item La {\sf variable aleatoria $X$ es discreta} si sólo toma una cantidad finita (o una sucesión) de valores numéricos $x_1,x_2,x_3,\ldots$, de manera que para cada uno de esos valores tenemos bien definida la probabilidad $P(X=x_i)$ de que $X$ tome el valor $x_i$.
%            \item La {\sf variable aleatoria $X$ es continua} si sus valores forman un cierto rango continuo dentro de los números reales, de manera que si nos dan un intervalo $I=(a,b)$ (aquí puede ser $a=-\infty$ o $b=+\infty$), tenemos bien definida la probabilidad $P(X\in I)$ de que el valor de $X$ esté dentro de ese intervalo $I$.
%        \end{enumerate}
%        \end{minipage}}}\\[3mm]
%        Veamos un ejemplo, muy parecido al Ejemplo \ref{ejem:VariableAleatoria:SumaDosDados}.
%        \begin{Ejemplo}\label{ejem:VariableAleatoria:RestaDosDados}
%            De nuevo lanzamos dos dados, pero ahora nos fijamos en la diferencia de los valores obtenidos (el menor menos el mayor, y cero si son iguales). Si llamamos $(a,b)$ al resultado de lanzar los dados, donde $a$ y $b$ son números del 1 al 6, entonces estamos definiendo una variable aleatoria mediante la expresión
%            \[X(a,b)=|a-b|.\]
%            Esta claro que la variable $X$ toma solamente los valores $0,1,2,3,4,5$. ¿Cuál es la probabilidad de que al calcular $X$ obtengamos $3$? El siguiente diagrama ayudará a entender la respuesta. Para cada punto del espacio muestral,  se muestra el valor de $X$:
%            \[
%            \begin{array}{cccccc}
%            X(1,1)=0&X(1,2)=1&X(1,3)=2&X(1,4)=3&X(1,5)=4&X(1,6)=5\\
%            X(2,1)=1&X(2,2)=0&X(2,3)=1&X(2,4)=2&X(2,5)=3&X(2,6)=4\\
%            X(3,1)=2&X(3,2)=1&X(3,3)=0&X(3,4)=1&X(3,5)=2&X(3,6)=3\\
%            X(4,1)=3&X(4,2)=2&X(4,3)=1&X(4,4)=0&X(4,5)=1&X(4,6)=2\\
%            X(5,1)=4&X(5,2)=3&X(5,3)=2&X(5,4)=1&X(5,5)=0&X(5,6)=1\\
%            X(6,1)=5&X(6,2)=4&X(6,3)=3&X(6,4)=2&X(6,5)=1&X(6,6)=0
%            \end{array}
%            \]
%            Y se observa que $P(X=3)=6/36=1/6$. De hecho, podemos repetir lo mismo para cada uno de los posibles valores de la variable aleatoria $X$. Se obtiene esta tabla:
%            \begin{center}
%            \begin{tabular}[t]{|c|c|c|c|c|c|c|}
%            \hline
%            \rule{0cm}{0.5cm}{\em Valor de $X$ (diferencia):}&0&1&2&3&4&5\\
%            \hline
%            \rule{0cm}{0.7cm}{\em Probabilidad de ese valor:}&$\dfrac{6}{36}$&$\dfrac{10}{36}$&$\dfrac{8}{36}$&$\dfrac{6}{36}$&$\dfrac{4}{36}$&$\dfrac{2}{36}$\\
%            &&&&&&\\
%            \hline
%            \end{tabular}
%            \end{center}
%        Y de hecho, esta tabla es lo que en este caso caracteriza a la variable aleatoria diferencia $X$.\qed
%        \end{Ejemplo}
%
%\end{itemize}
%
%\subsection{Variables aleatorias y sucesos.}
%
%\begin{itemize}
%
%        \item Al principio la diferencia entre suceso aleatorio y variable aleatoria puede resultar un poco confusa. Vamos a recordar lo que es cada uno de estos conceptos:
%            \begin{enumerate}
%                \item que un suceso es un {\em subconjunto}, y que una variable aleatoria es una {\em función}. Por ejemplo, un suceso al lanzar dos dados puede ser ``los dos resultados son pares'', y este enunciado no hay un valor numérico fácil de identificar. Lo que sí tenemos es una {\em probabilidad asociada a este suceso}.
%                \item Por el contrario, en la variable aleatoria $X(a,b)=|a-b|$, definida en el espacio muestral de los 36 posibles resultados al lanzar dos dados, el valor numérico está claramente definido: $|a-b|$.
%            \end{enumerate}
%            ¿Cuál es entonces el origen de la confusión? Probablemente la parte más confusa es que {\sf las variables aleatorias definen sucesos cuando se les asigna un valor}. Por ejemplo, si escribimos $X(a,b)=|a-b|=3$, estamos pensando en el suceso {\em ``la diferencia de los resultados de los dados es 3''}. Y hemos visto en el Ejemplo \ref{ejem:VariableAleatoria:RestaDosDados} que la probabilidad de ese suceso es \[P(X=3)=1/6.\]
%
%        \item ¿Para qué sirven entonces las variables aleatorias? Precisamente su utilidad es que representan {\sf modelos abstractos de asignación de probabilidad}. Es decir, la variable aleatoria nos permite concentrar nuestra atención en la forma en que la probabilidad se asigna a los posibles resultados numéricos de un experimento aleatorio, sin entrar en los detalles sobre el espacio muestral y los sucesos subyacentes a esa asignación de probabilidad.  Vamos a ver un ejemplo que tal vez ayude a aclarar el sentido en el que estas variables aleatorias son resúmenes que eliminan detalles (y por tanto información).
%            \begin{Ejemplo}
%            Ya hemos visto que en el espacio muestral correspondiente al lanzamiento de dos dados, la variable aleatoria $X(a,b)=|a-b|$ tiene esta tabla de valores y probabilidades:
%            \begin{center}
%            \begin{tabular}[t]{|c|c|c|c|c|c|c|}
%                \hline
%                \rule{0cm}{0.5cm}{\em Valor de $X$ (diferencia):}&0&1&2&3&4&5\\
%                \hline
%                \rule{0cm}{0.7cm}{\em Probabilidad de ese valor:}&$\dfrac{6}{36}$&$\dfrac{10}{36}$&$\dfrac{8}{36}$&$\dfrac{6}{36}$&$\dfrac{4}{36}$&$\dfrac{2}{36}$\\
%                &&&&&&\\
%                \hline
%            \end{tabular}
%            \end{center}
%            Y, por su parte, la variable aleatoria suma $Y(a,b)=a+b$ tiene esta tabla:
%            \begin{center}
%            \begin{tabular}[t]{|c|c|c|c|c|c|c|c|c|c|c|c|}
%                \hline
%                \rule{0cm}{0.5cm}{\em Valor de la suma:}&2&3&4&5&6&7&8&9&10&11&12\\
%                \hline
%                \rule{0cm}{0.7cm}{\em Probabilidad de ese valor:}&$\dfrac{1}{36}$&$\dfrac{2}{36}$&$\dfrac{3}{36}$&$\dfrac{4}{36}$&$\dfrac{5}{36}$&$\dfrac{6}{36}$&$\dfrac{5}{36}$&$\dfrac{4}{36}$&$\dfrac{3}{36}$&$\dfrac{2}{36}$&$\dfrac{1}{36}$\\
%                &&&&&&&&&&&\\
%            \hline
%            \end{tabular}
%            \end{center}
%            En el Ejemplo \ref{S1310-ejem:probabilidadCondicionadaLanzamientoDosDados} de la sesión del Jueves 13/10 nos hicimos la pregunta {\em `` ¿Cuál es la probabilidad de que la diferencia entre los valores de ambos dados (mayor-menor) sea menor que 4, sabiendo que la suma de los dados es 7?''} Está claro, con la notación que usamos ahora, que estamos preguntando cuál es la probabilidad del suceso
%            \[P(X<4)\cap P(Y=7).\]
%            ¿Podemos calcular este número usando sólo las tablas de probabilidad de $X$ e $Y$, sin utilizar más información sobre el suceso muestral subyacente?\qed
%            \end{Ejemplo}
%        \item En el caso de las variables aleatorias discretas, hemos visto que conocer la variable es esencialmente lo mismo que conocer la tabla de probabilidades asignadas a cada uno de los posibles valores de la variable (esta tabla se conoce como {\sf función de probabilidad o función de masa} de la variable aleatoria). En el caso de las variables aleatorias continuas, no podemos hacer la asignación de probabilidades de esta misma forma. Recordando que la probabilidad de las variables continuas es análoga al área, necesitamos un recurso técnicamente más complicado, y eso es lo que vamos a hacer en la próxima sesión al presentar los conceptos de función de densidad y función de distribución.
%\end{itemize}

%\section{Función de distribución.}
%
%
%\subsection{Variables aleatorias discretas.}
%
%\begin{itemize}
%
%    \item El concepto de probabilidad condicionada trata de reflejar los cambios en el valor de la Función Probabilidad que se producen cuando tenemos {\em información parcial} sobre el resultado de un experimento aleatorio. Por ejemplo, utilizando uno de esos casos en los que la Regla de Laplace es suficiente, si al lanzar dos dados sabemos que el resultado obtenido ha sido mayor que 3, entonces nuestro cálculo de la probabilidad de que haya sido un $7$  es distinto, porque el número de resultados posibles (el denominador en la fórmula de Laplace), ha cambiado.
%
%    \item Usando como guía la Regla de Laplace, estamos tratando de definir la {\em probabilidad del suceso $A$ sabiendo que ha ocurrido el suceso $B$}. Esto es lo que vamos a llamar la {\sf probabilidad de $A$ condicionada por $B$, y lo representamos por $P(A|B)$}. Pensemos en cuáles son los cambios en la aplicación de la Regla de Laplace (favorables/posibles) cuando sabemos que el suceso $B$ ha ocurrido. Antes que nada recordemos que, si el total de resultados elementales posibles es $n$ entonces
%        \[P(A)=\dfrac{\mbox{núm. de casos favorables a $A$}}{n}\mbox{, y también }P(B)=\dfrac{\mbox{núm. de casos favorables a $B$}}{n}.\]
%        Veamos ahora como deberíamos definir $P(A|B)$. Puesto que sabemos que $B$ ha ocurrido, los casos posibles ya no son todos los $n$ casos posibles originales: ahora los únicos casos posibles son los que corresponden al suceso $B$.  ¿Y cuáles son los casos favorables del suceso $A$, una vez que sabemos que $B$ ha ocurrido? Pues aquellos casos en los que $A$ y $B$ ocurren simultáneamente (o sea, el suceso $A\cap B$). En una fracción:
%        \[P(A|B)=\dfrac{\mbox{número de casos favorables a $A\cap B$}}{\mbox{número de casos favorables a $B$}}.\]
%        Si sólo estuviéramos interesados en la Regla de Laplace esto sería tal vez suficiente. Pero para poder generalizar esto hay una manera mejor de escribirlo. Dividimos el numerador y el denominador por $n$ y tenemos:
%        \[P(A|B)=\dfrac{\quad\left(\dfrac{\mbox{número de casos favorables a $A\cap B$}}{n}\right)\quad}{\left(\dfrac{\mbox{número de casos favorables a $B$}}{n}\right)}=\dfrac{P(A\cap B)}{P(B)}.\]
%        ¿Qué tiene de bueno esto? Pues que la expresión que hemos obtenido ya no hace ninguna referencia a casos favorables o posibles, nos hemos librado de la Regla de Laplace y hemos obtenido una expresión general que sólo usa la Función de Probabilidad (y por tanto podremos usarla, por ejemplo, en los problemas de probabilidad geométrica). Ya tenemos la definición:
%        \begin{center}
%        \fbox{\colorbox{Gris025}{\begin{minipage}{14cm}
%        \begin{center}
%        \vspace{2mm}
%        {\bf Probabilidad Condicionada:}
%        \end{center}
%        La probabilidad del suceso $A$ condicionada por el suceso $B$ se define así (y se supone que $P(B)\neq 0$.)
%            \[P(A|B)=\dfrac{P(A\cap B)}{P(B)}.\]
%        \end{minipage}}}
%        \end{center}
%
%
%\end{itemize}
%
%\subsection{Sucesos independientes.}
%
%\begin{itemize}
%
%    \item ¿Qué significado debería tener la frase {\em ``el suceso $A$ es independiente del suceso $B$''}\,? Parece evidente que, si los suceso son independientes, el hecho de saber que el suceso $B$ ha ocurrido no debería afectar para nada nuestro cálculo de la probabilidad de que ocurra $A$. Esta idea tiene una traducción inmediata en el lenguaje de la probabilidad condicionada, que es de hecho la definición de sucesos independientes:
%        \vspace{-3mm}
%        \begin{center}
%        \fbox{\colorbox{Gris025}{\begin{minipage}{14cm}
%        \begin{center}
%        \vspace{2mm}
%        {\bf Sucesos independientes:}
%        \end{center}
%        Los sucesos $A$ y $B$ son independientes si
%            \[P(A|B)=P(A).\]
%        Esto es equivalente a decir que
%        \[P(A\cap B)=P(A)P(B).\]
%        \end{minipage}}}
%        \end{center}
%
%    \item En general los sucesos $A_1,\ldots,A_n$ son independientes cuando para {\em cualquier colección} que tomemos de ellos, la probabilidad de la intersección es el producto de las probabilidades.
%
%\end{itemize}
%
%
%%\section{La regla de las probabilidad totales. Teorema de Bayes.}
%%
%%
%%\subsection{Probabilidad condicionada.}
%%
%%\begin{itemize}
%%
%%    \item
%%
%%\end{itemize}
%

\section{Otras fórmulas combinatorias}

\noindent{\textcolor{red}{\bf Atención:}
Aunque las incluimos aquí para complementar la información de este capítulo, estas fórmulas son mucho menos importantes para nosotros que las de los números combinatorios.}


\subsection{Permutaciones}

\begin{itemize}
\item {\bf Sin repetici\'on}

Se trata de obtener las {\sf distintas formas de ordenar} los elementos de un conjunto de $n$ elementos. Hay $\mbox{Per}(n)=n!$ de ellas.
\fobox{\operatorname{Per}(n)=n!}

\item {\bf Con repetici\'on}

El n\'umero de permutaciones (el orden es importante) que se pueden formar con $m$ objetos
entre los cuales hay $n_1$ iguales entre s\'{\i}, otros $n_2$ iguales
entre s\'{\i},\dots, y finalmente $n_k$ iguales entre s\'{\i}, es:
\fobox{\operatorname{PerRep}(n_1,n_2,\dots,n_k)=\dfrac{m!}{n_1!n_2!\dots n_k!}
\mbox{ con }n_1+n_2+\dots+n_k=m}

\end{itemize}

\subsection{Variaciones}

\begin{itemize}

\item {\bf Sin repetici\'on} Listas de $k$ elementos entre $n$ posibles, sin repetir
elementos y considerando distintas dos listas si el orden de los elementos es distinto.
\fobox{\operatorname{V}(n,k)=n\cdot(n-1)\cdot\dots\cdot(n-k+1)=\dfrac{n!}{(n-k)!}}

\item {\bf Con repetici\'on}

Si se permite que cada elemento aparezca tantas veces como se quiera, entonces tenemos simplemente:
\fobox{\operatorname{VRep}(n,k)=n^k}

\end{itemize}

\subsection{Combinaciones con repetici\'on}


\begin{itemize}

\item Selecciones de $k$ elementos entre $n$ posibles, admitiendo la repetici\'on de elementos, pero sin tener en cuenta el orden de la selecci\'on.
\fobox{\operatorname{CRep}(n,k)=\binom{n+k-1}{k}}

\end{itemize}



%\section*{Tareas asignadas para esta sesión.}
%
%\begin{enumerate}
%    \item Supongamos ahora que lanzamos un dado cuatro veces y nos preguntamos por la probabilidad de sacar exactamente dos seises. Esto es similar al problema del Caballero De Mere, que vimos en la sesión 7 y recuerda mucho al Ejemplo \ref{ejem:probabilidadLanzamientoMonedas}, hasta el punto de que es razonable preguntarse si la respuesta es la misma. Usa las ideas de esta sesión para obtener la respuesta. Escribe esa respuesta y el razonamiento que te conduce a ella en Moodle.
%%    \item Está claro que si en una habitación hay 367 personas, entonces hay al menos dos de ellas que cumplen años el mismo día, ¿verdad? ¿Cuál es el número mínimo de personas que debe haber en esa habitación para que la probabilidad sea superior al 50\%? Escribe tu respuesta y el razonamiento que te conduce a ella en Moodle.
%\end{enumerate}


%\section*{Recomendaciones.}
%
%\begin{enumerate}
%   \item El \link{http://www.ine.es/}{INE (Instituto Nacional de Estadística)} es el organismo oficial encargado, entre otras cosas del censo electoral, la elaboración del IPC (índice de precios de consumo), la EPA (encuesta de población activa), el PIB (producto interior bruto) etc. El instituto ofrece una enorme colección de datos estadísticos accesibles para cualquiera a través de la red (sistema INEbase). Además, tiene alojado en su página web un \link{http://www.ine.es/explica/explica.htm}{portal de divulgación estadística} en el que se pueden ver vídeos sobre estos y otros temas, que tal vez os interesen.
%\end{enumerate}

%\section*{\fbox{\colorbox{Gris025}{{Sesión 11. Probabilidad.}}}}
%
%\subsection*{\fbox{\colorbox{Gris025}{{Combinatoria.}}}}
%\subsection*{Fecha: Viernes, 21/10/2011, 14h.}
%
%\noindent{\bf Atención:
%En esta sesión vamos a ver también la regla de la probabilidad total y el teorema de Bayes, que aparecen en el resumen de la sesión del Jueves 13/10.
%}
%
%%\subsection*{\fbox{1. Ejemplos preliminares }}
%\setcounter{tocdepth}{1}
%%\tableofcontents
%\section*{Lectura recomendada}
%
%Las mismas de la sesión anterior.
%
%\section{Combinaciones}
%
%
%\subsection{Introducción.}
%
%\begin{itemize}
%
%    \item La Combinatoria es una parte de las matemáticas que estudia técnicas de recuento. En particular, estudia las posibles formas de seleccionar listas o subconjuntos de elementos de un conjunto dado siguiendo ciertos criterios (ordenados o no, con repetición o no, etcétera). Por esa razón es de mucha utilidad para el cálculo de probabilidades. La Combinatoria, no obstante, puede ser muy complicada, y en este curso vamos a concentrarnos en los resultados que necesitamos.
%
%    \item En particular estamos interesados en este problema. Dado un conjunto de $n$ elementos
%    \[A=\{x_1,x_2,\ldots,x_n\}\]
%    y un número $k$ con $0\leq k\leq n$, ¿cuántos subconjuntos distintos de $k$ elementos podemos formar con los elementos de $A$? Al usar la palabra subconjunto estamos diciendo que:
%    \begin{enumerate}
%        \item el {\sf orden de los elementos es irrelevante}. El subconjunto $\{x_1,x_2,x_3\}$ es el mismo que el subconjunto $\{x_3,x_1,x_2\}$.
%        \item los elementos del subconjunto {\sf no se repiten}.
%    \end{enumerate}
%    El número de subconjuntos posibles es el número de {\sf combinaciones de $n$ elementos tomados de $k$ en $k$} (cada uno de los subconjuntos es una combinación).
%    \begin{Ejemplo}\label{ejem:combinacionesCuatroDosEnDos}
%    Por ejemplo, en un conjunto de 4 elementos $A=\{a,b,c,d\}$, hay seis combinaciones distintas de elementos tomados de dos en dos :
%    \[\{a,b\},\{a,c\},\{a,d\},\{b,c\},\{b,d\},\{c,d\}\]
%    \qed
%    \end{Ejemplo}
%\end{itemize}
%
%\subsection{Números combinatorios}
%
%\begin{itemize}
%
%    \item Para calcular probabilidades de forma eficaz muchas veces necesitamos una manera de calcular el número de combinaciones posibles. El {\sf número combinatorio} $\binom{n}{k}$ es el número de combinaciones de $n$ elementos tomados de $k$ en $k$.
%
%    \item Para calcular ese número necesitamos el concepto de {\sf factorial}. El factorial del número $n$ es el producto
%        \[\fbox{$n!=n\cdot(n-1)\cdot(n-2)\cdot\,\ldots\,\cdot3\cdot 2\cdot 1.$}\]
%        Es decir, el producto de todos los números entre $1$ y $n$. Además definimos \fbox{$0!=1$}. Por ejemplo, se tiene:
%        \[6!=6\cdot 5\cdot 4\cdot 3\cdot 2\cdot 1=720.\]
%        La propiedad más importante del factorial es su crecimiento extremadamente rápido. Por ejemplo, $100!$ es del orden de $10^{57}$.
%
%    \item La fórmula de los números combinatorios es esta:
%    \\[3mm]
%        \fbox{\colorbox{Gris025}{\begin{minipage}{14cm}
%        \begin{center}
%        \vspace{2mm}
%        {\bf Números combinatorios:}
%        \end{center}
%        \[\binom{n}{k}=\dfrac{n!}{k!(n-k)!}\mbox{, \quad para $0\leq k\leq n$, y $n=0,1,2\ldots$ cualquier número natural}\]
%        \quad
%        \end{minipage}}}\\[3mm]
%        Por ejemplo, el número de combinaciones del Ejemplo \ref{ejem:combinacionesCuatroDosEnDos} es
%        \[\binom{4}{2}=\dfrac{4!}{2!(4-2)!}=\dfrac{24}{2\cdot 2}=6,\]
%        como era de esperar.
%
%    \item Hay dos observaciones que facilitan bastante el trabajo con estos números combinatorios.
%    \begin{enumerate}
%        \item Los números combinatorios se pueden representar en esta tabla de forma triangular, llamada el {\sf Triángulo de Pascal}:
%        \[
%        \begin{array}{l|llcccccccccccccc}
%        n=0&&&&&&&&&1\\
%        n=1&&&&&&&&1&&1\\
%        n=2&&&&&&&1&&2&&1\\
%        n=3&&&&&&1&&3&&3&&1\\
%        n=4&&&&&1&&4&&6&&4&&1\\
%        n=5&&&&1&&5&&10&&10&&5&&1\\
%        n=6&&&1&&6&&15&&20&&15&&6&&1\\
%        \vdots&&& &&\vdots&& &&\vdots&& &&\vdots&&
%        \end{array}
%        \]
%        El número $\binom{n}{k}$ ocupa la fila $n$ posición $k$ (se cuenta desde $0$). Por ejemplo en la $4$ fila, posición $2$ está nuestro viejo conocido $\binom{4}{2}=6$. ¿Cuánto vale $\binom{5}{3}$?
%
%        Los puntos suspensivos de la parte inferior están ahí para indicarnos qué podríamos seguir, y a la vez para servir de desafío. ¿Qué viene a continuación? ¿Qué hay en la línea $n=15$? Pues parece claro que empezará y acabará con un $1$. También parece claro que el segundo y el penúltimo  número valen $7$. ¿Pero y el resto? Lo que hace especial a esta tabla es que {\sf cada número que aparece en el interior de la tabla es la suma de los dos situados a su izquierda y derecha en la fila inmediatamente superior.} Por ejemplo, el $10$ que aparece en tercer lugar en la fila de $n=5$  es la suma del $4$ y el $6$ situados sobre él en la  segunda y tercera posiciones de la fila para $n=4$. Con esta información, podemos         obtener la séptima fila de la tabla, a partir de la sexta sumado según indican las flechas en este esquema:
%        \[
%        \begin{array}{l|llcccccccccccccccc}
%        n=6&&&&1&&6&&15&&20&&15&&6&&1\\
%           &&&&\swarrow\searrow&&\swarrow\searrow&&\swarrow\searrow&&\swarrow\searrow&&\swarrow\searrow&&
%           \swarrow\searrow&&\swarrow\searrow\\
%        n=7&&&1&&7&&21&&35&&35&&21&&7&&1
%        \end{array}
%        \]
%        \item La segunda observación importante sobre los números quedará más clara con un ejemplo:
%        \[
%        \binom{12}{7}=\dfrac{12!}{7!(12-7)!}=\dfrac{12!}{7!5!}.
%        \]
%        Ahora observamos que $12!=(12\cdot 11\cdot\cdots\cdot 6)\cdot(5\cdot\cdots\cdot 2\cdot 1)$, y los paréntesis muestran que esto
%        es igual a $(12\cdot 11\cdot\cdots\cdot 6)\cdot 5!$. Este factorial de $5$ se cancela con el del denominador y tenemos
%        \[
%        \binom{12}{7}=\dfrac{\overbrace{12\cdot 11\cdot 10\cdot 9\cdot 8\cdot 7\cdot 6}^{6\mbox{ factores}}}{7!}=792.
%        \]
%        Generalizando esta observación sobre la cancelación de factoriales, la forma en la que vamos a expresar los coeficientes binomiales será finalmente esta:
%        \begin{equation}\label{cap03:ecu:expresionPseudoFactorialCoeficientesBinomiales}
%        \dbinom{n}{k}=\frac{\overbrace{n\left( n-1\right) \left( n-2\right) \cdots \left( n-k+1\right) }^{k\mbox{ factores}}}{k!}
%        \end{equation}
%        Y, como hemos indicado, lo que caracteriza este esta expresión es que tanto el numerador como el denominador tienen $k$ factores.
%    \end{enumerate}
%
%    \item Los números combinatorios son importantes en muchos problemas de probabilidad. Veamos un par de ejemplos:
%        \begin{Ejemplo}
%        Tenemos una caja de 10 bombillas y sabemos que tres están fundidas. Si sacamos al azar tres bombillas de la caja\footnote{``al azar'' aquí significa que todos los subconjuntos de tres bombillas son equiprobables.}, ¿Cuál es la probabilidad de que hayamos sacado las tres que están fundidas?\\[2mm]
%        En este caso, al tratar de aplicar la Regla de Laplace, usamos los números combinatorios para establecer el número de casos posibles. ¿Cuántas formas distintas hay de seleccionar tres bombillas de un conjunto de 10? Evidentemente hay $\binom{10}{3}$ formas posibles. Este número es:
%        \[\binom{10}{3}=\dfrac{10\cdot 9\cdot 8}{3\cdot 2\cdot 1}=120.\]
%        Estos son los casos posibles. Está claro además que sólo hay un caso favorable, cuando elegimos las tres bombillas defectuosas. Así pues, la probabilidad pedida es:
%        \[\dfrac{1}{120}.\]
%        \qed
%        \end{Ejemplo}
%        El siguiente ejemplo es extremadamente importante para el resto del curso, porque nos abre la puerta que nos conducirá a la distribución binomial y a algunos de los resultados más profundos de la Estadística.
%        \begin{Ejemplo}\label{ejem:probabilidadLanzamientoMonedas}
%        Lanzamos una moneda al aire cuatro veces, y contamos el número de caras obtenidas en esos lanzamientos. ¿Cuál es la probabilidad de obtener exactamente dos caras en total?\\
%        Vamos a pensar en cuál es el espacio muestral. Se trata de listas de cuatro símbolos cara o cruz. Por ejemplo,
%        \[\smiley\smiley\dagger\smiley\]
%        es un resultado posible, con tres caras y una cruz. ¿Cuántas de estas listas de cara y cruz con cuatro símbolos hay? Enseguida se ve que hay $2^4$, así que ese es el número de casos posibles. ¿Y cuál es el número de casos favorables? Aquí es donde los números combinatorios. Podemos pensar así en los sucesos favorables: tenemos cuatro fichas, dos caras y dos cruces $\smiley,\smiley,\dagger,\dagger,$ y un casillero con cuatro casillas
%        \begin{center}
%        \begin{tabular}{|c|c|c|c|}
%        \hline
%         \rule{0cm}{0.5cm}\rule{1cm}{0cm}&\rule{1cm}{0cm}&\rule{1cm}{0cm} &\rule{1cm}{0cm}\\
%         \hline
%         \end{tabular}
%         \end{center}
%         en las que tenemos que colocar esas cuatro fichas. Cada manera de colocarlas corresponde a un suceso favorable. Y entonces está claro que lo que tenemos que hacer es elegir, de entre esas cuatro casillas, cuáles dos llevarán una cara (las restantes dos llevarán una cruz). Es decir, hay que elegir dos de entre cuatro. Y ya sabemos que la respuesta es $\binom{4}{2}=6$. Por lo tanto la probabilidad pedida es:
%         \[p(2 \mbox{ caras} )=\dfrac{\binom{4}{2}}{2^4}=\binom{4}{2}\left(\dfrac{1}{2}\right)^4=\dfrac{6}{16}.\]
%         Supongamos ahora que lanzamos la moneda $n$ veces y queremos saber cuál es la probabilidad de obtener $k$ veces cara. Un razonamiento similar produce la fórmula:
%        \[p(k \mbox{ caras})=\binom{n}{k}\left(\dfrac{1}{2}\right)^n.\]
%        \qed
%        \end{Ejemplo}
%
%        \item No podemos dejar de mencionar que los números combinatorios son también importantes en relación con el Teorema del Binomio, y que por eso se los conoce también como {\sf coeficientes binomiales}. En concreto, se tiene, para $a,b\in\R$, y $n\in\N$ esta {\sf Fórmula del Binomio}:
%             \[
%             (a+b)^n=\binom{n}{0}a^n+\binom{n}{1}a^{n-1}b+\binom{n}{2}a^{n-2}b^2+\cdots+\binom{n}{n-1}ab^{n-1}+\binom{n}{n}b^n
%              \]
%\end{itemize}
%
%
%
%\newpage
%\section{Otras fórmulas combinatorias}
%
%\noindent{\textcolor{red}{\bf Atención:}
%Aunque las incluimos aquí para complementar la información de la sesión de hoy, estas fórmulas son mucho menos importantes para nosotros que las de los números combinatorios.}
%
%
%\subsection{Permutaciones}
%
%\begin{itemize}
%\item {\bf Sin repetici\'on}
%
%Se trata de obtener las {\sf distintas formas de ordenar} los elementos de un conjunto de $n$ elementos. Hay $\mbox{Per}(n)=n!$ de ellas.
%\fobox{\operatorname{Per}(n)=n!}
%
%\item {\bf Con repetici\'on}
%
%El n\'umero de permutaciones (el orden es importante) que se pueden formar con $m$ objetos
%entre los cuales hay $n_1$ iguales entre s\'{\i}, otros $n_2$ iguales
%entre s\'{\i},\dots, y finalmente $n_k$ iguales entre s\'{\i}, es:
%\fobox{\operatorname{PerRep}(n_1,n_2,\dots,n_k)=\dfrac{m!}{n_1!n_2!\dots n_k!}
%\mbox{ con }n_1+n_2+\dots+n_k=m}
%
%\end{itemize}
%
%\subsection{Variaciones}
%
%\begin{itemize}
%
%\item {\bf Sin repetici\'on} Listas de $k$ elementos entre $n$ posibles, sin repetir
%elementos y considerando distintas dos listas si el orden de los elementos es distinto.
%\fobox{\operatorname{V}(n,k)=n\cdot(n-1)\cdot\dots\cdot(n-k+1)=\dfrac{n!}{(n-k)!}}
%
%\item {\bf Con repetici\'on}
%
%Si se permite que cada elemento aparezca tantas veces como se quiera, entonces tenemos simplemente:
%\fobox{\operatorname{VRep}(n,k)=n^k}
%
%\end{itemize}
%
%\subsection{Combinaciones con repetici\'on}
%
%
%\begin{itemize}
%
%\item Selecciones de $k$ elementos entre $n$ posibles, admitiendo la repetici\'on de elementos, pero sin tener en cuenta el orden de la selecci\'on.
%\fobox{\operatorname{CRep}(n,k)=\binom{n+k-1}{k}}
%
%\end{itemize}






    \chapter{Variables aleatorias}
    % !Mode:: "Tex:UTF-8"

\section{Variables aleatorias}\label{sec:variablesAletorias}


\subsection{¿Qué son las variables aleatorias?}

\begin{itemize}

    \item Hemos visto que cada suceso $A$ del espacio muestral $\Omega$ tiene asociado un valor $P(A)$ de la función probabilidad. Y sabemos que los valores de la función probabilidad son valores positivos, comprendidos entre $0$ y $1$. La idea de variable aleatoria es similar, pero generaliza este concepto, porque a menudo querremos asociar otros valores numéricos con los resultados de un experimento aleatorio.
        \begin{Ejemplo}\label{ejem:VariableAleatoria:SumaDosDados}
        Quizá uno de los ejemplos más sencillos sea lo que ocurre cuando lanzamos dos dados, y nos fijamos
        en la suma de los valores obtenidos. Esa suma es siempre un número del 2 al 12, y es perfectamente
        legítimo hacer preguntas como ¿cuál es la probabilidad de que  la suma valga $7$? Para responder a esa
        pregunta, iríamos al espacio muestral (formado por 36 resultados posibles), veríamos el valor de la suma
        en cada uno de ellos, para localizar aquellos en que la suma vale $7$. Así obtendríamos un suceso
        aleatorio $A=\{(1,6),(2,5),(3,4),(4,3),(5,2),(6,1)\}$, cuya probabilidad es $6/36$. De hecho podemos
        repetir lo mismo para cada uno de los posibles valores de la suma. Se obtiene esta tabla:\\[3mm]
        \begin{tabular}[t]{|c|c|c|c|c|c|c|c|c|c|c|c|}
        \hline
        \rule{0cm}{0.5cm}{\em Valor de la suma:}&2&3&4&5&6&7&8&9&10&11&12\\
        \hline
        \rule{0cm}{0.7cm}{\em Probabilidad de ese valor:}&$\dfrac{1}{36}$&$\dfrac{2}{36}$&$\dfrac{3}{36}$&$\dfrac{4}{36}$&$\dfrac{5}{36}$&$\dfrac{6}{36}$&$\dfrac{5}{36}$&$\dfrac{4}{36}$&$\dfrac{3}{36}$&$\dfrac{2}{36}$&$\dfrac{1}{36}$\\
        &&&&&&&&&&&\\
        \hline
        \end{tabular}\\[3mm]
        Y de hecho, esta tabla es lo que en este caso caracteriza a la variable aleatoria suma.\qed
        \end{Ejemplo}
        Vamos ahora a ver otro ejemplo inspirado en los problemas de probabilidad geométrica.
        \begin{Ejemplo}\label{ejem:ProbabilidadGeometricaSubconjuntosCirculo}
        Consideremos un círculo $C$ centrado en el origen y de radio 1. El espacio muestral $\Omega$ está formado por todos los subconjuntos\footnote{No excesivamente ``raros'', en el sentido que ya hemos discutido.} de puntos de $C$. Y la Función de Probabilidad se define así:
        \[P(A)=\mbox{área de }A.\]
        Consideremos ahora la variable $X(x,y)=x$, que a cada punto del círculo le asocia su coordenada $x$. En este caso la coordenada $x$ toma cualquier valor real entre $-1$ y $1$. Y si preguntamos {``¿cuál es la probabilidad de que tome por ejemplo el valor $1/2$?''}, la respuesta es $0$. Porque los puntos del círculo donde toma ese valor forman un segmento (una cuerda del círculo), y el segmento tiene área $0$. Las cosas cambian si preguntamos {``¿cuál es la probabilidad de que la coordenada $x$ esté entre $0$ y $1/2$?''} En este caso, como muestra la figura
        \begin{center}
        \includegraphics[height=7cm]{2011_10_25_Figura01-VariableAleatoriaContinua.png}
        \end{center}
        el conjunto de puntos del círculo cuyas coordenadas $x$ están entre $0$ y $1/2$ tiene un área bien definida y no nula. ¿Cuánto vale ese área? Aproximadamente $0.48$, y esa es la probabilidad que buscábamos. El cálculo del área se puede hacer de distintas maneras, pero el lector debe darse cuenta de que en ejemplos como este se necesita a veces recurrir al cálculo de integrales.\\
        Naturalmente, se pueden hacer preguntas más complicadas. Por ejemplo, dado un punto $(x,y)$ del círculo
         $C$ podemos calcular el valor de $f(x,y)=x^2+ 4y^2$. Y entonces nos preguntamos ¿cuál es la probabilidad de que, tomando un punto al azar en $C$, el valor de $f$ esté entre 0 y 1? La respuesta es, de nuevo, un área, pero más complicada: es el área que se muestra en esta figura:
        \begin{center}
        \includegraphics[height=7cm]{2011_10_25_Figura02-VariableAleatoriaContinua.png}
        \end{center}
        Lo que tienen en común ambos casos es que hay una función (o fórmula), que es $x$ en el primero y $f(x,y)$ en el segundo, y que nos preguntamos por la probabilidad de que los valores de esa fórmula caigan dentro de un cierto intervalo.
        \qed
        \end{Ejemplo}
        Los dos ejemplos que hemos visto contienen los ingredientes básicos de la noción de variable aleatoria. En el primer caso teníamos un conjunto finito de valores posibles, y a cada uno le asignábamos una probabilidad. En el segundo caso teníamos un rango continuo de valores posibles, y podíamos asignar probabilidades a intervalos. Lo que vamos a ver a continuación no se puede considerar de ninguna manera una definición rigurosa de variable aleatoria\footnote{La situación es similar a lo que ocurría al definir los sucesos aleatorios. Un suceso aleatorio $A$ es un subconjunto que tiene bien definida la probabilidad $P(A)$. Pero ya hemos dicho hay conjuntos tan {\em raros} que no es fácil asignarles un valor de la probabilidad (igual que a veces cuesta asignar un área). De la misma forma hay funciones tan raras que no se pueden considerar variables aleatorias. Se necesitan definiciones más rigurosas, pero que aquí sólo nos complicarían.}, pero servirá a nuestros propósitos.\\[3mm]
        \fbox{\colorbox{Gris025}{\begin{minipage}{14cm}
        \begin{center}
        \vspace{2mm}
        {\bf Variables aleatorias:}
        \end{center}
        Una variable aleatoria $X$ es una función (o fórmula) que le asigna, a cada elemento $p$ del espacio muestral $\Omega$, un número real $X(p)$. Distinguimos dos tipos de valores aleatorias:
        \begin{enumerate}
            \item La {\sf variable aleatoria $X$ es discreta} si sólo toma una cantidad finita (o una sucesión) de valores numéricos $x_1,x_2,x_3,\ldots$, de manera que para cada uno de esos valores tenemos bien definida la probabilidad $P(X=x_i)$ de que $X$ tome el valor $x_i$.
            \item La {\sf variable aleatoria $X$ es continua} si sus valores forman un cierto rango continuo dentro de los números reales, de manera que si nos dan un intervalo $I=(a,b)$ (aquí puede ser $a=-\infty$ o $b=+\infty$), tenemos bien definida la probabilidad $P(X\in I)$ de que el valor de $X$ esté dentro de ese intervalo $I$.
        \end{enumerate}
        \end{minipage}}}\\[3mm]
        Veamos un ejemplo, muy parecido al Ejemplo \ref{ejem:VariableAleatoria:SumaDosDados}.
        \begin{Ejemplo}\label{ejem:VariableAleatoria:RestaDosDados}
            De nuevo lanzamos dos dados, pero ahora nos fijamos en la diferencia de los valores obtenidos (el menor menos el mayor, y cero si son iguales). Si llamamos $(a,b)$ al resultado de lanzar los dados, donde $a$ y $b$ son números del 1 al 6, entonces estamos definiendo una variable aleatoria mediante la expresión
            \[X(a,b)=|a-b|.\]
            Esta claro que la variable $X$ toma solamente los valores $0,1,2,3,4,5$. ¿Cuál es la probabilidad de que al calcular $X$ obtengamos $3$? El siguiente diagrama ayudará a entender la respuesta. Para cada punto del espacio muestral,  se muestra el valor de $X$:
            \[
            \begin{array}{cccccc}
            X(1,1)=0&X(1,2)=1&X(1,3)=2&X(1,4)=3&X(1,5)=4&X(1,6)=5\\
            X(2,1)=1&X(2,2)=0&X(2,3)=1&X(2,4)=2&X(2,5)=3&X(2,6)=4\\
            X(3,1)=2&X(3,2)=1&X(3,3)=0&X(3,4)=1&X(3,5)=2&X(3,6)=3\\
            X(4,1)=3&X(4,2)=2&X(4,3)=1&X(4,4)=0&X(4,5)=1&X(4,6)=2\\
            X(5,1)=4&X(5,2)=3&X(5,3)=2&X(5,4)=1&X(5,5)=0&X(5,6)=1\\
            X(6,1)=5&X(6,2)=4&X(6,3)=3&X(6,4)=2&X(6,5)=1&X(6,6)=0
            \end{array}
            \]
            Y se observa que $P(X=3)=6/36=1/6$. De hecho, podemos repetir lo mismo para cada uno de los posibles valores de la variable aleatoria $X$. Se obtiene esta tabla:
            \begin{center}
            \begin{tabular}[t]{|c|c|c|c|c|c|c|}
            \hline
            \rule{0cm}{0.5cm}{\em Valor de $X$ (diferencia):}&0&1&2&3&4&5\\
            \hline
            \rule{0cm}{0.7cm}{\em Probabilidad de ese valor:}&$\dfrac{6}{36}$&$\dfrac{10}{36}$&$\dfrac{8}{36}$&$\dfrac{6}{36}$&$\dfrac{4}{36}$&$\dfrac{2}{36}$\\
            &&&&&&\\
            \hline
            \end{tabular}
            \end{center}
        Y de hecho, esta tabla es lo que en este caso caracteriza a la variable aleatoria diferencia $X$.\qed
        \end{Ejemplo}

\end{itemize}

\subsection{Variables aleatorias y sucesos.}

\begin{itemize}

        \item Al principio la diferencia entre suceso aleatorio y variable aleatoria puede resultar un poco confusa. Vamos a recordar lo que es cada uno de estos conceptos:
            \begin{enumerate}
                \item Un suceso es un {\em subconjunto}, mientras que una variable aleatoria es una {\em función}. Por ejemplo, un suceso al lanzar dos dados puede ser ``los dos resultados son pares'', y este enunciado no hay un valor numérico fácil de identificar. Lo que sí tenemos es una {\em probabilidad asociada a este suceso}.
                \item Por el contrario, en la variable aleatoria $X(a,b)=|a-b|$, definida en el espacio muestral de los 36 posibles resultados al lanzar dos dados, el valor numérico está claramente definido: $|a-b|$.
            \end{enumerate}
            ¿Cuál es entonces el origen de la confusión? Probablemente la parte más confusa es que {\sf las variables aleatorias definen sucesos cuando se les asigna un valor}. Por ejemplo, si escribimos $X(a,b)=|a-b|=3$, estamos pensando en el suceso {\em ``la diferencia de los resultados de los dados es 3''}. Y hemos visto en el Ejemplo \ref{ejem:VariableAleatoria:RestaDosDados} que la probabilidad de ese suceso es \[P(X=3)=1/6.\]

        \item ¿Para qué sirven entonces las variables aleatorias? Precisamente su utilidad es que representan {\sf modelos abstractos de asignación de probabilidad}. Es decir, la variable aleatoria nos permite concentrar nuestra atención en la forma en que la probabilidad se asigna a los posibles resultados numéricos de un experimento aleatorio, sin entrar en los detalles sobre el espacio muestral y los sucesos subyacentes a esa asignación de probabilidad.  Vamos a ver un ejemplo que tal vez ayude a aclarar el sentido en el que estas variables aleatorias son resúmenes que eliminan detalles (y por tanto información).
            \begin{Ejemplo}\label{ejem:VariablesAleatoriasEliminanInformacion}
            Ya hemos visto que en el espacio muestral correspondiente al lanzamiento de dos dados, la variable aleatoria $X(a,b)=|a-b|$ tiene esta tabla de valores y probabilidades:
            \begin{center}
            \begin{tabular}[t]{|c|c|c|c|c|c|c|}
                \hline
                \rule{0cm}{0.5cm}{\em Valor de $X$ (diferencia):}&0&1&2&3&4&5\\
                \hline
                \rule{0cm}{0.7cm}{\em Probabilidad de ese valor:}&$\dfrac{6}{36}$&$\dfrac{10}{36}$&$\dfrac{8}{36}$&$\dfrac{6}{36}$&$\dfrac{4}{36}$&$\dfrac{2}{36}$\\
                &&&&&&\\
                \hline
            \end{tabular}
            \end{center}
            Y, por su parte, la variable aleatoria suma $Y(a,b)=a+b$ tiene esta tabla:
            \begin{center}
            \begin{tabular}[t]{|c|c|c|c|c|c|c|c|c|c|c|c|}
                \hline
                \rule{0cm}{0.5cm}{\em Valor de la suma:}&2&3&4&5&6&7&8&9&10&11&12\\
                \hline
                \rule{0cm}{0.7cm}{\em Probabilidad de ese valor:}&$\dfrac{1}{36}$&$\dfrac{2}{36}$&$\dfrac{3}{36}$&$\dfrac{4}{36}$&$\dfrac{5}{36}$&$\dfrac{6}{36}$&$\dfrac{5}{36}$&$\dfrac{4}{36}$&$\dfrac{3}{36}$&$\dfrac{2}{36}$&$\dfrac{1}{36}$\\
                &&&&&&&&&&&\\
            \hline
            \end{tabular}
            \end{center}
            En el Ejemplo \ref{ejem:probabilidadCondicionadaLanzamientoDosDados} (página \pageref{ejem:probabilidadCondicionadaLanzamientoDosDados}) nos hicimos la pregunta {\em `` ¿Cuál es la probabilidad de que la diferencia entre los valores de ambos dados (mayor-menor) sea menor que 4, sabiendo que la suma de los dados es 7?''} Está claro, con la notación que usamos ahora, que estamos preguntando cuál es la probabilidad del suceso
            \[(X<4)\cap (Y=7).\]
            ¿Podemos calcular este número usando sólo las tablas de probabilidad de $X$ e $Y$, sin utilizar más información sobre el suceso muestral subyacente?\qed
            \end{Ejemplo}
        \item En el caso de las variables aleatorias discretas, hemos visto que conocer la variable es esencialmente lo mismo que conocer la tabla de probabilidades asignadas a cada uno de los posibles valores de la variable (esta tabla se conoce como {\sf función de probabilidad o función de masa} de la variable aleatoria). En el caso de las variables aleatorias continuas, no podemos hacer la asignación de probabilidades de esta misma forma. Recordando que la probabilidad de las variables continuas es análoga al área, necesitamos un recurso técnicamente más complicado, y eso es lo que vamos a hacer más adelante, al presentar los conceptos de función de densidad y función de distribución.
\end{itemize}


%\section*{Tareas asignadas para esta sesión.}
%
%\begin{enumerate}
%    \item Supongamos ahora que lanzamos un dado cuatro veces y nos preguntamos por la probabilidad de sacar exactamente dos seises. Esto es similar al problema del Caballero De Mere, que vimos en la sesión 7 y recuerda mucho al Ejemplo \ref{ejem:probabilidadLanzamientoMonedas}, hasta el punto de que es razonable preguntarse si la respuesta es la misma. Usa las ideas de esta sesión para obtener la respuesta. Escribe esa respuesta y el razonamiento que te conduce a ella en Moodle.
%    \item A lo largo del fin de semana aparecerá una nueva hoja de Ejercicios para trabajar con ella durante la semana que viene. Estad atentos a Moodle.
%%    \item Está claro que si en una habitación hay 367 personas, entonces hay al menos dos de ellas que cumplen años el mismo día, ¿verdad? ¿Cuál es el número mínimo de personas que debe haber en esa habitación para que la probabilidad sea superior al 50\%? Escribe tu respuesta y el razonamiento que te conduce a ella en Moodle.
%\end{enumerate}
%
%
%%\section*{Recomendaciones.}
%%
%%\begin{enumerate}
%%   \item El \link{http://www.ine.es/}{INE (Instituto Nacional de Estadística)} es el organismo oficial encargado, entre otras cosas del censo electoral, la elaboración del IPC (índice de precios de consumo), la EPA (encuesta de población activa), el PIB (producto interior bruto) etc. El instituto ofrece una enorme colección de datos estadísticos accesibles para cualquiera a través de la red (sistema INEbase). Además, tiene alojado en su página web un \link{http://www.ine.es/explica/explica.htm}{portal de divulgación estadística} en el que se pueden ver vídeos sobre estos y otros temas, que tal vez os interesen.
%%\end{enumerate}
%
%\section*{\fbox{\colorbox{Gris025}{{Sesión 12. Probabilidad.}}}}
%
%\subsection*{\fbox{\colorbox{Gris025}{{Media y varianza de variables aleatorias. Variable aleatoria binomial.}}}}
%\subsection*{Fecha: Martes, 25/10/2011, 14h.}
%
%\noindent{\bf Atención:
%\begin{enumerate}
%\item Este fichero pdf lleva adjuntos los ficheros de datos necesarios.
%\end{enumerate}
%}
%
%%\subsection*{\fbox{1. Ejemplos preliminares }}
%\setcounter{tocdepth}{1}
%%\tableofcontents
%\section*{Lectura recomendada}
%
%Al menos uno de los siguientes:
%    \begin{itemize}
%    \item Capítulo 4 de "La estadística en Comic".
%    \item Capítulo 5 de Head First Statistics.
%    \item Tema 5 de Bioestadística: Métodos y Aplicaciones, Univ. de Málaga (veremos las variables aleatorias continuas en próximas sesiones).
%    \item Apuntes de la sexta y comienzo de la séptima sesiones del Curso 2010-2011.
%
%    \end{itemize}
%
%\section{Variables aleatorias}
%
%
%\subsection{¿Qué son las variables aleatorias?}
%
%\begin{itemize}
%
%    \item Hemos visto que cada suceso $A$ del espacio muestral $\Omega$ tiene asociado un valor $P(A)$ de la función probabilidad. Y sabemos que los valores de la función probabilidad son valores positivos, comprendidos entre $0$ y $1$. La idea de variable aleatoria es similar, pero generaliza este concepto, porque a menudo querremos asociar otros valores numéricos con los resultados de un experimento aleatorio.
%        \begin{Ejemplo}\label{ejem:VariableAleatoria:SumaDosDados}
%        Quizá uno de los ejemplos más sencillos sea lo que ocurre cuando lanzamos dos dados, y nos fijamos en la suma de los valores obtenidos. Esa suma es siempre un número del 2 al 12, y es perfectamente legítimo hacer preguntas como ¿cuál es la probabilidad de que la suma valga $7$? Para responder a esa pregunta, iríamos al espacio muestral (formado por 36 resultados posibles), veríamos el valor de la suma en cada uno de ellos, para localizar aquellos en que la suma vale $7$. Así obtendríamos un suceso aleatorio $A=\{(1,6),(2,5),(3,4),(4,3),(5,2),(6,1)\}$, cuya probabilidad es $6/32$. De hecho podemos repetir lo mismo para cada uno de los posibles valores de la suma. Se obtiene esta tabla:\\[3mm]
%        \begin{tabular}[t]{|c|c|c|c|c|c|c|c|c|c|c|c|}
%        \hline
%        \rule{0cm}{0.5cm}{\em Valor de la suma:}&2&3&4&5&6&7&8&9&10&11&12\\
%        \hline
%        \rule{0cm}{0.7cm}{\em Probabilidad de ese valor:}&$\dfrac{1}{36}$&$\dfrac{2}{36}$&$\dfrac{3}{36}$&$\dfrac{4}{36}$&$\dfrac{5}{36}$&$\dfrac{6}{36}$&$\dfrac{5}{36}$&$\dfrac{4}{36}$&$\dfrac{3}{36}$&$\dfrac{2}{36}$&$\dfrac{1}{36}$\\
%        &&&&&&&&&&&\\
%        \hline
%        \end{tabular}\\[3mm]
%        Y de hecho, esta tabla es lo que en este caso caracteriza a la variable aleatoria suma.\qed
%        \end{Ejemplo}
%        Vamos ahora a ver otro ejemplo inspirado en los problemas de probabilidad geométrica.
%        \begin{Ejemplo}
%        Consideremos un círculo $C$ centrado en el origen y de radio 1. El espacio muestral $\Omega$ está formado por todos los subconjuntos\footnote{No excesivamente ``raros'', en el sentido que ya hemos discutido.} de puntos de $C$. Y la Función de Probabilidad se define así:
%        \[P(A)=\mbox{área de }A.\]
%        Consideremos ahora la variable que a cada punto del círculo le asocia su coordenada $x$. En este caso la coordenada $x$ toma cualquier valor real entre $-1$ y $1$. Y si preguntamos {``¿cuál es la probabilidad de que tome por ejemplo el valor $1/2$?''}, la respuesta es $0$. Porque los puntos del círculo donde toma ese valor forman un segmento (una cuerda del círculo), y el segmento tiene área $0$. Las cosas cambian si preguntamos {``¿cuál es la probabilidad de que la coordenada $x$ esté entre $0$ y $1/2$?''} En este caso, como muestra la figura
%        \begin{center}
%        \includegraphics[height=7cm]{2011_10_25_Figura01-VariableAleatoriaContinua.png}
%        \end{center}
%        el conjunto de puntos del círculo cuyas coordenadas $x$ están entre $0$ y $1/2$ tiene un área bien definida y no nula. ¿Cuánto vale ese área? Aproximadamente $0.48$, y esa es la probabilidad que buscábamos. El cálculo del área se puede hacer de distintas maneras, pero el lector debe darse cuenta de que en ejemplos como este se necesita a veces recurrir al cálculo de integrales.\\
%        Naturalmente, se pueden hacer preguntas más complicadas. Por ejemplo, dado un punto $(x,y)$ del círculo $C$ podemos calcular el valor de $f(x,y)=\frac{x^2}+ 4y^2$. Y entonces nos preguntamos ¿cuál es la probabilidad de que, tomando un punto al azar en $C$, el valor de $f$ esté entre 0 y 1? La respuesta es, de nuevo, un área, pero más complicada: es el área que se muestra en esta figura:
%        \begin{center}
%        \includegraphics[height=7cm]{2011_10_25_Figura02-VariableAleatoriaContinua.png}
%        \end{center}
%        Lo que tienen en común ambos casos es que hay una función (o fórmula), que es $x$ en el primero y $f(x,y)$ en el segundo, y que nos preguntamos por la probabilidad de que los valores de esa fórmula caigan dentro de un cierto intervalo.
%        \qed
%        \end{Ejemplo}
%        Los dos ejemplos que hemos visto contienen los ingredientes básicos de la noción de variable aleatoria. En el primer caso teníamos un conjunto finito de valores posibles, y a cada uno le asignábamos una probabilidad. En el segundo caso teníamos un rango continuo de valores posibles, y podíamos asignar probabilidades a intervalos. Lo que vamos a ver a continuación no se puede considerar de ninguna manera una definición rigurosa de variable aleatoria\footnote{La situación es similar a lo que ocurría al definir los sucesos aleatorios. Un suceso aleatorio $A$ es un subconjunto que tiene bien definida la probabilidad $P(A)$. Pero ya hemos dicho hay conjuntos tan {\em raros} que no es fácil asignarles un valor de la probabilidad (igual que a veces cuesta asignar un área). De la misma forma hay funciones tan raras que no se pueden considerar variables aleatorias. Se necesitan definiciones más rigurosas, pero que aquí sólo nos complicarían.}, pero servirá a nuestros propósitos.\\[3mm]
%        \fbox{\begin{minipage}{14cm}
%        \begin{center}
%        \vspace{2mm}
%        {\bf Variables aleatorias:}
%        \end{center}
%        Una variable aleatoria $X$ es una función (o fórmula) que le asigna, a cada elemento $p$ del espacio muestral $\Omega$, un número real $X(p)$. Distinguimos dos tipos de valores aleatorias:
%        \begin{enumerate}
%            \item La {\sf variable aleatoria $X$ es discreta} si sólo toma una cantidad finita (o una sucesión) de valores numéricos $x_1,x_2,x_3,\ldots$, de manera que para cada uno de esos valores tenemos bien definida la probabilidad $P(X=x_i)$ de que $X$ tome el valor $x_i$.
%            \item La {\sf variable aleatoria $X$ es continua} si sus valores forman un cierto rango continuo dentro de los números reales, de manera que si nos dan un intervalo $I=(a,b)$ (aquí puede ser $a=-\infty$ o $b=+\infty$), tenemos bien definida la probabilidad $P(X\in I)$ de que el valor de $X$ esté dentro de ese intervalo $I$.
%        \end{enumerate}
%        \end{minipage}}\\[3mm]
%        Veamos un ejemplo, muy parecido al Ejemplo \ref{ejem:VariableAleatoria:SumaDosDados}.
%        \begin{Ejemplo}\label{ejem:VariableAleatoria:RestaDosDados}
%            De nuevo lanzamos dos dados, pero ahora nos fijamos en la diferencia de los valores obtenidos (el menor menos el mayor, y cero si son iguales). Si llamamos $(a,b)$ al resultado de lanzar los dados, donde $a$ y $b$ son números del 1 al 6, entonces estamos definiendo una variable aleatoria mediante la expresión
%            \[X(a,b)=|a-b|.\]
%            Esta claro que la variable $X$ toma solamente los valores $0,1,2,3,4,5$. ¿Cuál es la probabilidad de que al calcular $X$ obtengamos $3$? El siguiente diagrama ayudará a entender la respuesta. Para cada punto del espacio muestral,  se muestra el valor de $X$:
%            \[
%            \begin{array}{cccccc}
%            X(1,1)=0&X(1,2)=1&X(1,3)=2&X(1,4)=3&X(1,5)=4&X(1,6)=5\\
%            X(2,1)=1&X(2,2)=0&X(2,3)=1&X(2,4)=2&X(2,5)=3&X(2,6)=4\\
%            X(3,1)=2&X(3,2)=1&X(3,3)=0&X(3,4)=1&X(3,5)=2&X(3,6)=3\\
%            X(4,1)=3&X(4,2)=2&X(4,3)=1&X(4,4)=0&X(4,5)=1&X(4,6)=2\\
%            X(5,1)=4&X(5,2)=3&X(5,3)=2&X(5,4)=1&X(5,5)=0&X(5,6)=1\\
%            X(6,1)=5&X(6,2)=4&X(6,3)=3&X(6,4)=2&X(6,5)=1&X(6,6)=0
%            \end{array}
%            \]
%            Y se observa que $P(X=3)=6/36=1/6$. De hecho, podemos repetir lo mismo para cada uno de los posibles valores de la variable aleatoria $X$. Se obtiene esta tabla:
%            \begin{center}
%            \begin{tabular}[t]{|c|c|c|c|c|c|c|}
%            \hline
%            \rule{0cm}{0.5cm}{\em Valor de $X$ (diferencia):}&0&1&2&3&4&5\\
%            \hline
%            \rule{0cm}{0.7cm}{\em Probabilidad de ese valor:}&$\dfrac{6}{36}$&$\dfrac{10}{36}$&$\dfrac{8}{36}$&$\dfrac{6}{36}$&$\dfrac{4}{36}$&$\dfrac{2}{36}$\\
%            &&&&&&\\
%            \hline
%            \end{tabular}
%            \end{center}
%        Y de hecho, esta tabla es lo que en este caso caracteriza a la variable aleatoria diferencia $X$.\qed
%        \end{Ejemplo}
%
%\end{itemize}

%\subsection{¿Para qué sirven las variables aleatorias?.}
%
%\begin{itemize}
%
%        \item Al principio la diferencia entre suceso aleatorio y variable aleatoria puede resultar un poco confusa. Vamos a recordar que un suceso es un {\em subconjunto}, y que una variable aleatoria es una {\em función}. Por ejemplo, un suceso al lanzar dos dados puede ser ``los dos resultados son pares'', y este enunciado no hay un valor numérico fácil de identificar. De la misma forma, la variable aleatoria
%
%
%
%\end{itemize}
%
%
%
\section{Media y varianza de variables aleatorias}


\subsection{Media de una variable aleatoria discreta}

\begin{itemize}

    \item Hemos dicho que las variables aleatorias son modelos teóricos de los resultados de un experimento aleatorio. Y de la misma forma que hemos aprendido a describir un conjunto de datos mediante su media aritmética y su desviación típica, podemos caracterizar a una variable aleatoria mediante valores similares. Empecemos por la media, en el caso de una variable aleatoria discreta. El caso de las variables aleatorias continuas requiere cálculo integral, y lo veremos un poco más adelante.

    \item El punto de partida es la fórmula que ya conocemos para calcular la media aritmética de una variable discreta a partir de su tabla de frecuencias, que escribimos de una forma ligeramente diferente, usando las frecuencias relativas:
        \[\fbox{\colorbox{Gris025}{$
        \bar x=\dfrac{\displaystyle\sum_{i=1}^k x_i\cdot f_i}{\displaystyle\sum_{i=1}^k f_i}
        =\dfrac{\displaystyle\sum_{i=1}^k x_i\cdot f_i}{n}
        =\displaystyle\sum_{i=1}^k x_i\cdot \dfrac{f_i}{n}
        $}}
        \]
        y aquí $\dfrac{f_i}{n}$ es la frecuencia relativa número $i$.\\[3mm]
        Para entender el siguiente paso, es importante entender que la probabilidad, como concepto teórico, es una idealización de lo que sucede en la realidad que estamos tratando de representar. Para centrar las ideas, volvamos al conocido caso del lanzamiento de dos dados, que ya vimos en el Ejemplo \ref{ejem:VariableAleatoria:SumaDosDados} (página \pageref{ejem:VariableAleatoria:SumaDosDados}).
        \begin{Ejemplo}\label{ejem:Cap04-VariableAleatoria:SumaDosDados}
        La tabla de probabilidades para los posibles valores de la suma es, como ya vimos,\\[3mm]
        \begin{tabular}[t]{|c|c|c|c|c|c|c|c|c|c|c|c|}
        \hline
        \rule{0cm}{0.5cm}{\em Valor de la suma:}&2&3&4&5&6&7&8&9&10&11&12\\
        \hline
        \rule{0cm}{0.7cm}{\em Probabilidad de ese valor:}&$\dfrac{1}{36}$&$\dfrac{2}{36}$&$\dfrac{3}{36}$&$\dfrac{4}{36}$&$\dfrac{5}{36}$&
        $\dfrac{6}{36}$&$\dfrac{5}{36}$&$\dfrac{4}{36}$&$\dfrac{3}{36}$&$\dfrac{2}{36}$&$\dfrac{1}{36}$\\
        &&&&&&&&&&&\\
        \hline
        \end{tabular}\\[3mm]
        Pero esto es un modelo teórico que describe a la variable aleatoria suma. Si hacemos un experimento en el mundo real, como el lanzamiento de 3000 pares de dados que se simula en esta \textattachfile{2011-10-25-Lanzamientos2Dados-FrecuenciasSumaVsProbabilidades.ods}{\textcolor{blue}{hoja de cálculo}}, lo que obtendremos es una tabla de frecuencias que son {\em aproximadamente} iguales a las probabilidades. ¿Y si en lugar de lanzar 3000 veces lo hiciéramos un millón de veces?
        \qed
        \end{Ejemplo}
        La idea que queremos subrayar es que los valores de las probabilidades son una especie de límite teórico de las frecuencias relativas, una idealización de lo que ocurre si lanzamos los dados muchísimas veces, tendiendo hacia infinito. Y por lo tanto, esto parece indicar que las fórmulas teóricas correctas se obtienen cambiando las frecuencias relativas por las correspondientes probabilidades. Eso conduce a esta definición para la media de una variable aleatoria:\\[3mm]
        \fbox{\colorbox{Gris025}{\begin{minipage}{14cm}
        \begin{center}
        \vspace{2mm}
        {\bf Media $\mu$ de una variable aleatoria discreta (valor esperado)}
        \end{center}
        Si $X$ es una variable aleatoria discreta, que toma los valores $x_1,x_2,\ldots,x_k$, con las probabilidades $p_1,p_2,\ldots,p_k$ (donde $p_i=P(X=x_i)$), entonces la {\sf media}, o {\sf valor esperado}, o {\sf esperanza matemática}  de $X$ es:
        \[
        \mu=\sum_{i=1}^k
        \left(x_i\cdot P(X=x_i)\right)=x_1p_1+x_2p_2+\cdots+x_kp_k.
        \]
        \end{minipage}}}\\[3mm]
        La media de una variable aleatoria discreta se suele representar con la letra griega $\mu$ para distinguirla de la media aritmética de una muestra $\bar x$. La media, como hemos indicado, también se suele llamar valor esperado o esperanza matemática de la variable $X$.\\
        (En el caso de que la variable aleatoria tome infinitos valores --ver el ejemplo \ref{Sesion08:ejem:LanzamientoMonedaHastPrimeraCara} (página \pageref{Sesion08:ejem:LanzamientoMonedaHastPrimeraCara}), en el que lanzábamos monedas hasta obtener la primera cara--, esta suma puede ser una suma infinita.)



        \item Vamos a aplicar esta definición al ejemplo de la suma de dos dados
        \begin{Ejemplo} {\bf Continuación del Ejemplo \ref{ejem:Cap04-VariableAleatoria:SumaDosDados}}\\
        A partir de la tabla tenemos:
        \[\mu=\sum x_i P(X=x_i)=\]
        \[\mbox{\small $
        2\cdot\dfrac{1}{36}+3\cdot\dfrac{2}{36}+4\cdot\dfrac{3}{36}+5\cdot\dfrac{4}{36}+6\cdot\dfrac{5}{36}+
        7\cdot\dfrac{6}{36}+8\cdot\dfrac{5}{36}+9\cdot\dfrac{4}{36}+10\cdot\dfrac{3}{36}+11\cdot\dfrac{2}{36}
        +12\cdot\dfrac{1}{36}$}=7.\]
        Así que, en este ejemplo, la media o valor esperado es $\mu=7$.\qed
        \end{Ejemplo}

        \item {\sf Valor esperado y juegos ``justos''.}
        Cuando se usa la probabilidad para analizar un juego de azar en el que cada jugador invierte una cierta cantidad de recursos (por ejemplo, dinero), es conveniente considerar la variable aleatoria
        \[X=\mbox{beneficio del jugador}=\mbox{(ganancia neta)}-\mbox{(recursos invertidos)}.\]
        Para que el juego sea justo la media de la variable beneficio (es decir, el beneficio esperado) debería ser $0$.
        \begin{Ejemplo}
        Cada uno de nosotros pone un euro, y lanzamos un dado. Si sale un uno ganas tú y te quedas los dos euros. Si sale cualquier otra cosa gano yo y me quedo los dos euros. ¿Es un juego justo? Parece claro que no. ¿Cuál es el valor esperado del beneficio para cada uno de nosotros?\\
        Una pregunta más interesante. Si tú sigues poniendo un euro, ¿cuántos euros tengo que poner yo para que el juego sea justo?\qed
        \end{Ejemplo}

\end{itemize}

\subsection{Varianza y desviación típica de una variable aleatoria discreta}

\begin{itemize}

        \item Ahora que hemos visto la definición de media, y como obtenerla a partir de la noción de frecuencias relativas, parece bastante evidente lo que tenemos que hacer para definir la varianza de una variable aleatoria discreta. Recordemos la fórmula para la varianza poblacional a partir de una tabla de frecuencias, y vamos a escribirla en términos de frecuencias relativas:
            \[
            \mbox{Var($x$)}=\dfrac{\displaystyle\sum_{i=1}^k{ f_i\cdot}(x_i-\bar x)^2}{{ \displaystyle\sum_{i=1}^k f_i}}=
            \dfrac{\displaystyle\sum_{i=1}^k{ f_i\cdot}(x_i-\bar x)^2}{n}=
            \displaystyle\sum_{i=1}^k{(x_i-\bar x)^2\cdot}\dfrac{f_i}{n}.
           \]
           Por lo tanto, definimos:\\[3mm]
           \fbox{\colorbox{Gris025}{\begin{minipage}{14cm}
           \begin{center}
           \vspace{2mm}
           {\bf Varianza $\sigma^2$ de una variables aleatoria discreta}
           \end{center}
           La {\sf varianza}  de una variable aleatoria discreta $X$, que toma los valores $x_1,x_2,x_3,\ldots,x_k$, con las probabilidades $p_1,p_2,\ldots,p_k$ (donde $p_i=P(X=x_i)$), es:
           \[
           \sigma^2=\sum_{i=1}^k
           \left((x_i-\mu)^2P(X=x_i)\right).
           \]
           \end{minipage}}}\\[3mm]

           \item Y por supuesto, esta definición va acompañada por la de la desviación típica:\\[3mm]
           \fbox{\colorbox{Gris025}{\begin{minipage}{14cm}
           \begin{center}
           \vspace{2mm}
           {\bf Desviación típica $\sigma$ de una variables aleatoria discreta}
           \end{center}
           La {\sf desviación típica}  de una variable aleatoria discreta $X$ es simplemente la raíz cuadrada $\sigma$ de su varianza.
           \[
           \sigma=\displaystyle\sqrt{\sum_{i=1}^k\left((x_i-\mu)^2P(X=x_i)\right)}.
           \]
           \end{minipage}}}\\[3mm]



\end{itemize}


%\section*{Tareas asignadas para esta sesión.}
%
%\begin{enumerate}
%   \item No hay tareas previstas para hoy. En la sesión del viernes sí habrá nuevas tareas.
%\end{enumerate}
%
%
%\section*{\fbox{\colorbox{Gris025}{{Sesión 13. Probabilidad.}}}}
%
%\subsection*{\fbox{\colorbox{Gris025}{{Variable aleatoria binomial.}}}}
%\subsection*{Fecha: Viernes, 28/10/2011, 14h.}
%
%\noindent{\bf Atención:
%\begin{enumerate}
%\item Este fichero pdf lleva adjuntos los ficheros de datos necesarios.
%\end{enumerate}
%}
%
%%\subsection*{\fbox{1. Ejemplos preliminares }}
%\setcounter{tocdepth}{1}
%%\tableofcontents
%\section*{Lectura recomendada}
%
%Al menos uno de los siguientes:
%    \begin{itemize}
%    \item Capítulo 5 de "La estadística en Comic".
%    \item Capítulo 7 de Head First Statistics.
%    \item Tema 5 de Bioestadística: Métodos y Aplicaciones, Univ. de Málaga (veremos las variables aleatorias continuas en próximas sesiones).
%    \item Apuntes de la sexta y comienzo de la séptima sesiones del Curso 2010-2011.
%
%    \end{itemize}

\section{Operaciones con variables aleatorias. Esperanza y varianza.}
\label{sec:OperacionesVariablesAleatorias}


\begin{itemize}

\item Aparte de los símbolos $\mu$ y $\sigma^2$ que ya vimos para la media y varianza de una variable aleatoria, en esta sección vamos a usar los símbolos
    \[E(X)=\mu, \operatorname{Var}(X)=\sigma^2,\]
    para la media y la varianza respectivamente. Estos símbolos son a veces más cómodos cuando se trabaja a la vez con varias variables aleatorias.

\item Una variable aleatoria $X$ es, al fin y al cabo, una fórmula que produce un resultado numérico. Y puesto que es un número, podemos hacer operaciones con ella. Por ejemplo, tiene sentido hablar de $2X$, $X+1$, $X^2$, etcétera.
    \begin{Ejemplo}
    En el caso del lanzamiento de dos dados, teníamos la variable aleatoria suma, definida mediante $X(a,b)=a+b$. En este caso:
    \[
    \begin{cases}
    2X(a,b)=2a+2b\\[2mm]
    X(a,b)+1=a+b+1\\[2mm]
    X^2(a,b)=(a+b)^2
    \end{cases}
    \]
    de manera que, por ejemplo, $X^2(3,4)=(3+4)^2=49$.\qed
    \end{Ejemplo}

    \item De la misma manera, si tenemos dos variables aleatorias $X_1$ y $X_2$ (dos fórmulas), podemos sumarlas para obtener una nueva variable $X=X_1+X_2$, o multiplicarlas, etcétera.
    \begin{Ejemplo}
    De nuevo en el lanzamiento de dos dados, si consideramos la variable aleatoria suma $X_1(a,b)=a+b$, y la variable aleatoria producto $X_2(a,b)=a\cdot b$, sería:
    \[X_1(a,b)+X_2(a,b)=(a+b)+a\cdot b.\]
    \qed
    \end{Ejemplo}

    \item Si hemos invertido algo de tiempo y esfuerzo en calcular las medias y las varianzas $X_1$ y $X_2$, nos gustaría poder aprovechar ese esfuerzo para obtener sin complicaciones las medias y varianzas de combinaciones como $X_1+X_2$, o $3X_1+5$, etcétera. Afortunadamente, eso es posible.\\[3mm]
           \fbox{\colorbox{Gris025}{\begin{minipage}{14cm}
           \begin{center}
           \vspace{2mm}
           {\bf Media y varianza de una combinación de variables aleatorias}
           \end{center}
           \begin{itemize}
           \item Si $X$ es una variable aleatoria, y $a, b$ son números cualesquiera, entonces
           \[E(a\cdot X+b)=a\cdot E(X)+b,\quad \operatorname{Var}(a\cdot X+b)=a^2\cdot \operatorname{Var}(X).\]
           \item Y si $X_1, X_2$ son dos variables aleatorias, se tiene:
           \[E(X_1+X_2)=E(X_1)+E(X_2).\]
           Si además $X_1$ y $X_2$ son {\em independientes}, entonces
           \[\operatorname{Var}(X_1+X_2)=\operatorname{Var}(X_1)+\operatorname{Var}(X_2).\]
           No entramos en este momento en la definición técnica de la independencia, pero es fácil intuir que se basa en la independencia de los sucesos subyacentes a los valores de las variables.\\
           \end{itemize}
           \end{minipage}}}\\[3mm]
        Con la notación de $\mu$ y $\sigma$ se obtienen estas fórmulas, algo más confusas:
        \[\mu_{aX+b}=a\cdot\mu_X,\quad \sigma^2_{aX+b}=a^2\sigma^2_X\]
        y
         \[\mu_{X_1+X_2}=\mu_{X_1}+\mu_{X_2},\quad \sigma^2_{X_1+X_2}=\sigma^2_{X_1}+\sigma^2_{X_2},\]
         donde la última fórmula, insistimos {\em es válida para variables independientes}.

\end{itemize}



    \chapter{Teorema central del límite}
    % !Mode:: "Tex:UTF-8"

\section{Experimentos de Bernouilli y distribución binomial.}\label{sec:ExperimentosBernouilliDistribucionBinomial}


\subsection{Experimento de Bernouilli}

\begin{itemize}

    \item En muchas situaciones, el resultado de un experimento sólo admite dos resultados posibles. Son las típicas situaciones de {\em cara o cruz, ``sí o no'', acierto o fallo, ganar o perder}.
    Por ejemplo:
    \begin{enumerate}
            \item Cuando lanzamos una moneda, y apostamos a que va a salir cara, entonces sólo podemos ganar la apuesta o perderla.
            \item Y si lanzamos un dado, y apostamos a que va a salir un seis, entonces sólo podemos ganar la apuesta o perderla.
    \end{enumerate}
        En ambas ocasiones sólo hay dos resultados posibles. La diferencia entre ellas es, naturalmente,  que la probabilidad de éxito o fracaso no es la misma. Al lanzar la moneda, la probabilidad de ganar la apuesta es $1/2$, mientras que en el caso del dado es $1/6$.

    \item Vamos a introducir la terminología que usaremos para describir este tipo de situaciones:\\[3mm]
        \fbox{\colorbox{Gris025}{\begin{minipage}{14cm}
        \begin{center}
        \vspace{2mm}
        {\bf Experimento de Bernouilli}
        \end{center}
        Un {\sf experimento de Bernouilli} es un experimento aleatorio que sólo tiene dos resultados posibles, que llamamos --arbitrariamente-- {\sf éxito} y {\sf fracaso}.\\[2mm]
        La {\sf probabilidad de éxito} se representa siempre con la letra $p$, mientras que la probabilidad de fracaso se representa con la letra $q$. Naturalmente, se tiene que cumplir que
        \[q=1-p.\vspace{2mm}\]
        \end{minipage}}}\\[3mm]
        Por ejemplo, en el caso de la moneda es $p=q=\frac{1}{2}$ (a menos, naturalmente, que la moneda esté trucada). Y en el caso del dado es $p=\frac{1}{6}$, mientras $q=\frac{5}{6}$.

    \item Una variable aleatoria $X$ es de tipo {\em Bernouilli$(p)$} si sólo puede tomar los valores $1$ y $0$ con probabilidades $p$ y $1-p$. En resumen, la tabla de esta variable es muy sencilla:
        \begin{center}{\bf
        \begin{tabular}[t]{|c|c|c|}
            \hline
            \rule{0cm}{0.5cm}{\em Valor de $X$:}&1&0\\
            \hline
            \rule{0cm}{0.7cm}{\em Probabilidad de ese valor:}&{\em p}&{\em q}\\
            \hline
        \end{tabular}}
        \end{center}
        Así que es muy fácil calcular la media y la varianza de una variable de tipo {\em Bernouilli$(p)$}:
        \[
        \mu=E(X)=1\cdot p+0\cdot q=p.
        \]
        \[
        \sigma^2=\operatorname{Var}(X)=(1-\mu)^2\cdot p+(0-\mu)^2\cdot q=\]
        \[=
        (1-p)^2\cdot p+(0-p)^2\cdot q=q^2p+p^2q=pq\cdot(p+q)=pq.
        \]

        Los experimentos de Bernouilli son muy importantes, porque los usamos como bloques básicos para construir otras situaciones más complejas. En particular, es la pieza básica para construir la distribución binomial.


\end{itemize}

\subsection{Variable aleatoria binomial}

\begin{itemize}

        \item Supongamos que tenemos un experimento de Bernouilli, con sus dos resultados posibles, éxito y fracaso, con probabilidades $p$ y $q$ respectivamente. Pero ahora {\em vamos a repetirlo una cierta cantidad de veces}. Y vamos a llamar $n$ al número de veces que lo repetimos. ¿Qué probabilidad hay de obtener exactamente $k$ éxitos en esos $n$ experimentos?

            Para fijar ideas, el experimento de Bernouilli puede ser lanzar un dado, y vamos a suponer que lo lanzamos $n=4$ veces. ¿Cuál es la probabilidad de sacar exactamente dos seises? Habrás reconocido que esta es la pregunta que habíamos dejado pendiente del útimo capítulo. Para obtener la respuesta, vamos a usar la fórmula de Laplace.
            \begin{Ejemplo}\label{ejem:BinomialDosSeisesCuatroTiradas}
            El conjunto de respuestas posibles (espacio muestral) tiene $6^4$ respuestas posibles (y equiprobables). ¿En cuántas de ellas se obtienen exactamente dos seises? (Dicho de otro modo ¿cuántas ``favorables'' hay?) Podemos representar los resultados de esas cuatro tiradas usando un casillero con cuatro casillas.
            \begin{center}
            \begin{tabular}{|c|c|c|c|}
            \hline
             \rule{0cm}{0.5cm}\rule{0.3cm}{0cm}&\rule{0.3cm}{0cm}&\rule{0.3cm}{0cm} &\rule{0.3cm}{0cm}\\
             \hline
             \end{tabular}
             \end{center}
             Los dos seises se pueden haber obtenido en la primera y segunda casillas, o en la primera y la tercera, etcétera. Marcamos con un $6$ las casillas en las que se han obtenido los seises:
            \begin{center}
            \begin{tabular}{|c|c|c|c|}
            \hline
             \rule{0cm}{0.5cm}\mbox{\large\bf 6}&\mbox{\large\bf 6}& &\\
            \hline
             \rule{0cm}{0.5cm}\mbox{\large\bf 6}&& \mbox{\large\bf 6}&\\
            \hline
             \rule{0cm}{0.5cm}\mbox{\large\bf 6}&& &\mbox{\large\bf 6}\\
            \hline
             \rule{0cm}{0.5cm}&\mbox{\large\bf 6}& \mbox{\large\bf 6}&\\
            \hline
             \rule{0cm}{0.5cm}&\mbox{\large\bf 6}& &\mbox{\large\bf 6}\\
            \hline
             \rule{0cm}{0.5cm}&& \mbox{\large\bf 6}&\mbox{\large\bf 6}\\
             \hline
             \end{tabular}
             \end{center}
             Hay seis posibilidades. Observa que estamos eligiendo 2 de entre cuatro casillas, y sabemos que:
             \[\binom{4}{2}=\dfrac{4\cdot 3}{2}=6.\]
             Una vez que hemos decidido donde colocar los seises, todavía tenemos que pensar en los resultados de los restantes lanzamientos. Si, por ejemplo, hemos obtenido los dos seises en el primer y segundo lanzamiento, tendremos:
            \[
            \left.\begin{array}{|c|c|c|c|}
            \hline
             \rule{0cm}{0.5cm}\mbox{\large\bf 6}&\mbox{\large\bf 6}&\mbox{\large\bf 1}&\mbox{\large\bf 1}\\
            \hline
            \rule{0cm}{0.5cm}\mbox{\large\bf 6}&\mbox{\large\bf 6}&\mbox{\large\bf 1}&\mbox{\large\bf 2}\\
            \hline
            \rule{0cm}{0.5cm}\vdots&\vdots&\vdots&\vdots\\
            \hline
            \rule{0cm}{0.5cm}\mbox{\large\bf 6}&\mbox{\large\bf 6}&\mbox{\large\bf 1}&\mbox{\large\bf 5}\\
            \hline
            \rule{0cm}{0.5cm}\mbox{\large\bf 6}&\mbox{\large\bf 6}&\mbox{\large\bf 2}&\mbox{\large\bf 1}\\
            \hline
            \rule{0cm}{0.5cm}\mbox{\large\bf 6}&\mbox{\large\bf 6}&\mbox{\large\bf 2}&\mbox{\large\bf 2}\\
            \hline
            \rule{0cm}{0.5cm}\vdots&\vdots&\vdots&\vdots\\
            \rule{0cm}{0.5cm}\mbox{\large\bf 6}&\mbox{\large\bf 6}&\mbox{\large\bf 5}&\mbox{\large\bf 5}\\
            \hline
             \end{array}\quad\right\}\,\,5^2=25\mbox{ posibilidades}
             \]
             Y ese número de posibilidades, 25, es el mismo si los dos seises se colocan en cualquier otro par de casillas. Por lo tanto, tenemos
             \[\binom{4}{2}5^2\]
             casos favorables, y la respuesta es:
             \[\dfrac{\displaystyle\binom{4}{2}5^2}{6^4}\approx 0.116\]
             ¿Y si hubiéramos lanzado los dados nueve veces, y de nuevo nos preguntáramos por la probabilidad de obtener dos seises? Sería:
             \[\dfrac{\displaystyle\binom{9}{2}5^{9-2}}{6^9},\]
             donde el $9-2=7$ corresponde a las cinco casillas que tenemos que rellenar con números distintos de $6$. Es interesante recordar que lo que hacemos es repetir $n=9$ veces un experimento de Bernouilli que tiene $p=\dfrac{1}{6}$ como probabilidad de éxito y $q=\dfrac{5}{6}$ como probabilidad de fracaso. Y lo que nos preguntamos es la probabilidad de obtener $k=2$ éxitos (y por lo tanto, claro, $9-2$ fracasos). Teniendo esto en cuenta, podemos escribir el resultado que acabamos de obtener de una forma más útil, que lo relaciona con los parámetros del experimento de Bernouilli subyacente. Separamos los nueve seises del denominador en dos grupos: dos corresponden a los éxitos, y siete a los fracasos. Obtenemos:
             \[\dfrac{\displaystyle\binom{9}{2}5^{9-2}}{6^9}=
             \mbox{\large $ \displaystyle\binom{9}{2}$}\cdot\left(\dfrac{1}{6}\right)^2\cdot\left(\dfrac{5}{6}\right)^{9-2}=
             \binom{n}{k}\cdot p^k\cdot q^{n-k}.
             \]
             \quad\qed
             \end{Ejemplo}

             Con este ejemplo ya estamos listos para la definición:\\[3mm]
           \fbox{\colorbox{Gris025}{\begin{minipage}{14cm}
           \begin{center}
           \vspace{2mm}
           {\bf Variable aleatoria binomial}
           \end{center}
            Una variable aleatoria discreta $X$ es de {\sf tipo binomial con parámetros $n$ y $p$}, lo que se representa con el símbolo $B(n,p)$, si $X$ representa el número de éxitos en la repetición de $n$ experimentos independientes de Bernouilli, con probabilidad $p$ de éxito en cada uno de ellos (y con $q=1-p$).\\[3mm]
            Si $X$ es una variable aleatoria binomial de tipo $B(n,p)$, la probabilidad $P(X=k)$, es decir la probabilidad de obtener $k$ éxitos viene dada por:
           \[\fbox{$\displaystyle
           P(X=k)=\binom{n}{k}\cdot p^k\cdot q^{n-k}.
           $}
           \]
           \vspace{1mm}
           \end{minipage}}}\\[3mm]

           \item Calcular ``a mano'' los valores de la distribución binomial puede ser un engorro, porque los números combinatorios son complicados de calcular, y además hay que incluir las potencias de $p$ y $q$. Por esta razón los valores de $P(X=k)$ en una variable de tipo $B(n,p)$ se encuentran tabulados en muchos libros de Estadística para distintos valores de $n, p$ y $k$. Por ejemplo, en \textattachfile{tablasUCA.pdf}{\textcolor{blue}{este fichero}} tienes unas tablas de la distribución binomial (y de otras muchas que irán apareciendo en las próximas secciones) para valores de $n\leq 10$, y distintos valores de $p$. Estas tablas pertenecen al libro \link{http://knuth.uca.es/moodle/course/view.php?id=21}{Inferencia Estadística}, elaborado por profesores de la Universidad de Cádiz.


            \item De momento, es bueno saber que el comando {\tt DISTR.BINOM(k;n;p;0)} de Calc permite calcular el valor $P(X=k)$ para $B(n,p)$. \footnote{El último parámetro de la función, al que hemos dado el valor $0$, puede ser 1 o 0. Si se usa 1 se obtienen probabilidades acumuladas $P(X\leq k)$.}

                En $R$ tenemos el comando {\tt dbinom(k,size=n,prob=p)} (que se explica solo) para calcular estos mismos valores (los acumulados se obtienen con {\tt dbinom}).
                Por ejemplo, la respuesta del Ejemplo \ref{ejem:BinomialDosSeisesCuatroTiradas}, se obtendría con estos comandos R:
                \begin{verbatim}
                p=1/6
                n=4
                k=2
                dbinom(k,size=n,prob=p)
                \end{verbatim}
                en los que basta modificar los valores de $n,p,k$ para calcular otros valores de la binomial.

                Aprovechamos para comentar que él número combinatorio $\binom{n}{k}$ se puede calcular en R con el comando {\tt choose(n,k)}.

           \end{itemize}

           \subsection*{Media y desviación típica de una variables aleatoria de tipo $B(n,p)$}

           \begin{itemize}

           \item Una variable aleatoria binomial $X$ de tipo $B(n,p)$ se puede considerar como la suma de $n$ variables independientes $X_1,\ldots,X_n$ de tipo Bernouilli$(p)$. Por lo tanto aplicando los  resultados que hemos visto en la sección \ref{sec:OperacionesVariablesAleatorias} se obtienen fórmulas para la media y la varianza de una variable binomial:\\[3mm]
           \fbox{\colorbox{Gris025}{\begin{minipage}{14cm}
           \begin{center}
           \vspace{2mm}
           {\bf Media y varianza de una variables aleatoria de tipo $B(n,p)$}
           \end{center}
           La media de una variable aleatoria discreta de tipo $B(n,p)$ es
           \[\mu=n\cdot p\]
           mientras que su {\sf desviación típica}  es
           \[
           \sigma=\sqrt{n\cdot p\cdot q}.
           \]
           \end{minipage}}}\\[3mm]



\end{itemize}

%
%\section*{Tareas asignadas para esta sesión.}
%
%\begin{enumerate}
%   \item Hay un nuevo cuestionario en Moodle, en este caso centrado sobre todo en preguntas de probabilidad. Tenéis hasta el viernes 4/11.
%\end{enumerate}
%
%
%\section*{\fbox{\colorbox{Gris025}{{Sesión 14. Probabilidad.}}}}
%
%\subsection*{\fbox{\colorbox{Gris025}{{Teorema Central del Límite, primera parte.}}}}
%\subsection*{Fecha: Miércoles, 02/11/2011, 16h.}
%
%\noindent{\bf Atención:
%\begin{enumerate}
%\item Este fichero pdf lleva adjuntos los ficheros de datos necesarios.
%\end{enumerate}
%}
%
%%\subsection*{\fbox{1. Ejemplos preliminares }}
%\setcounter{tocdepth}{1}
%%\tableofcontents
%\section*{Lectura recomendada}
%
%Al menos uno de los siguientes:
%    \begin{itemize}
%    \item Capítulo 5 de "La estadística en Comic".
%    \item Estos resultados están repartidos entre los Capítulos 10 y 11 de Head First Statistics.
%    \item Tema 5 de Bioestadística: Métodos y Aplicaciones, Univ. de Málaga (veremos las variables aleatorias continuas en próximas sesiones).
%    \item Apuntes de la sexta y comienzo de la séptima sesiones del Curso 2010-2011.
%
%    \end{itemize}

\section{Distribuciones Binomiales con n muy grande}


\noindent{\em ``If I have seen further, it is by standing upon the shoulders of giants''.\\ Isaac Newton, 1676}.

\begin{itemize}

\item Cuando los matemáticos empezaron a trabajar con la distribución binomial, no había ordenadores -- ni calculadoras-- disponibles. En esas condiciones, incluso el cálculo de un valor relativamente sencillo $P(X=30)$ para la distribución binomial $B(100,1/3)$, implicaba calcular números como $\binom{100}{30}$ (que es del orden de $10^{25}$). Ese cálculo podía resultar un inconveniente casi insufrible. Por esa razón, aquellos matemáticos empezaron a pensar sobre el comportamiento de la distribución binomial para valores de $n$ cada vez más grandes. Entre esos matemáticos estaba Abraham De Moivre, un hugonote francés refugiado en Londres, que había pasado a formar parte del selecto grupo de personas cercanas a Newton. Esa cercanía a uno de los fundadores del Cálculo ayuda sin duda a entender cómo llegó De Moivre a algunos de sus hallazgos.


\item De Moivre empezó pensando en los valores de una distribución binomial $B(n,p)$ para $n$ pequeño, por ejemplo $n=10$, y un valor cualquiera de $p$, por ejemplo $p=1/3$. Al representar los valores de probabilidad $P(X=0)$, $P(X=1)$,$P(X=2)$,\dots,$P(X=10)$ en un gráfico similar a un histograma se obtiene algo como esto\footnote{Fíjate en que las escalas en los ejes son muy distintas.}:
   \begin{center}
   \includegraphics[height=6cm]{2011_11_02_Figura01-Binomial.png}
   \end{center}
   De Moivre probablemente siguió pensando en este tipo de figuras para valores de $n$ cada vez más grandes. Por ejemplo, para $n=100$ se tendría:
   \begin{center}
   \includegraphics[height=10cm]{2011_11_02_Figura02-Binomial-a.png}
   \end{center}
   Atención a las escalas de nuevo. En esta figura la individualidad de cada uno de los rectángulos empieza a perderse, dando paso a la percepción de una cierta forma de {\em curva acampanada} que describe lo que ocurre, con una cima en el valor $\mu$, como en esta figura:
   \begin{center}
   \includegraphics[height=10cm]{2011_11_02_Figura03-Binomial.png}
   \end{center}
   Por su proximidad a Newton, estas situaciones en las que tenemos una curva y una aproximación de la curva mediante rectángulos no le podían resultar extrañas a De Moivre. Esas mismas ideas se estaban utilizando para sentar las bases del Cálculo Integral. En la siguiente figura hay un fragmento del libro \link{http://books.google.com/books?id=Tm0FAAAAQAAJ&pg=PA1\#v=onepage&q&f=false}{Principia Mathematica} (nos atrevemos a decir que es uno de los libros más importantes en la historia de la humanidad), en el que Newton sentó las bases del Cálculo Diferencial e Integral. Como puedes ver, en la parte que hemos destacado, Newton sugiere que se considere un número cada vez mayor de rectángulos (su número tiende hacia infinito), con bases que son cada vez más pequeñas en proporción al total de la figura.
   \begin{center}
   \includegraphics[height=8cm]{2011_11_02_Figura04-Newton.png}
   \end{center}
    Esos eran exactamente los ingredientes que aparecían en la situación en la que De Moivre se encontraba. Así que la pregunta, parecía evidente: {\em ¿cuáles serían esas curvas que De Moivre estaba empezando a entrever en sus reflexiones sobre la binomial?}.  Porque si tuviéramos la ecuación de esa curva podríamos usarla para aproximar los valores de la binomial sin necesidad de calcular los molestos números combinatorios. Por otra parte, aquellos matemáticos habían pensado mucho sobre fórmulas binomiales, así que De Moivre consiguió identificar esas curvas, y vio que las curvas que buscaba respondían todas a la misma fórmula. Para aproximar una binomial distribución binomial $B(n,p)$, con $n$ grande, y recordando que $\mu=np$ y $\sigma=\sqrt{npq}$, había que usar la curva:
    \begin{equation}\label{ecu:distribucionNormalGenerica}
    \fbox{\colorbox{Gris025}{$f_{\mu,\sigma}(x)=\displaystyle\dfrac{1}{\sigma\sqrt{2\pi}}e^{-\frac{1}{2}\left(\frac{x-\mu}{\sigma}\right)^2}
    $}}
    \end{equation}
    ¡Sí, esos son el número $e$ y el número $\pi$! Produce un cierto vértigo verlos aparecer aquí, cuando todo esto ha empezado lanzando dados... Veamos como funciona esta fórmula en un ejemplo.
    \begin{Ejemplo}\label{ejem:BinomialVsNormal}
    Volvamos al cálculo que proponíamos al principio de esta sección. Calculemos $P(X=30)$ para una distribución binomial $B(100,1/3)$ (es decir, que puedes pensar que estamos tirando un dado 100 veces y preguntándonos por la probabilidad de obtener 30 veces un número 1 o 2. Probabilidad $2/6=1/3$). Si usamos la definición, calcularíamos
   \[\displaystyle
   P(X=k)=\binom{n}{k}\cdot p^k\cdot q^{n-k},
   \]
   con $n=100, k=30, p=\frac{1}{3}$. Para calcular esto hay que obtener $\binom{100}{30}\approx 2.9372\cdot 10^{25}$. Con esto, finalmente se obtiene $P(X=30)\approx 0.06728$. Si usamos la función $f_{\mu,\sigma}(x)$ con $\mu=np=\frac{100}{3}$ y $\sigma=\sqrt{n\cdot p\cdot q}\approx4.714$ se obtiene
   \[f(30|\mu,\sigma)\approx 0.06591.\]
   La aproximación, como vemos, no está mal, {\em aunque no es espectacular}. Hay un detalle que podría mejorarla, pero lo dejamos para más adelante, cuando hayamos entendido esto mejor.
   \quad\qed
   \end{Ejemplo}
    Antes de seguir adelante, en \textattachfile{AproximacionBinomialPorNorrmal.R}{\textcolor{blue}{este fichero}} tienes las instrucciones en R para repetir esos cálculos.
\end{itemize}

\section{Las distribuciones continuas entran en escena...}\label{sec:distribucionesContinuasEntranEscena}

\begin{itemize}

    \item Por otra parte, si pensamos en valores de $n$ cada vez más grandes, las preguntas como $P(X=k)$ se vuelven cada vez menos relevantes. Si vas a lanzar un dado 10000 veces, la probabilidad de obtener exactamente $30$ veces 1 o 2 es prácticamente nula. Puedes usar el fichero de instrucciones anterior para calcularlo. En resultado es del orden de $10^{-128}$, inimaginablemente pequeño. Incluso los valores más probables (cercanos a la media $\mu$) tienen en este ejemplo  probabilidades de en torno a un $2\%$.  No, en casos como este, lo que tiene interés es preguntar por {\em intervalos de valores}. igual que hacíamos en la Estadística Descriptiva. Es decir, nos preguntamos ¿cuál es la probabilidad de obtener 300 éxitos o menos? O también, ¿cuál es la probabilidad de obtener entre 300 y 600 éxitos del total de 1000?\\

        Para entender la respuesta, volvamos por un momento a un valor de $n$ más moderado. Por ejemplo $n=21$, todavía con $p=1/3$. La media es $\mu=7$, y el diagrama correspondiente a la distribución $B(21,1/3)$ es
        \begin{center}
        \includegraphics[height=8cm]{2011_11_02_Figura05-ProbabilidadIntervalo1.png}
        \end{center}
        ¿Cuál es la probabilidad de obtener entre 5 y 9 éxitos? Pues la suma de áreas de los rectángulos coloreados en oscuro en esta figura (recuerda que la suma total de áreas de los rectángulos es 1, cuando se dibujan a escala):
        \begin{center}
        \includegraphics[height=8cm]{2011_11_02_Figura05-ProbabilidadIntervalo2.png}
        \end{center}
        Ese valor es $P(5\leq X\leq 9)=P(X=5)+P(X=6)+\cdots+P(X=9)$, y está cercano al $75\%$. Si ahora volvemos al problema para $B(1000,1/3)$  y nos preguntamos por $p(300\leq X\leq 600)$, vemos que tenemos que sumar el área de 301 rectángulos para calcular esa probabilidad. ¿No hay una forma mejor de hacer esto? Para De Moivre, en contacto con las ideas recién nacidas sobre cálculo integral y su aplicación al cálculo del área bajo una curva, la respuesta tuvo que ser evidente. Porque precisamente Newton había descubierto que, para definir el área bajo la gráfica de una función, para valores de $x$ entre $a$ y $b$, había que considerar una aproximación del área mediante $n$ rectángulos y estudiar el límite de esas aproximaciones para $n$ cada vez más grande, como se ilustra en esta figura:
        \begin{center}
        \includegraphics[height=8cm]{2011_11_02_Figura06-AreaMedianteIntegralesSumaInferior.png}
        \end{center}
        En \textattachfile{2011_11_02_AreaMedianteIntegralesSumaInferior.html}{\textcolor{blue}{este fichero html}} (se abre en el navegador) puedes explorar lo que ocurre con esas aproximaciones cuando $n$ aumenta.

        Volviendo a la distribución binomial, si $f_{\mu,\sigma}(x)$ es la curva que aproxima a $B(1000,1/3)$, entonces la probabilidad que buscamos será, aproximadamente
        \[\int_{300}^{600}f_{\mu,\sigma}(x)dx\]
        Y esta integral da como resultado aproximadamente $0.9868$. De hecho, si usamos R para sumar los valores $P(X=300)+\cdots+P(X=600)$ se obtiene aproximadamente $0.9888$. La aproximación está muy bien, y eso que aún tenemos pendiente mejorar el ajuste de la curva $f_{\mu,\sigma}(x)$ con los valores de $B(p,n)$.

    \item Recapitulemos: para calcular la probabilidad $P(a\leq X\leq b)$ de $B(n,p)$ hemos usado una cierta función $f_{\mu,\sigma}(x)$, y hemos visto que
        \[P(a\leq X\leq b)\approx \int_a^b f_{\mu,\sigma}(x)dx.\]
        Y en el proceso para llegar a esto hemos visto que el valor de probabilidad para un valor concreto es prácticamente nulo, es decir $P(X=a)\approx 0$. ¿Dónde hemos oído algo parecido? ¿Dónde decíamos que la probabilidad de un valor concreto era $0$? En los problemas de probabilidad geométrica en los que tratábamos con {\sf variables aleatorias continuas}\footnote{Ver el Ejemplo \ref{ejem:ProbabilidadGeometricaSubconjuntosCirculo} de la página \pageref{ejem:ProbabilidadGeometricaSubconjuntosCirculo}}.  De hecho, hemos dejado pendiente desde la sección \ref{sec:variablesAletorias} (pág. \pageref{sec:variablesAletorias}) el tratamiento general de las variables aleatorias continuas. Las ideas anteriores justifican la siguiente definición.\\[3mm]
           \fbox{\colorbox{Gris025}{\begin{minipage}{14cm}
           \begin{center}
           \vspace{2mm}
           {\bf Función de densidad de una variables aleatoria continua}
           \end{center}
           Si $X$ es una variable aleatoria continua, su {\sf función de densidad} es una función $f(x)$
           que tiene estas propiedades:
           \begin{itemize}
           \item $f(x)\geq 0$ para todo $x$; $f$ no toma valores negativos.
           \item El área total bajo la gráfica de $f$ es 1:
           \[\int_{-\infty}^{\infty}f(x)dx=1\]
           \item La función de densidad permite calcular probabilidades asociadas a $X$ mediante:
           \[P(a\leq X\leq b)=\int_a^b f(x)dx.\]
           \end{itemize}
           \end{minipage}}}\\[3mm]
        Ahora el descubrimiento de De Moivre se puede expresar en este nuevo lenguaje, más claramente. Lo que De Moivre descubrió es que para valores de $n$ grandes, la variable aleatoria binomial $B(n,p)$ (¡qué es discreta!) se puede aproximar bien por una variable aleatoria de tipo continuo, cuya función de densidad es la que aparece en la Ecuación \ref{ecu:distribucionNormalGenerica} de la página \pageref{ecu:distribucionNormalGenerica}. Esta relación con la binomial hace que esa variable aleatoria continua sea la más importante de todas, y es la razón por la que le vamos a dedicar una atención especial en la próxima sección.
\end{itemize}


%
%
%\section*{Tareas asignadas para esta sesión.}
%
%\begin{enumerate}
%   \item No hay tareas asignadas para esta sesión.
%\end{enumerate}
%
%\section*{\fbox{\colorbox{Gris025}{{Sesión 15. Probabilidad.}}}}
%
%\subsection*{\fbox{\colorbox{Gris025}{{Teorema Central del Límite, segunda parte. Distribución normal.}}}}
%\subsection*{Fecha: Viernes, 04/11/2011, 14h y 16h (sesión doble).}
%
%\noindent{\bf Atención:
%\begin{enumerate}
%\item Este fichero pdf lleva adjuntos los ficheros de datos necesarios.
%\end{enumerate}
%}
%
%%\subsection*{\fbox{1. Ejemplos preliminares }}
%\setcounter{tocdepth}{1}
%%\tableofcontents
%\section*{Lectura recomendada}
%
%Al menos uno de los siguientes:
%    \begin{itemize}
%    \item Capítulo 5 de "La estadística en Comic".
%    \item Estos resultados están repartidos entre los Capítulos 10 y 11 de Head First Statistics.
%    \item Tema 5 de Bioestadística: Métodos y Aplicaciones, Univ. de Málaga.
%    \item Apuntes de la octava y novena sesiones del Curso 2010-2011.
%
%    \end{itemize}

\section{Más sobre distribuciones continuas}

\begin{itemize}
    \item Vamos a empezar con un ejemplo que ilustre la definición de función de densidad una variable aleatoria continua.
        \begin{ejemplo}
        Consideremos la variable aleatoria continua $X$ cuya función de densidad es
        \[f(x)=\dfrac{1}{\pi(1+x^2)}.\]
        La gráfica de esta función se muestra en la siguiente figura:
       \begin{center}
       \includegraphics[width=14cm]{2011_11_04_Figura00-FuncionDensidad.png}
       \end{center}
        Podemos empezar comprobando que la probabilidad total calculada con esta función es $1$. En efecto,
        \[P(-\infty\leq X\leq\infty)=\int_{-\infty}^{\infty}f(x)dx=
        \int_{-\infty}^{\infty}\dfrac{1}{\pi(1+x^2)}dx=\]
        \[=\dfrac{1}{\pi}\left[\arctan x\right]_{-\infty}^{\infty}=
        \dfrac{1}{\pi}\left(\dfrac{\pi}{2}-\left(-\dfrac{\pi}{2}\right)\right)=1.
        \]
        Ahora nos preguntamos cuál es la probabilidad de que $X$ tome un valor entre 0 y 1. Tenemos que calcular el área sombreada en la figura:
       \begin{center}
       \includegraphics[width=10cm]{2011_11_04_Figura01-FuncionDensidad.png}
       \end{center}
        Y eso significa que debemos calcular esta integral:
        \[
        P(0\leq X\leq 1)=\int_0^1f(x)dx=\int_0^1\dfrac{1}{\pi(1+x^2)}dx=\dfrac{1}{\pi}\left[\arctan x\right]_0^1=
        \]\[
        =\dfrac{1}{\pi}\left(\arctan 1-\arctan 0\right)=\dfrac{1}{\pi}\left(\frac{\pi}{4}-0\right)=\dfrac{1}{4}.
        \]
        Si necesitas recordar las propiedades del arcotangente, \textattachfile{2011_11_04_GraficaArcoTangente.html}{\textcolor{blue}{aquí tienes su gráfica}} (se abre en el navegador). Si lo que queremos es calcular la probabilidad de que $X$ tome valores mayores que 1, debemos calcular el área que aparece en esta figura
       \begin{center}
       \includegraphics[width=14cm]{2011_11_04_Figura02-FuncionDensidad.png}
       \end{center}
        Y para ello hay que hacer este cálculo:
        \[
        P(X>1)=\int_1^{\infty}f(x)dx=\lim_{N\to\infty}\left(\int_1^N\dfrac{1}{\pi(1+x^2)}dx\right)=\lim_{N\to\infty}\left(\dfrac{1}{\pi}\left[\arctan x\right]_1^N\right)=
        \]\[
        =\lim_{N\to\infty}\left(\dfrac{1}{\pi}\left(\arctan N-\arctan 1\right)\right)=\dfrac{1}{\pi}\lim_{N\to\infty}\left(\arctan N-\dfrac{\pi}{4}\right)=\dfrac{1}{\pi}\left(\dfrac{\pi}{2}-\dfrac{\pi}{4}\right)=\dfrac{1}{4}.
        \]
        Por otra parte, a partir de la simetría de la función $f$, el cálculo que acabamos de hacer era innecesario, puesto que ya conocíamos $P(0\leq X\leq 1)$. ¿Ves cómo calcular $P(X>1)$ usando esto?\qed
        \end{ejemplo}


    \item A partir de este ejemplo, vemos que si $X$ es una variable aleatoria continua y  $f(x)$ es su función de densidad, la función $f$ representa una forma de repartir la probabilidad total (que siempre es uno) entre los puntos de la recta real, de maneras que las zonas donde $f(x)$ vale más son las zonas con mayor probabilidad. Esto se ilustra en la siguiente figura, para una función de densidad ficticia:
       \begin{center}
       \includegraphics[width=15cm]{2011_11_04_Figura04-FuncionDensidad.png}
       \end{center}

\end{itemize}

\subsection*{Media y varianza de una variable aleatoria continua}

\begin{itemize}
    \item Es fácil entender que uno de nuestros primeros objetivos es extender la definición de media y varianza al caso de variables aleatorias continuas. Recordemos que para una variable aleatoria discreta se definía
    \[\mu=\sum_{i=1}^{k}x_iP(X=x_i),\qquad  \sigma^2=\sum_{i=1}^{k}(x_i-\mu)^2P(X=x_i),\]
    siendo $x_1,x_2,\ldots,x_k$ los valores distintos que toma la variable. ¿Cómo podemos extender esto al caso de una variable continua con función de densidad $f(x)$? Bueno, siempre podemos {\em desandar el camino que tomó De Moivre}. Es decir, podemos pensar en reemplazar la función $f(x)$ por una (enorme) colección de rectángulos, como en esta figura:
   \begin{center}
   \includegraphics[width=15cm]{2011_11_04_Figura05-EsperanzaConFuncionDensidad.png}
   \end{center}
   Y ahora podemos {\em ``olvidar la curva $f(x)$ y simplemente sumar usando estos rectángulos como si se tratara de una variable discreta''.} A este proceso lo llamaremos {\sf discretización} de la variable aleatoria continua. ¿Cómo sería la suma correspondiente?
   \[\mu=E(X)=\sum_{\begin{minipage}{1.5cm}\tiny todos los\\ rectángulos\end{minipage}} x_iP(X=x_i)\]
   En esta suma $x_i$ es un punto dentro de uno de los rectángulos, algo así como las marcas de clase que veíamos al principio del curso. Puedes pensar, por ejemplo, que $x_i$ es el punto medio de la base de cada rectángulo.  Y ¿cuánto vale $P(X=x_i)$? Ese valor es la altura del rectángulo. Y en nuestro caso, esos valores vienen dados por $f(x_i)$, siendo $f$ la función de densidad. Así que la suma anterior se puede escribir
   \[\mu=E(X)=\sum_{\begin{minipage}{1.5cm}\tiny todos los\\ rectángulos\end{minipage}} x_if(x_i).\]
   Por otra parte, sabemos que la aproximación a $f$ mejora al hace más pequeños y numerosos los intervalos (lo puedes observar en  \textattachfile{2011_11_04_EsperanzaMatematicaParaFuncionDensidad.html}{\textcolor{blue}{este fichero html, que se abrirá en tu navegador}}). Este tipo de sumas, cuando los intervalos se hacen cada vez más finos y numerosos, se convierten en integrales al pasar al límite continuo. Y puesto que estamos sumando {\em todos los rectángulos del eje real}, esta integral recorre todo el eje, desde $-\infty$ hasta $\infty$. Con estas observaciones, no debería resultar demasiado extraña esta definición:\\[3mm]
           \fbox{\colorbox{Gris025}{\begin{minipage}{14cm}
           \begin{center}
           \vspace{2mm}
           {\bf Media (o valor esperado) de una variables aleatoria continua}
           \end{center}
           Si $X$ es una variable aleatoria continua con función de densidad $f(x)$, entonces la {\sf media} de $f$ es el valor
           \[\mu=\int_{-\infty}^{\infty} x\cdot f(x)dx.\]
           \end{minipage}}}\\[3mm]

\item La fórmula para la varianza se puede justificar mediante este mismo proceso de discretización. Se obtiene:\\[3mm]
           \fbox{\colorbox{Gris025}{\begin{minipage}{14cm}
           \begin{center}
           \vspace{2mm}
           {\bf Varianza y desviación típica de una variables aleatoria continua}
           \end{center}
           Si $X$ es una variable aleatoria continua con función de densidad $f(x)$ y media $\mu$, entonces la {\sf varianza} de $f$ es el valor
           \[\sigma^2=\int_{-\infty}^{\infty} (x-\mu)^2\cdot f(x)dx.\]
           Y la {\sf desviación típica} $\sigma$ es la raíz cuadrada de la varianza.
           \end{minipage}}}\\[3mm]


\item Resumimos la situación en una tabla para que puedas ver las analogías y las diferencias entre variables discretas y continuas
    \begin{center}
    \begin{tabular}{|c|c|c|}
    \hline
                        & $X$ Var. continua                                             & $X$ Var. discreta                               \\
    \hline
    Media $\mu$         & $\displaystyle \int_{-\infty}^{\infty} x\cdot f(x)dx$         & $\displaystyle\sum_{i=1}^k x_iP(X=x_i)$         \\
    \hline
    Varianza $\sigma^2$ & $\displaystyle \int_{-\infty}^{\infty} (x-\mu)^2\cdot f(x)dx$ & $\displaystyle\sum_{i=1}^k (x_i-\mu)^2P(X=x_i)$ \\
    \hline
    \end{tabular}
    \end{center}
    Como puede apreciarse, si se reemplaza $P(X=x_i)$ por $f(x)$, el paralelismo entre las dos fórmulas resulta evidente.

\end{itemize}

\subsection*{Variables continuas con soporte en un intervalo}

\begin{itemize}

\item A menudo sucede que una variable aleatoria continua $X$ sólo puede tomar valores dentro de un cierto intervalo $[a,b]$. En estos casos diremos que {\sf la variable $X$ tiene soporte en el intervalo $[a,b]$}. Por ejemplo, si definimos una variable continua $X$ que representa la altura de los ciudadanos españoles, entonces los valores de $X$ (en cm) están todos ellos comprendidos en el intervalo $[0,300]$. En casos como este, los valores de $f(x)$ fuera del intervalo $[a,b]$ son iguales a cero . Y eso incluso simplifica nuestros cálculos de medias y varianzas, porque, en lugar de una integral $\int_{-\infty}^{\infty}$, lo que tenemos que calcular es una integral $\int_a^b$. Veamos un ejemplo.
    \begin{ejemplo}
    Supongamos que $X$ es una variable aleatoria continua cuya función de densidad es
    \[f(x)=\begin{cases}2\cdot(2-x)&\mbox{ para }1\leq x\leq 2\\ 0&\mbox{en otro caso}\end{cases}\]
    como se ve en esta figura
    \begin{center}
    \includegraphics[width=15cm]{2011-11-04-EjemploFuncionDensidadIntervaloAcotado.png}
    \end{center}
    Empieza por comprobar (queda como ejercicio) que el área total bajo la gráfica de $f$ es 1. ¿Cuál es la media de $X$? Tenemos que calcular:
    \[
    \mu=\int_{1}^{2}xf(x)dx=\int_{1}^{2}x\cdot 2\cdot(2-x)dx=2\int_{1}^{2}(2x-x^2)dx=2\left[x^2-\frac{x^3}{3}\right]_1^2=2\left(4-\frac{8}{3}\right)-2\left(1-\frac{1}{3}\right)=\dfrac{4}{3}.
    \]
    Dejamos como ejercicio para el lector comprobar que
    \[
    \sigma^2=\int_{1}^{2}(x-\mu)^2f(x)dx=2\int_{1}^{2}\left(x-\dfrac{4}{3}\right)^2(2-x)dx=\dfrac{1}{18}.
    \]
    También puedes repetir el ejercicio con
    \[f(x)=\begin{cases}1&\mbox{ para }2\leq x\leq 2\\ 0&\mbox{en otro caso}\end{cases}\]
    \qed

    \end{ejemplo}

\end{itemize}

\subsection*{}



\section{Distribución normal}

\begin{itemize}

\item En la sección \ref{sec:distribucionesContinuasEntranEscena} (\pageref{sec:distribucionesContinuasEntranEscena}) hemos visto que De Moivre descubrió un tipo especialmente importante de variable aleatoria continua, a la que vamos a poner nombre.\\[3mm]
       \fbox{\colorbox{Gris025}{\begin{minipage}{14cm}
       \begin{center}
       \vspace{2mm}
       {\bf Variable aleatoria normal}
       \end{center}
       Una variable aleatoria continua $X$ es {\sf normal de tipo $N(\mu,\sigma)$} si su función de densidad es
       de la forma
       \[f_{\mu,\sigma}(x)=\dfrac{1}{\sigma\sqrt{2\pi}}e^{-\frac{1}{2}\left(\frac{x-\mu}{\sigma}\right)^2}\]
       donde $\mu$ y $\sigma>0$ son dos números reales.
       \end{minipage}}}\\[3mm]
       Las variables aleatorias normales son, insistimos, excepcionalmente importantes. Pero el trabajo con ellas tropieza con el inconveniente de que {\bf la función $f_{\mu,\sigma}(x)$ no tiene una primitiva elemental}, y por lo tanto el trabajo debe hacerse mediante integración numérica, tablas, o usando software como $R$ o una hoja de cálculo.


\item  Antes de seguir adelante vamos a ver el aspecto que tienen estas funciones, y como dependen de los valores de $\mu$ y $\sigma$. Lo puedes observar en \textattachfile{2011_11_04_DistribucionesNormales.html}{\textcolor{blue}{este documento html}}. Una propiedad especialmente significativa de esta familia de variables aleatorias es que
    \[P(\mu-\sigma<X<\mu+\sigma)\approx 0.68\mbox{, y también }P(\mu-2\sigma<X<\mu+2\sigma)\approx 0.95.\]
    Y además vamos a hacer constar lo que seguramente el lector intuye desde hace rato.\\[3mm]
       \fbox{\colorbox{Gris025}{\begin{minipage}{14cm}
       \begin{center}
       \vspace{2mm}
       {\bf Media y desviación típica de una variable aleatoria normal}
       \end{center}
       Si $X$ es una variable aleatoria normal de tipo $N(\mu,\sigma)$, cuya función de densidad es por tanto $f_{\mu,\sigma}(x)$, entonces
       \[\mu=E(X), \mbox{ es decir $\mu$ es la media de $X$},\]
       y
       \[\sigma^2=\operatorname{Var}(X), \mbox{ es decir $\sigma^2$ es la varianza de $X$}.\]
       Y, naturalmente, $\sigma$ es la desviación típica de $X$.
       \end{minipage}}}\\[3mm]
    No vamos a entrar en los detalles del cálculo de estas medias y varianzas para una variable aleatoria normal $N(\mu,\sigma)$, para no complicar la discusión.

\end{itemize}

\subsection{Distribución normal estándar. Tipificación.}

\begin{itemize}

\item  Una variable aleatoria $Z$, normal de tipo $N(0,1)$ (con media $\mu=0$ y desviación típica $\sigma=1$) es una {\sf variable normal estándar (o tipificada)}. Como hemos indicado, vamos a reservar la letra $Z$ para referirnos a una variable normal estándar; esta es una práctica generalizada en Estadística, que conviene respetar. La función de densidad $f_{0,1}(x)$ es especialmente simple:
       \[f_{0,1}(x)=\dfrac{1}{\sqrt{2\pi}}e^{-\frac{x^2}{2}}\]


\item ¿Por qué es importante la variable normal estándar? Pues porque todas las demás normales se pueden obtener de ella. Concretamente,\\[3mm]
       \fbox{\colorbox{Gris025}{\begin{minipage}{14cm}
       \begin{center}
       \vspace{2mm}
       {\bf Variable normal $N(\mu,\sigma)$ a partir de $N(0,1)$}
       \end{center}
       Si $X$ es una variable aleatoria normal de tipo $N(\mu,\sigma)$, entonces
       \[Z=\dfrac{X-\mu}{\sigma}\]
       es una variable normal estándar. Al proceso de obtener los valores de $Z$ a partir de los de $X$ se le llama {\sf tipificación.}\\
       En particular, para las funciones de densidad se cumple que:
       \[f_{\mu,\sigma}(x)=\dfrac{1}{\sigma}f\left(\left.\dfrac{x-\mu}{\sigma}\right|0,1\right),\]
       como puedes comprobar.
       \end{minipage}}}\\[3mm]
       De esta última igualdad se deduce, entre otras cosas,la propiedad que hemos comentado antes sobre el hecho de que
       \[P(\mu-\sigma<X<\mu+\sigma)\approx 0.68\] no depende de $\mu$ ni de $\sigma$.

       \item Este proceso de tipificación de las variables normales significa entre otras cosas que sólo necesitamos saber responder a las preguntas sobre probabilidad formuladas para el caso $N(0,1)$, porque todos los demás casos se reducen a este mediante la tipificación. Veamos un ejemplo.
           \begin{ejemplo}
           Una variable aleatoria continua $X$ es normal, de tipo $N(400,15)$. ¿Cuál es el valor de la probabilidad $P(380\leq X\leq 420)$?\\
           Consideremos la variable aleatoria
           \[Z=\dfrac{X-\mu}{\sigma}=\dfrac{X-400}{15}.\]
           Como sabemos, $Z$ es de tipo normal estándar $N(0,1)$. Y entonces:
           \[380\leq X\leq 420\]
           significa
           \[380-400\leq X-400\leq 420-400,\mbox{ es decir }-20\leq X\leq 20,\]
           y por tanto
           \[\dfrac{-20}{15}\leq \dfrac{X-400}{15}\leq\dfrac{20}{15},\mbox{ es decir }\dfrac{-4}{3}\leq Z\leq\dfrac{4}{3},\]
           por la construcción de $Z$. En resumen:
           \[P(380\leq X\leq 420)=P\left(\dfrac{-4}{3}\leq Z\leq\dfrac{4}{3}\right)\approx P(-1.33\leq Z\leq 1.33),\]
           y como se ve lo que necesitamos es saber responder preguntas para $Z$, que es de tipo $N(0,1)$. En este caso esa probabilidad es (leer más abajo) $\approx 0.82$.\qed
           \end{ejemplo}
       Este ejemplo explica porque los valores de $N(0,1)$ son especialmente importantes. De hecho, se encuentran tabulados al final de casi cualquier libro de estadística (salvo los más modernos, que dan por sentado que se va a usar un ordenador para estas operaciones). Ya aprenderemos, en la parte práctica del curso, a usar esas tablas y, sobre todo, el software para calcular los valores que necesitemos. Por el momento nos conformamos con saber que si $Z$ es de tipo $N(0,1)$, el comando R para calcular $P(a\leq Z\leq b)$ es:\\
       \begin{center}
       {\tt pnorm(b)-pnorm(a)}
       \end{center}
       mientras que en Calc se consigue lo mismo con
       \begin{center}
       {\tt DISTR.NORM.ESTAND(b)-DISTR.NORM.ESTAND(a)}
       \end{center}
       donde ahora $b$ y $a$ son los nombres de las celdas que almacenan los extremos del intervalo.
       \subsection*{Funciones de distribución}

       \item ¿Qué es lo restamos en estas dos fórmulas? Son valores de la forma $P(Z\leq k)$. Dada una variable aleatoria $X$ (discreta o continua), la función
       \[F(x)=P(X\leq x)\]
       se llama su {\sf función de distribución}. En $R$ esas funciones se reconocen por el prefijo {\tt p}. Por ejemplo, {\tt pnorm} es la función de distribución de una variable continua de tipo $N(\mu,\sigma)$, mientras que {\tt pbinom} es la de una variable discreta de tipo $B(n,p)$.

       \item La diferencia

\end{itemize}

\section{El teorema central del límite}\label{sec:teoremaCentralLimitePrimeraVersion}

\begin{itemize}

\item Para cerrar el intenso trabajo de este capítulo , queremos volver a la idea de De Moivre, que ahora podemos explicar con más claridad. Lo que De Moivre descubrió es que cuando se considera una variable binomial de $X$ de tipo $B(n,p)$ con valores de $n\emph{}$ muy grandes, sus valores se pueden calcular muy aproximadamente utilizando una variable $Y$ con distribución normal $N(\mu,\sigma)$, con
    \[\mu=n\cdot p,\sigma=\sqrt{n\cdot p\cdot q}.\]
    Pero esta idea contiene una pequeña trampa. Si volvemos a mirar los diagramas tipo histograma de la última sección, veremos que cada una de las barras está {\em centrada en $k$}. En la figura hemos ampliado la base para que esto resulte evidente:
   \begin{center}
   \includegraphics[width=12cm]{2011-11-04-CorreccionContinuidad.png}
   \end{center}
    Y eso significa que en realidad, como hemos indicado en la figura, esas barras incluyen la probabilidad de los valores desde $0.5$ unidades antes hasta $0.5$ unidades después del valor que aparece en su base. Al pasar a la aproximación por la normal debemos tener esto en cuenta si no queremos obtener valores poco precisos. Esto se conoce como {\sf corrección de continuidad}. En la práctica se traduce es esto:\\[3mm]
   \fbox{\colorbox{Gris025}{\begin{minipage}{14cm}
   \begin{center}
   \vspace{2mm}
   {\bf TEOREMA CENTRAL DEL LÍMITE, PRIMERA VERSIÓN}\\
   {\bf Aproximación de $X$ de tipo binomial $B(n,p)$ por $Y$ de tipo normal $N(\mu,\sigma)$}
   \end{center}
   Siendo
   \[\mu=n\cdot p,\sigma=\sqrt{n\cdot p\cdot q}\]
   y siempre que se cumpla $n\cdot p>5, n\cdot q>5$ (en caso contrario la aproximación no es muy buena),
   \begin{enumerate}
   \item para calcular $P(X=k)$, la aproximación por la normal que usamos es $P(k-0.5\leq Y\leq k+0.5)$.
   \item para calcular $P(X\leq k)$, la aproximación por la normal que usamos es $P(Y\leq k+0.5)$.
   \item para calcular $P(k_1\leq X\leq k_2)$, la aproximación por la normal que usamos es $P(k_1-0.5\leq Y\leq k_2+0.5)$.
   \end{enumerate}
   \end{minipage}}}\\[3mm]
   Y aquí tienes un \textattachfile{DistibucionBinomial.R}{\textcolor{blue}{fichero de instrucciones R}} con el que practicar estas aproximaciones (comparándolas con los valores calculados directamente  usando la binomial).


\item Esta es la primera ocasión en la que nos encontramos con que, para valores de $n$ grande, una distribución --en este caso la binomial $B(n,p)$-- se comporta cada vez más como si fuese una normal. Y la distribución binomial, recordémoslo, resulta del efecto combinado de $n$ ensayos independientes. Este comportamiento es el primer indicio de algo que iremos confirmando en el curso: cualquier fenómeno natural que resulte de la acción superpuesta (es decir, de la suma) de un número enorme de procesos independientes, tendrá una distribución aproximadamente normal. Y cuando se combina esta observación con el descubrimiento de la estructura atómica de la materia, o de la estructura celular de los seres vivos, se empieza a percibir el {\sf alcance universal de la distribución normal}, a través del Teorema Central del Límite, como una de las leyes fundamentales de la naturaleza. No encontramos mejor manera de resumirlo que la que Gonick y Smith incluyen en su libro al hablar de este teorema (pág. 83):
   \begin{center}
   \includegraphics[height=12cm]{2011-11-04-DeMoivre.png}
   \end{center}




\end{itemize}


%
%
%\section*{Tareas asignadas para esta sesión.}
%
%\begin{enumerate}
%   \item Aunque todavía no están disponibles, a lo largo del fin de semana aparecerán en Moodle una nueva hoja de problemas y un cuestionario.
%\end{enumerate}

%\section*{Lectura recomendada}
%
%Al menos uno de los siguientes:
%    \begin{itemize}
%    \item Capítulo 3 de "La estadística en Comic".
%    \item Capítulo 4 de Head First Statistics.
%    \item Tema 4 de Bioestadística: Métodos y Aplicaciones, Univ. de Málaga (no veremos la definición axiomática de probabilidad).
%    \item Apuntes de la tercera sesión del Curso 2010-2011, (las tres últimas páginas) y cuarta sesión íntegra.
%    \item Capítulo 4 de "La estadística en Comic".
%    \item Capítulo 5 de Head First Statistics.
%    \item Tema 5 de Bioestadística: Métodos y Aplicaciones, Univ. de Málaga (veremos las variables aleatorias continuas en próximas sesiones).
%    \item Apuntes de la sexta y comienzo de la séptima sesiones del Curso 2010-2011.
%    \item Capítulo 5 de "La estadística en Comic".
%    \item Capítulo 7 de Head First Statistics.
%    \item Tema 5 de Bioestadística: Métodos y Aplicaciones, Univ. de Málaga (veremos las variables aleatorias continuas en próximas sesiones).
%    \item Apuntes de la sexta y comienzo de la séptima sesiones del Curso 2010-2011.
%    \end{itemize}

%\section*{Recomendaciones.}
%
%\begin{enumerate}
%   \item El \link{http://www.ine.es/}{INE (Instituto Nacional de Estadística)} es el organismo oficial encargado, entre otras cosas del censo electoral, la elaboración del IPC (índice de precios de consumo), la EPA (encuesta de población activa), el PIB (producto interior bruto) etc. El instituto ofrece una enorme colección de datos estadísticos accesibles para cualquiera a través de la red (sistema INEbase). Además, tiene alojado en su página web un \link{http://www.ine.es/explica/explica.htm}{portal de divulgación estadística} en el que se pueden ver vídeos sobre estos y otros temas, que tal vez os interesen.
%\end{enumerate}




\part{Inferencia Estadística}
%  \input{003-InferenciaEstadistica}

    \chapter{Inferencia e intervalos de confianza}
    % !Mode:: "Tex:UTF-8"
%\section*{\fbox{\colorbox{Gris025}{{Sesión 16. Inferencia estadística.}}}}
%
%\subsection*{\fbox{\colorbox{Gris025}{{Muestreo.}}}}
%\subsection*{Fecha: Martes, 08/11/2011, 14h.}
%
%\noindent{\bf Atención:
%\begin{enumerate}
%\item Este fichero pdf lleva adjuntos los ficheros de datos necesarios.
%\end{enumerate}
%}
%
%%\subsection*{\fbox{1. Ejemplos preliminares }}
%\setcounter{tocdepth}{1}
%%\tableofcontents

\section{Distribución muestral}

\begin{itemize}
    \item En esta parte del curso, y después de nuestra incursión en el mundo de la Probabilidad, vamos a comenzar con la parte central de la Estadística, la {\sf Inferencia.} Recordemos que, en resumen, la inferencia Estadística consiste en la predicción de características de una población, a partir del estudio de una muestra tomada de esa población. Naturalmente, puesto que estamos haciendo Ciencia, queremos que nuestras predicciones sean precisas. Más concretamente, queremos poder decir cómo de fiables son nuestras predicciones. Y la Probabilidad nos va a permitir hacer esto, de manera que al final podemos hacer afirmaciones como, por ejemplo, {\em ``el valor que predecimos para la media de la población es $\mu$, {\sf y hay una probabilidad del 99\% de que esta predicción sea cierta}''}. Esta es la forma en la que las afirmaciones estadísticas se convierten en predicciones con validez y utilidad para la Ciencia. \\[3mm]
        Más adelante veremos como construir este tipo de afirmaciones en detalle. Pero para llegar hasta ellas, el primer paso es reflexionar sobre el proceso de obtención de las muestras. A su vez, en este proceso hay que distinguir dos aspectos:
        \begin{enumerate}
        \item un primer aspecto, de carácter más práctico: la propia forma en la que se obtiene una muestra. Este es uno de los pasos fundamentales para garantizar que los métodos producen resultados correctos, y que las predicciones de la Estadística son fiables. Vamos a dejar el análisis de este proceso de toma de muestras para más adelante, si tenemos tiempo para ocuparnos de él.
        \item el otro aspecto es más teórico. Tenemos que entender cómo es el conjunto de {\em todas las muestras posibles} que se pueden extraer, y que consecuencias estadísticas tienen las propiedades de ese conjunto de muestras. Es decir, tenemos que entender las {\sf distribuciones muestrales}. Esta parte todavía es esencialmente Teoría de Probabilidad, y es a lo que nos vamos a dedicar en este capítulo. Cuando la acabemos, habremos entrado, por fin, en el mundo de la Inferencia.
        \end{enumerate}

        \item De nuevo, vamos a empezar con un largo ejemplo, que nos va a ocupar casi toda la sesión de hoy. De hecho, usaremos el que es casi nuestro ejemplo canónico: vamos a lanzar dos dados.
        \begin{ejemplo}\label{ejem:DistribucionMediaMuestral}
        Consideremos la variable aleatoria $X(a,b)=a+b$ que representa la suma de puntos obtenidos al lanzar dos dados. Recordemos que el espacio muestral subyacente tiene 36 sucesos elementales equiprobables, que podemos representar como
        \[d_1=(1,1), d_2=(1,2),\ldots,d_6=(1,6),d_7=(2,1)\mbox{ y así hasta el }d_{36}=(6,6).\]
        Ya vimos (en el Ejemplo \ref{ejem:VariablesAleatoriasEliminanInformacion}, página \pageref{ejem:VariablesAleatoriasEliminanInformacion}) que la tabla de distribución de esta variable es:
        \begin{center}
            \begin{tabular}[t]{|c|c|c|c|c|c|c|c|c|c|c|c|}
                \hline
                \rule{0cm}{0.5cm}{\em Valor de la suma:}&2&3&4&5&6&7&8&9&10&11&12\\
                \hline
                \rule{0cm}{0.7cm}{\em Probabilidad de ese valor:}&$\dfrac{1}{36}$&$\dfrac{2}{36}$&$\dfrac{3}{36}$&$\dfrac{4}{36}$&$\dfrac{5}{36}$&$\dfrac{6}{36}$&$\dfrac{5}{36}$&$\dfrac{4}{36}$&$\dfrac{3}{36}$&$\dfrac{2}{36}$&$\dfrac{1}{36}$\\
                &&&&&&&&&&&\\
            \hline
            \end{tabular}
        \end{center}
        Y con esta tabla es fácil calcular la media y la desviación típica de $X$. Se obtiene $\mu_X=7$,  $\sigma_X=\dfrac{\sqrt{35}}{6}\approx 2.415$.

        Naturalmente, en un caso como este, en que el espacio muestral tiene sólo 36 elementos, y conocemos todos los detalles, el proceso de muestreo es innecesario. Pero precisamente por eso nos interesa este ejemplo, por ser tan sencillo. Vamos a usarlo como un modelo  ``de juguete'', como un laboratorio en el que aclarar nuestras ideas sobre las implicaciones del proceso de muestreo.

        Así pues, pensemos en muestras. En particular, vamos a pensar en muestras de tamaño 3. ¿Cuántas muestras distintas de tamaño 3 podemos obtener? Cada tirada de dos dados se identifica por un número del 1 al 36. Así que una muestra puede ser cualquier terna tal como
        \[(d_2,d_{15},d_{23})\]
        que corresponde a los tres resultados
        \[d_2=(1,2), d_{15}=(3,3), d_{23}=(4,5)\]
         de los dados. Pero, ¿qué sucede con, por ejemplo, $(d_4,d_4,d_4)$? ¿Es esta una muestra que debamos tomar en consideración? ¿Debemos admitir valores repetidos? Hay que andar con cuidado aquí: es importante, para empezar, que los tres valores de la muestra sean independientes entre sí. Y eso obliga a considerar extracción con reemplazamiento. ¿Qué queremos decir con esto? Es como si tuviéramos una urna con bolas marcadas del 1 al 36 y aleatoriamente extrajéramos tres. Si después de cada extracción no devolvemos la bola, ¡es evidente que los resultados de la segunda extracción no son independientes de los de la primera! Así que tenemos que devolver la bola a la caja cada vez. Y eso significa que sí, que tenemos que considerar muestras con repeticiones. Ahora ya podemos contestar a la pregunta de cuántas muestras de tres elementos hay. Son
        \[36^3=46656\]
        muestras distintas\footnote{Si no incluimos las repeticiones serían $\binom{36}{3}=7140$.} En este \textattachfile{Cap06-MuestreoSumaDosDados.R}{\textcolor{blue}{fichero de instrucciones R}} se construyen todas esas muestras y se comprueban todos los cálculos que vamos a realizar en este ejemplo. Las 46656 muestras de tamaño 3 van desde
        \[m_1=(d_1,d_1,d_1),m_2=(d_1,d_1,d_2),\ldots,\mbox{ pasando por }m_{1823}=(d_2,d_{15},d_{23}),\]\[\ldots\mbox{ hasta }m_{46656}=(d_{36},d_{36},d_{36}).\]
        Para cada una de ellas, hay tres valores de sumas (tres valores de $X$). Por ejemplo, para la muestra $(d_2,d_{15},d_{23})$ , que vimos antes, esos tres valores, que vamos a llamar $X_1, X_2$ y $X_3$,
        son ($X_1$ es la suma para la primera de las tres tiradas, $X_2$ y $X_3$ para la segunda y tercera):
        \[X_1=\hspace{-3mm}\underbrace{1+2=3}_{\mbox{\small valor de $X(d_2)$}},\quad X_2=\hspace{-3mm}\underbrace{3+3=6}_{\mbox{\small valor de $X(d_{15})$}},
        \quad X_3=\hspace{-3mm}\underbrace{4+5=9}_{\mbox{\small valor de $X(d_{23})$}}.\]
        Cada una de estas $X_1$, $X_2$ y  $X_3$ es una variable aleatoria, y cada una de ellas es una copia idéntica de $X$. El dato importante a retener es que tienen la misma media y varianza que $X$, y que son independientes (gracias al muestreo con reemplazamiento).

        Para ayudarte a seguir la discusión de este ejemplo, en la próxima página hay un diagrama que trata de aclarar los conceptos que van a ir apareciendo.
        \newpage
        \begin{center}\label{fig:DiagramaDistribucionMediaMuestral}
        \includegraphics[width=16cm]{2011-11-08-Diagrama.PNG}
        \end{center}
        {\LARGE La tabla de frecuencias de $\bar X$ está en la página \pageref{subsec:TablaFrecuenciasMediasMuestrales}}
        \newpage

        A continuación, puesto que tenemos tres valores ($X_1$, $X_2$ y  $X_3$), podemos hacer la media de estos tres:
        \[\mbox{media de $X$ en esa muestra }=\dfrac{X_1+X_2+X_3}{3}=\dfrac{3+6+9}{3}=\dfrac{18}{3}=6.\]
        Esta media es lo que vamos a llamar la {\sf media muestral}, que representamos por $\bar X$. Así pues
        \[\bar X=\dfrac{X_1+X_2+X_3}{3},\mbox{ y por tanto }\bar X(d_2,d_{15},d_{23})=6.\]
        ¡Es importante que no confundas los índices $(1,15,23)$ de la muestra $(d_2,d_{15},d_{23})$ --que son sólo eso, índices que identifican a la muestra concreta que estamos usando-- con $(3,6,9)$, que son los valores de las sumas para esas parejas de números! Asegúrate de entender esto antes de seguir adelante.

        Puesto que tenemos 46656 muestras distintas, podemos calcular 46656 de estas medias muestrales. No está mal, teniendo en cuenta que hemos empezado con 36 valores. El caso es que podemos ver esos 46656 valores como una nueva variable aleatoria, que llamaremos, naturalmente $\bar X$. Si agrupamos los valores de $\bar X$ por frecuencias se obtiene una tabla de frecuencias, que hemos incluido en la página \pageref{subsec:TablaFrecuenciasMediasMuestrales}. Y (por ejemplo, usando esa tabla),  esa nueva variable tiene una media, y una desviación típica, que representamos con $\mu_{\bar X}$ y $\sigma_{\bar X}$ respectivamente. Si no agrupamos los valores, $\mu_{\bar X}$ se calcularía así:
        \[\mu_{\bar X}=
        \dfrac{\overbrace{\bar X(d_1,d_1,d_1)+\bar X(d_1,d_1,d_2)+\cdots+\bar X(d_2,d_{15},d_{23})+\cdots+\bar X(d_{36},d_{36},d_{36})}^{\mbox{\small(Hay 46656 sumandos en el numerador)}} }{46656}=
        \]
        \[=\dfrac{\left(\dfrac{2+2+2}{3}\right)+\left(\dfrac{2+2+3}{3}\right)+\cdots+\left(\dfrac{3+6+9}{3}\right)+\cdots+\left(\dfrac{12+12+12}{3}\right)}{46656}.\]
        ¿Cuánto vale $\mu_{\bar X}$? Es decir, ¿cuánto vale la media de las medias muestrales? No hay sorpresas, es la media de la variable aleatoria original $X$:
        \[\mu_{\bar X}=\mu_X=7.\]
        Una vez calculada la media $\mu_{\bar X}$ de $\bar X$, podemos calcular su desviación típica $\sigma_{\bar X}$. Pero antes de ponernos manos a la obra queremos evitar desde el principio una posible confusión que a veces aparece, sobre el significado de $\sigma_{\bar X}$. El valor $\sigma_{\bar X}$ del que vamos a hablar es la raíz cuadrada de:
        \begin{equation}\label{eq:VarianzaDeMEdiasMuestrales}
        \begin{array}{c}
        \sigma^2_{\bar X}=
        \dfrac{\overbrace{
        (\bar X(d_1,d_1,d_1)\textcolor{red}{-\mu_{\bar X}})^{\textcolor{red}{2}}+
        (\bar X(d_1,d_1,d_2)\textcolor{red}{-\mu_{\bar X}})^{\textcolor{red}{2}}+
        \cdots+
        (\bar X(d_{36},d_{36},d_{36})\textcolor{red}{-\mu_{\bar X}})^{\textcolor{red}{2}}
        }^{\mbox{\small(Otra vez 46656 sumandos en el numerador)}} }{46656}=
        \\[8mm]
        =\dfrac{\left(\dfrac{2+2+2}{3}\textcolor{red}{-7}\right)^{\textcolor{red}{2}}+\left(\dfrac{2+2+3}{3}\textcolor{red}{-7}\right)^{\textcolor{red}{2}}+
        \cdots+\left(\dfrac{12+12+12}{3}\textcolor{red}{-7}\right)^{\textcolor{red}{2}}}{46656}.
        \end{array}
        \end{equation}
        Y \underline{no estamos hablando} de la varianza que se puede calcular para cada muestra individual. Para dejarlo más claro, igual que el cálculo
        \[\dfrac{3+6+9}{3}=\dfrac{18}{3}=6\]
        nos llevó a decir que
        \[\bar X(d_1,d_{15},d_{23})=6,\]
        ahora podríamos calcular:
        \[\dfrac{(3-6)^2+(6-6)^2+(9-6)^2}{3},\]
        y tendríamos 46656 de estos valores. Pero, insistimos, \underline{no es eso lo que estamos haciendo}, sino lo que indica la Ecuación \ref{eq:VarianzaDeMEdiasMuestrales}.



        ¿Qué crees que sucede con la desviación típica de las medias muestrales, es decir con $\sigma_{\bar X}$?  ¿Es más grande o más pequeña que la desviación típica de la variable original? Es decir, ¿están más dispersos los 36 valores originales de la suma de los dados, o los $46656$ valores que hemos obtenido a partir de las muestras?

        A bote pronto, es fácil pensar que, al aparecer tantísimas muestras, los valores se habrán dispersado. Todo lo contrario. Se obtiene:
        \[\sigma_{\bar X}=1.394\]
        qué es bastante más pequeño que el valor que ya obtuvimos de $\sigma_X=\frac{\sqrt{35}}{6}\approx 2.415$. De hecho, Si dividimos ambas desviaciones típicas y elevamos al cuadrado, tenemos
        \[\left(\dfrac{\sigma_{\bar X}}{\sigma_X}\right)^2=3.\]
        No es {\em aproximadamente} 3; es {\sf exactamente} $3$. ¿De dónde sale este $3$? Es fácil intuir que ese número es el tamaño de la muestra: estamos tomando muestras de tres valores de la variable original $X$.

        Al final de este ejemplo tan largo vamos a tratar de justificar teóricamente lo que ha sucedido, pero ahora queremos detenernos en otro tipo de explicación, más informal, pero quizá más intuitiva. Para ello, en la página final de este resumen hemos incluido la tabla de frecuencias de valores para la media muestral (sobre el conjunto de las $46656$ muestras). La Figura \ref{fig:distribucionXvsMediasMuestrales} muestra los correspondientes diagramas de barras.
        \begin{figure}[h]
        \begin{center}
        \caption{Distribución de $X$ (izda) y de $\bar X$ (dcha).\label{fig:distribucionXvsMediasMuestrales}}
        \includegraphics[width=16cm]{2011-11-08-DistribucionOriginalVsMuestral.png}
        \end{center}
        \end{figure}

       Y, como puede verse en la Figura, el efecto del muestreo ha sido una concentración mucho más acusada de la probabilidad sobre los valores centrales de la variable aleatoria. Una forma intuitiva de entender este fenómeno es pensando que en cada muestra de tres valores es más probable que haya dos cercanos a la media para compensar un posible valor alejado de la media. Es decir, que el proceso de muestreo matiza o lima las diferencias entre los distintos valores que toma la variable, empujándolos a todos hacia la media. Y si el fenómeno se observa incluso para un valor tan modesto como $n=3$ ¿qué pasará cuando, en otros ejemplos, se tomen muestras de, por ejemplo, $n=10000$? \qed

        \end{ejemplo}

\item Dejamos aquí el ejemplo, para retomar la última pregunta desde un punto de vista teórico, y explicar lo que sucede con la media muestral. La media muestral de tamaño $n$, es la suma de $n$ variables aleatorias {\sf independientes}, que corresponden a cada uno de los $n$ valores de la variable inicial $X$ que se han seleccionado para la muestra:
    \[\bar X=\dfrac{X_1+X_2+\cdots+X_n}{n}=\dfrac{X_1}{n}+\dfrac{X_2}{n}+\cdots+\dfrac{X_n}{n}\]
    Y las variables $X_i$ son copias de $X$, así que todas tienen media $\mu_X$ y desviación típica $\sigma_X$. Por lo tanto, en primer lugar,
    \[\mu_{\bar X}=E(\bar X)=\dfrac{E(X_1)+E(X_2)+\cdots+E(X_n)}{n}=\dfrac{n\cdot \mu_X}{n}=\mu_X.\]
    Y \underline{\sf puesto que son independientes}:
    \[\sigma^2_{\bar X}=\operatorname{Var}(\bar X)=\operatorname{Var}\left(\dfrac{X_1}{n}\right)+\operatorname{Var}\left(\dfrac{X_2}{n}\right)+\cdots+\operatorname{Var}\left(\dfrac{X_n}{n}\right)=
    n\cdot\dfrac{\operatorname{Var}(X)}{n^2}=\dfrac{\sigma^2_X}{n}.\]
    Y con esto se obtienen los resultados teóricos que explican lo que hemos constatado en el ejemplo anterior:\\[3mm]
        \fbox{\begin{minipage}{14cm}
        \begin{center}
        \vspace{2mm}
        {\bf La media muestral $\bar X$ y su distribución}
        \end{center}
        Sea $X$ una variable aleatoria cualquiera, con media $\mu_X$ y desviación típica $\sigma_X$.
       \begin{enumerate}
       \item Una {\sf muestra aleatoria de tamaño $n$} de $X$ es una lista $(X_1,X_2,\ldots,X_n)$ de $n$ copias independientes de la variable $X$.
       \item La {\sf media muestral} de $X$ es la variable aleatoria
       \[\bar X=\dfrac{X_1+\cdots+X_n}{n}\]
       \item Para la media y la desviación típica de la media muestral (muestras de tamaño $n$) de una variable aleatoria $X$ cualquiera se tiene:
        \[\mu_{\bar X}=\mu_X,\qquad \sigma_{\bar X}=\dfrac{\sigma_X}{\sqrt{n}}.\]
       \end{enumerate}
        \end{minipage}}\\[3mm]
        Y la última fórmula explica de donde proviene el $3$ que encontramos en el ejemplo al comparar $\sigma_{\bar X}$ con $\sigma_X$.

%\item No queremos dejar este tema sin atender a una confusión que surge con frecuencia al estudiar por primera vez la distribución muestral, y que tiene que ver con el significado del símbolo $\sigma_{\bar X}$. Este símbolo se refiere a la desviación típica de las medias muestrales.\\
%    A pesar  de que para cada muestra $m$ hemos obtenido tres números $X_1(m), X_2(m), X_3(m)$, \underline{no hemos calculado} la desviación típica de cada una de esas ternas. Por ejemplo, para la muestra
%    \[m_{522}=(d_1,d_{15},d_{23})\]
%    para la que
%    \[\]
%    cuya media muestral no hemos calculado

\end{itemize}



\subsection*{El Teorema Central del Límite, otra vez.}\label{subsec:teoremaCentralLimiteSegundaVersion}

\begin{itemize}
    \item En nuestra anterior sesión vimos que De Moivre había descubierto que, cuando $n$ se hace más y más grande, una variable de tipo binomial $B(n,p)$ se parece cada vez más a una variable de tipo normal $N(\mu, \sigma)$, para los valores correctos de $\mu$ y $\sigma$. Esta fue la primera versión del Teorema Central del Límite. Hoy, en el Ejemplo \ref{ejem:DistribucionMediaMuestral}, que acabamos de ver, hemos empezado con una variable aleatoria que, desde luego, no es binomial ({\em estamos sumando resultados, no midiendo éxitos de ningún tipo}). Y sin embargo, cuando se observa la parte izquierda de la Figura \ref{fig:distribucionXvsMediasMuestrales} de ese ejemplo, parece evidente que la distribución de la media muestral $\bar X$ se parece a la normal. ¡Y sólo estamos tomando $n=3$! Este fenómeno es otra nueva manifestación del Teorema Central del Límite, sobre el que ahora vamos a precisar más:\\[3mm]
       \fbox{\begin{minipage}{14cm}
       \begin{center}
       \vspace{2mm}
       {\bf TEOREMA CENTRAL DEL LÍMITE, SEGUNDA VERSIÓN}\\
       \end{center}
       Sea $X$ una variable aleatoria cualquiera, con media $\mu$ y desviación típica $\sigma$.
       \begin{enumerate}
       \item  {\sf Sea cual sea la forma de la distribución de $X$}, si se toman muestras de $X$ de tamaño $n$, entonces cuando $n$ se hace cada vez más grande la distribución de la media muestral
          $\bar X$ se aproxima cada vez más a la normal $N\left(\mu_X,\dfrac{\sigma_X}{\sqrt{n}}\right)$. En particular, para $n$ grande tenemos
          \[P(a\leq \bar X\leq b)\approx P\left(\dfrac{a-\mu_X}{\dfrac{\sigma_X}{\sqrt{n}}}\leq Z\leq \dfrac{b-\mu_X}{\dfrac{\sigma_X}{\sqrt{n}}}\right)\]
          siendo $Z$ de tipo  normal $N(0,1)$.
       \item Si además sabemos que la variable original es de tipo normal $N(\mu_X,\sigma_X)$, entonces, {\sf independientemente del tamaño $n$ de la muestra}, la media muestral también es normal, de tipo  $N\left(\mu_X,\dfrac{\sigma_X}{\sqrt{n}}\right)$.
       \end{enumerate}
       \end{minipage}}\\[3mm]
       \item En resumidas cuentas: para muestras grandes, las medias muestrales de todas las variables se comportan como variables normales, y si además empezamos con una variable normal, entonces el tamaño de la muestra es irrelevante. Esta última parte es muy importante cuando se tiene en cuenta la primera versión del Teorema Central del Límite que vimos. Aquella primera versión nos hizo pensar que las variables normales o muy aproximadamente normales debían ser extremadamente frecuentes en la naturaleza. Y esta versión nos asegura que el comportamiento en el muestreo de esas variables normales es especialmente bueno.

    \item  Esas son las buenas noticias. Las malas noticias son que, si no podemos suponer que la variable $X$ sea normal, entonces se necesitaría una muestra muy grande. Y en general, no tendremos muestras grandes. Además, para usar este resultado, necesitaríamos conocer la desviación típica de la población. Dejamos pendiente para la próxima sección este problema, junto con el problema de cómo obtener las muestras que mencionábamos al principio.


\end{itemize}

%
%\section*{Tareas asignadas para esta sesión.}
%No hay tareas asignadas para esta sesión.


%\newpage
\subsection*{Tabla de frecuencias de medias muestrales para el ejemplo de lanzamiento de dos dados}\label{subsec:TablaFrecuenciasMediasMuestrales}

        \begin{center}
        \begin{tabular}{|c|c|}
        \hline
        {\bf Valor de $\bar X$}\rule{0cm}{0.7cm}&{\bf Frecuencia}\\
        \hline
        2& 1 \\ \hline
        2$+\frac{1}{3}$\rule{0cm}{0.35cm}&6 \\ \hline
        2$+\frac{2}{3}$\rule{0cm}{0.35cm}& 21 \\ \hline
        3&56 \\ \hline
        3$+\frac{1}{3}$\rule{0cm}{0.35cm}&126 \\ \hline
        3$+\frac{2}{3}$\rule{0cm}{0.35cm}&252 \\ \hline
        4& 456 \\ \hline
        4$+\frac{1}{3}$\rule{0cm}{0.35cm}&756 \\ \hline
        4$+\frac{2}{3}$\rule{0cm}{0.35cm}& 1161 \\ \hline
        5&1666 \\ \hline
        5$+\frac{1}{3}$\rule{0cm}{0.35cm}&2247 \\ \hline
        5$+\frac{2}{3}$\rule{0cm}{0.35cm}& 2856 \\ \hline
        6&3431 \\ \hline
        6$+\frac{1}{3}$\rule{0cm}{0.35cm}&3906 \\ \hline
        6$+\frac{2}{3}$\rule{0cm}{0.35cm}& 4221 \\ \hline
        7&4332 \\ \hline
        7$+\frac{1}{3}$\rule{0cm}{0.35cm}&4221 \\ \hline
        7$+\frac{2}{3}$\rule{0cm}{0.35cm}& 3906 \\ \hline
        8&3431 \\ \hline
        8$+\frac{1}{3}$\rule{0cm}{0.35cm}& 2856 \\ \hline
        8$+\frac{2}{3}$\rule{0cm}{0.35cm}& 2247 \\ \hline
        9&1666 \\ \hline
        9$+\frac{1}{3}$\rule{0cm}{0.35cm}& 1161 \\ \hline
        9$+\frac{2}{3}$\rule{0cm}{0.35cm}&756 \\ \hline
        10&456 \\ \hline
        10$+\frac{1}{3}$\rule{0cm}{0.35cm}&252 \\ \hline
        10$+\frac{2}{3}$\rule{0cm}{0.35cm}&126 \\ \hline
        11& 56 \\ \hline
        11$+\frac{1}{3}$\rule{0cm}{0.35cm}& 21 \\ \hline
        11$+\frac{2}{3}$\rule{0cm}{0.35cm}&6 \\ \hline
        12&1 \\ \hline
        \end{tabular}
        \end{center}


%\section*{\fbox{\colorbox{Gris025}{{Sesión 17. Inferencia estadística.}}}}
%
%\subsection*{\fbox{\colorbox{Gris025}{{Intervalos de confianza para la media.}}}}
%\subsection*{Fecha: Viernes, 11/11/2011, 14h.}
%
%\noindent{\bf Atención:
%\begin{enumerate}
%\item Este fichero pdf lleva adjuntos los ficheros de datos necesarios.
%\end{enumerate}
%}
%
%%\subsection*{\fbox{1. Ejemplos preliminares }}
%\setcounter{tocdepth}{1}
%%\tableofcontents

\section{Intervalos de confianza para la media en poblaciones normales}

\begin{itemize}
    \item  Nuestro objetivo es, como decíamos al principio del resumen de la última sesión es usar el valor de $\bar X$ obtenido en una muestra, para  poder llegar a una predicción sobre $\mu_X$. El tipo de predicciónes que vamos a hacer sigue el esquema de esta frase:
       \begin{center}
       {\sf Hay una probabilidad del 90\% de que el valor de $\mu_X$ esté dentro del intervalo $(a,b)$.}
       \end{center}
       El intervalo $(a,b)$ será un {\sf intervalo de confianza} para $\mu_X$, y el porcentaje del $90\%$ es el {\sf nivel de confianza} que deseamos. Los niveles de confianza más frecuentes son $90\%$, $95\%$ y $99\%$. Hablando en general, el nivel de confianza está relacionado con la anchura del intervalo (cuánto mayor sea el margen de error, más segura es la predicción. ¡Pero también es menos precisa!) y con el tamaño de la muestra (cuántos más elementos incluya la muestra, mayor será la precisión de la predicción). Es decir, {\em conseguir un intervalo muy pequeño, con una probabilidad muy alta, requiere más trabajo (por ejemplo, en forma de muestras más grandes)}.

    \item ¿Cómo se construyen estos intervalos? Vamos a empezar recordando el último resultado de la anterior sesión, que es una de las formas que adopta el Teorema Central del Límite:\\[3mm]
       \fbox{\begin{minipage}{14cm}
       \begin{center}
       \vspace{2mm}
       {\bf TEOREMA CENTRAL DEL LÍMITE, POBLACIÓN NORMAL}\\
       \end{center}
       Sea $X$ una variable aleatoria normal, de tipo $N(\mu_X,\sigma_X)$. Entonces, {\sf independientemente del tamaño $n$ de la muestra}, la media muestral $\bar X$ también es normal, de tipo  $N\left(\mu_X,\dfrac{\sigma_X}{\sqrt{n}}\right)$.\\ Y por tanto, si definimos (tipificamos):
          \[Z=\dfrac{\bar X-\mu_X}{\dfrac{\sigma_X}{\sqrt{n}}},\]
          la variable $Z$ es de tipo normal estándar $N(0,1)$.
       \end{minipage}}\\[3mm]
       Y lo bueno de este resultado es que estamos muy bien preparados para responder a cualquier tipo de preguntas sobre la variable normal estándar $Z$. Antes de seguir con los intervalos de confianza, vamos a
       detenernos para dejar claros estos detalles referentes a la normal.

       \end{itemize}

       \subsection*{Valores críticos de la distribución normal estándar}

       \begin{itemize}


       \item Sea por lo tanto $Z$ una variable con distribución normal estándar $N(0,1)$. Si nos preguntamos por ejemplo, cuál es el valor de
       \[P(-K\leq Z\leq K)\]
       sabemos contestar con precisión. Usando tablas, o mediante los comandos de R ({\tt pnorm}) o de una hoja de cálculo como Calc ({\tt DISTRIB.NORM}) que permiten responder a este tipo de preguntas.  Pero además, usando esos mismos recursos, podemos responder a la pregunta contraria. Es decir, {\em fijando a priori un valor de la probabilidad}, por ejemplo $0.90$, podemos responder a la pregunta: ``¿cuál es el valor de $K$ para el que se cumple la siguiente desigualdad?''
       \begin{equation}\label{eq:inversaProbabilidadesNormales}
       P(-K\leq Z\leq K)=0.90
       \end{equation}
       Es decir, que estamos eligiendo $K$ de forma que el área que aparece en la figura sea $0.90$:
       \begin{center}
       \includegraphics[width=16cm]{2011-11-08-NormalEstandarArea90.png}
       \end{center}
        En las clases de prácticas estamos aprendiendo a usar R o una hoja de cálculo como Calc para contestar a estas preguntas. En R, por ejemplo, el comando {\tt qnorm(p)} nos devuelve el valor $z$ para el que se cumple
        \[P(Z\leq z)\leq p\]
        ¿Cómo usamos esto para calcular el valor $K$ en la Ecuación \ref{eq:inversaProbabilidadesNormales}? Empezamos por fijarnos en los valores de las {\em colas a izquierda y derecha} de la normal estándar, como se muestra en la figura:
       \begin{center}
       \includegraphics[width=16cm]{2011-11-08-NormalEstandarArea90-2.png}
       \end{center}
       Y como el área de la cola de la derecha es $0.05$, está claro que debemos buscar el valor $z$ tal que
        \[P(Z\leq z)\leq (1-0.05)=0.95\]
        Si introducimos en R el comando {\tt qnorm(0.95)} obtendremos $z\approx 1.645$. En Calc obtendríamos ese mismo valor usando {\tt DISTR.NORM.ESTAND.INV(0,95)}.  ¿Qué relación hay entre este cálculo y los intervalos de confianza? Está claro: el valor $0.90$ corresponde a uno de los posibles niveles de confianza que podemos fijar para el intervalo. Así que vamos a describir en detalle el proceso, desde que fijamos el nivel de confianza deseado.

        La notación tradicional en Estadística llama $1-\alpha$ al nivel de confianza. De esa forma, si el nivel de confianza es $1-\alpha=0.90$, entonces $\alpha=0.10$ representa la suma de las dos colas, a izquierda y derecha. Y en realidad, lo que necesitamos, como en el ejemplo de antes, es $\frac{\alpha}{2}=0.05$, para saber el área de una de las colas. Una vez que lo tenemos queremos saber cual es el valor $z$ que cumple
        \[P(Z\leq z)\leq \left(1-\dfrac{\alpha}{2}\right).\]
        Y ese valor es el que llamaremos {\sf valor crítico} para el nivel de confianza $\alpha$, y que se representa con el símbolo $z_{\alpha/2}$ (por ejemplo, $z_{0.05}\approx 1.645$).

        Al principio todo este asunto del $\alpha$, el $1-\alpha$ y el $\alpha/2$ resulta un poco confuso. Pero practicando un poco es fácil acostumbrarse. El esquema es este:
        \begin{center}
        \begin{tabular}{|c|c|c|c|c|c|c|}
        \hline
        Nivel de confianza:&&&&&Aquí usamos tablas, R, etc.&\\
        \hline
        $1-\alpha$\rule{0cm}{1cm}&$\longrightarrow$&$\alpha$&$\longrightarrow$&$\frac{\alpha}{2}$&$\longrightarrow$&$z_{\alpha/2}$\\[3mm]
        \hline
        0.90\rule{0cm}{1cm}&$\longrightarrow$&$0.10$&$\longrightarrow$&$0.05$&$\longrightarrow$&$1.645$\\[3mm]
        \hline
        \end{tabular}
        \end{center}

        Conviene familiarizarse con los valores críticos $z_{\alpha}{2}$ correspondientes a los niveles de confianza más utilizados:

        \begin{center}\label{tabla:valoresCriticosNormalEstandar}
        \begin{tabular}{|c|c|c|c|c|}
        \hline
        {\bf Nivel de confianza:}\rule{0cm}{0.5cm}&0.80&0.90&0.95&0.99\\
        \hline
        $z_{\alpha/2}$\rule{0cm}{0.5cm}&$1.28$&$1.64$&$1.96$&2.58\\[3mm]
        \hline
        \end{tabular}
        \end{center}

        Y la forma en que los vamos a usar es esta.\\[3mm]
       \fbox{\begin{minipage}{14cm}
       \begin{center}
       \vspace{2mm}
       {\bf CÓMO SE USAN LOS VALORES CRÍTICOS}\\
       \end{center}
       Si $Z$ es una variable normal estándar (de tipo $N(0,1)$), entonces para que sea:
        \[P(-K\leq Z\leq K)=1-\alpha\]
        debe ser
        \[-z_{\alpha/2}\leq Z\leq z_{\alpha/2}.\]
       \end{minipage}}\\[3mm]


       \end{itemize}

       \subsection*{Construcción del intervalo de confianza}

       \begin{itemize}

       \item Una vez que entendemos el papel que juegan los valores críticos, el plan para construir el         intervalo de confianza $(a,b)$ es muy sencillo de entender. Queremos llegar a un intervalo de la forma
       \[a\leq \mu_X\leq b\]
       de manera que podamos afirmar que
       \[P(a\leq \mu_{X}\leq b)=0.90\]
       Para conseguirlo vamos a utilizar la variable tipificada $Z$ construida a partir de la media muestral, es decir
       \[Z=\dfrac{\bar X-\mu_X}{\dfrac{\sigma_X}{\sqrt{n}}}\]
       que, como sabemos es de tipo normal estándar $N(0,1)$.

       Entonces, sabemos que para conseguir que sea
       \[P(-K\leq Z\leq K)=1-\alpha\]
       \[\mbox{debemos tomar: }-z_{\alpha/2}\leq Z\leq z_{\alpha/2}.\]
       Pero si traducimos esto a la media muestral $\bar X$, lo que estamos diciendo es que para que sea
       \[P\left(-K\leq \dfrac{\bar X-\mu_X}{\dfrac{\sigma_X}{\sqrt{n}}}\leq K\right)=1-\alpha\]
       tiene que ser:
       \[-z_{\alpha/2}\leq \dfrac{\bar X-\mu_X}{\dfrac{\sigma_X}{\sqrt{n}}} \leq z_{\alpha/2}.\]
       Ya casi lo tenemos. Ahora vamos a despejar $\mu$. Primer paso:
       \[-z_{\alpha/2}\dfrac{\sigma_X}{\sqrt{n}}\leq {\bar X-\mu_X} \leq z_{\alpha/2}\dfrac{\sigma_X}{\sqrt{n}}.\]
       A continuación restamos $\bar X$ en todos los términos:
       \[-\bar X-z_{\alpha/2}\dfrac{\sigma_X}{\sqrt{n}}\leq -\mu_X \leq -\bar X+z_{\alpha/2}\dfrac{\sigma_X}{\sqrt{n}}.\]
       Y finalmente cambiamos el signo de toda la desigualdad, con lo que las desigualdades se invierten (tras cambiar de signo, el término de la izquierda pasa a la derecha y viceversa). El resultado es:
       \[\bar X-z_{\alpha/2}\dfrac{\sigma_X}{\sqrt{n}}\leq \mu_X \leq \bar X+z_{\alpha/2}\dfrac{\sigma_X}{\sqrt{n}}.\]
       y este es, como queríamos, el intervalo \[a\leq \mu_X\leq b.\]
       En conclusión:\\[3mm]
       \fbox{\begin{minipage}{14cm}
       \begin{center}
       \vspace{2mm}
       {\bf Intervalo de confianza (nivel $(1-\alpha$)) para la media $\mu$.}\\{\bf Población normal, con desviación típica conocida.}\\
       \end{center}
       Sea $X$ una variable aleatoria normal, cuya desviación típica $\sigma_X)$ se conoce. Si consideramos muestras de tamaño $n$, entonces el intervalo de confianza al nivel $(1-\alpha)$  para la media $\mu_X$ es:
       \[\bar X-z_{\alpha/2}\dfrac{\sigma_X}{\sqrt{n}}\leq \mu_X \leq \bar X+z_{\alpha/2}\dfrac{\sigma_X}{\sqrt{n}}.\]
       que a veces escribiremos:
       \[\mu_X =\bar X \pm z_{\alpha/2}\dfrac{\sigma_X}{\sqrt{n}}.\]
       \end{minipage}}\\[3mm]

       \item {\sf Sobre el cálculo de intervalos:} El procedimiento para obtener estos intervalos de confianza es por lo tanto puramente mecánico, a partir de estos datos:
       \begin{enumerate}
       \item la media muestral $\bar X$,
       \item la desviación típica de la población, $\sigma_X$,
       \item el tamaño de la muestra $n$
       \item y el nivel de confianza deseado $1-\alpha$ (en la forma $0.90$, $0.95$, etc.)
       \end{enumerate}
       Y de hecho, en esta \textattachfile{Cap06-IntervaloConfianzaMediaPoblacionNormalVconocida.ods}{\textcolor{blue}{hoja de cálculo}}, y en este \textattachfile{Cap06-IntervaloConfianzaMediaPoblacionNormalVconocida.R}{\textcolor{blue}{fichero de instrucciones R}} basta con introducir los datos para obtener los extremos del intervalo de confianza buscado. Vamos a ver un ejemplo:
       \begin{ejemplo}
       Una muestra aleatoria de 50 individuos de una población normal con varianza conocida, e igual a $16$, presenta una media muestral de $320$. Calcular un intervalo de confianza al $99\%$ para la media de la población.\\
       Usando cualquiera de las dos herramientas de cálculo (o simplemente mirando la tabla de la página \pageref{tabla:valoresCriticosNormalEstandar}) comprobamos que el valor crítico correspondiente a este nivel de confianza es:
       \[z_{\alpha/2}=2.58\]
       Calculamos la anchura del intervalo:
       \[z_{\alpha/2}\dfrac{\sigma_X}{\sqrt{n}}=2.58\dfrac{4}{\sqrt{50}}\approx 1.46\]
       Y por lo tanto el intervalo de confianza buscado es:
       \[318.54\leq \mu_X\leq 321.46.\]
       o, escribiéndolo de otra forma:
       \[\mu=320\pm 1.46\]
       \quad\qed
       \end{ejemplo}

       \end{itemize}


       \section{Desviación típica muestral (y el misterio del $n-1$).}

       \begin{itemize}

       \item Se supone que estamos tratando de calcular la media $\mu_X$ de la población a partir del valor de $\bar X$ en una muestra, ¡porque no conocemos $\mu_X$, claro! Y sin embargo {\em ¿damos por conocida la desviación típica $\sigma_X$?} A primera vista, resulta al menos chocante dar por conocida la desviación típica, cuando ni siquiera conocemos la media.

           Una primera respuesta es que, en algunos contextos (por ejemplo, en los procesos de control de la calidad en fabricación industrial), la desviación típica de la población puede en ocasiones considerarse conocida. Pero es verdad que eso no siempre es así. ¿Y entonces? ¿Qué hacemos en esos otros casos en que $\sigma_X$ no es conocido? Pues lo que hacemos es utilizar un sustituto de la desviación típica de la población, pero calculado a partir de la muestra. Se trata de un viejo conocido (con el que nos encontramos en la sesión del 30/09):\\[3mm]
           \fbox{\begin{minipage}{14cm}
           \begin{center}
           \vspace{2mm}
           {\bf Varianza y desviación típica muestral.}\\
           \end{center}
           Dada una muestra de la variable $X$ de tamaño $n$, formada por los valores $x_1,\ldots,x_n$
           definimos la {\sf varianza muestral} (a veces se llama {\sf cuasivarianza muestral}) mediante:
           \[s^2=\dfrac{\displaystyle\sum_{i=1}^n(x_i-\bar x)^2}{\textcolor{red}{\bf\Large n-1}}.\]
           En el caso de valores agrupados por frecuencias, la fórmula es:
           \[s^2=\dfrac{\displaystyle\sum_{i=1}^k{\bf f_i\cdot}(x_i-\bar x)^2}{\bf\displaystyle\left(\sum_{i=1}^k f_i\right)\textcolor{red}{-1}}.
           \]
           Y la {\sf desviación típica muestral} es simplemente la raíz cuadrada $s$ de la varianza muestral.
           \end{minipage}}\\[3mm]

       \item Cuando $n$ es suficientemente grande, el uso del valor de $s$ como sustituto de $\sigma$ se puede justificar teóricamente por completo. ¿Cómo de grande debe ser $n$? Se suele utilizar
            \[\fbox{\textcolor{red}{\boldmath\large $n>30$}}\]
           como criterio para distinguir las muestras {\em grandes} de las pequeñas. De paso, podemos aprovechar la oportunidad para explicar un poco más este asunto del $n-1$ en el denominador. Cuando se estudia teóricamente esa aproximación de $\sigma$ por $s$, se descubre que, si se utiliza $n$ en el denominador, las aproximaciones a $\sigma$ resultan ser {\em sistemáticamente más pequeñas} de lo que deberían para que la aproximación funcione. Técnicamente, esto se resume diciendo que la fórmula con $n$:
           \[\dfrac{\displaystyle\sum_{i=1}^n(x_i-\bar x)^2}{{n}}\]
           es {\sf sesgada}, mientras que la fórmula de $s^2_X$ (que usa $n-1$) es {\sf insesgada}.

       \item Por cierto, en la anterior sesión, en el diagrama de la página \pageref{fig:DiagramaDistribucionMediaMuestral} (ver también la discusión en la página \pageref{eq:VarianzaDeMEdiasMuestrales}), decíamos {\em ``no estamos calculando cosas como''}. Bueno, pues ahora sí las estamos calculando. Precisamente, los números $s$ que estamos calculando son las desviaciones típicas calculadas para cada muestra, a partir de la media $\bar x$ de esa muestra concreta ({\em ¡salvo que, para que las cosas funcionen, hay que dividir por $n-1$ en lugar de por $n$, claro!}).
       \end{itemize}

       \subsection*{Intervalos de confianza para $\mu$ con muestra grande y varianza desconocida.}

       \begin{itemize}


       \item Una vez que decidimos utilizar $s$ como sustituto de $\sigma$, el cálculo del intervalo de confianza procede exactamente como antes:\\[3mm]
       \fbox{\begin{minipage}{14cm}
       \begin{center}
       \vspace{2mm}
       {\bf Intervalo de confianza (nivel $(1-\alpha$)) para la media $\mu$.}\\
       {\bf Población normal, con varianza desconocida, pero muestra grande $n>30$.}\\
       \end{center}
       Sea $X$ una variable aleatoria normal, Si consideramos muestras de tamaño $n$, entonces el intervalo de confianza al nivel $(1-\alpha)$  para la media $\mu_X$ es:
       \[\bar X-z_{\alpha/2}\dfrac{\textcolor{red}{\boldmath\mbox{\large $s$}}}{\sqrt{n}}\leq \mu_X \leq \bar X+z_{\alpha/2}\dfrac{\textcolor{red}{\boldmath\mbox{\large $s$}}}{\sqrt{n}}.\]
       que también escribiremos:
       \[\mu_X =\bar X \pm z_{\alpha/2}\dfrac{\textcolor{red}{\boldmath\mbox{\large $s$}}}{\sqrt{n}}.\]
       \end{minipage}}\\[3mm]
       Y aquí están la \textattachfile{Cap06-IntervaloConfianzaMediaPoblacionNormalVDesconocidaMuestraGrande.ods}{\textcolor{blue}{hoja de cálculo}}, y el \textattachfile{Cap06-IntervaloConfianzaMediaPoblacionNormalVDesconocidaMuestraGrande.R}{\textcolor{blue}{fichero de instrucciones R}} correspondientes a este caso.

       \item Al llegar a este punto, y reflexionar lo que hemos obtenido en esta sesión, comprobaremos que, como sucede siempre en la Ciencia, cada nueva respuesta conduce a más preguntas. ¿Qué sucede si el tamaño de la muestra es más pequeño, si $n<30$? Y ¿qué sucede si no sabemos si la población sigue una distribución normal? Veremos las respuestas a estas preguntas en la próxima sesión.

       \end{itemize}

%\section*{Tareas asignadas para esta sesión.}
%Este fin de semana habrá una tarea asignada en Moodle. Se os avisará a través de un mensaje en el foro cuando esté disponible.
%
%%\newpage
%
%
%\section*{\fbox{\colorbox{Gris025}{{Sesión 18. Inferencia estadística.}}}}
%
%\subsection*{\fbox{\colorbox{Gris025}{{Más sobre intervalos de confianza.}}}}
%\subsection*{Fecha: Martes, 15/11/2011, 14h.}
%
%\noindent{\bf Atención:
%\begin{enumerate}
%\item Este fichero pdf lleva adjuntos los ficheros de datos necesarios.
%\end{enumerate}
%}
%
%%\subsection*{\fbox{1. Ejemplos preliminares }}
%\setcounter{tocdepth}{1}
%%\tableofcontents


\section{Distribución $t$ de Student.}

\begin{itemize}

    \item Al final de la última sesión habíamos establecido un procedimiento para encontrar un intervalo de confianza para la media de una población normal, a partir de los valores de una muestra de tamaño $n$ de dicha población. En el caso de que la varianza de la población fuera desconocida, pero que el tamaño de la muestra fuera grande (mayor que $30$), tomábamos el intervalo
        \begin{equation}\label{eq:IntervaloConfianzaMediaConVarianzaDesconocida}
        \bar X-z_{\alpha/2}\dfrac{s}{\sqrt{n}}\leq \mu_X \leq \bar X+z_{\alpha/2}\dfrac{s}{\sqrt{n}},
        \end{equation}
        donde $s$ es la desviación típica muestral, es decir la raíz cuadrada de la varianza muestral:
        \[s^2=\dfrac{\displaystyle\sum_{i=1}^n(x_i-\bar x)^2}{{n-1}}.\]

    \item Estos resultados estaban basados en lo que el Teorema Central del Límite dice sobre la distribución de estas dos variables aleatorias:
        \[\begin{cases}
        Y_1=\dfrac{\bar X-\mu_X}{\dfrac{\sigma_X}{\sqrt{n}}}&\mbox{ para el caso de varianza $\sigma_X^2$ conocida, y}\\[8mm]
        \quad\\
        Y_2=\dfrac{\bar X-\mu_X}{\dfrac{s}{\sqrt{n}}}&\mbox{ para el caso de $\sigma_X^2$ desconocida.}
        \end{cases}
        \]
        El Teorema Central del Límite nos asegura que, en el primer caso, la variable $Y_1$ es {\em exactamente} una normal estándar $N(0,1)$. Y en el segundo caso, la variable $Y_2$ \textcolor{red}{\em para muestras grandes} ($n>30$), se puede aproximar muy bien por la normal estándar $N(0,1)$.

    \item A la vista de estos resultados, terminábamos la anterior sesión preguntándonos: ¿qué sucede si el tamaño de la muestra es más pequeño, si $n<30$? Y ¿qué sucede si no sabemos si la población sigue una distribución normal? Empecemos por la segunda pregunta, que tiene una respuesta relativamente fácil:
            \begin{enumerate}
            \item Aunque no sepamos si la población original tiene una distribución normal, si el tamaño de la muestra es suficientemente grande, el Teorema asegura que el propio proceso de muestreo se encargará de que $Y_2$ tenga un comportamiento muy parecido al de la normal. Así que también en este caso podemos utilizar la Ecuación \ref{eq:IntervaloConfianzaMediaConVarianzaDesconocida} para el intervalo de confianza.
            \item {\sf Si la muestra no es grande y no sabemos que la población sea normal, entonces \underline{no debemos} usar los métodos sencillos que estamos viendo.} Hablaremos de estos casos más adelante, cuando hablemos de Inferencia No Paramétrica.
            \end{enumerate}

    \item Esto nos deja con un caso pendiente. Si se cumplen estas condiciones:
    \[\begin{cases}
     \mbox{(1) la población original es normal (o al menos, aproximadamente normal)}\\[3mm]
     \mbox{(2) pero desconocemos la varianza de la población $\sigma^2_X$}\\[3mm]
     \mbox{(3) y el tamaño de la muestra es pequeño,}
     \end{cases}
     \]
     entonces ¿qué hacemos? Es importante entender que no podemos usar la Ecuación \ref{eq:IntervaloConfianzaMediaConVarianzaDesconocida}, porque {\bf para $n$ pequeño, la variable $Y_2$ (la que usa $s$) no sigue una distribución normal estándar}. La variable $Y_1$ sí que sigue una normal estándar, pero eso no nos sirve de gran cosa porque ignoramos el valor de $\sigma^2_X$.

     Afortunadamente, alguien buscó la respuesta para nosotros, estudiando el comportamiento de la variable $Y_2$ para $n$ pequeño.\\[3mm]
     \fbox{\begin{minipage}{14cm}
       \begin{center}
       \vspace{2mm}
       {\bf DISTRIBUCIÓN $t$ DE STUDENT}\\
       \end{center}
       Sea $X$ una variable aleatoria normal, de tipo $N(\mu_X,\sigma_X)$, y sea $\bar X$ la media muestral de $X$ en muestras de tamaño $n$. Entonces, la distribución de la variable aleatoria
        \[\dfrac{\bar X-\mu_X}{\dfrac{s}{\sqrt{n}}},\]
        recibe el nombre de {\sf distribución $t$ de Student con $k=n-1$ grados de libertad.}
       \end{minipage}}\\[3mm]
       Esta distribución fue estudiada por \link{http://en.wikipedia.org/wiki/William_Sealy_Gosset}{William S. Gosset}, que trabajaba para la fábrica de cerveza Guinness y que firmaba sus trabajos científicos bajo el pseudónimo de Student. Entre otras cosas, Student obtuvo la función de densidad de esta variable aleatoria continua, que para $k=n-1$ grados de libertad es:
       \[f(x)=\dfrac{1}{\sqrt{k}\cdot\beta(\frac{1}{2},\frac{k}{2})}\left(1+\dfrac{x^2}{k}\right)^{-\frac{k+1}{2}}.\]
       (Aquí $\beta$ es la \link{http://en.wikipedia.org/wiki/Beta_function}{función beta}, una función que se usa a menudo en matemáticas y que recuerda un poco a los números combinatorios.)

       {\em ¡No hay ninguna necesidad de que te aprendas esta función de densidad!} La escribimos aquí sólo para recordarte que la distribución de Student, como cualquier variable aleatoria continua, viene caracterizada por su función de densidad. Es mucho más interesante comparar los gráficos de la función de densidad de la normal estándar y la $t$ de Student para distintos valores de $k$. Eso es lo que se ha hecho en este \textattachfile{Cap06_StudentVsNormalEstandar.html}{\textcolor{blue}{documento html}} (se abre en el navegador, requiere Java). Como puedes comprobar, la distribución $t$ de Student tiene una forma de campana que recuerda a la normal, pero es más abierta. A medida que el valor de $k$ aumenta, no obstante, la $t$ se parece cada vez más a la normal estándar, de manera que para $k>30$ son esencialmente iguales.
    \end{itemize}

    \subsection*{Cálculo de valores asociados a la $t$ de Student con el ordenador}

    \begin{itemize}
     \item Aparte de estudiar sus propiedades, Student obtuvo tablas de valores de esta distribución, similares a las de la normal, para distintos valores de $n-1$ (lo que llamaremos los {\sf grados de libertad} de la muestra).  Tabla como esas aparecen como apéndices en la mayoría de los libros de estadística, y en particular en los que incluye la bibliografía de la asignatura.

     \item Por supuesto, también pueden obtenerse valores de la distribución $t$ de Student con la hoja de cálculo y con R. En Calc o Excel, por ejemplo, las funciones {\tt DISTR.T} y {\tt DISTR.T.INV} permiten cálculos de valores de probabilidad, similares a los que ya vimos para la distribución normal. Pero es importante entender las diferencias. El resultado de un comando como:
         \[\mbox{\tt DISTR.T(a;b;c)}\]
         es
         \[P(X>a)\quad\mbox{¡ATENCIÓN: la desigualdad es $>$, ver la siguiente figura!}\]
         donde $X$ es una variable aleatoria que sigue una distribución $t$ de Student con $b$ grados de libertad; si se usa el valor $c=1$ se obtiene el área de la cola derecha, mientras que si se usa $c=2$ se obtiene el área de ambas colas (es decir, se calcula $P(X>a)+P(X<-a)$, o lo que es lo mismo $P(|X|>a)$). Por ejemplo, para calcular
         \[P(X>\frac{1}{2}),\mbox{ con 3 grados de libertad,}\]
         se utiliza
         \[\mbox{\tt DISTR.T(0,5;3;1)=0,3257239824}\]
         que corresponde a esta figura
         \begin{center}
         \includegraphics[width=15cm]{2011-11-15-AreaColaDerechaTStudent.png}
         \end{center}
         Y si lo que queremos es calcular $P(X<\frac{1}{2})$, debemos utilizar
         \[\mbox{\tt 1-DISTR.T(0,5;3;1)=0,6742760176}\]

         {\sf Así que es importante tener en cuenta que el comportamiento de {\tt DISTR.NORM} y el de {\tt DISTR.T} son bastante distintos.} En general, siempre que vayamos a usar una función de la hoja de cálculo es necesario comprobar previamente que hemos entendido bien su funcionamiento, consultando la \link{http://office.microsoft.com/es-ar/excel-help/distr-t-funcion-distr-t-HP010335662.aspx}{documentación del programa}.

         Para calcular valores inversos y, por lo tanto, los valores críticos necesarios para los intervalos de confianza, la hoja de cálculo incluye la función {\tt DISTR.T.INV}. Y aquí, de nuevo, hay que ir con cuidado, porque el resultado de
         \[\mbox{\tt DISTR.T.INV(a;b)}\]
         es el valor $x$ tal que
         \[P(X>x)+P(X<-x)=a,\]
         para una variable aleatoria $X$ que sigue una distribución $t$ de Student con $b$ grados de libertad. Es decir, la función {\tt DISTR.T.INV} {\sf siempre usa el área de las dos colas, derecha e izquierda}. Por ejemplo,
         con 3 grados de libertad, para calcular el $x$ tal que
         \[\underbrace{P(X<-x)}_{\mbox{\tiny cola izda.}}+\underbrace{P(X>x)}_{\mbox{\tiny cola dcha.}}=\frac{1}{2},\]
         usamos
         \[\mbox{\tt DISTR.T.INV(0,5;3)=0,7648923284},\]
         que corresponde a esta figura
         \begin{center}
         \includegraphics[width=15cm]{2011-11-15-InversaTStudent.png}
         \end{center}
         A pesar de la posible confusión que genera, la ventaja de esto es que, si lo que quiero es encontrar el valor crítico para construir un intervalo de confianza, entonces a partir del nivel de confianza $1-\alpha$, precisamente lo que necesitamos saber es cuál es el valor de $x$ tal que
         \[P(X>x)+P(X<-x)=\alpha\]
         y eso, directamente, es lo que nos da {\tt DISTR.T.INV}. Por ejemplo, con $3$ grados de libertad, el valor crítico para un intervalo de confianza al $95\%$ (es decir $\alpha=0.05$) para la media se obtiene con
         \[\mbox{\tt DISTR.T.INV(0,05;3)=2,3533634348},\]


         \item Con R, en cambio, las cosas son distintas, y más parecidas a lo que vimos para la normal. La función
         \[\mbox{{\tt pt(a,df=b)}}\]
         devuelve el valor de $P(X<a)$ para una variable aleatoria que sigue una distribución $t$ de Student con $b$ grados de libertad (el comportamiento de {\tt pnorm} para la normal y el de {\tt pt} para la $t$ de Student son análogos).  Y, dada una probabilidad $a$, el valor de $x$ tal que
         \[P(X<x)=a,\]
         se calcula en R con la función {\tt qt}.  Así que, si antes, para $3$ grados de libertad, calculábamos $P(X<\frac{1}{2})$ en Calc usando
         \[\mbox{\tt 1-DISTR.T(0,5;3;1)=0,6742760176}\]
         en R usaríamos
         \[\mbox{\tt pt(0.5,df=3)}\]
         para obtener el mismo resultado. Para calcular el valor crítico al nivel $1-\alpha$ con $k$ grados de libertad usaríamos
         \[\mbox{\tt  -qt(alfa/2,df=k)}\]
         El valor resultante se denomina $t_{k;\alpha/2}$. Por ejemplo, el valor crítico al $95\%$ (es decir, $\alpha=0.05$) y con $k=10$ grados de libertad es $t_{10;0.025}$, que en R se calcularía con:
         \[\mbox{\tt  -qt(0.05/2,df=10)}\]
         obteniendo  $t_{10;0.025}\approx 2.228139$.
         %corrección hecha por mi

       \end{itemize}

       \subsection*{Intervalos de confianza para $\mu$ con muestras pequeñas y varianza desconocida.}

       \begin{itemize}


       \item Una vez entendido como calcular los valores críticos de la $t$ de Student, el cálculo del intervalo de confianza es muy parecido a los casos anteriores:\\[3mm]
       \fbox{\begin{minipage}{14cm}
       \begin{center}
       \vspace{2mm}
       {\bf Intervalo de confianza (nivel $(1-\alpha$)) para la media $\mu$.}\\[3mm]
       {\bf Población normal, varianza desconocida, muestras pequeñas $n<30$.}\\
       \end{center}
       Sea $X$ una variable aleatoria normal, Si consideramos muestras de tamaño $n$, y por lo tanto el número de grados de libertad es $k=n-1$, entonces el intervalo de confianza al nivel $(1-\alpha)$  para la media $\mu_X$ es:
       \[\bar X-\textcolor{red}{\boldmath\mbox{\large $t_{k;\alpha/2}$}}\dfrac{s}{\sqrt{n}}\leq \mu_X \leq \bar X+\textcolor{red}{\boldmath\mbox{\large $t_{k;\alpha/2}$}}\dfrac{s}{\sqrt{n}}.\]
       que también escribiremos:
       \[\mu_X =\bar X \pm \textcolor{red}{\boldmath\mbox{\large $t_{k;\alpha/2}$}}\dfrac{s}{\sqrt{n}}.\]
       \end{minipage}}\\[3mm]
       Y aquí están la \textattachfile{Cap06-IntervaloConfianzaMediaPoblacionNormalVDesconocidaMuestraPequenna.ods}{\textcolor{blue}{hoja de cálculo}}, y el \textattachfile{Cap06-IntervaloConfianzaMediaPoblacionNormalVDesconocidaMuestraPequenna.R}{\textcolor{blue}{fichero de instrucciones R}} correspondientes a este caso.

       \item Veamos un ejemplo (tomado de {\em Estadística, 2a. edición}, M. Spiegel, Ed.MacGraw-Hill(1991)).
       \begin{ejemplo}
       Una muestra de 10 medidas del diámetro de una esfera dan una media $\bar X=4.38$cm y una desviación típica $s=0.06$cm. Hallar intervalos de confianza al $95\%$ y al $99\%$ para el diámetro de la esfera.\\[3mm]

       Puesto que $n=10$, usamos la distribución $t$ y tomamos $k=9$ grados de libertad. Al nivel $1-\alpha=0.95$ (es decir, $\alpha/2=0.025$) Calculamos
       \[t_{9;0.025}\approx 2.26\]
       %corrección hecha por mi
       El intervalo al $95\%$ es:
       \[\bar X \pm \textcolor{black}{\mbox{$t_{k;\alpha/2}$}}\dfrac{s}{\sqrt{n}}=4.38\pm 2.26\cdot\dfrac{0.06}{\sqrt{10}}=4.38\pm 0.04.\]
       Para el intervalo al $99\%$ calculamos:
       \[t_{9;0.005}\approx 3.25\]
       %corrección hecha por mi
       y se obtiene:
       \[\bar X \pm \textcolor{black}{\mbox{$t_{k;\alpha/2}$}}\dfrac{s}{\sqrt{n}}=4.38\pm 3.25\cdot\dfrac{0.06}{\sqrt{10}}=4.38\pm 0.06,\]
       naturalmente más ancho que el anterior.\qed
       \end{ejemplo}

       \end{itemize}

%\section*{Tareas asignadas para esta sesión.}
%La tarea prevista para este fin de semana todavía no está disponible. Se os avisará a través de un mensaje en el foro cuando lo esté.
%
%\section*{\fbox{\colorbox{Gris025}{{Sesión 19. Inferencia estadística.}}}}
%
%\subsection*{\fbox{\colorbox{Gris025}{{Introducción al Contraste de hipótesis.}}}}
%\subsection*{Fecha: Martes, 22/11/2011, 14h.}
%
%\noindent{\bf Atención:
%\begin{enumerate}
%\item Este fichero pdf lleva adjuntos los ficheros de datos necesarios.
%\end{enumerate}
%}
%
%%\subsection*{\fbox{1. Ejemplos preliminares }}
%\setcounter{tocdepth}{1}
%%\tableofcontents


    \chapter{Contraste de hipótesis}
    % !Mode:: "Tex:UTF-8"
\section{El lenguaje del contraste de hipótesis}

\begin{itemize}

    \item En los próximos capítulos del curso vamos a ver cómo obtener intervalos de confianza para otros parámetros y distribuciones. El Contraste de Hipótesis, que vamos a conocer en este capítulo, está íntimamente relacionado con los intervalos de confianza, de manera que podemos decir que, por cada nuevo intervalo de confianza que aprendamos a calcular, habrá un contraste de hipótesis asociado. En este capítulo vamos a centrarnos en contrastes de hipótesis asociados con los intervalos de confianza que ya hemos estudiado (para la media, y en poblaciones normales).
\end{itemize}



\subsection*{Hipótesis nula y alternativa.}

\begin{itemize}

    \item Con el cálculo de intervalos de confianza hemos empezado a utilizar la Estadística para hacer predicciones sobre la población a partir de la muestra. Ahora estamos en condiciones de hacer afirmaciones como {\em ``con una fiabilidad del 95\%, la media de la población está en el intervalo $(a,b)$''}. ¿Cómo se usan estas afirmaciones en el trabajo científico?
        Recordemos el esquema básico del trabajo científico. Como siempre sucede en estos casos, es {\em
        realmente esquemático}, y la realidad es muchas veces algo más compleja, pero como guión inicial sirve a nuestros propósitos:
        \begin{enumerate}
        \item un científico propone una {\sf hipótesis}. Por ejemplo: ``Hemos desarrollado un nuevo medicamento, {\em Pildorín Complex}, para tratar la depresión severa en el \link{http://es.wikipedia.org/wiki/Canguro_rojo}{canguro rojo australiano}\marginpar{\includegraphics*[scale=1,width=1.5cm,keepaspectratio=true]{2011-11-22-Canguro.png}}. Y sostenemos que el medicamento es tan bueno, que después de administrárselo, los pacientes darán saltos de alegría. De hecho, afirmamos que la altura de esos saltos será mucho mayor de lo que era, antes del tratamiento''.
        \item Esa afirmación debe probarse mediante unos experimentos, de los cuales se extrae una colección de datos, una {\em muestra}.  En esta fase es importante que el diseño del experimento sea cuidadoso, de manera que la muestra sea representativa de la población, y útil para el análisis estadístico. En  el ejemplo, entre otras cosas, está claro que habrá que seleccionar minuciosamente un grupo de canguros depresivos, a los que se administrará el medicamento, y se medirá con cuidado la altura de sus saltos, antes y después de tratarlos. Hacia el final del curso hablaremos algo más sobre esta fase (habrá que hacer grupos de control, probar placebos, etc.)
        \item Y a continuación, viene la fase de {\sf análisis estadístico}. El objetivo es usar la estadística para ver si los datos avalan nuestras afirmaciones, nuestra hipótesis de que ``la altura media de los saltos es significativamente mayor que antes del tratamiento.''. Porque el laboratorio de la competencia (que lleva años vendiendo su medicamento {\em Saltaplus}), enseguida dirá que nuestro medicamento no tiene efectos, y que los saltos que hemos observado en nuestros canguros depresivos son, simplemente, sus saltos habituales, que los canguros a veces saltan más y a veces menos, y que nuestras medidas son {\sf fruto del  azar}.
        \end{enumerate}
        La última frase es esencial. Porque es verdad que los canguros ya daban saltos, aleatoriamente más o menos altos, antes de tomar nuestro medicamento. ¿Podemos usar la estadística para demostrar que el uso de {\em {\em Pildorín Complex}} ha tenido realmente un efecto sobre la altura de los saltos de los canguros depresivos? Bueno, naturalmente necesitamos saber algo sobre la altura típica de los saltos de los canguros depresivos (sin medicar). Así que le preguntamos a un experto independiente, ¿cuánto saltan los canguros depresivos? Vamos a suponer que el experto dice que la altura (en metros) de los saltos se puede representar mediante una variable aleatoria que sigue una distribución normal, con media $\mu_0=2.5$ (en metros). Nosotros hemos observado en nuestra muestra de 100 canguros depresivos tratados con {\em {\em Pildorín Complex}} una altura de salto media $\bar X=2.65$ (en metros), con desviación típica muestral $s=0.5$. Esto podría ser fruto del azar, claro está. Pero la pregunta es ¿cómo de sorprendente, cómo de rara o excepcional le parece esa muestra al experto? Para exagerarlo: si después de darles el tratamiento los canguros dieran saltos de 10m en promedio, al experto --y a la competencia-- le costaría mucho decir ``bueno, será cosa del azar''.

    \item El objetivo del contraste de hipótesis consiste, en una explicación informal, en establecer cómo de sorprendentes,  inesperados  o inexplicables le parecen los resultados de la muestra a alguien {\em que no acepta, o no se cree, nuestra hipótesis de trabajo}. Así pues, para entender lo que supone el contraste de hipótesis, nos servirá de ayuda pensar en una confrontación, en la que, por un lado, estamos nosotros con la hipótesis que defendemos y enfrente se sitúa un escéptico, que no se cree nuestra hipótesis y que por tanto defiende la hipótesis contraria. Empecemos por la terminología:
        \begin{enumerate}
        \item La hipótesis que defiende el escéptico (la competencia) es la {\sf hipótesis nula}, y se representa con $H_0$.
        \item Nuestra hipótesis, la que defendemos, se llamará {\sf hipótesis alternativa}, y se representa $H_a$.
        \end{enumerate}
        Veámoslo en el ejemplo de los canguros depresivos:
        \begin{ejemplo}
        En este caso las hipótesis son:
        \begin{enumerate}
        \item {\bf Hipótesis nula \boldmath $H_0$:} la altura media de los saltos de los canguros depresivos tratados con {\em Pildorín Complex} no es mayor que la de los canguros sin tratar. Es decir, la altura media de esos saltos no es mayor que $2.5$.  Esta hipótesis equivale a decir que nuestro tratamiento no ha tenido el efecto deseado, o que ha tenido un {\sc efecto nulo} sobre los canguros depresivos.
        \item {\bf Hipótesis alternativa \boldmath $H_a$:} la altura media de los saltos de los canguros tratados con {\em Pildorín Complex} es mayor que la de los canguros sin tratar. Es decir, nuestra hipótesis es que la variable aleatoria {\em altura de los saltos} sigue una distribución normal $N(\mu,0.5)$, donde {\sf la media $\mu$ es mayor que $\mu_0$}.
        \end{enumerate}
        \qed
        \end{ejemplo}



%     Afortunadamente, alguien buscó la respuesta para nosotros, estudiando el comportamiento de la variable $Y_2$ para $n$ pequeño.\\[3mm]
%     \fbox{\begin{minipage}{14cm}
%       \begin{center}
%       \vspace{2mm}
%       {\bf DISTRIBUCIÓN $t$ DE STUDENT}\\
%       \end{center}
%       Sea $X$ una variable aleatoria normal, de tipo $N(\mu_X,\sigma_X)$, y sea $\bar X$ la media muestral de $X$ en muestras de tamaño $n$. Entonces, la distribución de la variable aleatoria
%        \[\dfrac{\bar X-\mu_X}{\dfrac{s}{\sqrt{n}}},\]
%        recibe el nombre de {\sf distribución $t$ de Student con $k=n-1$ grados de libertad.}
%       \end{minipage}}\\[3mm]
%       Esta distribución fue estudiada por \link{http://en.wikipedia.org/wiki/William_Sealy_Gosset}{William S. Gosset}, que trabajaba para la fábrica de cerveza Guinness y que firmaba sus trabajos científicos bajo el pseudónimo de Student. Entre otras cosas, Student obtuvo la función de densidad de esta variable aleatoria continua, que para $k=n-1$ grados de libertad es:
%       \[f(x)=\dfrac{1}{\sqrt{k}\cdot\beta(\frac{1}{2},\frac{k}{2})}\left(1+\dfrac{x^2}{k}\right)^{-\frac{k+1}{2}}.\]
%       (Aquí $\beta$ es la \link{http://en.wikipedia.org/wiki/Beta_function}{función beta}, una función que se usa a menudo en matemáticas y que recuerda un poco a los números combinatorios.)
%
%       {\em ¡No hay ninguna necesidad de que te aprendas esta función de densidad!} La escribimos aquí sólo para recordarte que la distribución de Student, como cualquier variable aleatoria continua, viene caracterizada por su función de densidad. Es mucho más interesante comparar los gráficos de la función de densidad de la normal estándar y la $t$ de Student para distintos valores de $k$. Eso es lo que se ha hecho en este \textattachfile{2011_11_15_StudentVsNormalEstandar.html}{\textcolor{blue}{documento html}} (se abre en el navegador, requiere Java). Como puedes comprobar, la distribución $t$ de Student tiene una forma de campana que recuerda a la normal, pero es más abierta. A medida que el valor de $k$ aumenta, no obstante, la $t$ se parece cada vez más a la normal estándar, de manera que para $k>30$ son esencialmente iguales.
    \end{itemize}

    \subsection*{Errores de tipo I y tipo II.}



    \begin{itemize}

    \item En la próxima sección veremos como se utilizan los resultados experimentales (los valores muestrales) para decidir entre las dos hipótesis. Pero antes de hacer esto, y todavía en el terreno de la terminología, vamos a pensar un poco en la decisión que debemos tomar, y en las consecuencias de esa decisión: tenemos que decidir entre la hipótesis nula y la hipótesis alternativa. Como se trata de variables aleatorias, y sólo disponemos de datos muestrales, tomemos la decisión que tomemos, podemos estar equivocándonos. En seguida nos daremos cuenta de que, puesto que hay dos hipótesis enfrentadas, pueden darse estas cuatro situaciones:
        \begin{center}
        \begin{tabular}{cccc}
        \cline{3-4}
        &&\multicolumn{2}{|c|}{\bf ¿Qué hipótesis es correcta?}\\[3mm]
        \cline{3-4}
                                                      &&\multicolumn{1}{|c|}{\bf $H_0$ (nula) es correcta}&\multicolumn{1}{|c|}{\bf $H_a$ (alternativa) es correcta}\\[3mm]
        \cline{2-4}
                                        &\multicolumn{1}{|c|}{\bf Rechazar $H_0$}&\multicolumn{1}{|c|}{\bf Error tipo I}&\multicolumn{1}{|c|}{\bf Decisión correcta}\\[3mm]
        \cline{2-4}
                                        &\multicolumn{1}{|c|}{\bf Rechazar $H_a$}&\multicolumn{1}{|c|}{\bf Decisión correcta}&\multicolumn{1}{|c|}{\bf Error tipo II}\\[3mm]
        \cline{2-4}
        \end{tabular}
        \end{center}
        \item Un {\sf error de tipo I} significa que la hipótesis nula se rechaza, {\em a pesar de que es cierta}, y la hipótesis alternativa (la que nosotros defendemos) se acepta en su lugar. En  muchos casos, este es el tipo de error que se considera más grave (puedes pensarlo así: el error tiene agravante, porque encima de ser un error favorece a nuestra hipótesis.)

        \item El {\sf error de tipo II} significa que la hipótesis alternativa (la que defendemos) se rechaza, {\em a pesar de ser cierta}. Es también, naturalmente, un error, aunque como hemos dicho, en algunos casos se considera el mal menor, frente al error de tipo I. La importancia relativa de esos errores, sin embargo, depende mucho del contexto, y del significado (¡y la valoración de los riesgos!) que tenga para nosotros aceptar o rechazar cada una  de las hipótesis. Volveremos sobre esto en breve.

        \item Más adelante nos interesarán estas preguntas: ¿cuál es la probabilidad de cometer un error de tipo I? ¿Y un error de tipo II? Por el momento, nos conformamos con subrayar que ambas preguntas se pueden formular en términos de probabilidades condicionadas. En este sentido, a probabilidad de cometer un error de tipo I es:
            \[P(\mbox{error tipo I})=P(\mbox{rechazar $H_0$}|\mbox{$H_0$ es correcta})\]
            Mientras que para el tipo II es:
            \[P(\mbox{error tipo II})=P(\mbox{rechazar $H_a$}|\mbox{$H_a$ es correcta})\]

        \item Si esta terminología te recuerda a la de las pruebas diagnósticas (sensibilidad y especificidad) es porque, en efecto, una prueba diagnóstica es, hablando en general, un cierto tipo de contraste de hipótesis.

    \end{itemize}

    \section{Un contraste de hipótesis, paso a paso. Región de rechazo y p-valor.}

    \begin{itemize}

     \item La decisión de rechazar la hipótesis nula  se basa en los mismos métodos de cálculo que ya hemos conocido al obtener intervalos de confianza. El esquema básico de los contrastes de hipótesis que vamos a realizar se puede resumir así:
         \begin{enumerate}
         \item Definimos claramente lo que significan las hipótesis nula $H_0$, y alternativa $H_a$. El significado será una cierta igualdad o desigualdad sobre un parámetro de la distribución de una variable aleatoria en la población; por ejemplo, como hipótesis nula podemos decir que la media de la variable es menor o igual que $\mu_0$. En este caso la media de la población es el parámetro elegido.

         \item  Debemos elegir un estadístico que permita estimar ese parámetro. Es decir, elegir un número que podamos calcular a partir de las muestras que tomemos de la población, y que nos permita hacer estimaciones de probabilidad sobre el parámetro que hemos elegido en el paso anterior. Por ejemplo, si hemos elegido la media, y estamos en un ejemplo en el que las muestras son grandes, y conocemos la varianza poblacional, podemos tomar como estadístico el valor
             \[\dfrac{\bar X-\mu_0}{\frac{\sigma}{\sqrt{n}}},\]
             que sabemos que sigue una distribución normal estándar $N(0,1)$, \textcolor{red}{\sf siempre que la hipótesis nula sea cierta} (esta frase es la clave del proceso).
         \item Fijamos un nivel de significación $1-\alpha$ (típicamente al $90\%$, $95\%$ o $99\%$; es decir que $\alpha$ es $0.1$, $0.05$ o $0.01$), con una interpretación similar a la que vimos para los intervalos de confianza. De hecho, como veremos más abajo, $\alpha$ es la máxima probabilidad que estamos dispuestos a tolerar de cometer un error de tipo I .
         \item Obtenemos una muestra de la población y calculamos el valor del estadístico del segundo paso en esa muestra.
         \item Usando lo que sabemos de su distribución muestral, calculamos la probabilidad de obtener ese valor del estadístico, \textcolor{red}{\sf suponiendo que la hipótesis nula sea cierta}. Los cálculos de estos dos últimos pasos (este y el anterior) son esencialmente los mismos que para realizar un intervalo de confianza (y son los únicos necesarios para el contraste). El valor de probabilidad que obtenemos en este paso se conoce como el {\sf p-valor del contraste}, y es independiente de $\alpha$.
         \item Si esa probabilidad, el p-valor, es menor que $\alpha$, concluimos que los datos apuntan a que la hipótesis nula es falsa y la {\sf rechazamos}. En este caso se dice que hemos obtenido un {\sf resultado significativo}  (a favor de la hipótesis alternativa).
        \item Si, por el contrario, el p-valor es mayor que $\alpha$, diremos que los datos no permiten rechazar la hipótesis nula. En este caso se dice que hemos obtenido un {\sf resultado no significativo}.
         \end{enumerate}

         Antes de seguir, vamos a ver en detalle como se lleva a cabo cada uno de estos pasos del contraste de hipótesis en el ejemplo de los canguros.

       \begin{ejemplo}

         \begin{enumerate}
         \item[]
         \item En este ejemplo, tanto la competencia como nosotros estamos de acuerdo en que la variable aleatoria
         \[X=\mbox{\{\em altura de los saltos de los canguros depresivos, tratados con {\em Pildorín Complex}\}}\]
         sigue una distribución del tipo normal con media $\mu$. Nuestra discrepancia se refiere al valor de la media:
             \begin{itemize}
             \item La hipótesis nula, que defiende la competencia, es $H_0=\{\mu\leq 2.5\}$
             \item La hipótesis alternativa, que defendemos nosotros, es $H_a=\{\mu>2.5\}$.
             \end{itemize}
            El valor que aparece en ambas hipótesis y que marca la frontera entre lo que afirma la competencia y lo que afirmamos nosotros es el que llamamos $\mu_0$, y en este ejemplo es $\mu_0=2.5$. Por cierto, el igual en la desigualdad siempre está del lado de la hipótesis nula.
         \item  Puesto que el tamaño de las muestras ($n=100$) es bastante grande, y {\em no conocemos la desviación típica de la población}, revisamos lo que hemos aprendido sobre la distribución muestral de la media, y concluimos que un estadístico apropiado para este problema es:
             \[\dfrac{\bar X-\mu_0}{\frac{s}{\sqrt{n}}}.\]
             La utilidad de este estadístico para el contraste es que {\sf si la hipótesis nula es cierta}, su distribución será una normal $N(0,1)$. Nuestro objetivo, en tanto que partidarios de la hipótesis alternativa, es demostrar que los datos de la muestra no encajan con esa predicción, son {\em demasiado raros}, más allá de las dudas razonables.
         \item Elegimos el  nivel de significación $1-\alpha=95\%$.
         \item Ya hemos dicho que en nuestra muestra de la población es $\bar X=2.65$ y $s=0.5$, y que la muestra se compone de 100 observaciones. Calculamos el valor del estadístico, como si la hipótesis nula ($\mu=2.5$) fuera cierta :
         \[\dfrac{\bar X-\mu_0}{\frac{s}{\sqrt{n}}}=\dfrac{2.65-2.5}{\frac{0.5}{\sqrt{100}}}=3.\]
         \item Usando lo que sabemos de la distribución estándar, calculamos la probabilidad de obtener este valor $3$ (o uno mayor), en una variable de tipo $N(0,1)$. Se obtiene:
          \[P(Z\geq 3)\approx 0.001350\]
          Esta probabilidad es el \textcolor{red}{p-valor} del contraste.
         \item Puesto que $0.001350<0.05=\alpha$, el resultado es significativo al nivel de confianza elegido, y por eso {\sf rechazamos la hipótesis nula}, dando por probada experimentalmente la eficacia de {\em Pildorín Complex}.
         \end{enumerate}\qed
       \end{ejemplo}

     \item Vamos a fijarnos en el quinto y sexto pasos de este ejemplo, porque ahí se concentra el núcleo de la decisión que tomamos. Hemos calculado el p-valor:
          \[P(Z\geq 3)\approx 0.001350<\alpha\]
          y por otro lado sabemos que $1-\alpha=P(Z\leq z_{\alpha})$, es decir, $\alpha=P(Z\geq z_{\alpha})$. Combinando estas dos últimas expresiones se tiene
          \[P(Z\geq 3)<P(Z\geq z_{\alpha}),\]
          siendo $z_{\alpha}=z_{0.10}=1.645$
          Y una forma razonable de leer esto es diciendo que hemos rechazado la hipótesis nula $H_0$, porque el valor $3$ del estadístico obtenido a partir de la muestra es mayor que $z_{\alpha}$ para el nivel de confianza que habíamos fijado. Al verlo así queda claro que habríamos rechazado la hipótesis nula para cualquier valor del estadístico mayor que $z_{\alpha}$. Eso nos permite decir que los valores $z$ que cumplen $z>z_{\alpha}$ forman la {\sf región de rechazo} de la hipótesis nula.

     \item La región de rechazo se define con independencia de que la hipótesis nula $H_0$ sea cierta o no. Pero si además sabemos que $H_0$ es cierta, entonces la distribución del estadístico
     \[\dfrac{\bar X-\mu_0}{\frac{s}{\sqrt{n}}}\]
     es realmente la normal estándar. En ese caso, si obtenemos un valor de este estadístico en la región de rechazo, habremos rechazado la hipótesis nula, {\em a pesar de que es cierta}. Es decir, habremos cometido un error de tipo I. Y puesto que la probabilidad de obtener uno de esos valores es $\alpha$\footnote{aquí, en esta frase es donde precisamente usamos el hecho de que $H_0$ es cierta.}, comprobamos que, como habíamos anunciado:
     \[\alpha=P(\mbox{cometer un error de tipo I})\]

     \item Análogamente, se puede definir un valor
     \[\beta=P(\mbox{cometer un error de tipo II})\]
     que es la probabilidad de la región de rechazo de la hipótesis alternativa. Los dos tipos de errores van fuertemente emparejados. A primera vista podríamos pensar que lo mejor es tratar de hacer ambos errores pequeños simultáneamente. Pero esto es, en general, inviable, porque al disminuir la probabilidad de cometer un error de tipo I (al disminuir $\alpha$) estamos aumentando la probabilidad de cometer uno de tipo II (aumentamos $\beta$). Esta relación entre ambos tipos de errores, se ilustra en este \textattachfile{Cap07_ContrasteHipotesis_TiposDeError.html}{\textcolor{blue}{documento html}} para el caso de un contraste sobre el valor de la media. Como hemos dicho, la decisión depende mucho del contexto: los errores de tipo I se consideran más relevantes cuando, como en nuestro ejemplo, se está estudiando un nuevo procedimiento terapéutico o se propone una nueva teoría. Sin embargo, en otras aplicaciones, como por ejemplo en control de calidad, en seguridad alimentaria, o en los estudios medioambientales para detectar niveles altos de sustancias contaminantes, los errores de tipo II son los más preocupantes, porque cometer uno de estos errores significaría no detectar una situación potencialmente peligrosa.
    \end{itemize}

    \section{Contrastes de una y dos colas}

    \begin{itemize}

    \item En la discusión anterior hemos presentado los elementos básicos del lenguaje del contraste de hipótesis, tomando siempre como referencia un ejemplo sobre la media, en el que la hipótesis alternativa era
        \[H_a=\{\mu>\mu_0\},\quad \mbox{(en el ejemplo era }\mu_0=2.5).\]
        y la hipótesis nula era de la forma
        \[H_0=\{\mu\leq \mu_0\}.\]
        Y elegíamos esta hipótesis nula porque nuestra intención era mostrar que el tratamiento {\em aumentaba} el valor de la media. La región de rechazo de la hipótesis nula tiene el aspecto que se muestra en la figura:
        \begin{center}
        \includegraphics[width=15cm]{2011-11-22-ContrasteHipotesis-ColaDerecha.png}
        \end{center}
        En el caso del contraste de hipótesis para la media que hemos visto, para muestras grandes con varianza desconocida, eso quiere decir que la región de rechazo $R$ es de la forma:
        \[R=\left\{\dfrac{\bar X-\mu_0}{\frac{s}{\sqrt{n}}}>z_{\alpha}\right\},\]
        siendo $z_{\alpha}$ el valor crítico que, en la normal estándar  $N(0,1)$ deja una probabilidad $1-\alpha$ a su izquierda (y por tanto $\alpha$ a la derecha como queremos).


    \item En otros problemas, sin embargo, puede que nuestra hipótesis sea distinta. Evidentemente, habrá ocasiones en que lo que queremos analizar el si el tratamiento ha disminuido la media. Y en ese caso la hipótesis nula (que siempre es lo contrario de lo que queremos probar) será de la forma:
        \[H_0=\{\mu\geq \mu_0\},\quad \mbox{(mientras que  } H_a=\{\mu<\mu_0\}.\]
        Ahora la región de rechazo de la hipótesis nula tiene este aspecto:
        \begin{center}
        \includegraphics[width=15cm]{2011-11-22-ContrasteHipotesis-ColaIzda.png}
        \end{center}
        Y la región de rechazo $R$ es de la forma:
        \[R=\left\{\dfrac{\bar X-\mu_0}{\frac{s}{\sqrt{n}}}<z_{1-\alpha}\right\},\]
        siendo $z_{1-\alpha}=-z_{\alpha}$ el valor crítico que, en la normal estándar  $N(0,1)$ deja una probabilidad $\alpha$ a su izquierda.

    \item En ambos casos, la región de rechazo es una de las colas de la distribución (a derecha o a izquierda). Sin embargo, no siempre será así. Es posible que pensemos que el tratamiento tiene algún efecto, pero no sepamos a priori si ese efecto va a hacer que la media sea más alta o más baja. En este caso, nuestra hipótesis alternativa es de la forma:
        \[H_a=\{\mu\neq\mu_0\}.\]
        y la hipótesis nula es:
        \[H_0=\{\mu=\mu_0\}.\]
        A diferencia de los dos casos anteriores, ahora la región de rechazo de la hipótesis nula la forman dos colas de la distribución. En concreto, la región de rechazo $R$ es de la forma:
        \[R=\left\{\left|\dfrac{\bar X-\mu_0}{\frac{s}{\sqrt{n}}}\right|>z_{\alpha/2}\right\},\]
        siendo $z_{\alpha/2}$ el valor crítico que, en la normal estándar  $N(0,1)$ deja una probabilidad $1-\alpha/2$ a su izquierda (y por lo tanto, cada cola tiene probabilidad $\alpha/2$, como queremos).
        \begin{center}
        \includegraphics[width=15cm]{2011-11-22-ContrasteHipotesis-DosColas.png}
        \end{center}
        Este último caso es el que más recuerda a los intervalos de confianza, porque la región de rechazo es exactamente el exterior del intervalo de confianza para la media al nivel $1-\alpha$.


    \end{itemize}

       \section{Contraste de hipótesis para $\mu$ con muestras pequeñas y varianza desconocida.}\label{subsec:contrasteHipotesisMediaMuestrasPequennasVarianzaDesconocida}

       \begin{itemize}


       \item Al igual que sucedía con los intervalos de confianza, si el tamaño de la muestra es pequeño (recordemos, $n<30$), los contrasted de hipótesis son similares, pero utilizando la distribución $t$ de Student para calcular los valores críticos. Se obtienen estos resultados:
        \begin{enumerate}
        \item Hipótesis alternativa: $H_a=\{\mu>\mu_0\}$, hipótesis nula: $H_0=\{\mu\leq \mu_0\}.$.\\[3mm]
            Región de rechazo $R$ de la forma:
            \[R=\left\{\dfrac{\bar X-\mu_0}{\frac{s}{\sqrt{n}}}>t_{k;\alpha}\right\},\]
            siendo $t_{k;\alpha}$ el valor crítico para la distribución $t$ de Student con $k=n-1$ grados de libertad, que deja una probabilidad $1-\alpha$ a su izquierda (y por tanto $\alpha$ a la derecha).


    \item Hipótesis alternativa: $H_a=\{\mu<\mu_0\}$, hipótesis nula: $H_0=\{\mu\geq \mu_0\}.$.\\[3mm]
            Región de rechazo $R$ de la forma:
            \[R=\left\{\dfrac{\bar X-\mu_0}{\frac{s}{\sqrt{n}}}<t_{k;1-\alpha}\right\},\]
            siendo $t_{k;1-\alpha}=-t_{k;\alpha}$ el valor crítico para la distribución $t$ de Student con $k=n-1$ grados de libertad, que deja una probabilidad $\alpha$ a su izquierda.



    \item Hipótesis alternativa: $H_a=\{\mu\neq\mu_0\}$, hipótesis nula: $H_0=\{\mu=\mu_0\}.$.\\[3mm]
            Región de rechazo $R$ de la forma:
        \[R=\left\{\left|\dfrac{\bar X-\mu_0}{\frac{s}{\sqrt{n}}}\right|>t_{k;\alpha/2}\right\},\]
        siendo $t_{k;\alpha/2}$ el valor crítico para la distribución $t$ de Student con $k=n-1$ grados de libertad, que deja una probabilidad $1-\alpha/2$ a su izquierda (y por lo tanto, cada cola tiene probabilidad $\alpha/2$).

    \end{enumerate}

       \end{itemize}



    \chapter{Distribuciones relacionadas con la binomial}\label{cap:DistribucionesRelacionadasBinomial}
    % !Mode:: "Tex:UTF-8"


\section{Proporciones y su distribución muestral}

\begin{itemize}

    \item Hasta ahora, todas nuestras incursiones en el terreno de la inferencia se han centrado en el problema de la estimación de la media de la población. Sin embargo, en algunos casos la media no es el valor que más nos interesa. En esta sección vamos a tratar un ejemplo especialmente importante de una de estas situaciones. De esa forma empezaremos a ver inferencias sobre parámetros distintos de la media.

     \item  Para centrar el problema, supongamos que tenemos una población $\Omega$, y que en los individuos de esa población hay definida cierta característica que puede estar presente o no en esos individuos. Por ejemplo, podemos fijarnos en la población de araos comunes (\link{http://www.seo.org/aves_espana.cfm}{\em Uria aalge}, en inglés common guillemot). Esta especie presenta un polimorfismo, que consiste en la existencia, en algunos ejemplares de un anillo ocular blanco (estos ejemplares se denominan {\em embridados (bridled, en inglés)}).
         \begin{center}
         \includegraphics[width=11cm]{2011-11-29-araos.jpg}
         \end{center}
         En esta imagen de una colonia de cría en Escocia puede verse en el centro uno de estos ejemplares embridados rodeado de ejemplares sin esa característica. Una pregunta natural es ¿cuál es la proporción de ejemplares embridados sobre el total de individuos de la especie? Hay muchas otras preguntas que encajan con este modelo: ¿qué proporción de personas han contraído una determinada enfermedad? ¿que porcentaje de ejemplares albinos hay en una especie?, etcétera.

    \item Vamos a fijar la terminología necesaria para responder a preguntas como esta. Llamaremos $p$ a la proporción de individuos de la especie que presentan la característica que es objeto de estudio. Para estimar el valor de $p$, naturalmente, tomaremos una muestra aleatoria de la población, formada por $n$ observaciones. Recordemos que eso significa que tenemos $n$ variables aleatorias independientes,
        \[X_1,X_2,\ldots,X_n\]
        y que la distribución de probabilidad para cada una de ellas es una copia de la distribución de probabilidad de la población original. ¿Cómo usaremos esta muestra para estimar $p$?

        Así que debemos empezar por preguntarnos ¿cuál es esa distribución de la población original? Afortunadamente, la respuesta es fácil. La población original se puede describir mediante una variable aleatoria $X$ que toma sólo dos valores: 1 cuando la característica está presente, y 0 cuando está ausente. Y la probabilidad de que $X$ tome el valor $1$ es precisamente $p$, la proporción que queremos estimar (la probabilidad de $0$ es $q=1-p$). Es decir, que la variable $X$ es una variable de tipo Bernouilli, o dicho de otra forma, es una binomial $B(1,p)$. Su media es $\mu_X=1\cdot p=p$ y su desviación típica es $\sigma_X=\sqrt{1\cdot p\cdot q}=\sqrt{p\cdot q}$.

        Recapitulando: las variables $X_1$,\ldots,$X_n$ son independientes, cada una de ellas sólo puede tomar los valores $1$ o $0$, y la probabilidad de que $X_i$ tome el valor $1$ es $p$ (y toma ese valor cuando el $i$-ésimo individuo de la muestra presenta la característica). Por lo tanto,
        \[\mu_{X_1}=\mu_{X_2}=\cdots=\mu_{X_n}=p,\qquad \sigma_{X_1}=\sigma_{X_2}=\cdots=\sigma_{X_n}=\sqrt{p\cdot q}\]

        \item ¿Cómo usamos la muestra para estimar $p$? Pues contamos el número de individuos de la muestra que presentan esa característica, y dividimos entre el número $n$ de elementos de la muestra. Está claro que esto significa que calculamos la {\sf proporción muestral:}
            \[\hat p=\dfrac{X_1+X_2+\cdots+X_n}{n}\]
            Obsérvese que usamos el símbolo $\hat p$ para distinguir la proporción muestral de la de poblacional, que es $p$. Y por lo tanto, la proporción muestral es simplemente la media de una lista de variables independientes de tipo $B(1,p)$
            ¿Qué sucede al sumar $n$ variables independientes de tipo $B(1,p)$? Pues, pensándolo un poco, nos daremos cuenta de que se obtiene una binomial $B(n,p)$. Por lo tanto la variable $\hat p$ es una binomial $B(n,p)$ {\em pero dividida por $n$}. Esto lo representamos así:
             \[\hat p \sim \textcolor{red}{\dfrac{1}{n}}B(n,p).\]
            Ahora basta con recordar lo que dijimos en su momento en la página \pageref{sec:ExperimentosBernouilliDistribucionBinomial}, sobre la media y varianza de una combinación de variables aleatorias. Para la media resulta:
            \[E\left(\dfrac{1}{n}B(n,p)\right)=\dfrac{1}{n}\cdot E(B(n,p))=\dfrac{n\cdot p}{n}=p,\]
            Mientras que para la varianza (recordando que los números salen al cuadrado) es:
            \[\operatorname{Var}\left(\dfrac{1}{n}B(n,p)\right)=\left(\dfrac{1}{n}\right)^2\cdot \operatorname{Var}(B(n,p))=\dfrac{n\cdot p\cdot q}{n^2}=\dfrac{p\cdot q}{n}.\]
            Y por lo tanto hemos obtenido este resultado:\\[3mm]
            \fbox{
            \colorbox{Gris025}{
            \begin{minipage}{14cm}
            \begin{center}
            \vspace{2mm}
            {\bf Proporción muestral $\hat p$ y su distribución}
            \end{center}
            Sea $X$ una variable aleatoria de tipo $B(1,p)$, y sea $(X_1,X_2,\ldots,X_n)$ una muestra aleatoria independiente de tamaño $n$ de $X$. Si llamamos
            \[\hat p=\dfrac{X_1+X_2+\cdots+X_n}{n}\]
            entonces \[\hat p \sim \dfrac{1}{n}B(n,p)\]
            y por lo tanto:
           \[\mu_{\hat p}=p, \qquad \sigma_{\hat p}=\sqrt{\dfrac{p\cdot q}{n}}.\]
            \end{minipage}
            }
            }\\[3mm]

        \end{itemize}

\section{Inferencia estadística sobre proporciones}\label{sec:InferenciaEstadisticaSobreProporciones}

\begin{itemize}

    \item  Al disponer de toda la información sobre la variable $\hat p$, estamos en condiciones de construir intervalos de confianza y realizar contrastes de hipótesis que involucren a esa variable. Empezamos por repetir algo que, a estas alturas\footnote{Y, en cualquier caso, se puede consultar la segunda versión del Teorema Central del Límite, en la página \pageref{subsec:teoremaCentralLimiteSegundaVersion}}, debería ser evidente: para valores grandes de $n$, la distribución de $\hat p$ es muy aproximadamente normal. Y usando entonces los valores críticos de la normal estándar, está claro que podemos obtener fácilmente intervalos de confianza y contrastes. Sólo nos queda un pequeño problema, que ya tuvimos en su momento en el caso de la media. La desviación típica de $\hat p$ es:
        \[\sqrt{\dfrac{p\cdot q}{n}},\]
        pero no podemos usar esto directamente, porque desconocemos el valor de $p$. Así que lo que vamos a hacer es reemplazarlo con
        \[\sqrt{\dfrac{\hat p\cdot \hat q}{n}},\]
        (donde $\hat q=1-\hat p$), que es el valor que podemos calcular a partir de la muestra. Se puede demostrar rigurosamente que entonces se cumple esto:\\[3mm]
       \fbox{
       \colorbox{Gris025}{
       \begin{minipage}{14cm}
       \begin{center}
       \vspace{2mm}
       {\bf Teorema central del límite para proporciones}\\
       \end{center}
       Sea $X$ una variable aleatoria de tipo $B(1,p)$. Si se toman muestras independientes de $X$ de tamaño $n$, entonces cuando $n$ se hace cada vez más grande la distribución de la proporción muestral
          $\hat p$ se aproxima cada vez más a la normal $N\left(p,\sqrt{\dfrac{\hat p\cdot\hat q}{n}}\right)$.\\
          En particular, para $n$ grande tenemos
          \[Z=\dfrac{\hat p-p}{\sqrt{\dfrac{\hat p\cdot \hat q}{n}}}\sim N(0,1).\]
          Esta aproximación se considera válida cuando se cumplen, a la vez:
          \[n\cdot\hat p>5, n\cdot\hat q>5.\]
       \end{minipage}
       }
       }\\[3mm]
        Y a partir de aquí es inmediato obtener el intervalo de confianza:\\[3mm]
       \fbox{
       \colorbox{Gris025}{
       \begin{minipage}{14cm}
       \begin{center}
       \vspace{2mm}
       {\bf Intervalo de confianza (nivel $(1-\alpha$)) para la proporción $p$, con muestra grande.}\\
       \end{center}
       Si consideramos muestras de tamaño $n$ suficientemente grandes, entonces el intervalo de confianza al nivel $(1-\alpha)$  para la proporción $p$ es:
       \[\hat p-z_{\alpha/2}\sqrt{\dfrac{\hat p\cdot \hat q}{n}}\leq p \leq \hat p +z_{\alpha/2}\sqrt{\dfrac{\hat p\cdot \hat q}{n}}.\]
       que también escribiremos:
       \[p =\hat p \pm z_{\alpha/2}\sqrt{\dfrac{\hat p\cdot \hat q}{n}}.\]
       \end{minipage}
       }
       }\\[3mm]
       Y los contrastes de hipótesis unilaterales y bilaterales {\em (¡atención a los valores de $z$ utilizados en cada caso!)}:
       \\[3mm]
       \fbox{\colorbox{Gris025}{\begin{minipage}{14cm}
       \begin{center}
       \vspace{2mm}
       {\bf Contraste de hipótesis (nivel $(1-\alpha$)) para la proporción $p$, con muestra grande.}\\
       \end{center}
       Si consideramos muestras de tamaño $n$ suficientemente grandes, entonces se tienen los siguientes contrastes de hipótesis:
       \begin{enumerate}
       \item[(a)] Hipótesis nula: $H_0=\{p\leq p_0\}$.\\
            Región de rechazo:
            \[\hat p>p_0+z_{\alpha}\sqrt{\dfrac{p_0\cdot q_0}{n}}.\]
       \item[(b)] Hipótesis nula: $H_0=\{p\geq p_0\}$.\\
            Región de rechazo:
            \[\hat p<p_0+z_{1-\alpha}\sqrt{\dfrac{p_0\cdot q_0}{n}}.\]
       \item[(a)] Hipótesis nula: $H_0=\{p=p_0\}$.\\
            Región de rechazo:
            \[|\hat p-p_0|>z_{\alpha/2}\sqrt{\dfrac{p_0\cdot q_0}{n}}.\]
       \end{enumerate}
       \end{minipage}}}\\[3mm]
       Es importante observar una diferencia con el caso de la media. Puesto que el contraste se basa en suponer que la hipótesis nula es cierta, hemos utilizado $p_0$ y $q_0=1-p_0$ en lugar de $\hat p$ y $\hat q$. La razón de hacer esto es que, como hemos dicho, si suponemos que la hipótesis nula es cierta, entonces la desviación típica de la proporción muestral sería $\sqrt{\dfrac{p_0\cdot q_0}{n}}$. En el caso de la media, sin embargo, suponer conocida la media $\mu_0$ de la población no nos servía para saber cuál es la desviación típica de la población, y por eso usábamos $s$ como sustituto.

       Veamos como funciona esto en un ejemplo (tomado de {\em Biostatistics, 7th edition}, W.W.Daniel, Ed.John Wiley (1999), ejemplo 7.5.2 en la página 250).
       \begin{ejemplo}
       En un estudio de drogadictos por vía intravenosa en una gran ciudad, Coates et al. descubrieron que, para un muestra de 423 drogadictos, 18 eran seropositivos para el VIH. Se desea saber si estos datos permiten concluir que la proporción de drogadictos seropositivos en esa población es inferior al 5\%.\\
       La hipótesis nula y alternativa son, para este ejemplo (en el que $p_0=0.05$, $q_0=0.95$):
       \[
       H_0=\{p\geq 0.05\},\qquad H_a=\{p<0.05\}
       \]
       Y hemos obtenido una proporción muestral:
       \[\hat p=\dfrac{18}{423}\approx 0.04255\]
       Elegimos un nivel de confianza del 95\%, es decir $1-\alpha=0.95$, con lo que
       \[z_{1-\alpha}=z_{0.95}=-1.6449.\]
        Calculamos:
       \[p_0+z_{1-\alpha}\sqrt{\dfrac{p_0\cdot q_0}{n}}=0.05+(-1.6449)\sqrt{\dfrac{0.05\cdot 0.95}{423}}\approx 0.03257\]
       Y puesto que (recuerda que $\hat p=0.04255$):
       \[0.03257<0.04255\]
       concluimos que {\em no se puede rechazar la hipótesis nula}. Es decir, el porcentaje de seropositivos puede ser mayor que el 5\% conjeturado.\\

       Otra manera de organizar esta misma cuenta es utilizar el estadístico
       \[Z=\dfrac{\hat p-p_0}{\sqrt{\dfrac{p_0\cdot q_0}{n}}}\]
       que, {\em si se acepta la hipótesis nula}, tiene distribución normal estándar. Por lo tanto, calculamos $Z$ para nuestros datos:
       \[Z=\dfrac{\hat p-p_0}{\sqrt{\dfrac{p_0\cdot q_0}{n}}}=
       \dfrac{0.04255-0.05}{\sqrt{\dfrac{0.05\cdot 0.95}{423}}}\approx -0.7030
       \]
       Como este valor cumple
       \[-0.7030>z_{0.95}=-1.645\]
       vemos que no se puede rechazar la hipótesis nula.
       La probabilidad de la cola izquierda para este valor en la normal estándar:
       \[P(Z<-0.7030)\approx 0.2410\]
       es el $p$-valor del contraste. \qed
       \end{ejemplo}

    \end{itemize}


\section{Distribución de Poisson}

\begin{itemize}

    \item Las distribuciones binomial y normal, que (junto con la $t$ de Student) han centrado nuestro trabajo hasta ahora, no agotan, ni mucho menos, el repertorio de las distribuciones que se usan en
    Estadística. En las próximas secciones vamos a ampliar nuestro repertorio de distribuciones, presentando las distribuciones más destacadas y sus aplicaciones. Entre esas aplicaciones veremos de nuevo algunos de los conceptos que ya hemos discutido, como las distribuciones muestrales y sus aplicaciones para calcular intervalos de confianza, realizar contrastes de hipótesis, etcétera.


     \item Dijimos, en su momento que la distribución binomial $B(n,p)$ era, sin duda, la más importante de todas las distribuciones discretas. Y al pasar al límite para $n\to\infty$, obtuvimos como límite la distribución normal. Pero ese límite no se obtenía sin condiciones. Como vimos al enunciar la primera versión del Teorema Central del Límite (en la página \pageref{sec:teoremaCentralLimitePrimeraVersion} ), la aproximación de la binomial normal por la normal se comporta bien en tanto se cumplan las condiciones:
         \[n\cdot p>5, n\cdot q>5.\]
         Sin embargo, es frecuente encontrase con situaciones que, aunque se dejan enunciar en el lenguaje de éxitos y fracasos de los ensayos de Bernouilli (como pasaba con la binomial), tienen asociados valores de $p$ extremadamente bajos. Si, por ejemplo, $p=0.001$, entonces la condición $n\cdot p>5$ no empieza a cumplirse hasta que han transcurrido 5000 ensayos. Y sin embargo, si queremos calcular
         \[P(X=34),\quad\mbox{para $X$ del tipo }B(150,0.001)\]
         el cálculo usando la definición con la binomial resulta bastante complicado:
         \[P(X=34)=\binom{150}{34}\left(0.001\right)^{34}(0.999)^{116}.\]
         Vamos a ver, en esta sección, una distribución que permite aproximar a la binomial en estos casos. Antes de la definición, una observación importante. La variable que vamos a definir es {\em discreta, pero puede tomar cualquier valor entre los números naturales $0,1,2,\ldots$ (infinitos valores).}\\[3mm]
         \fbox{\colorbox{Gris025}{\begin{minipage}{14cm}
         \begin{center}
         \vspace{2mm}
         {\bf DISTRIBUCIÓN DE POISSON}\\
         \end{center}
          Sea $\lambda$ un número positivo. Una variable aleatoria discreta $X$, es de tipo {\sf Poisson $\operatorname{Pois}(\lambda)$}, si $X$ puede tomar cualquier valor natural $0,1,2,3,\ldots$, con esta distribución de probabilidad:
          \[P(X=k)=\dfrac{\lambda^k}{k!}e^{-\lambda}\]
         \end{minipage}}}\\[3mm]
         El valor $k$ se va a seguir interpretando como el número de éxitos obtenidos. Y enseguida vamos a ver qué significa el parámetro $\lambda$, y cómo se usa esta distribución de Poisson para aproximar la binomial. Pero para practicar un poco la definición, veamos que, por ejemplo, si $\lambda=2$, y $k=3$, se tiene
         \[P(X=3)=\dfrac{2^3}{3!}e^{-2}\approx  0.180447\]
         Pero los valores de probabilidad decaen rápidamente. Por ejemplo, con el mismo valor $\lambda=2$, pero con $k=10$ se obtiene:
         \[P(X=10)=\dfrac{2^{10}}{10!}e^{-2}\approx 1.2811\cdot 10^{-8}.\]
         Este tipo de comportamiento es de esperar, porque si vamos a aproximar binomiales con probabilidades $p$ (de éxito en cada ensayo) bajas, esperamos que la probabilidad de un número alto de éxitos sea muy pequeña.
         En la siguiente figura puedes ver representados algunos valores de probabilidad de la distribución de Poisson para $\lambda=2$:
         \begin{center}
         \includegraphics[width=12cm]{2011-11-25-Poisson01.png}
         \end{center}
         Y en este \textattachfile{Cap08-Poisson.html}{\textcolor{blue}{documento html}} (se abre en el navegador, requiere Java) puedes observar el comportamiento de la distribución de Poisson a medida que $n$ cambia.

    \item El parámetro $\lambda$ de la distribución de Poisson determina todas sus propiedades estadísticas:\\[3mm]
         \fbox{\colorbox{Gris025}{\begin{minipage}{14cm}
         \begin{center}
         \vspace{2mm}
         {\bf MEDIA Y VARIANZA DE UNA DISTRIBUCIÓN DE POISSON}\\
         \end{center}
          Sea $X$ una variable aleatoria discreta de tipo {\sf Poisson $\operatorname{Pois}(\lambda)$}. Entonces su media y varianza vienen dadas por:
          \[\mu_X=\lambda,\qquad \sigma^2_X=\lambda\]
         \end{minipage}}}\\[3mm]
         Este valor de la media se calcula aplicando la definición, que ya conocemos,  sólo que en este caso se trata de una suma infinita (serie):
         \[\mu_X=0\cdot P(X=0)+1\cdot P(X=1)+2\cdot P(X=2)+\cdots=
         0\cdot\dfrac{\lambda^0}{0!}e^{-\lambda}+1\cdot\dfrac{\lambda^1}{1!}e^{-\lambda}+2\cdot\dfrac{\lambda^2}{2!}e^{-\lambda}+\cdots\]
         y hay que usar matemáticas algo más complicadas para ver que el valor de esta suma infinita es $\lambda$. La varianza se obtiene de una serie similar.

    \item Esta distribución fue introducida por  \link{http://en.wikipedia.org/wiki/Sim\%C3\%A9on\_Denis_Poisson}{Siméon Denis Poisson}, un físico y matemático francés del siglo XIX, discípulo de Laplace.

  \item Hemos dicho que la distribución de Poisson se utiliza para aproximar la binomial en el caso de probabilidades pequeñas. Por esa razón la distribución de Poisson se llama a veces la {\sf distribución de los sucesos raros}. Concretamente, el resultado de aproximación es este:\\[3mm]
         \fbox{\colorbox{Gris025}{\begin{minipage}{14cm}
         \begin{center}
         \vspace{2mm}
         {\bf APROXIMACIÓN DE LA BINOMIAL POR UNA DISTRIBUCIÓN DE POISSON}\\
         \end{center}
          Si $X$ es una variable aleatoria discreta de tipo binomial $B(n,p)$ y se cumplen a la vez estas dos condiciones\footnote{Algunos autores sugieren $n>30$.}:
          \[n>50,\qquad n\cdot p<5,\]
          entonces los valores de probabilidad de $X$ se pueden aproximar por los de una distribución de tipo Poisson, concretamente por una $\operatorname{Pois}(\lambda)$, con
          \[\lambda=n\cdot p.\]
         \end{minipage}}}\\[3mm]
         Esta aproximación se puede obtener a partir de la binomial mediante el paso al límite cuando $p\to 0$, $n\to\infty$, pero $n\cdot p=\lambda$ se mantiene constante. Los límites necesarios no son muy complicados, pero tampoco aportan gran cosa a nuestra comprensión. Si estás interesado puedes encontrarlos en el libro {\em Estadística Básica para estudiantes de ciencias}, de Gorgas, Cardiel y Zamorano (profesores de la Univ. Complutense de Madrid), que puedes descargar en versión electrónica desde \link{http://www.ucm.es/info/Astrof/users/jaz/estadistica.html}{esta página}.

         Naturalmente, como habrás podido observar al analizar la distribución de Poisson para distintos valores de $\lambda$, al aumentar $\lambda$ se obtienen distribuciones cada vez más parecidas a la normal. Esta es otra manifestación más del Teorema Central del Límite. Esto permite aproximar la distribución de Poisson por una normal:\\[3mm]
         \fbox{\colorbox{Gris025}{
         \begin{minipage}{14cm}
         \begin{center}
         \vspace{2mm}
         {\bf APROXIMACIÓN DE LA DISTRIBUCIÓN DE POISSON POR UNA NORMAL}\\
         \end{center}
          Si $X$ es una variable aleatoria discreta de tipo $\operatorname{Pois}(\lambda)$, y se cumple:
          \[\lambda>5,\]    
          entonces los valores de probabilidad de $X$ se pueden aproximar por los de una distribución de tipo normal, concretamente:
          \[N(\lambda,\sqrt{\lambda}).\]
         \end{minipage}}}
         \\[3mm]

%         ¿Hasta qué punto es coherente esto con la aproximación binomial-normal que hemos visto (suponiendo que se cumplan todas las condiciones)? Si aproximamos una binomial $B(n,p)$ directamente por una normal usaríamos $N\left(n\cdot p,\sqrt{n\cdot p\cdot q}\right)$. Pero si primero aproximamos la binomial por una Poisson, usaríamos $\operatorname{Pois}(\lambda)=\operatorname{Pois}(n\cdot p)$. Y si ahora aproximamos esta Poisson por una normal, tendríamos

         Vamos a usar esto, más adelante en el curso, como justificación teórica de algunos métodos. Pero, en la práctica, para el cálculo de valores de probabilidad, hay que tener en cuenta que los valores de la distribución de Poisson son relativamente fáciles de calcular.


  \item Otra de las aplicaciones frecuentes de la distribución de Poisson es a procesos en los que un determinado fenómeno (un éxito, en el lenguaje de la binomial) puede ocurrir una o más veces a lo largo de un intervalo de tiempo (o en una parcela de espacio) y que reúnen estas características:
      \begin{enumerate}
        \item el número de sucesos en un intervalo es independiente del número de sucesos en cualquier otro intervalo.
        \item cuánto mayor es el intervalo de tiempo, mayor es la probabilidad de que ocurra un suceso individual en ese intervalo. Hay proporcionalidad entre longitud del intervalo y probabilidad del suceso.
        \item la probabilidad de que ocurran dos sucesos en un mismo intervalo es muy pequeña.
      \end{enumerate}
      En tal caso, si llamamos $X=${\em \{número de sucesos en un intervalo\}}, la variable $X$ sigue una distribución de tipo $\operatorname{Pois}(\lambda)$, siendo $\lambda$ naturalmente el número medio de sucesos en cada intervalo.

  \item Para calcular los valores de probabilidad asociados a la distribución de Poisson disponemos de bastantes recursos. Naturalmente, se pueden usar las tablas de esta distribución que incluyen casi todos los libros de Estadística. En las hojas de cálculo como Calc o Excel disponemos de la función {\tt POISSON}, que reemplaza ventajosamente a esas tablas (y aquí tienes un \textattachfile{Cap08-ProbabilidadPoisson.ods}{\textcolor{blue}{fichero Calc}} preparado para que puedas practicar el cálculo de valores de probabilidad).

      En R disponemos de:
      \begin{enumerate}
        \item la función {\tt dpois(k,lambda)} devuelve el valor de $P(X=k)$ para una distribución de Poisson de parámetro {\tt lambda}.
        \item la función {\tt ppois(k,lambda)} devuelve el valor de $P(X\leq k)$ para una distribución de Poisson de parámetro {\tt lambda}.
        \item la función {\tt qpois(p, lambda)} devuelve, para un valor de probabilidad {\tt p}, el {\sf menor número $k$} para que el que se cumple $P(X\leq k)\geq p$.
      \end{enumerate}
      Y en este \textattachfile{Cap08-DistribucionPoisson.R}{\textcolor{blue}{fichero de instrucciones R}} tienes preparadas esas funciones para practicar con ellas.

      Finalmente, para hacer esos cálculos, puedes usar el entorno Wiris con este \textattachfile{Cap08-Poisson-Wiris.html}{\textcolor{blue}{documento html}} (se abre en el navegador, requiere Java y conexión activa a internet).

    \end{itemize}


\section{Inferencia estadística para la distribución de Poisson}

\begin{itemize}

    \item Debería estar empezando a convertirse en una costumbre: cada nueva distribución que aparece (en este caso la de Poisson) lleva aparejados los correspondientes resultados inferenciales, que se traducen en el cálculo de intervalos de confianza y contrastes de hipótesis. Con la distribución de Poisson, naturalmente, sucede lo mismo. No vamos a considerar el caso de muestras pequeñas, porque sería demasiado complicado. Al limitarnos a muestras grandes, podemos contar con el Teorema Central del Límite, que nos garantiza este resultado:\\[3mm]
       \fbox{\colorbox{Gris025}{\begin{minipage}{14cm}
       \begin{center}
       \vspace{2mm}
       {\bf Teorema central del límite para la distribución de Poisson}\\
       \end{center}
       Sea $X$ una variable aleatoria de tipo $\operatorname{Pois}(\lambda)$. Si se toman muestras independientes de $X$ de tamaño $n$, entonces cuando $n$ se hace cada vez más grande la distribución de la media muestral
       $\hat p$ se aproxima cada vez más a la normal $N\left(\lambda,\sqrt{\dfrac{\lambda}{n}}\right)$.\\
       En particular, para $n$ grande tenemos $Z=\dfrac{\bar X-\lambda}{\sqrt{\dfrac{\mbox{\boldmath$\textcolor{red}{\bar X}$}}{n}}}\sim N(0,1)$.
       \end{minipage}}}\\[3mm]
       {\em Obsérvese} que, con vistas a la inferencia, hemos utilizado $\bar X$ como sustituto de $\lambda$ en el cálculo de la desviación típica muestral. Esta sustitución es similar a otras que ya hemos encontrado.

    \item Como en el caso de las proporciones, tras establecer la distribución de la media muestral, es inmediato obtener el intervalo de confianza:\\[3mm]
       \fbox{\colorbox{Gris025}{\begin{minipage}{14cm}
       \begin{center}
       \vspace{2mm}
       {\bf Intervalo de confianza (nivel $(1-\alpha$)) para $\lambda$ (en $\operatorname{Pois}(\lambda)$), con muestra grande.}\\
       \end{center}
       Si consideramos muestras de tamaño $n$ suficientemente grandes, entonces el intervalo de confianza al nivel $(1-\alpha)$  para $\lambda$  es:
       \[\bar X-z_{\alpha/2}\sqrt{\dfrac{\bar X }{n}}\leq \lambda \leq \bar X +z_{\alpha/2}\sqrt{\dfrac{\bar X}{n}}.\]
       que también escribiremos $\lambda =\bar X \pm z_{\alpha/2}\sqrt{\dfrac{\bar X}{n}}$.
       \end{minipage}}}\\[3mm]
       Y los contrastes de hipótesis unilaterales y bilaterales {\em (¡de nuevo, hay que prestar atención a los valores de $z$ utilizados en cada caso!)}:
       \\[3mm]
       \fbox{\colorbox{Gris025}{\begin{minipage}{14cm}
       \begin{center}
       \vspace{2mm}
       {\bf Contraste de hipótesis (nivel $(1-\alpha$)) para $\lambda$ (en $\operatorname{Pois}(\lambda)$), con muestra grande.}\\
       \end{center}
       Si consideramos muestras de tamaño $n$ suficientemente grandes, entonces se tienen los siguientes contrastes de hipótesis:
       \begin{enumerate}
       \item[(a)] Hipótesis nula: $H_0=\{\lambda\leq \lambda_0\}$.\\
            Región de rechazo $\bar X>\lambda_0+z_{\alpha}\sqrt{\dfrac{\lambda_0}{n}}.$
       \item[(b)] Hipótesis nula: $H_0=\{\lambda\geq \lambda_0\}$.\\
            Región de rechazo: $\bar X<\lambda_0+z_{1-\alpha}\sqrt{\dfrac{\lambda_0}{n}}.$
       \item[(a)] Hipótesis nula: $H_0=\{\lambda=\lambda_0\}$.\\
            Región de rechazo: $|\bar X-\lambda_0|>z_{\alpha/2}\sqrt{\dfrac{\lambda_0}{n}}.$
       \end{enumerate}
       \end{minipage}}}\\[3mm]
       Una observación, muy parecida a la que hicimos en el caso de los contrastes sobre proporciones. Puesto que el contraste se basa en suponer que la hipótesis nula es cierta, hemos utilizado $\lambda_0$ en lugar de $\bar X$. Y de nuevo, la razón de hacer esto es que, como hemos dicho, si suponemos que la hipótesis nula es cierta, entonces la desviación típica de la media muestral sería $\sqrt{\dfrac{\lambda_0}{n}}$.

\end{itemize}



    \chapter{Inferencia sobre la varianza}
    % !Mode:: "Tex:UTF-8"

\section{Inferencia sobre la varianza y $\chi^2$}

\begin{itemize}

    \item Ya hemos obtenido intervalos de confianza y contrastes de hipótesis para (a) la media $\mu$ de una población normal, (b) la proporción $p$ en una población de tipo Bernouilli y (c) $\lambda$ en una población de tipo Poisson. Estos dos últimos casos (b) y (c) corresponden a distribuciones que quedan completamente identificadas por un parámetro: $p$ y $\lambda$ respectivamente. Pero si volvemos la vista al caso (a), sabemos que una población normal se caracteriza por dos parámetros: $\mu$ y $\sigma$ (o $\sigma^2$). Y aunque hemos dedicado bastante esfuerzo a $\mu$, no hemos dicho nada todavía sobre el problema de cómo estimar $\sigma$, a partir de una muestra de la población.

    \item Vamos a intentar evitar una posible confusión que puede estar apareciendo: desde luego, tenemos un {\sf candidato natural} a servir de estimador para $\sigma^2$, que no es otro que $s^2$, la {\sf cuasivarianza muestral}. Lo que estamos diciendo es que hasta ahora hemos usado $s^2$ como una herramienta auxiliar, {\em con el objetivo de estimar $\mu$ mediante $\bar X$}. El protagonista de aquella estimación, por así decirlo, era $\bar X$. Pero ahora queremos centrar nuestra atención en $s^2$ (sin nadie que le robe protagonismo) y preguntarnos ¿cómo se puede usar $s^2$ {\em para estimar $\sigma^2$}, la varianza poblacional?

    \item Con la experiencia que ya vamos acumulando, sabemos que el primer paso es obtener algún resultado sobre la distribución de $s^2$, la cuasivarianza muestral. Sea por lo tanto $X$ una variable aleatoria con distribución de tipo $N(\mu,\sigma)$ (que representa a la población), y sea $X_1,X_2,\ldots,X_n$ una muestra aleatoria de $X$ (como siempre, las $X_i$ son $n$ copias independientes de $X$).  Recordemos que entonces:
        \[s^2=\dfrac{\displaystyle\sum_{i=1}^n(X_i-\bar X)^2}{{n-1}}.\]
        Como siempre, vamos a tratar de relacionar esto con la normal estándar $N(0,1)$. Para conseguirlo, vamos a dividir esta expresión por $\sigma^2$, y la reorganizaremos (vamos buscando tipificar las $X_i$):
        \begin{equation}\label{eq:ObtenerDistribucionCuasivarianzaMuestral}
        \dfrac{s^2}{\sigma^2}=\dfrac{\displaystyle\sum_{i=1}^n(X_i-\bar X)^2}{\sigma^2\cdot(n-1)}=
        \dfrac{1}{(n-1)}\dfrac{\displaystyle\sum_{i=1}^n(X_i-\bar X)^2}{\sigma^2}=\dfrac{1}{(n-1)}\sum_{i=1}^n\left(\dfrac{(X_i-\bar X)^2}{\sigma^2}\right)=
        \end{equation}
        \[
        =\textcolor{red}{\dfrac{1}{(n-1)}\sum_{i=1}^n\left(\dfrac{X_i-\bar X}{\sigma}\right)^2}=\dfrac{1}{(n-1)}\sum_{i=1}^n Z_i^2=\dfrac{1}{n-1}(Z_1^2+Z_2^2+\cdots+Z_n^2).\]
        El paso que hemos destacado coloreándolo en rojo es el paso en el que tipificamos las $X_i$, y obtenemos las $Z_i$ que son, cada una de ellas, copias de la normal estándar.

    \item Lo que hace que esta situación sea diferente de las que nos hemos encontrado hasta ahora, es que las $Z_i$ están elevadas al cuadrado. Vamos despacio: aunque no vamos a entrar en los detalles, porque son demasiado técnicos, se puede demostrar que la suma de variables normales es una variable normal\footnote{Si $X_1\sim N(\mu_1,\sigma_1)$ y $X_2\sim N(\mu_2,\sigma_2)$, entonces $(X_1+X_2)\sim N\left(\mu_1+\mu_2,\sqrt{\sigma_1^2+\sigma_2^2}\right)$ }. Así que si simplemente estuviéramos sumando las $Z_i$, al final obtendríamos una variable normal. La dificultad estriba en que cada $Z_i$ aparece elevada al cuadrado, {\sf y el cuadrado de una variable con distribución normal no es una variable con distribución normal}. Esto es relativamente fácil de entender: la normal estándar $Z$ toma valores positivos y negativos, como de hecho sucede con cualquier otra normal. Pero en cuanto la elevamos al cuadrado, deja de tomar valores negativos. Así que, como decíamos, el cuadrado de una normal estándar no puede ser una normal, y la suma de unos cuantos cuadrados de normales estándar tampoco resulta ser una normal (ni estándar, ni no estándar, simplemente no es normal). Lo cual es un problema, claro, porque precisamente eso es lo que necesitamos para entender la distribución de la cuasivarianza muestral $s^2$.

    \item La pregunta es: ¿cuál es, entonces, la distribución de una suma de cuadrados de normales estándar independientes?\\[3mm]
       \fbox{\begin{minipage}{14cm}
       \begin{center}
       \vspace{2mm}
       {\bf Distribución $\chi^2$. Media y varianza.}\\
       \end{center}
       Si la variable aleatoria $Y$ es la suma de los cuadrados de una familia de $n$ copias independientes de la distribución normal estándar, entonces diremos que $Y$ es de tipo $\chi^2_k$ (de \link{http://en.wikipedia.org/wiki/Karl_Pearson}{Pearson}), con $k=n-1$ grados de libertad.\\
       La media de $\chi^2_k$ es $\mu_{\chi^2_k}=k$, y su desviación típica es $\sigma_{\chi^2_k}=\sqrt{2k}$.
       \end{minipage}}\\[3mm]

    \item La función de densidad de la distribución $\chi^2$ (para $k=4$) tiene este aspecto:
         \begin{center}
         \includegraphics[width=15cm]{2011-12-02-ChiCuadrado.png}
         \end{center}
         Observa, en primer lugar, que la función vale $0$ (no hay probabilidad asociada) para los valores negativos, como ya habíamos adelantado. La fórmula de esta función de densidad es:
         \[f(x;n)=
         \begin{cases}
         \dfrac{1}{2^{k/2}\Gamma(k/2)}x^{(k/2)-1}e^{-x/2}&\mbox{ si }x\geq 0\\
         0&\mbox{ si }x<0
         \end{cases}
         \]
         donde $\Gamma$ es la \link{http://es.wikipedia.org/wiki/Funci\%C3\%B3n_gamma}{función Gamma}. Como en el caso de la $t$ de Student, no es necesario, ni mucho menos, que te aprendas esta fórmula. Pero es bueno tener una idea general del aspecto que tiene $\chi^2_k$ para distintos valores de $k$. Puedes observarlo por ti mismo usando este \textattachfile{Cap09_ChiCuadrado.html}{\textcolor{blue}{documento html}} (se abre en el navegador, requiere Java).

    \item Para calcular los valores de $\chi^2_k$, en $R$ disponemos de las funciones
        \[\mbox{{\tt pchisq(x,df=k)},\qquad y \qquad {\tt qchisq(p,df=k)},}\]
        que -como en los otros casos que ya hemos encontrado-- resuelven respectivamente los problemas directos e inversos asociados con esta distribución, usando siempre colas izquierdas. Por ejemplo, el resultado de    \[\mbox{\tt qchisq(0.95,df=4)}\]
        es $9.487729$, lo cual significa que si $Y\sim\chi^2_4$:
        \[P(Y\leq 9,487729)=0.95,\]
        como se ilustra en la figura:
        \begin{center}
        \includegraphics[width=15cm]{2011-12-02-ChiCuadrado-2.png}
        \end{center}
        En Calc disponemos de las funciones {\tt CHISQDIST} y {\tt CHISQINV}, de nuevo para calcular probabilidades y problemas inversos, usando en ambos casos la cola izquierda, Por lo tanto, como en R, se tiene este resultado:
        \[\mbox{\tt CHISQINV(0,95;4)=9,487729}\]
        y la interpretación es la misma figura que acabamos de ver. Pero además, en Calc,  existe también la función {\tt DISTR.CHI}, para problemas directos, que usa la cola derecha de la distribución, de manera que, por ejemplo, el resultado:
        \[\mbox{\tt DISTR.CHI(9;4)=0,061099481}\]
        significa que si $Y\sim\chi^2_4$ (con cuatro grados de libertad), entonces:
        \[P(Y\geq 9)=0.0061099481\]
        como en esta figura:
        \begin{center}
        \includegraphics[width=15cm]{2011-12-02-ChiCuadrado-coladerecha.png}
        \end{center}

    \item Ahora que disponemos de la información necesaria sobre la distribución $\chi^2_k$, podemos volver al problema de la distribución de la cuasivarianza muestral. Recuerda que en la Ecuación \ref{eq:ObtenerDistribucionCuasivarianzaMuestral} (de la página \pageref{eq:ObtenerDistribucionCuasivarianzaMuestral}) hemos obtenido
        \[(n-1)\dfrac{s^2}{\sigma^2}=\dfrac{1}{n-1}(Z_1^2+Z_2^2+\cdots+Z_n^2),\]
        y que las $Z_i$ eran todas normales estándar independientes. Con esto tenemos este resultado:\\[3mm]
       \fbox{\begin{minipage}{14cm}
       \begin{center}
       \vspace{2mm}
       {\bf $\chi^2_k$ y la distribución de la cuasivarianza muestral}\\
       \end{center}
       Sea $X$ una variable aleatoria de tipo normal $N(\mu,\sigma)$. Si se toman muestras independientes de $X$ de tamaño $n$, y $s^2$ es la cuasivarianza muestral de $X$, entonces:
        \[(n-1)\dfrac{s^2}{\sigma^2}\sim\chi^2_k,\mbox{ con }k=n-1.\]
       \end{minipage}}\\[3mm]

    \item Como en casos anteriores, la distribución de un estadístico muestral lleva de forma directa a obtener el intervalo de confianza:\\[3mm]
       \fbox{\begin{minipage}{14cm}
       \begin{center}
       \vspace{2mm}
       {\bf Intervalo de confianza (nivel $(1-\alpha$)) para $\sigma^2$ \textcolor{red}{(la varianza)},\\
       en una población $N(\mu,\sigma)$), con muestras de tamaño $n$.}\\
       \end{center}
       Si consideramos muestras de tamaño $n$, entonces el intervalo de confianza al nivel $(1-\alpha)$  para $\sigma^2$  es:
       \[\dfrac{(n-1)s^2}{\chi^2_{k,\alpha/2}}\leq\sigma^2\leq\dfrac{(n-1)s^2}{\chi^2_{k,1-\alpha/2}} ,\mbox{ con }k=n-1.\]
       donde, para cualquier número $u$, $\chi^2_{k,u}$ es el valor que verifica:
       \[P(Y>\chi^2_{k,u})=u.\]
       \end{minipage}}\\[3mm]
       La construcción de este intervalo parte de una idea muy sencilla: como queremos un nivel de confianza $1-\alpha$, ponemos $\alpha/2$ en cada una de las dos colas de $\chi^2_k$, y buscamos los valores críticos correspondientes, $\chi^2_{k,1-\alpha/2}$ y $\chi^2_{k,\alpha/2}$, como muestra la figura:
       \begin{center}
       \includegraphics[width=15cm]{2011-12-02-ChiCuadrado-ValoresCriticosIntervalo.png}
       \end{center}
       Sabemos entonces que:
       \[P(\chi^2_{k,1-\alpha/2}<\chi^2_k<\chi^2_{k,\alpha/2})=1-\alpha\]
       Y como
       \[(n-1)\dfrac{s^2}{\sigma^2}\sim\chi^2_k,\]
       tenemos:
       \[P\left(\chi^2_{k,1-\alpha/2}<(n-1)\dfrac{s^2}{\sigma^2}<\chi^2_{k,\alpha/2}\right)=1-\alpha\]
       y ahora podemos despejar $\sigma^2$ en estas desigualdades, teniendo en cuenta que al dar la vuelta a la fracción, las desigualdades también se invierten:
       \[P\left(
       \dfrac{1}{\chi^2_{k,\alpha/2}}<\dfrac{\sigma^2}{(n-1)s^2}<\dfrac{1}{\chi^2_{k,1-\alpha/2}}
       \right)=1-\alpha,\]
       y finalmente:
       \[P\left(
       \dfrac{(n-1)s^2}{\chi^2_{k,\alpha/2}}<\sigma^2<\dfrac{(n-1)s^2}{\chi^2_{k,1-\alpha/2}}
       \right)=1-\alpha,\]
       \begin{ejemplo}
       La desviación típica de las estaturas de 16 estudiantes seleccionados aleatoriamente en una universidad de 10000 estudiantes es de 2.40cm. Hallar un intervalo de confianza al 95\% para la desviación típica de la altura de la población de estudiantes de esa universidad.\\
       Tenemos $k=16-1=15$. Como $1-\alpha=0.95$, es $\alpha=0.05$, con lo que $\alpha/2=0.025$ y $1-\alpha/2=0.975$. Entonces, usando $R$, por ejemplo:
       \[\chi^2_{k,1-\alpha/2}\approx 6.262138, \chi^2_{k,\alpha/2}\approx27.48839\]
       Como sabemos que $s^2=(2.4)^2=5.76$, se obtiene entonces:
       \[\dfrac{15\cdot 5.76}{27.49}<\sigma^2<\dfrac{15\cdot 5.76}{6.26}\]
       o lo que es lo mismo (sacando raíces cuadradas):
       \[1.77<\sigma<3.71\]
       \qed
       \end{ejemplo}


       \item También podemos ahora establecer los contrastes de hipótesis unilaterales y bilaterales {\em (¡de nuevo, hay que prestar atención a los valores de $\chi^2_{k;u}$ utilizados en cada caso!)}:
       \\[3mm]
       \fbox{\begin{minipage}{14cm}
       \begin{center}
       \vspace{2mm}
       {\bf Contraste de hipótesis (nivel $(1-\alpha$)) para $\sigma^2$ (en $N(\mu,\sigma)$), con muestras de tamaño $n$ y $k=n-1$.}\\
       \end{center}
       Se tienen los siguientes contrastes de hipótesis:
       \begin{enumerate}
       \item[(a)] Hipótesis nula: $H_0=\{\sigma^2\leq \sigma^2_0\}$. Región de rechazo: \[\sigma_0^2<\dfrac{(n-1)s^2}{\chi^2_{k,\alpha}}.\]
       \item[(b)] Hipótesis nula: $H_0=\{\sigma^2\geq \sigma^2_0\}$. Región de rechazo: \[\sigma_0^2>\dfrac{(n-1)s^2}{\chi^2_{k,1-\alpha}}.\]
       \item[(a)] Hipótesis nula: $H_0=\{\sigma^2=\sigma^2_0\}$. Región de rechazo: \[(n-1)\dfrac{s^2}{\sigma_0^2}\mbox{ no pertenece al intervalo:}
            \left(\chi^2_{k,1-\alpha/2},\chi^2_{k,\alpha/2}\right).\]
            \quad\\
       \end{enumerate}
       \end{minipage}}\\[3mm]
       Y debería estar ya claro porque, para el contraste, hemos utilizado $\sigma^2_0$.

\end{itemize}






    \chapter{Inferencia sobre dos poblaciones}\label{cap:Inferencia2Poblaciones}
    % !Mode:: "Tex:UTF-8"

\section{Diferencia de proporciones en dos poblaciones}

\begin{itemize}

    \item En todos los problemas que hemos estudiado hasta ahora, hemos supuesto que nuestro interés se reducía a una única población, cuyas características estudiábamos. Sin embargo, en las aplicaciones de la Estadística, a menudo nos encontramos con situaciones en las que lo natural es comparar los datos procedentes de dos poblaciones, {\em precisamente para ver si existen diferencias entre ellas.} Los ejemplos son numerosos: un nuevo tratamiento que se prueba en dos grupos, mediante ensayos de tipo doble ciego, administrando el tratamiento a un grupo y un placebo al grupo de control. Lo que nos interesa es, por ejemplo, saber si la proporción de pacientes que experimentan mejoría es la misma en ambos grupos. En otro ejemplo tenemos dos poblaciones de una misma especie de árboles, y queremos estudiar si la proporción de entre ellas que están infectadas con un determinado hongo es distinta. Podríamos seguir con otros muchos ejemplos, pero lo que todos ellos tienen en común es que:
        \begin{enumerate}
            \item tenemos dos poblaciones (que llamaremos población 1 y población 2), y una misma variable aleatoria $X$, definida en ambas poblaciones. La variable $X$ representa la proporción de individuos de cada población que presentan una determinada característica. Se trata por tanto de una variable de tipo Bernouilli, pero el parámetro $p$ (la proporción) puede ser distinto en las dos poblaciones. Así que tenemos que usar dos símbolos, $p_1$ y $p_2$, para referirnos a las proporciones en cada una de las poblaciones.
            \item Tomamos dos muestras aleatorias, una en cada población, de tamaños $n_1$ y $n_2$ respectivamente. Y para cada una de esas muestras calculamos la proporción muestral; se obtendrán, de nuevo, dos valores $\hat p_1$ y $\hat p_2$.
            \item El objetivo de nuestro estudio es comparar ambas proporciones, analizando la diferencia $p_1-p_2$. Y, como en secciones precedentes, lo que queremos es obtener intervalos de confianza para $p_1-p_2$, y poder realizar contrastes de hipótesis sobre esa diferencia.
        \end{enumerate}
    \item Una vez planteado el problema, los pasos que hay que dar están claros. Empezamos haciendo alguna suposición sobre el comportamiento estadístico de las dos poblaciones. En concreto vamos a suponer, para empezar, que ambas muestras son suficientemente grandes, y que $\hat p_1$ y $\hat p_2$ no son demasiado pequeñas (ni demasiado cercanas a $1$).{\sf Si se cumplen las condiciones
        \[n_1\cdot\hat p_1>5,\qquad n_1\cdot\hat q_1>5,\qquad  n_2\cdot\hat p_2>5,\qquad n_2\cdot\hat q_2>5,\]
        entonces las dos poblaciones se comportan aproximadamente como las normales $N(n_1p_1,\sqrt{n_1p_1q_1})$ y $N(n_2p_2,\sqrt{n_2p_2q_2})$ respectivamente.}

        A partir de esta información, obtenemos una distribución para el estadístico muestral que nos interesa, que es la diferencia $\hat p_1-\hat p_2$. Vimos, en el Capítulo 8, (pág. \pageref{sec:InferenciaEstadisticaSobreProporciones}), que en estas condiciones las proporciones muestrales tienen una distribución muy aproximadamente normal, concretamente:
        \[\hat p_1\sim N\left(p_1,\sqrt{\dfrac{\hat p_1\cdot\hat q_1}{n_1}}\right),\qquad\mbox{ y análogamente }\qquad \hat p_2\sim N\left(p_2,\sqrt{\dfrac{\hat p_2\cdot\hat q_2}{n_2}}\right).\]
        Eso significa que la diferencia $\hat p_1-\hat p_2$ es --muy aproximadamente-- la diferencia de dos distribuciones normales. Y (como en la discusión de la página \pageref{ObtenerDistribucionCuasivarianzaMuestral} (Capítulo 9, ver nota a pie de página), eso significa que la diferencia se puede aproximar ella misma por una normal, de modo que es:\\[3mm]
        \fbox{\begin{minipage}{14cm}
         \begin{center}
         \vspace{2mm}
         {\bf Distribución muestral de la diferencia de proporciones}\\
         \end{center}
         Si se cumplen las condiciones
        \[n_1\cdot\hat p_1>5,\qquad n_1\cdot\hat q_1>5, \qquad n_2\cdot\hat p_2>5,\qquad n_2\cdot\hat q_2>5,\]
        entonces la diferencia de proporciones se puede aproximar por esta distribución normal:
        \[\hat p_1-\hat p_2\sim N\left(p_1-p_2,\sqrt{\dfrac{\hat p_1\cdot\hat q_1}{n_1}+\dfrac{\hat p_2\cdot\hat q_2}{n_2}}\right)\]
         \end{minipage}}\\[3mm]

    \item Y, como de costumbre, con la distribución muestral obtenemos nuestros objetivos, en forma de intervalos y contrastes:\\[3mm]
       \fbox{\begin{minipage}{14cm}
       \begin{center}
       \vspace{2mm}
       {\bf Intervalo de confianza (nivel $(1-\alpha$)) para la diferencia de proporciones $p_1-p_2$, \\
       con muestras de tamaño grande.}\\
       \end{center}
       Si se cumplen las condiciones
        \[n_1\cdot\hat p_1>5,\qquad n_1\cdot\hat q_1>5, \qquad n_2\cdot\hat p_2>5,\qquad n_2\cdot\hat q_2>5,\]
       entonces el intervalo de confianza al nivel $(1-\alpha)$  para $p_1-p_2$  es:
       \[(p_1-p_2)=(\hat p_1-\hat p_2)\pm z_{\alpha/2}\sqrt{\dfrac{\hat p_1\cdot\hat q_1}{n_1}+\dfrac{\hat p_2\cdot\hat q_2}{n_2}}\]
       siendo $z_{\alpha/2}$ el valor crítico de la normal estándar, que cumple $P(Z>z_{\alpha/2})=\alpha/2$.
       \begin{center}
       \vspace{2mm}
       {\bf Contraste de hipótesis (nivel $(1-\alpha$)) para la diferencia de proporciones $p_1-p_2$, \\
       con muestras de tamaño grande.}
       \end{center}
       Si se cumplen las condiciones
        \[n_1\cdot\hat p_1>5,\qquad n_1\cdot\hat q_1>5, \qquad n_2\cdot\hat p_2>5,\qquad n_2\cdot\hat q_2>5,\]
       y se define:
       \[\hat p=\dfrac{n_1\hat p_1+n_2\hat p_2}{n_1+n_2},\quad \hat q=1-\hat p\]
       (es decir, $\hat p$ es la media ponderada de las proporciones muestrales), entonces se tienen los siguientes contrastes de hipótesis:
       \begin{enumerate}
       \item[(a)] Hipótesis nula: $H_0=\{p_1\leq p_2\}$. Región de rechazo: \[\hat p_1>\hat p_2+z_{\alpha}\sqrt{\hat p\hat q\left(\dfrac{1}{n_1}+\dfrac{1}{n_2}\right)}.\]
       \item[(b)] Hipótesis nula: $H_0=\{p_1\geq p_2\}$. Región de rechazo: cambia $p_1$ por $p_2$ y usa el anterior.
       \item[(a)] Hipótesis nula: $H_0=\{p_1=p_2\}$. Región de rechazo: cuando $\hat p_1-\hat p_2$ no pertenece al intervalo de confianza calculado más arriba.
            \quad\\
       \end{enumerate}
       \end{minipage}}\\[3mm]
       \begin{ejemplo}(Tomado de {\em Estadística para Biología y Ciencias de la Salud, 3a. ed.}, J.Susan Milton, Ed.MacGraw-Hill, págs. 276 a 278.)
       En un estudio sobre el uso de la prednisona en el tratamiento de pacientes renales, se utilizaron 72 sujetos en 19 hospitales. De los 34 pacientes tratados con prednisona, sólo uno sufrió insuficiencia renal. Sin embargo, de los 38 que recibieron un placebo, se produjo insuficiencia renal en 10. Construir  un intervalo de confianza al $95\%$ de la diferencia de tasas de insuficiencia renal entre quienes recibieron prednisona y quienes recibieron un placebo.\\
       En este ejemplo tenemos
       \[n_1=34,\quad \hat p_1=\dfrac{1}{34}\approx 0.02941,\quad n_2=38,\quad \hat p_2=\dfrac{10}{38}\approx 0.2632\]
       Y puesto que $1-\alpha=0.95$, es $\alpha/2=0.025$ y $z_{\alpha/2}=1.9600$. Por lo tanto, el intervalo pedido es:
       \[(p_1-p_2)=(0.02941-0.2632)\pm 1.9600\sqrt{\dfrac{0.02941\cdot(1-0.2941)}{34}+\dfrac{0.2632\cdot(1-0.2632)}{38}}=\]
       \[=-0.2338\pm 0.1511\]
       o lo que es lo mismo,
       \[-0.3848<p_1-p_2<-0.0827\]
       El hecho de que este intervalo no incluya al cero se puede considerar como una prueba estadística (al 95\%) de que las dos proporciones son realmente diferentes. ¿Lo son al $99\%$?\qed
       \end{ejemplo}

       \item En Calc puedes usar esta \textattachfile{Cap10-IntervaloConfianzaDiferenciaProporcionesMuestrasGrandes.ods}{\textcolor{blue}{hoja de Cálculo}} para realizar las cuentas necesarias en este ejemplo, y otros similares. En R, además de este  \textattachfile{Cap10--IntervaloConfianzaDiferenciaProporcionesPoblacionesNormalesMuestrasGrandes.R}{\textcolor{blue}{fichero de instrucciones}}, disponemos del comando {\tt prop.test} que permite obtener intervalos de confianza (y contrastes de hipótesis) sobre proporciones. Para calcular este ejemplo teclearíamos
       \[\mbox{\tt prop.test(c(1,10),c(34,38),conf.level=0.95,correct=F)}\]
       Como se ve, tenemos que introducir dos vectores, que representa respectivamente los éxitos (el vector {\tt c(1,10)}) y el tamaño de las muestras (el vector {\tt c(34,38)}). Indicamos el nivel de confianza mediante {\tt conf.level=0.95},y finalmente la opción {\tt correct=F} desactiva ciertas correcciones estadísticas avanzadas (que no hemos visto en este curso), para que obtengamos el intervalo en la forma que queremos.
       La parte de la respuesta que se obtiene que nos interesa es:
       \begin{verbatim}
        95 percent confidence interval:
         -0.38483372 -0.08265854
       \end{verbatim}



\end{itemize}

\section{Diferencia de medias en dos poblaciones}\label{sec:diferenciaMediasDosPoblaciones}

\begin{itemize}

    \item Vamos a estudiar ahora un problema similar al anterior. De nuevo tenemos dos poblaciones, y una variable aleatoria $X$ definida en ambas, pero ahora --en lugar de la diferencia de proporciones-- lo que queremos es estudiar la diferencia entre las medias $\mu_1$ y $\mu_2$. Este problema también aparece muy a menudo en el mismo tipo de aplicaciones que hemos visto en el caso de proporciones. Por ejemplo, después de aplicar un tratamiento, queremos saber si el nivel medio de azúcar en sangre de los pacientes ha disminuido, comparado con los del grupo de control que han recibido un placebo. Este problema se formula de manera natural como una pregunta sobre la diferencia de valores medios en ambos grupos.

    \item Empezamos suponiendo que, en ambas poblaciones, la variable $X$ tiene un comportamiento aproximadamente normal (por ejemplo, esto sucede si ambas muestras son grandes, $n_1>30$ y $n_2>30$). Sean $X_1$ y $X_2$, para distinguirlas, las distribuciones de $X$ respectivamente en cada una de las poblaciones. Estamos suponiendo que
        \[X_1\sim N\left(\mu_1,\sigma_1\right),\qquad\mbox{ y que }\qquad X_2\sim N\left(\mu_2,\sigma_2\right).\]
        Entonces, si llamamos $\bar X_1$ y $\bar X_2$ respectivamente a las medias muestrales en cada una de las poblaciones, el Teorema Central del Límite  nos permite afirmar que
        \[\bar X_1\sim N\left(\mu_1,\dfrac{\sigma_1}{\sqrt{n_1}}\right),\qquad\mbox{ y que }\qquad \bar X_2\sim N\left(\mu_2,\dfrac{\sigma_2}{\sqrt{n_2}}\right).\]
        Por lo tanto, sin la menor duda, la diferencia $\bar X_1-\bar X_2$ es una normal. Concretamente:
        \[\bar X_1-\bar X_2\sim N\left(\mu_1-\mu_2,\sqrt{\dfrac{\sigma_1^2}{n_1}+\dfrac{\sigma_2^2}{n_2}}\right)\]


        El problema, como ya nos sucedió en el caso de una única población, consiste en saber {si las varianzas de las poblaciones originales pueden considerarse conocidas}. Si es así, entonces los intervalos de confianza y contrastes se pueden obtener directamente a partir de esta distribución muestral de la diferencia de medias.  Si no es así, se hace necesario aplicar una serie de modificaciones que vamos a enumerar:
        \begin{enumerate}
            \item {\sf Si ambas  muestras son grandes}, debemos recurrir a reemplazar las varianzas $\sigma_1^2$ y $\sigma_2^2$ por las cuasivarianzas muestrales $s_1^2$ y $s_2^2$;  en este caso se usa:
                \[\sqrt{\dfrac{s_1^2}{n_1}+\dfrac{s_2^2}{n_2}}\]
                en lugar de
                \[\sqrt{\dfrac{\sigma_1^2}{n_1}+\dfrac{\sigma_2^2}{n_2}}\]
                y podemos recurrir todavía a los valores críticos de la normal estándar $Z$.
            \item {\sf Si las muestras no son suficientemente grandes, pero sabemos que las poblaciones son normales, y (aunque no las conozcamos) podemos suponer que las varianzas son iguales}, entonces podemos usar la distribución $t$ de Student con $n_1+n_2-2$ grados de libertad, y además debemos recurrir a reemplazar las varianzas $\sigma_1^2$ y $\sigma_2^2$ por una combinación de las cuasivarianzas muestrales $s_1^2$ y $s_2^2$;  concretamente usamos:
                \[\sqrt{\left(\dfrac{(n_1-1)s_1^2+(n_2-1)s_2^2}{n_1+n_2-2}\right)\left(\dfrac{1}{n_1}+\dfrac{1}{n_2}\right)}\]
                en lugar de \[\sqrt{\dfrac{\sigma_1^2}{n_1}+\dfrac{\sigma_2^2}{n_2}}\]
            \item {\sf Si las muestras no son suficientemente grandes, pero sabemos que las poblaciones son normales, y {\sc no} podemos suponer que las varianzas son iguales}, entonces se usa:
                \[\sqrt{\dfrac{s_1^2}{n_1}+\dfrac{s_2^2}{n_2}}\]
                en lugar de
                \[\sqrt{\dfrac{\sigma_1^2}{n_1}+\dfrac{\sigma_2^2}{n_2}}\]
                y todavía podemos usar la distribución $t$ de Student. Pero los grados de libertad son más complicados de obtener. Se suele utilizar $t_f$, donde $f$ el número entero más próximo a
                \begin{equation}\label{ecu:aproximacionWelch}
                \dfrac{\left(\dfrac{s_1^2}{n_1}+\dfrac{s_2^2}{n_2}\right)^2}{\dfrac{1}{n_1+1}\left(\dfrac{s_1^2}{n_1}\right)^2+\dfrac{1}{n_2+1}\left(\dfrac{s_2^2}{n_2}\right)^2}-2
                \end{equation}
                Esta expresión se conoce como {\em aproximación de Welch.}
            \item Finalmente, si las muestras son pequeñas, y no podemos asegurar que las poblaciones sean normales, entonces debemos utilizar {\em métodos de inferencia no paramétricos}, más complicados que lo que vamos a ver en este curso.
        \end{enumerate}

        \item El resultado de toda esta información muestral se puede ver en la Tabla \ref{tabla:IntervalosConfianzaContrastesParaDiferenciaMedias} (página \pageref{tabla:IntervalosConfianzaContrastesParaDiferenciaMedias}).        Queremos dejar claro que {\em no es necesario, desde luego, recordar todas estas fórmulas}. Lo que debemos tener claro es la existencia de esta división en casos (a), (b), (c) y (d), y, llegado el caso, buscar las fórmulas adecuadas para cada uno de ellos.


       \begin{table}[h]
       \caption{\small Tabla de intervalos de confianza y contrastes para la diferencia de medias\label{tabla:IntervalosConfianzaContrastesParaDiferenciaMedias}}
       \fbox{\begin{minipage}{14cm}
       \begin{center}
       \vspace{2mm}
       {\bf Intervalos de confianza y contraste de hipótesis (nivel $(1-\alpha$))\\ para la diferencia de medias $\mu_1-\mu_2$}
       \end{center}
       %%%%%%%%%%%%%%%%%%%%%%%%%%%%%%%%%%
       {\bf (a) Poblaciones normales, varianzas conocidas}.\\
       {\bf Intervalo:}  $\displaystyle(\mu_1-\mu_2)=(\bar X_1-\bar X_2)\pm z_{\alpha/2}\sqrt{\dfrac{\sigma_1^2}{n_1}+\dfrac{\sigma_2^2}{n_2}}$\\
       {\bf Contraste de hipótesis:\\}
       Hipótesis nula: $H_0=\{\mu_1\leq\mu_2\}$. Región de rechazo: $\bar X_1>\bar X_2+\displaystyle z_{\alpha}{\sqrt{\frac{\sigma_1^2}{n_1}+\frac{\sigma_2^2}{n_2}}}$.\\
       Para contrastar $\mu_2\leq\mu_1$, simplemente intercambiar las poblaciones.\\
       Hipótesis nula: $H_0=\{\mu_1=\mu_2\}$. Región de rechazo: $|\bar X_1-\bar X_2|>\displaystyle{z_{\alpha/2}}{\sqrt{\frac{\sigma_1^2}{n_1}+\frac{\sigma_2^2}{n_2}}}$.\\
       %%%%%%%%%%%%%%%%%%%%%%%%%%%%%%%%%%%%
       {\bf (b) Ambas muestras grandes ($>30$), varianzas desconocidas}:\\
       {\bf Intervalo:}      $\displaystyle(\mu_1-\mu_2)=(\bar X_1-\bar X_2)\pm z_{\alpha/2}\sqrt{\dfrac{s_1^2}{n_1}+\dfrac{s_2^2}{n_2}}$\\
       {\bf Contraste de hipótesis:\\}
       Hipótesis nula: $H_0=\{\mu_1\leq\mu_2\}$. Región de rechazo: $\bar X_1>\bar X_2+\displaystyle{z_{\alpha}}{\sqrt{\frac{s_1^2}{n_1}+\frac{s_2^2}{n_2}}}$.\\
       Para contrastar $\mu_2\leq\mu_1$, simplemente intercambiar las poblaciones.\\
       Hipótesis nula: $H_0=\{\mu_1=\mu_2\}$. Región de rechazo: $|\bar X_1-\bar X_2|>\displaystyle{z_{\alpha/2}}{\sqrt{\frac{s_1^2}{n_1}+\frac{s_2^2}{n_2}}}$.\\
       %%%%%%%%%%%%%%%%%%%%%%%%
       {\bf (c) Muestras pequeñas, varianzas desconocidas pero iguales}:\\
       {\bf Intervalo:}\\
       \[(\mu_1-\mu_2)=(\bar X_1-\bar X_2)\pm \textcolor{red}{t_{n_1+n_2-2;\alpha/2}}\sqrt{\left(\dfrac{(n_1-1)s_1^2+(n_2-1)s_2^2}{n_1+n_2-2}\right)\left(\dfrac{1}{n_1}+\dfrac{1}{n_2}\right)}\]
       {\bf Contraste de hipótesis:\\}
       Hipótesis nula: $H_0=\{\mu_1\leq\mu_2\}$. Región de rechazo:
       \[\bar X_1>\bar X_2+\displaystyle{\textcolor{red}{t_{n_1+n_2-2;\alpha}}}{\sqrt{\left(\dfrac{(n_1-1)s_1^2+(n_2-1)s_2^2}{n_1+n_2-2}\right)\left(\dfrac{1}{n_1}+\dfrac{1}{n_2}\right)}}.\]
       Para contrastar $\mu_2\leq\mu_1$, simplemente intercambiar las poblaciones.\\
       Hipótesis nula: $H_0=\{\mu_1=\mu_2\}$. Región de rechazo:
       \[|\bar X_1-\bar X_2|>\displaystyle{\textcolor{red}{t_{n_1+n_2-2;\alpha/2}}}{\sqrt{\left(\dfrac{(n_1-1)s_1^2+(n_2-1)s_2^2}{n_1+n_2-2}\right)\left(\dfrac{1}{n_1}+\dfrac{1}{n_2}\right)}}.\]
       %%%%%%%%%%%%%%%%%%%%%%%%%
       {\bf (d) Muestras pequeñas, varianzas desconocidas y distintas}:\\
       {\bf Intervalo:}       $\displaystyle(\mu_1-\mu_2)=(\bar X_1-\bar X_2)\pm \textcolor{red}{t_{f;\alpha/2}}\sqrt{\dfrac{s_1^2}{n_1}+\dfrac{s_2^2}{n_2}}$\\
       {\bf Contraste de hipótesis:\\}
       Hipótesis nula: $H_0=\{\mu_1\leq\mu_2\}$. Región de rechazo: $\bar X_1>\bar X_2+\textcolor{red}{t_{f;\alpha}}\sqrt{\dfrac{s_1^2}{n_1}+\dfrac{s_2^2}{n_2}}$.\\
       Para contrastar $\mu_2\leq\mu_1$, simplemente intercambiar las poblaciones.\\
       Hipótesis nula: $H_0=\{\mu_1=\mu_2\}$. Región de rechazo: $|\bar X_1-\bar X_2|>\textcolor{red}{t_{f;\alpha/2}}\sqrt{\dfrac{s_1^2}{n_1}+\dfrac{s_2^2}{n_2}}$.\\
       siendo $f$ el entero más próximo a (aproximación de Welch, Ecuación \ref{ecu:aproximacionWelch}):
       \[\dfrac{\left(\dfrac{s_1^2}{n_1}+\dfrac{s_2^2}{n_2}\right)^2}{\dfrac{1}{n_1+1}\left(\dfrac{s_1^2}{n_1}\right)^2+\dfrac{1}{n_2+1}\left(\dfrac{s_2^2}{n_2}\right)^2}-2.\]
       \end{minipage}}\\[3mm]
       \end{table}






    \end{itemize}



\section{Inferencias para comparar varianzas en poblaciones normales. Distribución $F$ de Fisher-Snedecor}

\begin{itemize}

    \item Hemos visto en el apartado anterior que, para hacer inferencia sobre la diferencia de medias entre dos poblaciones normales, es necesario en ocasiones saber si las varianzas de ambas poblaciones son iguales (aunque no sepamos sus valores). Naturalmente, si conocemos esas varianzas, basta con compararlas. Pero, como hemos discutido en otras ocasiones, a menudo es poco realista asumir, para la inferencia, que se conocen esas varianzas, o incluso si coinciden o no.

    \item Necesitamos por lo tanto pensar en algún tipo de pregunta que nos permita saber si los dos números $\sigma_1^2$ y $\sigma_2^2$ son o no iguales. A poco que se piense sobre ello, hay dos candidatos naturales:
        \begin{enumerate}
        \item Podemos estudiar la diferencia $\sigma_1^2-\sigma_1^2$ y ver si está cerca de $0$.
        \item O podemos estudiar el cociente $\dfrac{\sigma_1^2}{\sigma_2^2}$.
        \end{enumerate}
        ¿Cuál de los dos es el más adecuado? Esta segunda reflexión ya es un poco más sutil. Es conveniente pensar sobre un ejemplo. Supongamos que $\sigma_1^2=\dfrac{1}{1000}$, $\sigma_2^2=\dfrac{1}{1000000}$. Entonces
        \[\sigma_1^2-\sigma_1^2=0.000999,\quad\mbox{mientras que }\quad\dfrac{\sigma_1^2}{\sigma_2^2}=1000.\]
        A la vista de este ejemplo, la situación empieza a estar más clara. La diferencia  $\sigma_1^2-\sigma_1^2$ tiene el inconveniente de la {\em sensibilidad a la escala} en la comparación. Si empezamos con dos números {\em pequeños} (en las unidades del problema), entonces su diferencia es asimismo {\em pequeña} en esas unidades. Pero eso no impide que uno de los números sea órdenes de magnitud (miles de veces) más grande que el otro. En cambio, el cociente no tiene esta dificultad. Si el cociente de dos números es cercano a uno, podemos asegurar que los dos números son realmente parecidos, con independencia de su tamaño.

    \item Por las razones expuestas, vamos a utilizar el cociente
        \[\dfrac{\sigma_1^2}{\sigma_2^2},\]
        y trataremos de estimar si este cociente es un número cercano a uno. ¿Cómo podemos estimar ese cociente? Parece que el candidato natural para la estimación sería el cociente de las cuasivarianzas muestrales:
        \[\dfrac{s_1^2}{s_2^2}.\]
        Y el siguiente paso para la inferencia está claro: {\em ¿cuál es la distribución muestral de este cociente?}

    \item Para responder a esta pregunta, recordemos que si $n_1$ y $n_2$ son los tamaños muestrales en ambas poblaciones, entonces  \[k_1\dfrac{s_1^2}{\sigma_1^2}\sim\chi^2_{k_1},\quad\mbox{y análogamente }k_2\dfrac{s_2^2}{\sigma_2^2}\sim\chi^2_{k_2},\quad\mbox{ con }k_1=n_1-1,\quad k_2=n_2-1.\]
        Y por lo tanto, dividiendo:
        \[\dfrac{s_1^2/s_2^2}{\sigma_1^2/\sigma_2^2}\sim\dfrac{\chi^2_{k_1}/k_1}{\chi^2_{k_2}/k_2}.\]
        Esta relación está a un paso de lo que necesitamos para empezar la inferencia (intervalos y contrastes), ...si supiéramos cómo se comporta el cociente de dos distribuciones de tipo $\chi^2$. Para describir estos cocientes necesitamos la última de las grandes distribuciones clásicas de la Estadística.\\[3mm]
        \fbox{\begin{minipage}{14cm}
        \begin{center}
        \vspace{2mm}
        {\bf Distribución $F$ de Fisher-Snedecor.}\\
        \end{center}
        Una variable aleatoria $Y$ de la forma
        \[\dfrac{\chi^2_{k_1}/k_1}{\chi^2_{k_2}/k_2}\]
        es una variable de tipo Fisher-Snedecor $F_{k_1,k_2}$. A veces escribimos  $F(k_1,k_2)$ si necesitamos una notación más clara.
        \end{minipage}}\\[3mm]
        Esta función recibe su nombre de los dos matemáticos que contribuyeron a establecer su uso en Estadística, \link{http://en.wikipedia.org/wiki/Ronald_Fisher}{R. Fisher} y \link{http://en.wikipedia.org/wiki/George_W._Snedecor}{G.W.Snedecor}. La función de densidad de $F_{n_1,n_2}$ (que, como en casos anteriores, no vamos a necesitar para la inferencia) es esta:
        \[f_{n_1,n_2}(x)=
        \begin{cases}
        \dfrac{1}{\beta\left(\dfrac{k_1}{2},\dfrac{k_2}{2}\right)}\left(\dfrac{k_1}{k_2}\right)^{k_1/2}\dfrac{x^{\frac{k_1}{2}-1}}{\left(1+\frac{k_1}{k_2}x\right)^{\frac{k_1+k_2}{2}}}&x\geq 0\\[6mm]
        0&x<0
        \end{cases}
        \]
        donde $\beta$ es, de nuevo, la \link{http://en.wikipedia.org/wiki/Beta_function}{función beta} que ya apareció en relación con la $t$ de Student. En este \textattachfile{Cap10_FisherSnedecor.html}{\textcolor{blue}{documento html}} (se abre en el navegador, requiere Java) puedes observar la forma de la distribución $F_{k_1,k_2}$ para distintos valores de $k_1$ y $k_2$. Su aspecto típico es como el que se muestra en esta figura:
        \begin{center}
        \includegraphics[width=15cm]{2011-12-13-FisherSnedecor.png}
        \end{center}
        y, como puede verse, es otra vez --como sucedía con la $\chi^2$-- una distribución asimétrica, claramente no normal.


    \item En Calc y Excel \link{http://office.microsoft.com/es-es/excel-help/distr-f-HP005209087.aspx}{disponemos de las funciones} {\tt DISTR.F} y {\tt DISTR.F.INV} para resolver problemas directos e inversos relacionados con la distribución $F_{k_1,k_2}$. Ambas funciones trabajan con colas derechas de la distribución, de manera que, por ejemplo, el resultado de Calc:
        \[\mbox{\tt DISTR.F(1,5;10;60)=0,161863136}\]
        significa que
        \[P( F_{10,60}>1.5)=0,161863136\]
        Y, recíprocamente, el resultado
        \[\mbox{\tt DISTR.F.INV(0,05;10;60)=1,9925919966}\]
        significa que $y=1,9925919966$ es la solución del problema inverso:
        \[P( F_{10,60}>y)=0,05.\]
        En R se pueden utilizar las funciones {\tt pf} y {\tt qf} para resolver problemas directos e inversos, usando, como siempre en R, la cola izquierda de la distribución. Por lo tanto en R se tiene:
        \[\mbox{\tt 1-pf(1.5,df1=10,df2=60)=0.1618631}\]
        para resolver el mismo problema directo que hicimos con Calc (observa el {\tt 1-pf} que hemos utilizado, al tratarse de un problema de cola derecha). Y se tiene
        \[\mbox{\tt qf(1-0.05,df1=10,df=60)=1.992592}\]
        para resolver el anterior problema inverso. Observa que, en este caso, hemos usado {\tt qf(1-) } al tratarse de un problema de cola derecha.

    \item Ahora que ya nos hemos familiarizado con la distribución $F$, podemos volver a la inferencia sobre la diferencia de varianzas en el punto en el que la habíamos dejado. Podemos resumir lo que hemos estado haciendo diciendo que:
        \[\dfrac{s_1^2/s_2^2}{\sigma_1^2/\sigma_2^2}\sim\dfrac{\chi^2_{k_1}/k_1}{\chi^2_{k_2}/k_2}=F_{k_1,k_2}\]
        Observa que si fuese $\sigma_1^2=\sigma_2^2$, entonces $s1^2/s_2^2$ se comportaría directamente como $F_{k_1,k_2}$. A partir de esta información vamos a obtener directamente los intervalos de confianza y contrastes de hipótesis necesarios. Como en casos anteriores, {\sf llamaremos $f_{k_1,k_2;\alpha}$ al valor que tiene la propiedad de que:}
        \[P(F_{k_1,k_2}\leq f_{k_1,k_2;\alpha})=1-\alpha\]
        como en esta figura:
        \begin{center}
        \includegraphics[width=15cm]{2011-12-13-FisherSnedecor-ValoresCriticos.png}
        \end{center}
        Y para calcular estos valores usamos, como hemos dicho, {\tt DISTR.F.INV} en Calc, o {\tt qf} en R. Con estos ingredientes
        \\[3mm]
       \fbox{\begin{minipage}{14cm}
       \begin{center}
       \vspace{2mm}
       {\bf Intervalo de confianza (nivel $(1-\alpha$)) para $\frac{\sigma_1^2}{\sigma_2^2}$, en dos poblaciones normales.}\\
       \end{center}
       Si consideramos muestras independientes de tamaños $n_1$ y $n_2$ respectivamente, entonces el intervalo de confianza al nivel $(1-\alpha)$  para $\frac{\sigma_1^2}{\sigma_2^2}$  es:
       \[\dfrac{s_1^2}{s_2^2}\cdot\dfrac{1}{f_{k_1,k_2;\alpha/2}}\leq\frac{\sigma_1^2}{\sigma_2^2}\leq \dfrac{s_1^2}{s_2^2}\cdot\dfrac{1}{f_{k_1,k_2;1-\alpha/2}}.\]
       con $k_1=n_1-1$, $k_2=n_2-1$.
       \end{minipage}}\\[3mm]
        Aquí tienes una \textattachfile{Cap10-IntervaloConfianzaCocienteVarianzas.ods}{\textcolor{blue}{hoja de cálculo}}, y también un \textattachfile{Cap10-IntervaloConfianzaCocienteVarianzas2PoblacionesNormales.R}{\textcolor{blue}{fichero de instrucciones R}} preparados para obtener estos intervalos de confianza.

       \begin{ejemplo}(Adaptado --y corregido-- de {\em Curso y ejercicios de estadística}, V.Quesada, A.Isidoro, L.A.López. Ed. Alhambra 1984.) Se está haciendo un estudio sobre hipertensión. Se toma una muestra de trece pacientes de una ciudad, y en otra ciudad se toma una muestra de dieciséis pacientes. Se obtienen las siguientes medias y cuasidesviaciones típicas muestrales (en mm de mercurio):
       \[\bar X_1=166,\quad  \bar X_2=164.7,\quad s_1=28,\quad s_2=7.\]
       Si queremos calcular un intervalo de confianza (o hacer un contraste) para la diferencia de medias, puesto que las poblaciones son pequeñas, necesitamos saber si las varianzas de ambas poblaciones se pueden suponer iguales. Para ello calculamos el intervalo de confianza de $\frac{\sigma_1^2}{\sigma_2^2}$, usando Calc o R. Se obtiene (al 99\%)
       \[3.3889\leq\dfrac{\sigma_1^2}{\sigma_2^2}\leq 67.9960\]
       Y, como este intervalo no contiene al $1$, podemos concluir con un 99\% de confianza que las varianzas de ambas poblaciones no son iguales. Esta información serviría de guía en la inferencia sobre $\mu_1-\mu_2$.
       \qed
       \end{ejemplo}


        Una observación: en algunos libros, para expresar ese intervalo de confianza, se utiliza esta propiedad de los valores críticos de la distribución $F$
        \[f_{k_1,k_2;\alpha}=\dfrac{1}{f_{k_2,k_1;1-\alpha}}.\]
        Esta propiedad permitía disminuir el volumen de las tablas que se incluyen en esos libros. Pero, dado que nosotros vamos a calcular esos valores usando el ordenador, preferimos la expresión que aparece más arriba.


       \item Finalmente, aquí están los contrastes de hipótesis unilaterales y bilaterales:
       \\[3mm]
       \fbox{\begin{minipage}{14cm}
       \begin{center}
       \vspace{2mm}
       {\bf Contraste de hipótesis (nivel $(1-\alpha$)) para $\frac{\sigma_1^2}{\sigma_2^2}$, en dos poblaciones normales.}\\
       \end{center}
       Se tienen los siguientes contrastes de hipótesis:
       \begin{enumerate}
       \item[(a)] Hipótesis nula: $H_0=\{\sigma_1^2\leq \sigma_2^2\}$. Región de rechazo: \[\dfrac{s_1^2}{s_2^2}>f_{k_1,k_2;\alpha}.\]
       \item[(b)] Para $H_0=\{\sigma_1^2\geq \sigma_2^2\}$, intercambiar los papeles de ambas poblaciones.
       \item[(a)] Hipótesis nula: $H_0=\{\sigma_1^2=\sigma_2^2\}$. Región de rechazo:
       \[\dfrac{s_1^2}{s_2^2}\mbox{ no pertenece al intervalo:}
            \left(f_{k_1,k_2;1-\alpha/2},f_{k_1,k_2;\alpha/2}\right).\]
            \quad\\
       \end{enumerate}
       \end{minipage}}\\[3mm]


\end{itemize}




\part{Inferencia sobre la relación entre dos variables}
    % !Mode:: "Tex:UTF-8"

\section*{Introducción}\label{part04:intro}
Todo nuestro trabajo, hasta ahora, ha consistido en el estudio de una única variable aleatoria. Incluso en el anterior capítulo, hemos partido de la idea de que teníamos dos poblaciones, pero la variable que observábamos en ambas era la misma. Sin embargo, está claro que en muchos problemas, nos interesan simultáneamente varias variables distintas de una población. Y más concretamente, {\sf nos interesan las relaciones que pueden existir entre esas variables.}

El modelo matemático ideal de relación entre dos variables se refleja en la noción de {\sf función}. La idea intuitiva de función, en matemáticas, es que tenemos una expresión como:
\[y=f(x),\]
donde $x$ e $y$ representan {\sf variables}, y $f$ es una {\em fórmula} o {\em procedimiento}, que permite calcular valores de la variable $y$ a partir de valores de la variable $x$. Un ejemplo típico sería una expresión como
\[y=\frac{x}{x^2+1}.\]
Aquí la fórmula que define la función es
\[f(x)=\frac{x}{x^2+1}.\]
Dada un valor de $x$, sea un número real cualquiera, como por ejemplo $x=2$, sustituimos ese valor en la fórmula y obtenemos
\[y=\frac{2}{2^2+1}=\frac{2}{5}.\]
En este contexto la variable $x$ se llama {\sf independiente}, mientras que la $y$ es la {\sf variable dependiente}. Y en este concepto de función: el valor de $y$ que se obtiene está absolutamente {\em determinado} por el valor de $x$. No hay ninguna incertidumbre, nada aleatorio, en relación con el vínculo entre la $y$ y la $x$.

Sin embargo, cuando se estudian problemas del mundo real, las relaciones entre variables son mucho menos simples. Todos sabemos que, en general, {\em la edad de un bebé (en días) y su peso (en gramos)} están relacionados. En esa frase aparecen dos variables, edad y peso, y afirmamos que existe una relación entre ellas. Pero desde luego, no existe una fórmula que nos permita, dada la edad de un bebé, calcular su peso exacto, en el mismo sentido en el que antes hemos sustituido $2$ para obtener $2/5$. La idea de relación de la que estamos empezando a hablar tiene mucho que ver con la aleatoriedad y la incertidumbre típicas de la Estadística. Y para reflejar este tipo de {\sf relaciones inciertas} vamos a usar la notación
\[y \sim x.\]
Esta notación indica dos cosas:
\begin{enumerate}
  \item Que hablamos de la posible relación entre las variables $x$ e $y$, como hemos dicho.
  \item Pero, además, al escribirla en este orden queremos señalar que se desea utilizar los valores de la variable $x$ para, de alguna manera, predecir o explicar los valores de la variable $y$. Volveremos sobre esto muy pronto, y más detalladamente. Pero por el momento conviene irse familiarizando con la terminología. Cuando tratamos de predecir $y$ a partir de $x$, decimos que $y$ es la {\sf variable respuesta (response variable)}, y que $x$ es la {\sf variable explicativa (explanatory variable)}.
\end{enumerate}

En esta parte del curso vamos a extender los métodos que hemos aprendido al estudio de este tipo de relaciones entre dos variables aleatorias. Pero, como sabemos desde el principio del curso, las variables se clasifican en dos grandes tipos: cuantitativas y cualitativas (también llamadas factores). En lo que sigue, y para abreviar, usaremos una letra C mayúscula para indicar que la variable es cuantitativa y una letra F mayúscula para indicar que es un factor (cualitativa). Atendiendo al tipo de variables que aparezcan en el problema que estamos estudiando, y al papel (respuesta o explicativa) que las variables jueguen en el problema, nos vamos a encontrar con cuatro situaciones posibles, que hemos representado en la Tabla \ref{tabla:MetodosInferencia2Variables} (página \pageref{tabla:MetodosInferencia2Variables}).

        \begin{center}
        \begin{table}[h]
        \begin{tabular}{cccc}
        \cline{3-4}
        &
        &
        \multicolumn{2}{|c|}{\bf {\rule{0mm}{0.5cm}Tipo var. respuesta.}}
        \\[3mm]
        \cline{3-4}
        &
        &
        \multicolumn{1}{|c|}{\bf \rule{0mm}{0.5cm}Cuantitativa (C)}&
        \multicolumn{1}{|c|}{\bf \rule{0mm}{0.5cm}Cualitativa (F)}
        \\[3mm]
        \cline{1-4}
        \multicolumn{1}{|c}{\multirow{2}{*}[-1em]{\bf \begin{tabular}{cc}Tipo variable\\explicativa\end{tabular}}}&
        \multicolumn{1}{|c|}{\bf Cuantitativa (C)}&
        \multicolumn{1}{|c|}{\bf \rule{0mm}{0.7cm}(11) \begin{tabular}{cc}C $\sim$ C\\Regresión\end{tabular}}&
        \multicolumn{1}{|c|}{\bf (14) \begin{tabular}{cc}F $\sim$ C\\ \begin{minipage}{4cm}Regresión Logística\\y similares.\end{minipage}\end{tabular}}
        \\[3mm]
        \cline{2-4}
         \multicolumn{1}{|c}{}&
         \multicolumn{1}{|c|}{\bf Cualitativa (F)}&
         \multicolumn{1}{|c|}{\bf \rule{0mm}{0.7cm}(12) \begin{tabular}{cc} C $\sim$ F\\ ANOVA\end{tabular}}&
         \multicolumn{1}{|c|}{\bf (13) \begin{tabular}{cc} F $\sim$ F \\Contraste $\chi^2$\end{tabular}}\\[3mm]
        \cline{1-4}
        \end{tabular}
        \caption{Casos posibles en la inferencia sobre la relación entre  dos variables}
        \label{tabla:MetodosInferencia2Variables}
        \end{table}
        \end{center}


Por ejemplo, en la relación entre edad en días de un bebé y su peso en gramos, ambas variables son cuantitativas, y diremos que es una situación C $\sim$ C. Cada una de esas situaciones requiere el uso de técnicas estadísticas distintas. Hemos indicado, de forma abreviada, bajo cada una de las entradas de la tabla, el nombre de la técnica correspondiente. Y en esta parte del curso, le dedicaremos un capítulo a cada una de las técnicas; los números de esos capítulos, que aparecen entre paréntesis en la tabla, indican el orden en que vamos a proceder.

Empezaremos, en el siguiente capítulo, por la situación C $\sim$ C, porque es la más cercana al concepto familiar de función $y=f(x)$, que el lector ya conoce. Pero antes de empezar, sin embargo, queremos advertir al lector de un problema con el que vamos a tropezar varias veces en esta parte del curso. Cuando, en la segunda parte del curso, estudiamos la Probabilidad y las variables aleatorias, ya dijimos que el tratamiento que vamos a hacer de esos temas pretende mostrar al lector sólo lo necesario para hacer comprensibles las ideas fundamentales de la Estadística. Ahora, al estudiar la relación entre dos variables aleatorias, nos ocurre algo  similar. Pero las técnicas matemáticas necesarias son más complicadas; esencialmente, es como el paso de funciones de una variable (que se estudian en la matemática elemental) a las funciones de varias variables (que sólo se estudian en cursos avanzados). Afortunadamente, la intuición, que a estas alturas del curso hemos adquirido, nos va a permitir avanzar sin atascarnos en esos detalles. Pero en algunos momentos notaremos cierta resistencia a ese avance, porque nos faltan los fundamentos teóricos que se requieren. En esta ocasión vamos a aplicar a rajatabla la máxima de que no sirve de nada tener la solución antes de tener el problema. Y dejaremos para el último capítulo de esta parte del curso la discusión, somera y sin detalles técnicos, del armazón teórico necesario. Ese capítulo es una lectura opcional, y está pensado para complementar la información de los que lo preceden.


    \chapter{Regresión lineal simple.}
    % !Mode:: "Tex:UTF-8"

%\section*{\fbox{\colorbox{Gris025}{{Sesión 26. Inferencia estadística.}}}}
%
%\subsection*{\fbox{\colorbox{Gris025}{{Inferencia en la regresión lineal.}}}}
%\subsection*{Fecha: Martes, 20/12/2011, 14h.}
%
%\noindent{\bf Atención: este fichero pdf lleva adjuntos los ficheros de datos necesarios.}
%
%%\subsection*{\fbox{1. Ejemplos preliminares }}
%\setcounter{tocdepth}{1}
%%\tableofcontents

\section{Variables correlacionadas}

\begin{itemize}

    \item Por ejemplo, en el \textattachfile{p0470-p0478-Herrerillos.pdf}{\textcolor{blue}{artículo adjunto}} (ver la figura 5, de la página 473\footnote{{\em The effect of temperature and clutch size on the energetic cost of incubation in a free-living Blue Tit (Parus Caeruleus)}. S. Haftorn, R. E. Reinertsen. The Auk (102), pp.470-478, 1985.}), los investigadores estudian --entre otras cosas-- la relación entre el consumo de oxígeno y la temperatura del aire en una hembra de Herrerillo común (Parus Caeruleus), tanto cuando está incubando, como cuando no lo está.
        \begin{center}
        \includegraphics[width=12cm]{2011_12_16_Herrerillo.jpg}
        \end{center}
        Como puede verse en esa gráfica, tenemos una serie (muestra) de {\sf parejas} de datos
        \[(x_1,y_1),(x_2,y_2),(x_3,y_3),\ldots,(x_n,y_n),\]
        donde cada uno de los datos corresponde a una de las dos variables, $X$ para la coordenada horizontal, e $Y$ para la coordenada vertical. En este ejemplo $X$ representa la temperatura del aire, e $Y$ el consumo de oxígeno. Al investigador le interesa estudiar si hay alguna relación entre ambas variables. A primera vista, parece evidente que a menor temperatura, mayor consumo de oxígeno. Pero querríamos disponer de una herramienta más precisa. Algo que nos permitiera hacer predicciones. Algo como una fórmula, en la que introducir la temperatura del aire, y poder calcular el consumo de oxígeno. No se trata de hacer muchas, muchísimas medidas hasta tener cubiertas todas las temperaturas posibles, sino de usar las medidas que tenemos para establecer la relación entre esas dos variables.


    \item En matemáticas, el ejemplo más común de una relación entre dos variables $x$ e $y$ es una función $y=f(x)$, de las que en primer curso se han estudiado numerosos ejemplos: funciones polinómicas, funciones r racionales, exponenciales, logaritmos, trigonométricas, etcétera. Una de estas funciones, como por ejemplo,
        \[y=f(x)=x^2/(1+x^4)\]
        representa una relación exacta entre las variables $x$ e $y$. Este tipo de relaciones exactas se utilizan, en las aplicaciones de las matemáticas, como modelos teóricos. El modelo clásico son las leyes de la Física, como las leyes de Newton, Maxwell, etcétera. Si queremos calcular la fuerza de atracción gravitatoria $F$ entre dos cuerpos de masas $m_1$ y $m_2$, situados a distancia $r$, sabemos que, con las unidades correctas, esta fuerza viene dada por la ley de Newton:
        \[F(r)=G\dfrac{m_1\cdot m_2}{r^2}\]
        Es decir, que sustituimos aquí un valor de $r$ y obtenemos un valor de $F$, en principio -teóricamente- con toda la precisión que queramos. Pero, claro está, esa visión es una simplificación, un modelo teórico. Cuando vayamos al mundo real y tratemos de aplicar esta fórmula, surgen varios inconvenientes:
        \begin{enumerate}
            \item Ni las masas, ni las distancias, se pueden medir con una precisión infinita. (Y no es sólo porque haya errores experimentales de medida, es que además hay límites teóricos a la precisión de las medidas.)
            \item Incluso aceptando como correctas las leyes de Newton, en nuestro modelo estamos introduciendo muchas simplificaciones. Estamos considerando que esos dos cuerpos que se atraen se pueden considerar como partículas puntuales (Porque, de otra forma ¿cómo se define la distancia entre ellos? ¿cómo se define su masa?).
            \item Y además, ahora sabemos que la ley de la gravedad de Newton sólo es precisa dentro de un determinado rango de valores de los parámetros. Para escalas espaciales muy grandes o muy pequeñas, o para objetos enormemente masivos (agujeros negros, por ejemplo) o extremadamente ligeros (partículas subatómicas), sus predicciones son incorrectas, y tenemos que usar las correcciones que hizo Einstein, o las últimas teorías de gravedad cuántica, si queremos resultados precisos.
        \end{enumerate}

        Por --entre otras-- estas razones, sabemos que estas leyes son modelos teóricos, y no esperamos que sus predicciones se cumplan con precisión absoluta. Ni siquiera lo esperábamos cuando el modelo predominante en ciencia era el determinismo de Newton y Laplace
        %(nota al final, pág. \pageref{sec:notaSobreDeterminismo})
        . No es realista esperar que las observaciones se correspondan exactamente con un modelo teórico como el que refleja una ecuación del tipo $y=f(x)$. En el caso de la Biología, que estudia fenómenos y procesos muy complejos, a menudo no posible aislar las variables bajo estudio de su entorno, sin perturbar irremediablemente el propio objeto de estudio. Así que tenemos que aceptar como un hecho que la relación entre variables, en Biología, nunca es tan nítida como sucede con muchos ejemplos de la Física o la Química.

    \item Volvamos al problema, tenemos una lista de parejas de datos,
    \[(x_1,y_1),(x_2,y_2),(x_3,y_3),\ldots,(x_n,y_n),\]
    como los de la figura:
    \begin{center}
    \includegraphics[height=6cm]{2011-12-16-Regresion01.png}
    \end{center}

    que se corresponden con $n$ puntos del plano, y nos preguntamos si podemos utilizar esos puntos para establecer una relación $y=f(x)$.  Si somos capaces de hacer esto, cuando nos llegue otro valor de la variable $x$, podremos utilizar esa relación para predecir el correspondiente valor de la $y$.

    \item Naturalmente, fórmulas (es decir, funciones) hay muchas... y los matemáticos saben fabricar fórmulas distintas para distintas necesidades. Por ejemplo, usando un procedimiento que se llama interpolación, podemos fabricar un polinomio que pase por todos y cada uno de los puntos\footnote{Hay un detalle técnico: no debe haber dos puntos con la misma coordenada $x$}. Puedes ver el resultado, aplicado a la colección de puntos de la figura anterior, en este \textattachfile{Cap11_Regresion01.html}{\textcolor{blue}{documento html}} (se abre en el navegador, requiere Java). Pero si juegas un poco con esa idea, enseguida descubrirás que:
        \begin{enumerate}
        \item[(a)] la fórmula es demasiado complicada, y el grado del polinomio aumenta demasiado rápido.
        \item[(b)] peor aún: la capacidad de predicción de esta fórmula es {\em nula}: si añadimos un punto más, la curva que produce la fórmula cambia por completo, y los valores que predice no tienen nada que ver con los anteriores. Ese comportamiento es claramente indeseable. Querríamos una fórmula que fuese bastante estable al añadir o quitar un punto.
        \end{enumerate}
         ¿Cómo podemos elegir una buena fórmula? Una que, a la vez, sea sencilla, estable, y que represente bien al conjunto de puntos. Para obtener algo sencillo, conviene empezar con cosas sencillas. Así que nos preguntamos ¿cuáles son las funciones más sencillas de todas? Dejando de lado las constantes --que se pasan de sencillez-- está claro que las rectas son las funciones con las gráficas, y las ecuaciones, más simples de todas. Una recta es una función de la forma:
         \[y=a+b\cdot x\]
         donde $a$ y $b$ son dos números, la {\sf ordenada en el origen} y la  {\sf pendiente} respectivamente, cuyo significado geométrico puedes recordar en este \textattachfile{Cap11_EcuacionRectaPendienteOrdenadaOrigen.html}{\textcolor{blue}{documento html}}. Cambiando los valores de $m$ y $n$ podemos obtener todas las rectas del plano (salvo las verticales, que no necesitaremos). Y entonces podemos hacer la siguiente pregunta: de entre todas esas infinitas rectas, ¿cuál es la que mejor representa a nuestro conjunto de puntos (desde el punto de vista de la Estadística)? En esta figura
         \begin{center}
         \includegraphics[width=15cm]{2011_12_16_Regresion02.png}
         \end{center}
         puedes ver dos intentos de ajustar una recta a los datos, con bastante acierto a la izquierda, y considerablemente peor a la derecha. Y en este \textattachfile{Cap11_Regresion02a.html}{\textcolor{blue}{documento html}} puedes mover los puntos $M$ y $N$ para tratar de intuir cual es la mejor recta. Después, puedes ver la respuesta que proporciona la Estadística, y ver si has acertado:

    \item Antes de enfrascarnos en los detalles técnicos de esa pregunta, un momento de reflexión. ¿Por qué usamos rectas? Desde luego, porque son sencillas. En segundo lugar, porque hay muchas otras situaciones en las que podemos hacer un cambio de variable, y resolver el problema en las nuevas variables usando una recta. Para que se entienda mejor, si tenemos una función de la forma:
        \[y=4\cdot e^{3x+2}\]
        y pasamos el $4$ al miembro izquierdo y tomamos logaritmos, se convierte en:
        \[\ln\left(\dfrac{y}{4}\right)=3x+2\]
        Y si ahora hacemos el cambio de variables \[u=\ln\frac{y}{4},\]
        obtenemos
        \[u=3x+2\]
        que es una recta en las nuevas variables $x,u$. Hay muchas funciones --pero no todas-- que se pueden convertir en rectas mediante trucos de cambio de variable similares a este. Y hay otra propiedad de las rectas que las hace especialmente importantes. Fíjate en lo que sucede en este \textattachfile{Cap11_Regresion02.html}{\textcolor{blue}{documento html}}, y haz zoom para ver cada vez más lejos la ``presunta'' recta. ¿Sorprendido? Lo que sucede aquí es algo que deberíamos haber aprendido en Cálculo: las funciones ``razonables'', vistas de cerca (como si las mirásemos a través de un microscopio) se parecen cada vez más y más a una recta (su recta tangente en el punto en el que aplicamos el microscopio, claro). Cuando se estudia la dependencia entre dos variables, en un rango reducido de valores, lo previsible es encontrar una recta. Pero también es importante aprender la lección inversa: lo que a cierta escala parece una recta, puede ser sólo una visión demasiado local, demasiado limitada, de la relación entre las dos variables.
    \end{itemize}

\section{Error cuadrático medio, recta de regresión, correlación.}

    \begin{itemize}
    \item Vamos, entonces, a concretar. La pregunta es: de entre todas las rectas $y=a\cdot x+b$, ¿cuál es la que mejor representa --estadísticamente-- a nuestro conjunto de puntos? Para entender la respuesta, tenemos que reflexionar un poco sobre el uso que pensamos darle a la recta que vamos a obtener. El objetivo es que, una vez que tengamos la ecuación
        \[y=a+b\cdot x,\]
        cada vez que obtengamos un valor de la variable $x$ podemos utilizar esta ecuación para {\sf predecir el valor de $y$ sin medirlo}. Y esto es interesante porque, en muchos casos, pensamos en la variable $x$ como la variable {\em fácil} de medir, mientras que $y$ puede ser complicada. En el ejemplo de la hembra de herrerillo incubando, es muy fácil medir la temperatura del aire, basta con usar un termómetro, y esa medida perturba muy poco los restantes parámetros del experimento. En cambio, medir el consumo de oxígeno del pobre pajarillo obliga a colocarle algún tipo de aparato de medida. Esa operación no sólo es laboriosa, sino que debe realizarse con mucho esmero para que el propio diseño experimental no perturbe los propios parámetros que estamos tratando de medir. En resumidas cuentas, medir $y$ no es fácil, y preferiríamos poder predecirla a a partir de $x$. Por eso vamos a llamar a $x$ la {\sf variable independiente}, {\sf variable predictora}, o {\sf regresora}, mientras que $y$ es la {\sf variable dependiente} o {\sf respuesta}.

        Con esta reflexión podemos avanzar un poco más en la determinación de la recta. Lo que esperamos de esa recta es que sea buena prediciendo los valores de $y$. Nosotros la obtenemos a partir del conjunto de puntos
        \[(x_1,y_1),(x_2,y_2),(x_3,y_3),\ldots,(x_n,y_n),\]
        Pero si consideramos {\em por separado} los valores de la coordenada $x$, que son:
        \[x_1, x_2,\ldots, x_n,\]
        y los sustituimos en la ecuación de la recta, obtendremos una colección de {\em valores predichos}:
        \[\hat y_1,\hat y_2,\ldots,\hat y_n,\]
        donde, por supuesto,
        \[\hat y_i=a+b\cdot x_i,\quad\mbox{ para }i=1,\ldots,n.\]
        Y ahora podemos precisar lo que queremos: la recta será la mejor posible si estos valores predichos se parecen lo más posible a los valores iniciales de la coordenada $y$
        \[y_1, y_2,\ldots, y_n.\]
        Estamos en terreno conocido: para medir cómo se parecen esos dos conjuntos de valores consideramos las diferencias o {\sf residuos}:
        \[y_1-\hat y_1,y_2-\hat y_2,\ldots,y_n-\hat y_n,\]
        Y ¿qué hacemos, las promediamos? No, a estas alturas ya sabemos que promediar diferencias, sin más, no es una buena idea, {\em porque las diferencias positivas muy grandes pueden compensarse con diferencias negativas muy grandes, y engañarnos}. Para conseguir una información fiable, tenemos que pagar el peaje de elevar al cuadrado las diferencias, y entonces promediar:\\[3mm]
        \fbox{\colorbox{Gris025}{\begin{minipage}{14cm}
         \begin{center}
         \vspace{2mm}
         {\bf Error cuadrático medio}\\
         \end{center}
         Dado el conjunto de puntos
         \[(x_1,y_1),(x_2,y_2),(x_3,y_3),\ldots,(x_n,y_n),\]
         si consideramos
         \[\hat y_1,\hat y_2,\ldots,\hat y_n,\]
         siendo,
        \[\hat y_i=a+b\cdot x_i,\quad\mbox{ para }i=1,\ldots,n.\]
        entonces el {\sf error cuadrático medio} de la recta $y=a\cdot x+b$ es:
        \[\mbox{ECM}(y=a+b\cdot x)=\dfrac{1}{n}\sum_{i=1}^n(y_i-\hat y_i)^2=\dfrac{1}{n}\sum_{i=1}^n(y_i-a-b\cdot x_i)^2.\]
         \end{minipage}}}\\[3mm]
         El error cuadrático medio depende de los puntos $(x_i,y_i)$, y por supuesto, de la recta que se utilice. En este \textattachfile{Cap11_Regresion03.html}{\textcolor{blue}{documento html}} puedes visualizar el significado del error cuadrático medio.


    \item Una vez definido el error cuadrático medio, la búsqueda de la mejor recta se puede formular de una manera mucho más precisa: ¿cuáles son los valores $a$ y $b$ para los que la recta
    \[y=a+b\cdot x\]
    produce el valor mínimo posible de $ECM(y=a+b\cdot x)$? Una vez fijados los puntos $(x_i,y_i)$, el error ECM depende sólo de $a$ y de $b$. Así que este es un problema de máximos y mínimos, como los que se estudian en Cálculo, para una función de dos variables. Y, conociendo las herramientas necesarias, es muy fácil de resolver: se calcula un par de derivadas parciales, se igualan a cero, se resuelve el sistema, y --básicamente-- ya está. Para entender mejor la expresión de la recta que se obtiene, es conveniente introducir primero un poco de notación. Si pensamos {\em por separado} en los valores de la coordenada $x$,
    \[x_1, x_2,\ldots, x_n,\]
    y en los valores iniciales de la coordenada $y$:
    \[y_1, y_2,\ldots, y_n.\]
    podemos definir sus medias y varianzas. Para $x$ es\footnote{Usamos el símbolo $\operatorname{Var}_n$ para indicar que se divide por $n$, no por $n-1$.}:
    \[\bar x=\dfrac{\displaystyle\sum_{i=1}{n}x_i}{n},\qquad {\operatorname{Var}_n}(x)=\dfrac{\displaystyle\sum_{i=1}^{n}(x_i-\bar x)^2}{n}\]
    Y, análogamente, para la $y$ se tiene:
    \[\bar y=\dfrac{\displaystyle\sum_{i=1}{n}y_i}{n},\qquad {\operatorname{Var}_n}(y)=\dfrac{\displaystyle\sum_{i=1}^{n}(y_i-\bar y)^2}{n}\]
    Vamos a utilizar estos valores para escribir la solución del problema. Pero queremos empezar por señalar que con ellos se puede construir un punto interesante, el que tiene por coordenadas $(\bar x,\bar y)$, las medias por separado. Si $\bar x$ es un buen representante de las coordenadas $x$, y $\bar y$ es un buen representante de las coordenadas $\bar y$, ¿será verdad que la mejor recta posible tiene que pasar por ese punto $(\bar x,\bar y)$? Como vamos a ver la respuesta es afirmativa:\\[3mm]
    \fbox{\colorbox{Gris025}{\begin{minipage}{14cm}
    \begin{center}
    \vspace{2mm}
    {\bf Recta de regresión (o de mínimos cuadrados. Covarianza)}\\
    \end{center}
    Dado el conjunto de puntos
    \[(x_1,y_1),(x_2,y_2),(x_3,y_3),\ldots,(x_n,y_n),\]
    la {\sf recta de regresión o de mínimos cuadrados} es la recta que minimiza el error cuadrático medio ECM. Esa recta puede escribirse en la forma:
    \[(y-\bar y)=\dfrac{\textcolor{red}{\operatorname{cov}(x,y)}}{{\operatorname{Var}_n}(x)}(x-\bar x)\]
    siendo
    \[\operatorname{cov}(x,y)=\dfrac{\displaystyle\sum_{i=1}^{n}(x_i-\bar x)(y_i-\bar y)}{n}\]
    una nueva cantidad, que llamaremos la {\sf covarianza} de $x$ e $y$.
    \end{minipage}}}\\[3mm]
    Como no podía ser de otra manera, Calc y Excel incluyen funciones para obtener los valores necesarios. En concreto, aparte de {\tt PROMEDIO} y {\tt VARP} que ya conocemos, disponemos de las funciones\footnote{En inglés son respectivamente {\tt COVAR}, {\tt SLOPE}, {\tt INTERCEPT},}:
    \begin{enumerate}
    \item {\tt COVAR(Datos1;Datos2)}, que calcula la covarianza del conjunto de puntos.
    \item {\tt PENDIENTE(Datos\_y;Datos\_x)}, que calcula la pendiente de la recta de regresión. {\em ¡Atención al orden de las variables!}
    \item {\tt INTERSECCION.EJE(Datos\_y;Datos\_x)}, que calcula la pendiente de la recta de regresión. {\em ¡Atención al orden de las variables!}
    \end{enumerate}
    En esta \textattachfile{Cap11_RectaRegresion.ods}{\textcolor{blue}{hoja Calc}} puedes ver como se aplican todos esos cálculos a una colección aleatoria de datos, y modificándola ligeramente puedes utilizarla para otros ejemplos. También puedes usar este \textattachfile{Cap11_Wiris_RegresionLineal.html}{\textcolor{blue}{documento html con Wiris}}, en el que se calcula la recta de regresión, covarianza y correlación para los mismos datos. Y, finalmente, aquí tienes un \textattachfile{Cap11_RegresionLinealConR.R}{\textcolor{blue}{documento de instrucciones R}}
     para realizar los mismos cálculos.





\end{itemize}


\section{Análisis de la varianza. Coeficiente $r$ de correlación lineal de Pearson.}\label{sec:Anova}

    \begin{itemize}

    \item Ahora que ya sabemos obtener la recta, es obligatorio que empecemos a hacernos preguntas sobre la calidad de los resultados que hemos obtenido. La recta de regresión siempre se puede calcular, pero hay conjuntos de puntos para los que, incluso la mejor recta es bastante mala. ¿Cómo podemos estar seguros de que el ajuste de la recta a los datos es de buena calidad? Naturalmente, tenemos que tener en cuenta el error cuadrático medio ECM que hemos usado para definir la recta. Si ese error es pequeño, la recta será buena...¿ves la dificultad?  Ya nos hemos tropezado con situaciones parecidas. Es un problema de escala ¿pequeño comparado con qué? El tamaño absoluto del ECM depende de las unidades de medida que se estén utilizando, y por eso es difícil usarlo directamente como un indicador fiable de calidad. Queremos obtener un indicador de calidad que no dependa de la escala. Para eso vamos a hacer un análisis más detallado del error cuadrático medio.

    \item Recordemos que el objetivo básico es medir la diferencia entre los valores iniciales de la coordenada $y$:
    \[y_1, y_2,\ldots, y_n,\]
    y los valores que predice la recta de regresión:
    \[\hat y_1,\hat y_2,\ldots,\hat y_n,\]
    Además, tenemos la media $\bar y$ de los valores iniciales. Con esta media podemos calcular la varianza de $y$:
    \[{\operatorname{Var}_n}(y)=\dfrac{1}{n}\displaystyle\sum_{i=1}^{n}(y_i-\bar y)^2\]
    y al ver esta fórmula, nos damos cuenta de que recuerda bastante al ECM:
    \[\mbox{ECM}(y=a+b\cdot x)=\dfrac{1}{n}\sum_{i=1}^n(y_i-\hat y_i)^2\]
    De hecho, al compararlas está claro que podemos escribir una tercera fórmula, en la que comparamos la media con los valores que predice la regresión:
    \[\dfrac{1}{n}\sum_{i=1}^n(\bar y_i-\hat y_i)^2\]
    Con esta tercera fórmula, estamos en condiciones de hacer una descomposición o Análisis de la Varianza de $y$ (en inglés, ANalysis Of VAriance, abreviado ANOVA). Se puede demostrar (no es difícil) que siempre se cumple.
    \[{\operatorname{Var}_n}(y)=\mbox{ECM}+\dfrac{1}{n}\sum_{i=1}^n(\bar y_i-\hat y_i)^2\]
    Y ahora, para obtener una estimación de calidad, independiente de la escala, dividimos por la varianza de $y$ ambos miembros, obteniendo:
    \[1=\dfrac{\mbox{ECM}}{{\operatorname{Var}_n}(y)}+\dfrac{\dfrac{1}{n}\sum_{i=1}^n(\bar y_i-\hat y_i)^2}{{\operatorname{Var}_n}(y)}\]
    Sustituyendo aquí, en  $\bar y_i-\hat y_i$, la expresión de la recta de regresión, se obtiene:
    \[1=\dfrac{\mbox{ECM}}{{\operatorname{Var}_n}(y)}+r^2\]
    donde $r$ es el:\\[3mm]
        \fbox{\colorbox{Gris025}{\begin{minipage}{14cm}
         \begin{center}
         \vspace{2mm}
         {\bf Coeficiente de correlación lineal de Pearson}\\
         \end{center}
         Es el valor $r$ que cumple:
         \[r=\dfrac{\operatorname{cov}(x,y)}{\sqrt{{\operatorname{Var}_n}(x)\cdot{\operatorname{Var}_n}(y)}}\]
         Recuerda que $\operatorname{cov}(x,y)$ es la covarianza de $x$ e $y$.
         \end{minipage}}}\\[3mm]

    Por lo tanto, la ecuación que regula la calidad del ajuste se puede escribir:
    \[\dfrac{\mbox{ECM}}{{\operatorname{Var}_n}(y)}=1-r^2\]
    Y ahora es fácil interpretar $r$. Es un número, entre $-1$ y $1$, que tiene la propiedad de que {\sf cuanto más cerca de 1 está $r^2$, mejor es el ajuste de la recta de regresión a los datos. Por contra, si $r=0$, el ajuste es muy malo.} El signo de $r$ se corresponde con el de la pendiente de la recta de regresión, y tiene la misma interpretación.

    \end{itemize}

\section{Regresión lineal y muestreo}

\begin{itemize}

    \item Empecemos recordando que la recta de regresión $y=a+b\cdot x$ que hemos localizado en la anterior sección es,
    \[(y-\bar y)=\dfrac{{\operatorname{cov}(x,y)}}{V_x}(x-\bar x), \quad\mbox{siendo }\operatorname{cov}(x,y)=\dfrac{\displaystyle\sum_{i=1}^{n}(x_i-\bar x)(y_i-\bar y)}{n}.\]
    En esta sección,
    \[V_x=\dfrac{1}{n}\sum_{i=1}^n(x_i-\bar x)^2,\qquad V_y=\dfrac{1}{n}\sum_{i=1}^n(y_i-\bar y)^2\]
    (se usa la definición poblacional). Por lo tanto,\\[3mm]
    \fbox{\colorbox{Gris025}{\begin{minipage}{14cm}
    \begin{center}
    \vspace{2mm}
    {\bf Coeficientes $a$ y $b$, y relación con $r$}\\
    \end{center}
    Los coeficientes $a$ y $b$ de la recta de regresión son:
    \begin{equation}\label{ec:coeficientesRectaRegresion}
    b=\dfrac{{\operatorname{cov}(x,y)}}{V_x},\qquad a=\bar y-b\bar x.
    \end{equation}
    y en particular,
    \[r^2=b^2\dfrac{V_x}{V_y}\]
    de donde
    \[r=b\dfrac{s_x}{s_y}\]
    (aquí hay truco; recuerda que $s_x$ y $s_y$ proceden de fórmulas que usan $n-1$ en el denominador).
    \end{minipage}}}\\[3mm]
    Como hemos visto, esta recta es, de entre todas las rectas posibles, la que mejor representa, desde el punto de vista estadístico, al conjunto de $n$ puntos del plano:
    \[(x_1,y_1),(x_2,y_2),(x_3,y_3),\ldots,(x_n,y_n),\]
    Y hemos aprendido a medir la calidad de esa recta, {\sf para describir esos $n$ puntos}. Pero, naturalmente, eso es sólo un primer paso. Es fácil comprender que, en general, esos $n$ puntos serán sólo una muestra, tomada de una población que nos interesa estudiar. Y, como cabe suponer, cada muestra diferente que tomemos producirá una recta distinta.  En la siguiente figura pueden verse dos muestras de una misma población (una representada por los puntos azules redondos, y otra por las cruces rojas) y las correspondientes rectas de regresión (azul, trazo continuo la de la primera población, y rojo discontinua la de la segunda). Y en este \textattachfile{Cap11_InferenciaMuestreoRegresion.html}{\textcolor{blue}{documento html}} puedes comprobar como distintas muestras producen rectas distintas
    \begin{center}
    \includegraphics[width=14cm]{2011_12_20_RegresionDosMuestras.png}
    \end{center}
    En general, cada una de esas rectas es distinta de las demás, y además {\em distinta de la {\sf recta teórica}, que de momento podemos pensar que es la que obtendríamos si pudiéramos usar todos los datos de la población.} Vamos a llamar
    \[y=\alpha +\beta\cdot x\]
    a esa recta teórica. Como es habitual, usamos letras griegas para referirnos a los parámetros poblacionales, para distinguirlos de los parámetros $a$ y $b$ que corresponden a la muestra.

    \textcolor{red}{\underline{ATENCIÓN:}} Ya hemos advertido en otras ocasiones de lo {\em decepcionante} que resulta la notación que se usa en Estadística. Aquí vamos a usar $\alpha$ porque es lo que hacen todos los libros de Estadística, pero esa notación resulta confusa porque vamos a usar también $\alpha$ para el nivel de confianza en contrastes de hipótesis e intervalos de confianza.

    \item A la vista de estas reflexiones, parece evidente que nos debemos hacer --para empezar, entre otras-- estas preguntas: ¿Podemos usar la teoría de muestreo para contrastar la existencia de una relación lineal entre $x$ e $y$? Y si suponemos que esa relación existe, ¿podemos usar $a$ y $b$ para estimar $\alpha$ y $\beta$ (es decir, obtener intervalos de confianza)?
%        \begin{enumerate}
%            \item Una pregunta que hemos dejado pendiente de la anterior sesión. Si nos dan un nuevo valor de $x$, y usamos la recta de regresión para calcular $y$, ¿cómo de buena es esa estimación del valor de $y$?
%            \item
%        \end{enumerate}

    \item En capítulos anteriores, para realizar estimaciones por inferencia, siempre hemos tenido que hacer alguna suposición sobre la distribución de la población, y a partir de aquí, obteníamos información sobre la distribución muestral del estadístico adecuado a cada problema. La situación en la que estamos ahora es análoga, pero hay algunas diferencias. Y de hecho, hay varias opciones disponibles para continuar. Pero algunas de ellas utilizan el concepto de {\sf variable aleatoria continua bidimensional}, que no hemos estudiado.


    \item Vamos a quedarnos, por tanto con un modelo muy básico, pero aún así útil. Vamos a interpretar los parámetros $\alpha$ y $\beta$ así: supondremos que para cada valor fijo $x_0$ de la variable $x$ tenemos una variable aleatoria normal $Y_{x_0}$ de tipo $N(\alpha+\beta\cdot x,\sigma)$, donde $\sigma$ es la misma, independientemente de $x_0$ (esto desde luego, es una simplificación) . Y entonces interpretamos el punto        $(x_1,y_1)$ suponiendo que $y_1$ es una observación de $Y_{x_1}$, el punto $(x_2,y_2)$ suponiendo que $y_2$ es una observación de $Y_{x_2}$,etcétera, hasta el punto $(x_n,y_n)$, para el que suponemos igualmente que $y_n$ es una observación de $Y_{x_n}$. Esto es equivalente a suponer que nuestras observaciones se explican mediante este modelo:
        \[y=\alpha +\beta\cdot x+\epsilon,\quad\mbox{ siendo }\epsilon\sim N(0,\sigma).\]
        La siguiente figura ilustra la forma en la que se suele entender esto. Como se ve en la figura, para cada valor $x_0$ hay asociada una copia local de la normal $N(0,\sigma)$, que permite calcular las probabilidades condicionadas del tipo
        \[P(Y\leq y|x=x_0)\]
        En este modelo, $a$ y $b$ son, evidentemente, estimadores de los parámetros $\alpha$ y $\beta$ de la distribución teórica.
        \begin{center}
        \includegraphics[width=14.5cm]{2011-12-20-RegresionDistribucionesNormalesMarginales.png}
        \end{center}

    \subsection*{}\label{sec:anova}
    \item Vamos a recordar el ANOVA (análisis de la varianza) que hicimos en la sección anterior. Justo antes de la definición del coeficiente de correlación de Pearson habíamos obtenido:
        \[1=\dfrac{\mbox{ECM}}{V_y}+r^2\]
        o, lo que es lo mismo (recuerda que $\operatorname{Var}(y)=V_y$):
        \[V_y=\mbox{ECM}+r^2V_y\]
        Y como
        \[r^2=\dfrac{(\operatorname{cov}(x,y))^2}{V_x\cdot V_y}\]
        esto es:
        \[V_y=\mbox{ECM}+\dfrac{(\operatorname{cov}(x,y))^2}{V_x}=\mbox{ECM}+b\cdot\operatorname{cov}(x,y),\]
        donde $b$ es la pendiente de la recta de regresión (recuerda la Ecuación \ref{ec:coeficientesRectaRegresion}, página \pageref{ec:coeficientesRectaRegresion}). Conceptualmente, hemos descompuesto $V_y$, que mide la dispersión total de los valores de $y$, de esta forma:
        \[
        \underbrace{\left(\mbox{dispersión total de }y\right)}_{V_y}=
        \underbrace{\left(\mbox{dispersión aleatoria }N(0,\sigma)\right)}_{\mbox{ECM}}+
        \underbrace{\left(\mbox{dispersión debida a la regresión}\right)}_{b\cdot\operatorname{cov}(x,y)}
        \]
        Esta ecuación nos va a servir para entender la forma en la que se contrasta la existencia de una relación lineal entre $x$ e $y$. Si esa relación no existe, entonces al obtener una muestra de puntos
        \[(x_1,y_1),(x_2,y_2),(x_3,y_3),\ldots,(x_n,y_n),\]
        prácticamente toda la variabilidad que observemos se puede achacar al término aleatorio $\mbox{ECM}$, y el término $b\cdot\operatorname{cov}(x,y)$ prácticamente no existe. Es decir, que si alguien sostiene (como {\em hipótesis nula}) que no existe relación lineal entre las variables $x$ e $y$, eso es equivalente a afirmar que $b$ está muy próximo a $0$, que a su vez es equivalente a afirmar que $\operatorname{cov}(x,y)$ es prácticamente $0$. Sólo nos falta recordar que, en este modelo, $b$ es en realidad un estimador de $\beta$, la pendiente de la recta teórica. Decir que no existe dependencia lineal entre $x$ e $y$ equivale a decir que $\beta=0$.

    \item Ahora el plan está claro. Usamos la hipótesis nula:
    \[H_0=\{\beta=0\}\]
    y necesitamos un estadístico que nos permita estimar $\beta$. Si la muestra produce un valor del estadístico muy alejado de cero (es decir, un valor muy improbable), podremos rechazar la hipótesis nula. El estadístico que se utiliza es una especie de tipificación\footnote{Puedes ver los detalles en el capítulo 12 de {\em Estadística Aplicada}, de J. de la Horra, o en el capítulo 17 de {\em Estadística básica para estudiantes de ciencias}, de Gorgas, Cardiel y Zamorano.} de $b$, que conduce a:\\[3mm]
    \fbox{\colorbox{Gris025}{\begin{minipage}{14cm}
    \begin{center}
    \vspace{2mm}
    {\bf Contraste de hipótesis nula $\beta=0$ para la existencia de una dependencia lineal entre $x$ e $y$}\\
    \end{center}
    Si se cumple la hipótesis nula $H_0=\{\beta=0\}$, entonces el estadístico:
    \[\dfrac{\operatorname{cov}(x,y)}{\sqrt{\dfrac{V_x V_y-\operatorname{cov}^2(x,y)}{n-2}}}\]
    sigue una distribución $t$ de Student con \textcolor{red}{$n-2$ grados de libertad.} Por lo tanto, al realizar el contraste, rechazaremos la hipótesis nula si en la muestra obtenida se cumple que:
    \[\left|\dfrac{\operatorname{cov}(x,y)}{\sqrt{\dfrac{V_x V_y-\operatorname{cov}^2(x,y)}{n-2}}}\right|>t_{n-2;\alpha/2}.\]
    \end{minipage}}}\\[3mm]
    Más adelante, cuando hayamos aprendido algo sobre el método ANOVA, podremos volver sobre el resultado de este contraste y expresarlo en términos de ese método. Allí usaremos la distribución F de Fisher que ya concoemos, y veremos la relación entre ambos planteamientos.

    \item Supongamos que hemos rechazado la hipótesis nula, y por lo tanto trabajamos con la hipótesis de que, en efecto, existe un modelo lineal $y=\alpha+x\cdot\beta$ como el que hemos descrito, que relaciona los valores de $x$ e $y$. Hemos dicho que $a$ y $b$ son estimadores de $\alpha$ y $\beta$. Por lo tanto es razonable usarlos para obtener intervalos de confianza sobre esos parámetros, Las cuentas son parecidas a las que hemos presentado para el contraste de hipótesis (y los detalles se pueden consultar en las mismas fuentes), y se obtienen estos intervalos:\\[3mm]
        \fbox{\colorbox{Gris025}{\begin{minipage}{14cm}
        \begin{center}
        \vspace{2mm}
        {\bf Intervalos de confianza (nivel $(1-\alpha)$) para los parámetros $\alpha$ y $\beta$ del modelo de dependencia lineal}\\
        \end{center}
        La recta del modelo es $y=\alpha+\beta\cdot x$. Para el parámetro $\beta$ se tiene este intervalo de confianza:
        \[\beta=
%        b\pm t_{n-2;\alpha/2}\left(\sqrt{\dfrac{1}{n-2}\cdot\dfrac{SS_y-b\cdot SS_{xy}}{SS_x}}\right)
%        =
        b\pm t_{n-2;\alpha/2}\sqrt{\dfrac{1}{n-2}\cdot\dfrac{V_y-b\cdot\operatorname{cov}_{xy}}{V_x}}
        .\]
%        donde:
%        \[
%        \begin{cases}
%        SS_x=\sum_{i=1}^n(x_i-\bar x)^2=n V_x\\[3mm]
%        SS_y=\sum_{i=1}^n(y_i-\bar y)^2=n V_y\\[3mm]
%        SS_{xy}=\sum_{i=1}^n(x_i-\bar x)(y_i-\bar y)=n \operatorname{cov}(x,y)
%        \end{cases}
%        \]
        Para el parámetro $\alpha$ se obtiene una expresión algo más complicada:
        \[\alpha=a\pm t_{n-2;\alpha/2}
        \sqrt{        \left(\dfrac{1}{n-2}\cdot\dfrac{V_y-b\cdot\operatorname{cov}_{xy}}{V_x}\right)\cdot
        \left(V_x+\bar x^2\right)
        },\]
        Como ya hemos advertido, hay que tener cuidado porque el mismo símbolo $\alpha$ aparece aquí \underline{representando dos cosas distintas}.
        \end{minipage}}}\\[3mm]





    \end{itemize}


%\section{Verificando la calidad del ajuste lineal *}
%
%
%\begin{itemize}
%
%    \item Las condiciones descritas al comienzo de la sección sobre inferencia hay que verificarlas.
%    \item Residuos
%    \item Ver la sección 6.2 de Daalgard
%    \item Q-Q plots
%
%
%
%    \end{itemize}
%
%\section{Regresión no lineal (unidimensional) *}
%\begin{itemize}
%
%    \item
%
%
%    \end{itemize}
%

%\section*{Tareas asignadas para esta sesión.}
%No hay tareas asignadas para esta sesión.
%
%\section*{Lectura recomendada}
%
%Las mismas de la anterior sesión.
%
%
%
%\section*{\fbox{\colorbox{Gris025}{{Sesión 26. Inferencia estadística.}}}}
%
%\subsection*{\fbox{\colorbox{Gris025}{{Inferencia en la regresión lineal.}}}}
%\subsection*{Fecha: Martes, 20/12/2011, 14h.}
%
%\noindent{\bf Atención: este fichero pdf lleva adjuntos los ficheros de datos necesarios.}
%
%%\subsection*{\fbox{1. Ejemplos preliminares }}
%\setcounter{tocdepth}{1}
%%\tableofcontents

%\section{Regresión lineal y muestreo}
%
%\begin{itemize}
%
%    \item Empecemos recordando que la recta de regresión $y=a+b\cdot x$ que hemos localizado en la anterior sesión es,
%    \[(y-\bar y)=\dfrac{{\operatorname{cov}(x,y)}}{V_x}(x-\bar x), \quad\mbox{siendo }\operatorname{cov}(x,y)=\dfrac{\displaystyle\sum_{i=1}^{n}(x_i-\bar x)(y_i-\bar y)}{n}.\]
%    En esta sesión,
%    \[V_x=\dfrac{1}{n}\sum_{i=1}^n(x_i-\bar x)^2,\qquad V_y=\dfrac{1}{n}\sum_{i=1}^n(y_i-\bar y)^2\]
%    (se usa la definición poblacional). Por lo tanto,\\[3mm]
%    \fbox{\colorbox{Gris025}{\begin{minipage}{14cm}
%    \begin{center}
%    \vspace{2mm}
%    {\bf Coeficientes $a$ y $b$, y relación con $r$}\\
%    \end{center}
%    Los coeficientes $a$ y $b$ de la recta de regresión son:
%    \begin{equation}\label{ec:coeficientesRectaRegresion}
%    b=\dfrac{{\operatorname{cov}(x,y)}}{V_x},\qquad a=\bar y-b\bar x.
%    \end{equation}
%    y en particular,
%    \[r^2=b^2\dfrac{V_x}{V_y}\]
%    de donde
%    \[r=b\dfrac{s_x}{s_y}\]
%    (aquí hay truco; recuerda que $s_x$ y $s_y$ proceden de fórmulas que usan $n-1$ en el denominador).
%    \end{minipage}}}\\[3mm]
%    Como hemos visto, esta recta es, de entre todas las rectas posibles, la que mejor representa, desde el punto de vista estadístico, al conjunto de $n$ puntos del plano:
%    \[(x_1,y_1),(x_2,y_2),(x_3,y_3),\ldots,(x_n,y_n),\]
%    Y hemos aprendido a medir la calidad de esa recta, {\sf para describir esos $n$ puntos}. Pero, naturalmente, eso es sólo un primer paso. Es fácil comprender que, en general, esos $n$ puntos serán sólo una muestra, tomada de una población que nos interesa estudiar. Y, como cabe suponer, cada muestra diferente que tomemos producirá una recta distinta.  En la siguiente figura pueden verse dos muestras de una misma población (una representada por los puntos azules redondos, y otra por las cruces rojas) y las correspondientes rectas de regresión (azul, trazo continuo la de la primera población, y rojo discontinua la de la segunda). Y en este \textattachfile{2011_12_20_InferenciaMuestreoRegresion.html}{\textcolor{blue}{documento html}} puedes comprobar como distintas muestras producen rectas distintas
%    \begin{center}
%    \includegraphics[width=14cm]{2011_12_20_RegresionDosMuestras.png}
%    \end{center}
%    En general, cada una de esas rectas es distinta de las demás, y además {\em distinta de la {\sf recta teórica}, que de momento podemos pensar que es la que obtendríamos si pudiéramos usar todos los datos de la población.} Vamos a llamar
%    \[y=\alpha +\beta\cdot x\]
%    a esa recta teórica. Como es habitual, usamos letras griegas para referirnos a los parámetros poblacionales, para distinguirlos de los parámetros $a$ y $b$ que corresponden a la muestra.
%
%    \textcolor{red}{\underline{ATENCIÓN:}} Ya hemos advertido en otras ocasiones de lo {\em decepcionante} que resulta la notación que se usa en Estadística. Aquí vamos a usar $\alpha$ porque es lo que hacen todos los libros de Estadística, pero esa notación resulta confusa porque vamos a usar también $\alpha$ para el nivel de confianza en contrastes de hipótesis e intervalos de confianza.
%
%    \item A la vista de estas reflexiones, parece evidente que nos debemos hacer --para empezar, entre otras-- estas preguntas: ¿Podemos usar la teoría de muestreo para contrastar la existencia de una relación lineal entre $x$ e $y$? Y si suponemos que esa relación existe, ¿podemos usar $a$ y $b$ para estimar $\alpha$ y $\beta$ (es decir, obtener intervalos de confianza)?
%%        \begin{enumerate}
%%            \item Una pregunta que hemos dejado pendiente de la anterior sesión. Si nos dan un nuevo valor de $x$, y usamos la recta de regresión para calcular $y$, ¿cómo de buena es esa estimación del valor de $y$?
%%            \item
%%        \end{enumerate}
%
%    \item En sesiones anteriores, para realizar estimaciones por inferencia, siempre hemos tenido que hacer alguna suposición sobre la distribución de la población, y a partir de aquí, obteníamos información sobre la distribución muestral del estadístico adecuado a cada problema. La situación en la que estamos ahora es análoga, pero hay algunas diferencias. Y de hecho, hay varias opciones disponibles para continuar. Pero algunas de ellas utilizan el concepto de {\sf variable aleatoria continua bidimensional}, que no hemos estudiado.
%
%
%    \item Vamos a quedarnos, por tanto con un modelo muy básico, pero aún así útil. Vamos a interpretar los parámetros $\alpha$ y $\beta$ así: supondremos que para cada valor fijo $x_0$ de la variable $x$ tenemos una variable aleatoria normal $Y_{x_0}$ de tipo $N(\alpha+\beta\cdot x,\sigma)$, donde $\sigma$ es la misma, independientemente de $x_0$ (esto desde luego, es una simplificación) . Y entonces interpretamos el punto        $(x_1,y_1)$ suponiendo que $y_1$ es una observación de $Y_{x_1}$, el punto $(x_2,y_2)$ suponiendo que $y_2$ es una observación de $Y_{x_2}$,etcétera, hasta el punto $(x_n,y_n)$, para el que suponemos igualmente que $y_n$ es una observación de $Y_{x_n}$. Esto es equivalente a suponer que nuestras observaciones se explican mediante este modelo:
%        \[y=\alpha +\beta\cdot x+\epsilon,\quad\mbox{ siendo }\epsilon\sim N(0,\sigma).\]
%        La siguiente figura ilustra la forma en la que se suele entender esto. Como se ve en la figura, para cada valor $x_0$ hay asociada una copia local de la normal $N(0,\sigma)$, que permite calcular las probabilidades condicionadas del tipo
%        \[P(Y\leq y|x=x_0)\]
%        En este modelo, $a$ y $b$ son, evidentemente, estimadores de los parámetros $\alpha$ y $\beta$ de la distribución teórica.
%        \begin{center}
%        \includegraphics[width=14.5cm]{2011-12-20-RegresionDistribucionesNormalesMarginales.png}
%        \end{center}
%
%    \subsection*{}\label{sec:anova}
%    \item Vamos a recordar el ANOVA (análisis de la varianza) que hicimos en la sesión anterior. Justo antes de la definición del coeficiente de correlación de Pearson habíamos obtenido:
%        \[1=\dfrac{\mbox{ECM}}{V_y}+r^2\]
%        o, lo que es lo mismo (recuerda que $\operatorname{Var}(y)=V_y$):
%        \[V_y=\mbox{ECM}+r^2V_y\]
%        Y como
%        \[r^2=\dfrac{(\operatorname{cov}(x,y))^2}{V_x\cdot V_y}\]
%        esto es:
%        \[V_y=\mbox{ECM}+\dfrac{(\operatorname{cov}(x,y))^2}{V_x}=\mbox{ECM}+b\cdot\operatorname{cov}(x,y),\]
%        donde $b$ es la pendiente de la recta de regresión (recuerda la Ecuación \ref{ec:coeficientesRectaRegresion}, página \pageref{ec:coeficientesRectaRegresion}). Conceptualmente, hemos descompuesto $V_y$, que mide la dispersión total de los valores de $y$, de esta forma:
%        \[
%        \underbrace{\left(\mbox{dispersión total de }y\right)}_{V_y}=
%        \underbrace{\left(\mbox{dispersión aleatoria }N(0,\sigma)\right)}_{\mbox{ECM}}+
%        \underbrace{\left(\mbox{dispersión debida a la regresión}\right)}_{b\cdot\operatorname{cov}(x,y)}
%        \]
%        Esta ecuación nos va a servir para entender la forma en la que se contrasta la existencia de una relación lineal entre $x$ e $y$. Si esa relación no existe, entonces al obtener una muestra de puntos
%        \[(x_1,y_1),(x_2,y_2),(x_3,y_3),\ldots,(x_n,y_n),\]
%        prácticamente toda la variabilidad que observemos se puede achacar al término aleatorio $\mbox{ECM}$, y el término $b\cdot\operatorname{cov}(x,y)$ prácticamente no existe. Es decir, que si alguien sostiene (como {\em hipótesis nula}) que no existe relación lineal entre las variables $x$ e $y$, eso es equivalente a afirmar que $b$ está muy próximo a $0$, que a su vez es equivalente a afirmar que $\operatorname{cov}(x,y)$ es prácticamente $0$. Sólo nos falta recordar que, en este modelo, $b$ es en realidad un estimador de $\beta$, la pendiente de la recta teórica. Decir que no existe dependencia lineal entre $x$ e $y$ equivale a decir que $\beta=0$.
%
%    \item Ahora el plan está claro. Usamos la hipótesis nula:
%    \[H_0=\{\beta=0\}\]
%    y necesitamos un estadístico que nos permita estimar $\beta$. Si la muestra produce un valor del estadístico muy alejado de cero (es decir, un valor muy improbable), podremos rechazar la hipótesis nula. El estadístico que se utiliza es una especie de tipificación\footnote{Puedes ver los detalles en el capítulo 12 de {\em Estadística Aplicada}, de J. de la Horra, o en el capítulo 17 de {\em Estadística básica para estudiantes de ciencias}, de Gorgas, Cardiel y Zamorano.} de $b$, que conduce a:\\[3mm]
%    \fbox{\colorbox{Gris025}{\begin{minipage}{14cm}
%    \begin{center}
%    \vspace{2mm}
%    {\bf Contraste de hipótesis nula $\beta=0$ para la existencia de una dependencia lineal entre $x$ e $y$}\\
%    \end{center}
%    Si se cumple la hipótesis nula $H_0=\{\beta=0\}$, entonces el estadístico:
%    \[\dfrac{\operatorname{cov}(x,y)}{\sqrt{\dfrac{V_x V_y-\operatorname{cov}^2(x,y)}{n-2}}}\]
%    sigue una distribución $t$ de Student con \textcolor{red}{$n-2$ grados de libertad.} Por lo tanto, al realizar el contraste, rechazaremos la hipótesis nula si en la muestra obtenida se cumple que:
%    \[\left|\dfrac{\operatorname{cov}(x,y)}{\sqrt{\dfrac{V_x V_y-\operatorname{cov}^2(x,y)}{n-2}}}\right|>t_{n-2;\alpha/2}.\]
%    \end{minipage}}}\\[3mm]
%
%    \item Supongamos que hemos rechazado la hipótesis nula, y por lo tanto trabajamos con la hipótesis de que, en efecto, existe una modelo lineal $y=\alpha+\cdot\beta$ como el que hemos descrito, que relaciona los valores de $x$ e $y$. Hemos dicho que $a$ y $b$ son estimadores de $\alpha$ y $beta$. Por lo tanto es razonable usarlos para obtener intervalos de confianza sobre esos parámetros, Las cuentas son parecidas a las que hemos presentado para el contraste de hipótesis (y los detalles se pueden consultar en las mismas fuentes), y se obtienen estos intervalos:\\[3mm]
%        \fbox{\colorbox{Gris025}{\begin{minipage}{14cm}
%        \begin{center}
%        \vspace{2mm}
%        {\bf Intervalos de confianza (nivel $(1-\alpha)$) para los parámetros $\alpha$ y $\beta$ del modelo de dependencia lineal}\\
%        \end{center}
%        La recta del modelo es $y=\alpha+\beta\cdot x$. Para el parámetro $\beta$ se tiene este intervalo de confianza:
%        \[\beta=
%%        b\pm t_{n-2;\alpha/2}\left(\sqrt{\dfrac{1}{n-2}\cdot\dfrac{SS_y-b\cdot SS_{xy}}{SS_x}}\right)
%%        =
%        b\pm t_{n-2;\alpha/2}\sqrt{\dfrac{1}{n-2}\cdot\dfrac{V_y-b\cdot\operatorname{cov}_{xy}}{V_x}}
%        .\]
%%        donde:
%%        \[
%%        \begin{cases}
%%        SS_x=\sum_{i=1}^n(x_i-\bar x)^2=n V_x\\[3mm]
%%        SS_y=\sum_{i=1}^n(y_i-\bar y)^2=n V_y\\[3mm]
%%        SS_{xy}=\sum_{i=1}^n(x_i-\bar x)(y_i-\bar y)=n \operatorname{cov}(x,y)
%%        \end{cases}
%%        \]
%        Para el parámetro $\alpha$ se obtiene una expresión algo más complicada:
%        \[\alpha=a\pm t_{n-2;\alpha/2}
%        \sqrt{        \left(\dfrac{1}{n-2}\cdot\dfrac{V_y-b\cdot\operatorname{cov}_{xy}}{V_x}\right)\cdot
%        \left(V_x+\bar x^2\right)
%        },\]
%        Como ya hemos advertido, hay que tener cuidado porque el mismo símbolo $\alpha$ aparece aquí \underline{representando dos cosas distintas}.
%        \end{minipage}}}\\[3mm]
%
%
%
%
%
%
%
%
%
%
%    \end{itemize}



    %\section{Otros modelos de regresión. (*)}
    %\begin{itemize}
    %
    %    \item Modelo de regresión lineal múltiple:
    %    \[y=c_0+c_1x_1+c_2x_2+\cdots+c_kx_k\]
    %
    %    \item El esquema de la pág. 85 de Estadística II para dummies.
    %
    %    \item Residuos.
    %
    %\end{itemize}
    %
    %Pendiente


    \chapter{Introducción al ANOVA}\label{cap:IntroduccionANOVA}
    % !Mode:: "Tex:UTF-8"

%\chapter{Diseño de experimentos}
%
%\section*{\fbox{\colorbox{Gris025}{{Sesión 27. Diseño experimental.}}}}\label{sesion:27}
%
%\subsection*{\fbox{\colorbox{Gris025}{{Análisis de la varianza.}}}}
%\subsection*{Fecha: Martes, 10/01/2012, 14h.}
%
%\noindent{\bf Atención: este fichero pdf lleva adjuntos los ficheros de datos necesarios.}
%
%%\subsection*{\fbox{1. Ejemplos preliminares }}
%\setcounter{tocdepth}{1}
%%\tableofcontents

\section{ANOVA unifactorial.}

\begin{itemize}

    \item El análisis de la varianza (ANOVA) es una de las técnicas básicas de la estadística, y es el fundamento de la parte de la estadística aplicada conocida como diseño experimental.  {\em El análisis de la varianza consiste en descomponer la variabilidad presente en un conjunto de datos en una suma de términos, de manera que cada uno de esos términos se pueda atribuir a una fuente específica de variación, que influye en el fenómeno que estamos estudiando.} El diseño de experimentos, y el propio análisis de la varianza, tienen como objetivo maximizar la calidad estadística de la información que se obtiene a partir de un experimento, y requieren un estudio mucho más detallado que el que vamos a poder hacer en un curso de introducción a la estadística como es este. Aquí nos vamos a limitar a dar los primeros pasos en esa dirección.

    \item Para comenzar a caminar, vamos a combinar varias ideas que ya hemos visto en anteriores capítulos. En realidad, ya hemos visto un ejemplo de análisis de la varianza. Cuando, en el Capítulo 11, ver página \pageref{sec:anova}), tratábamos de entender la dispersión de los puntos en un modelo sencillo de regresión lineal, obtuvimos esta descomposición de la varianza para la variable $y$
        \[V_y=\mbox{ECM}+b\cdot\operatorname{cov}(x,y),\]
        (aquí ECM es el error cuadrático medio, y $b$ es la pendiente de la recta de regresión). Dijimos entonces que esto se podía interpretar conceptualmente como una descomposición de $V_y$, la dispersión total de los valores de $y$, de esta forma:
        \[
        \underbrace{\left(\mbox{dispersión total de }y\right)}_{V_y}=
        \underbrace{\left(\mbox{dispersión aleatoria }N(0,\sigma)\right)}_{\mbox{ECM}}+
        \underbrace{\left(\mbox{dispersión debida a la regresión}\right)}_{b\cdot\operatorname{cov}(x,y)}
        \]
        En este caso, la varianza de $y$ se descompone en una componente puramente aleatoria (ECM), y una componente que corresponde al modelo lineal $y=a+bx$ que estamos utilizando. Esta segunda componente nos permite explicar parte de la diferencia entre los valores de $y$, diciendo que corresponden a distintos valores de la variable $x$, que es la variable predictora, o regresora; es decir, es la variable que en muchos casos, resulta más fácil de medir o de manipular al diseñar el experimento. Y, en este modelo de regresión lineal, en principio es posible cualquier valor de la variable $x$.

    \item El otro ingrediente que queremos traer a la memoria es la inferencia sobre diferencia de medias en poblaciones normales que hemos estudiado; es decir, la estimación por intervalos y contrastes de hipótesis de una hipótesis nula tal como
        \[H_0=\{\mu_1=\mu_2\}.\]
        En este caso nos interesaba una misma variable aleatoria $X$ en ambas poblaciones, y tratábamos de detectar alguna diferencia en las medias, que se pudiera atribuir a diferencias entre ambas poblaciones. ¿Qué tiene esto que ver con el modelo de regresión lineal? Bueno, en realidad aquí también hay dos variables. Por un lado la variable $X$, y por otro lado la {\sf variable población} $P$, una variable discreta, que sólo puede tomar dos valores, 1 y 2, para distinguir de que población proceden los datos. Y estamos tratando de usar la variable $P$ como predictora para analizar las diferencias (para empezar, la diferencia de medias) de los valores de $X$. Esa es la conexión: el valor (1 o 2) de $P$ permite analizar diferencias entre los valores de $X$, de modo análogo a como los distintos valores de $x$ permitían entender los distintos valores de $y$ en el modelo de regresión. Y una diferencia evidente entre ambos casos es que, mientras que la variable población sólo toma dos valores, la variable $x$ del modelo de regresión es una variable continua con infinitos valores posibles.

    \item El problema que vamos a estudiar en esta sección es una generalización del problema de la diferencia de medias. Para centrar las ideas, vamos a pensar en un ejemplo concreto. Supongamos que estamos comparando la eficacia de $k$ tratamientos diferentes para una misma enfermedad. Para ello, vamos a realizar un estudio en el que intervienen un total de $n$ pacientes. Esos pacientes se dividen aleatoriamente en $k$ grupos, y a cada uno de los grupos se le asigna uno de los tratamientos. La variable aleatoria $X$ que nos interesa es la respuesta al tratamiento, pero está claro que en este ejemplo interviene también una {\em variable población} $P$, que puede tomar $k$ valores, y que indica el grupo al que pertenece el paciente, y por lo tanto el tratamiento que se le ha asignado. Cada uno de esos $k$ grupos se puede considerar como una muestra de una población (la de los pacientes a los que se les aplica el tratamiento número $k$), y las medias de los valores de $X$ en cada una de las poblaciones serían:
        \[\mu_1,\mu_2,\ldots,\mu_k.\]
        Naturalmente, estamos interesados en comparar esos tratamientos según la respuesta que producen. Y eso significa que, como primer paso, queremos contrastar la hipótesis nula
        \[H_0=\{\mu_1=\mu_2=\cdots=\mu_k\}\]
        Esta hipótesis nula indica que no hay diferencias en la respuesta producida por los distintos tratamientos (es decir, $X$ no depende de $k$, por decirlo en un lenguaje que recuerde al de la regresión lineal).

    \item La situación que hemos descrito se presta al tipo de análisis estadístico conocido como {\sf ANOVA de un factor (o de una vía, o de clasificación simple), completamente aleatorio y de efectos fijos.} Vamos a explicar uno por uno estos términos:
    \begin{enumerate}
        \item Decimos que es un modelo de un factor, porque sólo tenemos en cuenta cómo depende $X$ del tratamiento aplicado, sin tener en cuenta otras variables que pueden influir (la edad, el género de los pacientes, su dieta y estilo de vida, etcétera).
            \item Es completamente aleatorio porque los pacientes se asignan de forma aleatoria a cada grupo de tratamiento, sin tratar de agruparlos de ninguna manera.
            \item Y es de efectos fijos, porque nosotros hemos seleccionado cuáles son los tratamientos que queremos analizar, no los hemos elegido al azar de entre un conjunto posible de tratamientos.
    \end{enumerate}

    \item A cada uno de los $k$ tratamientos le hemos asignado un cierto grupo de pacientes. Vamos a llamar $n_j$ al número de pacientes que se han asignado al tratamiento número $j$, donde $j$ va desde 1 hasta $k$. Si llamamos $X_{ij}$ al valor de la variable $X$ en el paciente número $i$ del grupo número $j$, entonces podemos anotar los resultados experimentales en forma de tabla:
        \begin{equation}\label{ec:tablaValoresParaAnova}
        \begin{array}{cccccc}
        &\multicolumn{5}{r}\mbox{\bf Tratamiento aplicado ($j$ de 1 a $k$)}\\
        \cline{2-6}
        &1&2&3&\cdots&k\\
        \hline
        \multirow{5}{*}{\mbox{\bf Paciente ($i$ de 1 a $n_j$)}}
        &X_{11}&X_{12}&X_{13}&\cdots&X_{1k}\\
        &X_{21}&X_{22}&X_{23}&\cdots&X_{2k}\\
        &X_{31}&X_{32}&X_{33}&\cdots&X_{3k}\\
        &\vdots&\vdots&\vdots&\ddots&\vdots\\
        &X_{n_11}&X_{n_22}&X_{n_33}&\cdots&X_{n_kk}\\
        \hline
        \end{array}
        \end{equation}
        {\sf Un par de observaciones importantes sobre la notación:} aunque la tabla anterior parece indicar que todos los tratamientos han sido probados en el mismo número de pacientes, {\em en general no es así}, de modo que cada columna de la Tabla (\ref{ec:tablaValoresParaAnova}) puede tener distinta longitud. Es decir, {\sf no estamos suponiendo} que sea $n_1=n_2=\cdots=n_k$. Y, en segundo lugar, conviene comprobar, cuando se usan las fórmulas de un libro de texto, cuáles son los significados de $i$ y $j$ en $X_{ij}$, porque algunos autores los cambian de orden. Nosotros vamos a utilizar la notación más coherente con la notación matricial de uso general en matemáticas, donde en una tabla, $i$ indica la fila, y $j$ indica la columna.

\end{itemize}

\section{Identidad ANOVA. Residuos e hipótesis necesarias.}

\begin{itemize}

    \item Para contrastar la hipótesis nula de igualdad de medias entre todos los grupos (entre tratamientos, por tanto), necesitamos como siempre
        \begin{enumerate}
        \item hacer algunas hipótesis sobre distribuciones muestrales y
        \item un estadístico adecuado, cuyo comportamiento sea conocido a partir de esas hipótesis muestrales.
        \end{enumerate}
        Y aquí es donde entra en juego el análisis de la varianza, y su descomposición en componentes. Necesitamos introducir un poco de notación, para organizar los cálculos, y para conocer además la notación habitual en los textos de estadística relativos a estos problemas. Por ejemplo, para representar la suma de valores del grupo $j$ (columna $j$)
        \[X_{\textcolor{red}{\mbox{\bf\large $\cdot$}}j}=\sum_{i=1}^{n_j}X_{ij}\]
        Observa el punto que aparece como subíndice (y que, en esta ocasión, hemos coloreado de rojo para destacarlo). Ese punto indica sumación sobre la variable $i$ a la que sustituye. Por ejemplo, la media del grupo número $j$ sería
        \[\bar X_{\textcolor{red}{\mbox{\bf\large $\cdot$}}}j=\dfrac{X_{\textcolor{red}{\mbox{\bf\large $\cdot$}}}j}{n_j}\]
        Y la suma de todos los valores de la tabla se indicará con dos puntos:
        \[X_{\textcolor{red}{\mbox{\bf\large $\cdot\cdot$}}}=\sum_{j=1}^k\sum_{i=1}^{n_j}X_{ij}\]
        La media de todos los valores de la tabla se indica simplemente colocando una barra sobre $X$, como hemos hecho en otros capítulos:
        \[\bar{X}=\dfrac{X_{\textcolor{red}{\mbox{\bf\large $\cdot\cdot$}}}}{n}\]


    \item Con esta notación, empecemos el trabajo necesario para obtener el contraste de igualdad de medias. En primer lugar, para cualquier valor de la Tabla \ref{ec:tablaValoresParaAnova} (pág. \pageref{ec:tablaValoresParaAnova}) podemos escribir esta igualdad:
        \[X_{ij}-\mu=(X_{ij}-\mu_j)+(\mu_j-\mu)\]
        donde $\mu_j$ es la media (teórica o poblacional) de la población número $j$ (los pacientes tratados con el tratamiento $j$), y $\mu$ representa la media teórica de la población que resulta de combinar todas las poblaciones en una sola población. De la misma forma podemos escribir una ecuación de estimadores:

        \begin{equation}\label{cap14:ecu:AnovaIdentidadEstimadores}
        X_{ij}-\bar X=(X_{ij}-\bar X_{\mbox{\bf\large $\cdot$}j})+(\bar X_{\mbox{\bf\large $\cdot$}j}-\bar X)
        \end{equation}

        Supongamos que estamos tratando de estimar la varianza de todos los valores de la Tabla \ref{ec:tablaValoresParaAnova},  olvidando por un momento la separación en $k$ poblaciones (grupos), y usando $\mu$ como media teórica. Naturalmente, como $\mu$ no es conocido, usaríamos $\bar X$ para estimarlo. Entonces elevaríamos al cuadrado los términos de la forma
        \[X_{ij}-\bar X\]
        y sumaríamos para todos los valores de $i$ y $j$. El resultado interesante  se obtiene cuando se sustituye la anterior identidad (\ref{cap14:ecu:AnovaIdentidadEstimadores}) en esa suma (y se hacen unas cuantas simplificaciones algebraicas, que nos vamos a ahorrar para abreviar). Se obtiene:\\[3mm]
        \fbox{\colorbox{Gris025}{\begin{minipage}{14cm}
        \begin{center}
        \vspace{2mm}
        {\bf Identidad de la suma de cuadrados para ANOVA}\\
        \end{center}
        \begin{equation}\label{ec:identidadSumaCuadradosANOVA}
        \underbrace{\sum_{j=1}^k\sum_{i=1}^{n_j}(X_{ij}-\bar X)^2}_{(I)}=
        \underbrace{\sum_{j=1}^k n_j(\bar X_{\mbox{\bf\large $\cdot$}j}-\bar X)^2}_{(II)}+
        \underbrace{\sum_{j=1}^k\sum_{i=1}^{n_j}(X_{ij}-\bar X_{\mbox{\bf\large $\cdot$}j})^2}_{(III)}
        \end{equation}
        y el análisis de la varianza consiste en la interpretación de cada uno de los tres términos de esta ecuación:
        \begin{enumerate}
        \item El término $(I)$ representa, como hemos dicho, la dispersión total de los datos cuando se consideran como si procedieran de una única población combinada.
        \item El término $(II)$ representa la dispersión en los datos que se atribuye al hecho de que se utilizan $k$ tratamientos distintos. Es la dispersión {\sf entre grupos}.
        \item Finalmente, el término $(III)$ representa la dispersión en los datos que se atribuye al factor aleatorio {\sf dentro de los grupos}, porque cada {\sf individuo} responde de una forma distinta al tratamiento por razones que en este modelo se consideran puramente aleatorias.
        \end{enumerate}
        \end{minipage}}}\\[3mm]
        Esta descomposición o análisis de la varianza es justo lo que se necesita para poder introducir las hipótesis que proporcionan soporte teórico al contraste de igualdad de medias que vamos a hacer.

\end{itemize}

\subsection{Residuos e hipótesis necesarias para aplicar ANOVA.}\label{subsec:residuosAnova}

\begin{itemize}

    \item Volvamos al asunto de las condiciones que el modelo tiene que cumplir, para que el ANOVA funcione correctamente. Recordemos que estamos trabajando con un modelo ANOVA de una vía (o unifactorial, o de clasificación simple), completamente aleatorio y de efectos fijos. Además, vamos a suponer que:
        \begin{enumerate}
                \item las $k$ muestras (es decir, las $k$ columnas de la Tabla (\ref{ec:tablaValoresParaAnova}), página \pageref{ec:tablaValoresParaAnova}) son muestras independientes.
                \item cada una de esas muestras procede de una población normal (las poblaciones corresponden a los diferentes grupos de tratamiento), con media $\mu_j$ para la población número $j$.
                \item las $k$ poblaciones {\em tienen la misma varianza $\sigma^2$ (homocedasticidad).}
        \end{enumerate}
        %Por razones de agilidad, para llegar lo antes posible al núcleo del ANOVA, vamos a dejar para las próximas secciones varias preguntas importantes: ¿cómo comprobar estas condiciones? ¿qué se puede hacer cuando falla alguna de ellas?
        Al igual que sucedía en capítulos anteriores, donde nos planteábamos este problema para el caso de dos poblaciones, ya sabemos que la primera hipótesis depende de un diseño experimental correcto, y a veces es difícil de garantizar. En este capítulo simplemente supondremos que esa independencia está garantizada, posponiendo la discusión detallada para más adelante.

    \item La segunda hipótesis también va a quedar pospuesta a otro capítulo, cuando veamos distintas maneras de contrastar la normalidad de un conjunto de datos. Por el momento, sólo queremos destacar algunas ideas:
        \begin{enumerate}
                \item El contraste ANOVA de un factor es robusto frente a las desviaciones moderadas respecto a la normalidad. Es decir, que si se verifican las otras dos hipótesis (independencia e igualdad de varianzas), ANOVA seguirá funcionando aunque los datos sean un poco {\em no normales}.
                \item Si las muestras (los grupos a los que se aplica cada uno de los tratamientos) son de un tamaño muy pequeño, es muy difícil contrastar esta hipótesis de normalidad.
                \item Aunque ya hemos dicho que posponemos los detalles sobre los métodos más avanzados, para empezar debemos representar el diagrama de cajas (boxplot) de cada uno de los grupos por separado, y estudiar si se corresponde con el de una población normal. Siempre teniendo en cuenta, como hemos dicho, que si el tamaño de la muestra es pequeño, es muy posible que el diagrama de cajas no sea representativo de la población. Un poco más adelante, en este mismo capítulo, veremos como hacer estos diagramas usando R.
        \end{enumerate}

    \item La tercera hipótesis, la de la igualdad de varianzas (homocedasticidad), es, en cambio, más delicada: si los grupos son todos del mismo tamaño ({\em diseño equilibrado}), ANOVA es bastante robusto frente a cambios en las varianzas. Pero con grupos de distinto tamaño, el método pierde potencia rápidamente. Parte del trabajo que no vamos a poder hacer en este curso consiste en aprender lo que hay que hacer cuando no se puede justificar esa hipótesis de igualdad de varianzas. ¿Cómo se puede verificar si se cumple esa homogeneidad de las varianzas? Usando un concepto que ya encontramos al hablar de regresión: los residuos.

    \item[]{\bf Residuos}

    \item Al igual que sucedía en el caso de la regresión, con la que ANOVA está emparentado, el análisis de la condición de igualdad de varianzas del ANOVA utiliza el concepto de {\sf residuo}. Recordemos que en el caso de la regresión, dado un valor observado $(x_i,y_i)$ y la recta de regresión, de ecuación
        \[y=a+b\cdot x\]
        sustituimos $x_i$ en la recta y obtenemos el valor predicho por el modelo $\hat y_i=a+b\cdot x_i$. El residuo en este caso es la diferencia
        \[e_i=y_i-\hat y_i\]
        entre el valor observado y el que predice el modelo.

        ¿Qué significa residuo (y {\em valor predicho}) en el contexto de ANOVA? Vamos a escribir una ecuación similar a la \ref{cap14:ecu:AnovaIdentidadEstimadores} (página \pageref{cap14:ecu:AnovaIdentidadEstimadores}):
        \[
        X_{ij}=\bar X_{\mbox{\bf\large $\cdot$}j}+\left(X_{ij}-\bar X_{\mbox{\bf\large $\cdot$}j}\right)
        \]
        Esta expresión muestra el valor $X_{ij}$ como el resultado de dos términos:
        \begin{enumerate}
            \item la media $\bar X_{\mbox{\bf\large $\cdot$}j}$ del grupo al que pertenece el elemento en cuestión y
            \item el término $(X_{ij}-\bar X_{\mbox{\bf\large $\cdot$}j})$, que es el que vamos a llamar el {\sf residuo}, y que compara a cada elemento con la media de su grupo.
        \end{enumerate}

    \item En el caso de la regresión, vimos que los residuos representaban la componente aleatorio del modelo. Es decir, la parte de la muestra de puntos que no se explicaba mediante la contribución de la recta de regresión. En este caso, sucede algo similar: si el residuo es cero, el valor $X_{ij}$ se convierte en:
        \[
        X_{ij}=\bar X_{\mbox{\bf\large $\cdot$}j}
        \]
        la media de su grupo. Y si eso sucediera para todos los valores, entonces lo único que haría distintos unos valores de otros sería su adscripción a uno u otro grupo. Eso permite pensar en los residuos como la parte puramente aleatoria del modelo, en el sentido de que corresponde a valores que no se explican simplemente sabiendo a que grupo pertenece el elemento que examinamos. Otra manera conceptual de expresar lo que estamos diciendo es que se cumple:
        \[\mbox{Respuesta}=\mbox{(media del grupo)}\,+\,\mbox{(residuo)},\]
        y por lo tanto, en el modelo ANOVA, la media del grupo se considera como el valor {\em teórico (o ajustado) que predice el modelo}, de la misma forma que en la regresión el valor teórico era el que se obtenía sustituyendo $x$ en la ecuación de la recta.\\
        En este lenguaje, la Ecuación \ref{cap14:ecu:AnovaIdentidadEstimadores} se deja expresar así:
        \[
        X_{ij}=\bar X+(\bar X_{\mbox{\bf\large $\cdot$}j}-\bar X)+(X_{ij}-\bar X_{\mbox{\bf\large $\cdot$}j}),
        \]
        es decir:
        \[\mbox{Respuesta}=\mbox{(respuesta media general)}\,+\,\mbox{(diferencia debida al grupo)}\,+\,\mbox{(residuo).}\]

    \item Una vez entendido el significado de los residuos en ANOVA, ¿cómo podemos usarlos para verificar la condición de homogeneidad de la varianza? A menudo, los residuos se analizan gráficamente. Por ejemplo, usando un gráfico de los residuos frente a los valores que predice el modelo (ordenados por tamaño, claro; recordemos que los valores predichos por el modelo ANOVA son las medias de los grupos). Si en ese gráfico los puntos aparecen con forman de cuña (o con algún otro patrón claramente definido), podemos sospechar que hay una dependencia entre la media y la varianza, y por lo tanto, concluiremos que no se cumple la hipótesis de homogeneidad de varianzas. Por otra parte, si esa hipótesis se cumple, los residuos seguirán una distribución normal $N(0,\sigma)$, siendo $\sigma$ la desviación típica común a todas las poblaciones. Así que podemos representar los percentiles de los residuos (tipificados) frente a los correspondientes percentiles de una normal estándar, en lo que se conoce como un gráfico Q-Q (en inglés, Q-Q plot; la Q procede de {\em quantile}). Si la hipótesis de homogeneidad de varianzas se cumple, los puntos de ese gráfico deben estar aproximadamente situados en una recta.

\end{itemize}


\section{Tablas ANOVA.}

\begin{itemize}

    \item Ahora que hemos aprendido algo sobre las condiciones del modelo, ya estamos en condiciones de presentar el estadístico que usaremos para el contraste de la igualdad de medias:\\[3mm]
        \fbox{\colorbox{Gris025}{\begin{minipage}{14cm}
        \begin{center}
        \vspace{2mm}
        {\bf Distribución muestral de los componentes del ANOVA unifactorial}\\
        \end{center}
        Si la hipótesis nula $H_0=\{\mu=\mu_1=\mu_2=\cdots=\mu_k\}$ es cierta, entonces:
        \begin{equation}\label{ec:distribucionMuestralComponentesANOVA}
        Y=\dfrac{\quad
        \dfrac{\sum_{j=1}^k n_j(\bar X_{\mbox{\bf\large $\cdot$}j}-\bar X)^2}{k-1}
        \quad
        }{
        \quad\dfrac{\sum_{j=1}^k\sum_{i=1}^{n_j}(X_{ij}-\bar X_{\mbox{\bf\large $\cdot$}j})^2}{n-k}\quad
        }\sim F_{k-1;n-k}
        \end{equation}
        siendo $F_{k-1;n-k}$ la distribución de Fisher-Snedecor con $k-1$ y $n-k$ grados de libertad.
        \end{minipage}}}\\[3mm]
        Si la hipótesis nula fuese cierta, esperaríamos valores del estadístico cercanos a $1$. Por lo tanto, rechazaremos la hipótesis nula si se obtienen valores del estadístico suficientemente mayores que $1$, de manera que la variabilidad debida a diferencias entre grupos es claramente mayor que la que podemos explicar mediante las diferencias dentro de los grupos. Se trata, en definitiva, {\sf de un contraste unilateral} (cola derecha).


    \item La forma habitual de presentar los cálculos del contraste de igualdad de medias en el modelo ANOVA unifactorial, completamente aleatorio y de efectos fijos, que hemos descrito, es mediante una tabla como esta:
    \end{itemize}

    \hspace{-1cm}
    \begin{minipage}{15cm}{
    \begin{tabular}{lccccc}
    \hline
    {\small Fuente de variación}&{\small Suma cuadrados}&{\small Grados libertad}&{\small Cuadrado medio}&{\small Estadístico}&\textcolor{red}{\small P-valor}\\
    \hline
    {\small Entre grupos}&{\scriptsize $\displaystyle\sum_{j=1}^k n_j(\bar X_{\mbox{\bf\large $\cdot$}j}-\bar X)^2$}&$k-1$&{\scriptsize $\dfrac{\displaystyle\sum_{j=1}^k n_j(\bar X_{\mbox{\bf\large $\cdot$}j}-\bar X)^2}{k-1}$}&$Y$&\textcolor{red}{$P(F>Y)$}\\

    \begin{minipage}{2cm}{\small Dentro de\\ los grupos}\end{minipage}&{\scriptsize $\displaystyle\sum_{j=1}^k\sum_{i=1}^{n_j}(X_{ij}-\bar X_{\mbox{\bf\large $\cdot$}j})^2$}&$n-k$&{\scriptsize $\dfrac{\displaystyle\sum_{j=1}^k\sum_{i=1}^{n_j}(X_{ij}-\bar X_{\mbox{\bf\large $\cdot$}j})^2}{n-k}$}\\
    \hline
    \end{tabular}
    }
    \end{minipage}
\begin{itemize}    
    \item Hemos resaltado en color rojo el que probablemente es el valor más relevante de la tabla, el p-valor del contraste.  Esta tabla ANOVA se utiliza, como venimos diciendo, para contrastar la hipótesis nula
    \[H_0=\{\mu_1=\mu_2=\cdots=\mu_k\}\]
    En concreto, si el $p$-valor que aparece en la tabla ANOVA  es menor que el valor de $\alpha$ correspondiente al nivel de confianza, entonces el contraste es {\sf significativo} a ese nivel, y rechazaremos la hipótesis nula. Es decir, consideraremos que los datos obtenidos no son --insistimos, a ese nivel-- estadísticamente compatibles con la hipótesis de que las medias son iguales. Otra manera de pensar esto, es que el contraste es tanto más significativo cuanto mayor sea el valor del estadístico $Y$.


\subsection{Cálculos con Calc y R}

    \item Naturalmente, no tiene mucho sentido tratar de realizar estas cuentas a mano. En esta \textattachfile{Cap14-Tabla-ANOVA.ods}{\textcolor{blue}{hoja de cálculo (Calc)}}, y en
    \textattachfile{Cap14-ANOVA.R}{\textcolor{blue}{este documento de instrucciones R}},
    se calculan automáticamente estos valores. Ambos ficheros asumen que los datos están contenidos en un fichero de tipo csv (como
    \textattachfile{Cap14-Datos-ANOVA.csv}{\textcolor{blue}{este de ejemplo}}, con cuatro grupos), y que se cumplen estas condiciones:
        \begin{itemize}
        \item[] {\bf si se va a usar el fichero con R:}
        \item cada grupo de datos está en una columna.
        \item se usan puntos para indicar los decimales, no comas.
        \item en cada fila los elementos están separados por espacios (no tabuladores).
        \item los títulos de las columnas son G1, G2, G3, etcétera. Si hay más (o menos) de cuatro grupos, se deben modificar --de forma evidente-- las líneas que empiezan
        \begin{verbatim}
Grupos<-c(G1,G2,G3,G4)
Nombres<-c(rep("G1",length(G1)),rep("G2",length(G2)),rep("G3",length(
        \end{verbatim}
        para adaptarlas al número de grupos con el que estemos trabajando.
        \item todas las columnas tienen la misma longitud. Si faltan elementos, se sustituyen por la cadena de caracteres {\tt 'NA'} (comillas incluidas).
        \item[] {\bf si se va a usar con Calc}
        \item los puntos deben reemplazarse por comas.
        \item los datos (sin los títulos G1, G2, G3, pero incluyendo los {\tt 'NA'}) se deben cortar y pegar en la primera hoja (tiene tres) del documento Calc, titulada {\tt Datos}. La tabla ANOVA se obtiene entonces en la tercera hoja del documento.
        \item se pueden introducir un máximo de 10 grupos, con un máximo de 50 observaciones por grupo. Para conjuntos de datos mayores, recomendamos el uso de herramientas especializadas como R.
        \end{itemize}
    \item En este caso, hay diferencias entre Calc y Excel. Mientras que Calc no incluye las tablas ANOVA entre sus funciones estadísticas predefinidas, Excel sí tiene una función ANOVA, incluida en el Analysis ToolPak (podéis ampliar la información en \link{http://office.microsoft.com/en-us/excel-help/use-the-analysis-toolpak-to-perform-complex-data-analysis-HP010342762.aspx?CTT=1}{este enlace}).

    \item El fichero de comandos R incluye muchas otras opciones de análisis, que tienen que ver tanto con la verificación de las condiciones del modelo, como con los métodos que se aplican si el contraste ANOVA resulta significativo, y que comentaremos brevemente en el próximo apartado. Para no extendernos, dejamos la discusión detallada del uso de ese fichero de comandos para una entrada del blog de la asignatura.

    \item Y aquí se incluyen varios  \textattachfile{datos-ANOVA.zip}{\textcolor{blue}{ficheros de datos}} con los que poner a prueba estos métodos.
    \end{itemize}


\subsection{¿Qué hacer si el contraste ANOVA es significativo?}
    \begin{itemize}
        \item Si el contraste ANOVA es significativo  (es decir, si el p-valor es bajo), concluiremos que hay evidencia estadística para rechazar la hipótesis nula. Por lo tanto, la conclusión es que las medias $\mu_i$ no son todas iguales. ¿Y ahora qué? Necesitaríamos, desde luego, saber qué medias son distintas entre sí.  La primera idea es, evidentemente, que podemos hacer comparaciones por parejas, grupo a grupo, usando la teoría que aprendimos en el Capítulo 10. Si tenemos $n$ grupos, el número de parejas se calcula mediante el número combinatorio
            \[\binom{n}{2}=\dfrac{n(n-1)}{2}.\]
            Por ejemplo, para $n=6$ hay que hacer 15 comparaciones. Supongamos que decidimos trabajar a un nivel de significación $\alpha=0.05$. Recordemos que $\alpha$ indica la probabilidad de cometer un error de tipo I, y por lo tanto, la probabilidad de afirmar que existe una diferencia entre las medias de dos grupos, cuando en realidad no es así. Si en cada una de las 15 comparaciones necesarias corremos el riesgo de cometer un error de tipo I con una probabilidad del 5\%, entonces es fácil (ya que las comparaciones son independientes entre sí) ver que la probabilidad total de cometer ese error al menos una vez en la serie completa de comparaciones es:
            \[P(\mbox{error tipo I en 15 comparaciones})=1-(0.95)^{15}\approx 0.537\]
            Es decir, que tenemos una probabilidad mayor del 50\% de cometer un error de tipo I. Con menos grupos el problema es menor, pero aún así grave. Y, por supuesto, a medida que aumenta el número de grupos, esta probabilidad aumenta hasta hacerse casi una certeza a partir de diez o más grupos. La conclusión evidente es que no podemos lanzarnos a hacer las comparaciones sin más.

        \item Uno de los remedios tradicionales para este problema es utilizar lo que se conoce como {\sf ajuste de Bonferroni}. Con este método, el nivel de significación se reparte entre las distintas comparaciones que debemos realizar, de manera que se garantiza el control de los errores de tipo I. Este ajuste se puede realizar fácilmente en cualquier programa estadístico (aunque no las hojas de cálculo, al menos no en las versiones que he examinado), y desde luego, está incluido en R (de nuevo nos remitimos al blog de la asignatura para los detalles). El problema con el ajuste de Bonferroni es que es demasiado {\em conservador}. Y como ya discutimos en su momento, tratar de reducir la probabilidad de cometer un error de tipo I, lleva aparejado un aumento de la probabilidad de cometer errores de tipo II.

        \item Para paliar estos problemas, los estadísticos han diseñado bastantes métodos con el objetivo de comparar las medias de los distintos grupos. Estos métodos se caracterizan, a menudo, por estar diseñados con un tipo específico de comparación de medias en mente. Por ejemplo, uno de esos métodos, el contraste de Dunnett, se utiliza en aquellas situaciones en que, desde el principio, se dispone de un {\em grupo de control}, y lo que se desea es comparar la media de los demas grupos con la del grupo de control; no nos interesa comparar todos con todos, sino medir las posibles diferencias con el control. Y hay otros tipos de contrastes (SNK, Ryan, Duncan, Tukey, Scheffe, Peritz...), cada uno con su finalidad, sus pros y sus contras, que los hacen más adecuados (o populares) en distintos campos de trabajo (medicina, ecología, etc.)

        \item En lugar de ponerles remedios como los que hemos comentado, hay otra forma de evitar los problemas del método ANOVA que hemos descrito; ¡usar otros métodos, claro! El ANOVA y todas las estrategias de comparaciones múltiples asociadas con él, asumen que los datos proceden de poblaciones normales o al menos aproximadamente normales. Podemos librarnos de esa condición utilizando contrastes de hipótesis no paramétricos, como el de Kruskal-Wallis, del que hablaremos en otro capítulo.

        \item Y despedimos este capítulo avanzando algo del próximo. Está claro que la situación que hemos estudiado en este capítulo, en la que la respuesta depende de un sólo factor, es la más sencilla de un conjunto mucho más general de situaciones. Situaciones en las que la respuesta depende de varios factores, y nos veremos obligados a tener cuenta las posibles interacciones entre los distintos niveles de esos factores. Como decíamos, eso es contenido correspondiente al próximo capítulo.
    \end{itemize}


%\section*{Tareas asignadas para esta sesión.}
%No hay tareas asignadas para esta sesión.
%
%\section*{Lectura recomendada}
%
%Los textos que hemos utilizado en el curso no cubren este tema, o no al menos con el suficiente detalle. En la próxima y última sesión del curso comentaremos la bibliografía en la que se pueden ampliar los métodos ANOVA y en general, las técnicas de diseño experimental.

%\section*{\fbox{\colorbox{Gris025}{{Sesión 28. Diseño experimental.}}}}
%
%\subsection*{\fbox{\colorbox{Gris025}{{Última sesión del curso}}}}
%\subsection*{Fecha: Viernes, 13/01/2012, 14h.}
%
%\noindent{\bf Atención: este fichero pdf lleva adjuntos los ficheros de datos necesarios.}
%
%%\subsection*{\fbox{1. Ejemplos preliminares }}
%\setcounter{tocdepth}{1}
%%\tableofcontents

%\section{Variables intrusas, comparaciones emparejadas y más sobre diseño experimental.}
%
%\begin{itemize}
%
%    \item A lo largo del curso, en los ejemplos de inferencia y contraste de hipótesis, hemos usado a menudo la idea de muestras independientes cuando estábamos tratando de estudiar las características de una variable aleatoria $X$ en una o más poblaciones. En las aplicaciones de la Estadística, y muy especialmente en el diseño experimental, a menudo esa suposición de independencia en las muestras es imposible. Lo normal, por el contrario, es que el resultado del experimento se vea afectado por la presencia de otras variables. Por ejemplo, al estudiar los efectos de dos medicamentos $A$ y $B$, podemos formar dos grupos de pacientes y aplicar a cada uno de ellos uno de los tratamientos. Obtendremos como resultado dos muestras,
%        \[X^A_1,X^A_2,\ldots,X^A_n\mbox{ y por otro lado }X^B_1,X^B_2,\ldots,X^B_m.\]
%        y podemos aplicar los métodos que hemos visto para comparar la respuesta a ambos tratamientos (por ejemplo, haciendo un contraste de hipótesis para la diferencia de medias $\mu_A-\mu_B$ a partir de esas dos muestras). Observa que no estamos suponiendo que ambas muestras sean del mismo tamaño. Pero, si hacemos esto,  debemos tener en cuenta que el resultado del experimento se verá afectado por muchos otros factores: el género de cada paciente, su variabilidad genética, su modo de vida, su edad, etcétera. Todas estas {\em variables intrusas} influyen sobre la respuesta al tratamiento; y a consecuencia de esto, a menudo sería necesario considerar muestras de tamaños inaceptablemente grandes para paliar los efectos de esas variables intrusas. En lugar de hacer esto, hay una manera de diseñar el experimento que nos permite obtener resultados estadísticamente relevantes sin tener que recurrir a muestras enormes.
%
%    \item La idea es muy sencilla: puesto que lo que está alterando el resultado del experimento es la variabilidad individual de los pacientes, lo que hacemos es probar los dos tratamientos {\em en todos los pacientes} (consecutiva o simultáneamente, dependiendo del tipo de experimento; esta parte del diseño debe ser objeto también de ua reflexión minuciosa).  Ahora obtendremos dos muestras del mismo tamaño, y podemos considerar las {\sf diferencias emparejadas}:
%        \[
%        \begin{cases}
%        D_1=X^A_1-X^B_1\\
%        D_2=X^A_2-X^B_2\\
%        D_3=X^A_3-X^B_3\\
%        \qquad\vdots\\
%        D_n=X^A_n-X^B_n
%        \end{cases}
%        \]
%        Estos valores se pueden considerar como una muestra de la variable aleatoria $D$ (diferencia entre las respuestas a ambos tratamientos de {\em un mismo paciente}). Y ahora podemos utilizarlas para realizar un contraste de hipótesis sobre $\mu_D$ (que sustituye a $\mu_A-\mu_B$). Por ejemplo, para contrastar la hipótesis nula $H_0=\{\mu_D\leq \mu_D^0\}$, con una muestra de tamaño pequeño, y suponiendo que la población tiene una distribución normal, podríamos utilizar los métodos que vimos en la sesión 9 (del 22/11/2011, página \pageref{S2211-subsec:contrasteHipotesisMediaMuestrasPequennasVarianzaDesconocida}) el estadístico
%        \[\dfrac{\bar D-\mu_D^0}{\dfrac{s}{\sqrt{n}}}\]
%        Sabemos que la distribución muestral de este estadístico es una $t$ de Student con $n-1$ grados de libertad, y a partir de esto podemos realizar el contraste.
%
%    \item Lo que hemos hecho, por tanto, es emparejar los resultados experimentales utilizando algunas unidades naturales (cada paciente es una unidad) del fenómeno que estamos analizando. Esta idea tan simple es una primera indicación de por dónde discurre el trabajo en el diseño experimental, en el que se trata de obtener la máxima información estadística posible a partir de muestras de tamaños moderados, y teniendo en cuenta la presencia de otras variables que intervienen en ese fenómeno.
%
%    \item Siguiendo con este tipo de ideas, el modelo ANOVA (unifactorial, completamente aleatorio y de efectos fijos), que hemos descrito en la \link{http://moodle.mat.uah.es/file.php?file=\%2F93\%2FEsquemas\_Curso_2011-2012\%2F2012\_01\_10\_Estadistica.pdf}{anterior sesión} nos permitía contrastar una hipótesis nula de la forma
%        \[H_0=\{\mu_1=\mu_2=\cdots=\mu_k\}\]
%        entre grupos de pacientes que han sido sometidos a $k$ tratamientos diferentes.  Si el resultado del contraste no es significativo (es decir, si no rechazamos la hipótesis nula), entonces no hay argumentos estadísticos para concluir que los tratamientos produzcan resultados distintos. Pero si la hipótesis nula se rechaza, entonces lo único que el contraste nos permite afirmar es que, de entre todos los $k$ tratamientos probados, al menos hay una pareja de medias $\mu_i$ y $\mu_j$ que son distintas. La pregunta natural es ¿cómo localizamos esas parejas con medias distintas, para comparar los tratamientos, ahora que sabemos que no son todos igual de efectivos?
%
%    \item La primera respuesta es comparar todas las parejas posibles de tratamientos entre sí. Es decir, comparamos el tratamiento 1 con el dos, el uno con el 3, el 2 con el 5, etcétera, para todas las {\em combinaciones} posibles. Y decimos combinaciones porque sabemos que hay
%        \[\binom{k}{2}=\dfrac{k(k-1)}{2}\]
%        parejas posibles. Por ejemplo, para comparar cinco tratamientos esto significa que debemos realizar $10$ contrastes de hipótesis distintos. Si son 20 los tratamientos que comparamos, el número de contrastes se eleva a $190$. Cada uno de esos contrastes se puede llevar a cabo por los métodos que vimos (sesión 24, del 13/12/2011) para comparar la media de dos poblaciones. No obstante hay que hacer algunas consideraciones importantes:
%        \begin{enumerate}
%        \item a la hora de seleccionar el número de grados de libertad que usamos en la $t$ de Student, (ver la Sesión 24, sección \ref{S1312-sec:diferenciaMediasDosPoblaciones}, página \pageref{S1312-sec:diferenciaMediasDosPoblaciones}), debemos tener en cuenta que el modelo ANOVA que hemos usado supone que las varianzas de todas las poblaciones son iguales. Pero {\em además}, cuando estamos comparando $\mu_i$ con $\mu_j$, para estimar $s$ (la cuasidesviación típica muestral), podemos utilizar el hecho de que tenemos a nuestra disposición todos los datos de la tabla ANOVA, y no sólo los de las muestras $i$ y $j$.
%        \item aunque no podemos entrar en la demostración detallada, se puede comprobar que el riesgo de cometer errores de tipo I puede volverse muy alto si se realizan comparaciones entre un número (incluso no muy alto) de tratamientos\footnote{es algo así como la versión estadística de la sabiduría popular que afirma que si los médicos te hacen suficientes pruebas, terminan encontrándote algo...}. Hay varios esquemas de diseño experimental (constrastes de Bonferroni, Duncan, y esquemas similares, ver la bibliografía) que se utilizan para mitigar este problema.
%        \item A diferencia con el caso de las comparaciones emparejadas, aquí no hemos supuesto que se aplican todos los tratamientos a todos los pacientes. Se puede modificar el diseño experimental para utilizar {\sf bloques}, que es una generalización de la idea de pares de datos. El bloque es una colección de unidades experimentales, que tienen valores similares de las variables intrusas, cuyos efectos estamos tratando de mitigar.
%        \end{enumerate}
%
%    \item Finalmente, en esta descripción (superficial, desde luego) de algunos de los primeros problemas que debe abordar el diseño de experimentos, queremos mencionar que existen también muchos métodos para realizar contrastes de hipótesis cuando se tienen en cuenta {\em a la vez} varios factores. Recordemos que el ANOVA que hemos visto era unifactorial porque clasificábamos los datos teniendo en cuenta tan sólo un factor, el tratamiento recibido. Si queremos tener en cuenta otros factores (que de lo contrario tendríamos que considerar como variables intrusas), podemos utilizar los diseños en bloques completos aleatorizados, combinados con la técnica de los cuadrados latinos para diseñar el experimento. Un \link{http://en.wikipedia.org/wiki/Latin_square}{\sf cuadrado latino} es una especie de generalización del conocido \link{http://en.wikipedia.org/wiki/Mathematics_of_Sudoku}{Sudoku}, en la que tenemos un cuadrado de $n$ filas y $n$ columnas, relleno de números del $1$ al $n$, con la condición de que cada uno de estos números aparece sólo una vez en cada fila y en cada columna. Por ejemplo,
%        \[
%        \begin{array}{|c|c|c|c|}
%        \hline
%        1&2&3&4\\
%        \hline
%        2&1&4&3\\\hline
%        3&4&1&2\\
%        \hline
%        4&3&2&1\\
%        \hline
%        \end{array}
%        \]
%        es uno de los $576$ cuadrado latino de orden 4 que existen. Si se consideran cuadrados latinos de tamaño $9$ (como en el Sudoku), el número aumenta a 5524751496156892842531225600
%        \footnote{sólo una fracción de estos, en concreto 6670903752021072936960, son Sudokus}. El primer paso de uno de estos diseños experimentales consiste en elegir aleatoriamente uno de esos cuadrados latinos para asignar los factores a las unidades experimentales (a su vez, esto puede repetirse al nivel de bloques, en diseños más elaborados). Como puedes ver, la combinatoria que acompaña al diseño experimental puede ser bastante complicada.
%
%\end{itemize}
%



    \chapter{Tablas de contingencia y test $\chi^2$}\label{cap:TablasContingenciaTestChi2}
    % !Mode:: "Tex:UTF-8"

Continuando con el tema de la relación entre dos variables, le llega el turno al caso C $\sim$ C, cuando tanto la variable respuesta como la explicativa son cualitativas (también llamadas categóricas, o factoriales). La técnica nueva que vamos a aprender en este capítulo se conoce habitualmente como test $\chi^2$ (léase {\em ji cuadrado o chi cuadrado}).



\section{Tablas de contingencia}

\subsection{El caso $2\times 2$}
\begin{itemize}

    \item El \link{http://www.cis.es/cis/opencms/ES/index.html}{{\em Barómetro} del CIS} (Centro de Investigaciones Sociológicas) permite, entre otras muchas cosas, obtener datos sobre las creencias religiosas de la población en España. Una pregunta que puede interesarnos es ¿hay alguna diferencia al respecto entre hombres y mujeres? Vamos a utilizar los datos del {\em Barómetro} para intentar contestar.

        Por ejemplo, en el mes de enero de 2013 el {\em Barómetro} recoge las respuestas de $n=2452$ personas sobre sus creencias religiosas\footnote{En realidad son 2483, pero para simplificar vamos a eliminar de nuestra consideración a las 19 mujeres y a los 12 hombres que decidieron no contestar.}. Observa que, como de costumbre, vamos a usar $n$ para el número total de personas encuestadas. Agrupamos a todos los creyentes de distintas religiones por un lado y a los que se declaran no creyentes o ateos por otro. Y así tenemos una tabla de doble entrada:
        \begin{center}
        \begin{tabular}{|c|c|c|c|}
          \hline
          % after \\: \hline or \cline{col1-col2} \cline{col3-col4} ...
           & Hombres & Mujeres & Total \\
           \hline
          Creyentes & ?? & ?? & 1864 \\
          \hline
          No creyentes & ?? & ?? & 588 \\
          \hline
          Total & 1205 & 1247 & 2452 \\
          \hline
        \end{tabular}
        \end{center}
        Los valores que aparecen aquí se denominan {\sf valores marginales} (porque aparecen en los márgenes de la tabla, claro).

        Hemos dejado sin rellenar el resto de la tabla porque es el momento de hacerse una pregunta, que dará comienzo a nuestro trabajo de este capítulo: si {\em suponemos} que no hay diferencia entre hombres y mujeres, en lo referente a las creencias religiosas, ¿qué números {\em esperaríamos} ver en esa tabla? Si las creencias religiosas fuesen {\em independientes} del género, esperaríamos encontrar en el grupo de mujeres la misma proporción $p$ de creyentes que existe en la población en conjunto. Y tenemos una estimación muestral de esa proporción poblacional de creyentes declarados, que es:
        \[\hat p=\dfrac{1864}{2452}\approx 0.7602\]
        Así que podemos utilizar esto para rellenar una tabla de {\em valores esperados} (redondeados a enteros):
        \begin{center}
        \begin{tabular}{|c|c|c|c|}
          \hline
          % after \\: \hline or \cline{col1-col2} \cline{col3-col4} ...
           & Hombres & Mujeres & Total \\
           \hline
          Creyentes & $e_{11}=916$ & $e_{12}=948$ & 1864 \\
          \hline
          No creyentes & $e_{21}=289$ & $e_{22}=299$ & 588 \\
          \hline
          Total & 1205 & 1247 & 2452 \\
          \hline
        \end{tabular}
        \end{center}
        Los valores que aparecen aquí se han calculado de la forma evidente. Por ejemplo, nuestra estimación del número de mujeres creyentes es:
        \[e_{12}=1247\cdot\hat p=1247\cdot\dfrac{1864}{2452}\approx 948.\]
        La notación $e_{ij}$ que hemos usado es la habitual en este tipo de situaciones. El valor $e_{ij}$ es el valor {\em esperado} en la fila $i$ y columna $j$.


    \item Con esto estamos listos para ver los datos reales del {\em Barómetro}. Se obtuvo esta tabla:
        \begin{center}
        \begin{tabular}{|c|c|c|c|}
          \hline
          % after \\: \hline or \cline{col1-col2} \cline{col3-col4} ...
           &{\bf Hombres }&{\bf Mujeres }& Total \\
           \hline
          {\bf Creyentes }& $o_{11}=849$ & $o_{12}=1015$ & 1864 \\
          \hline
          {\bf No creyentes }& $o_{21}=356$ & $o_{22}=232$ & 588 \\
          \hline
          Total & 1205 & 1247 & 2452 \\
          \hline
        \end{tabular}
        \end{center}
        De nuevo, la notación $o_{ij}$ es la que se utiliza habitualmente en estos casos para los {\em valores observados}. Y ahora que hemos visto varios ejemplos, podemos hacer oficial el nombre de este tipo de tablas. Las tablas que estamos viendo, que reflejan las frecuencias (observadas o esperadas) de las posibles combinaciones de dos variables cualitativas se llaman {\sf tablas de contingencia}. En particular, estamos trabajando contablas de contingencia $2\times 2$, porque ambas variables toman dos valores (hombres/mujeres, creyentes/no creyentes). Pronto veremos ejemplos más generales de tablas de contingencia con cualquier número de filas o columnas.

    \item   A la vista de las dos tablas de valores $e_{ij}$ y $o_{ij}$, resulta evidente que los valores observados no coinciden con los esperados. De hecho, el número de hombres no creyentes es más alto de lo que habíamos estimado a partir de la población en conjunto (y, lógicamente, el número de mujeres no creyentes es más bajo que la estimación). Pero ese número de hombres no creyentes, ¿es {\sf significativamente} más alto?\\
    La palabra {\em ``significativamente''}, a estas alturas del curso, debería ponernos en guardia. Claro, es que esta situación tiene todos los ingredientes de un contraste de hipótesis. Hay una hipótesis nula, que podemos describir así:
        \[H_0=\{\mbox{Las creencias religiosas no dependen del género.} \}\]
        o también
        \[H_0=\left\{\mbox{Los valores esperados $e_{ij}$ describen correctamente la distribución de probabilidad.}\right\}\]
        Y al obtener unos valores muestrales, distintos de los que predice la hipótesis nula, nos preguntamos si esos valores son tan distintos de los esperados como para que, a alguien que cree en la hipótesis nula, le resulte muy difícil aceptar que son fruto del azar.\\

    \item Antes de seguir adelante, vamos a hacer un par de observaciones:
        \begin{enumerate}
            \item es posible que el lector haya pensado: ``están intentando liarme, cuando esto es mucho más sencillo: ¡nada de dos variables! Estamos estudiando una única variable (la creencia religiosa), con dos resultados posibles (cree/ no cree). Y estudiamos la proporción de creyentes en {\em dos poblaciones:} hombres y mujeres. Así que esto es un problema de inferencia con la Binomial, del tipo que ya hemos estudiado en el Capítulo \ref{cap:Inferencia2Poblaciones}.'' Si el lector ha pensado esto: enhorabuena. Es cierto. En el caso en el que tanto la variable respuesta como la variable explicativa son ambas categóricas y con dos valores posibles (tenemos una tabla $2\times 2$), el problema se puede abordar con los métodos del Capítulo \ref{cap:Inferencia2Poblaciones}, usando la Distribución Binomial y viendo los dos valores posibles de la variable explicativa como si correspondiesen a dos poblaciones. Y los resultados --en ese caso-- son equivalentes a los que vamos a obtener aquí. Y hemos empezado por este ejemplo, del caso más sencillo, precisamente para establecer esa conexión. Pero enseguida vamos a ocuparnos de casos en los que las variables toman más de dos valores, y se necesitan los métodos de este capítulo.

            \item si la frase {\em distribución de probabilidad} te ha intrigado, enhorabuena otra vez. Este es uno de esos momentos sobre los que nos pusimos en guardia en la introducción de esta parte del curso (ver página \pageref{part04:intro}). Para entender con precisión lo que significa {\em distribución de probabilidad} en este contexto, necesitaríamos discutir la {\em distribución multinomial}; se trata de un análogo de la distribución binomial, cuando el experimento puede tener varios resultados, en lugar de sólo dos, como en los experimentos de Bernouilli que sirven de base a la binomial. En el último capítulo de esta parte del curso daremos algunos detalles más.

        \end{enumerate}


    \item Volvamos al asunto de cómo contrastar si las creencias religiosas dependen del género. Ya sabemos, por nuestra experiencia en capítulos previos, que para hacer un contrate de hipótesis, necesitamos un estadístico, y además, información sobre la distribución muestral de ese estadístico cuando $H_0$ es cierta. Como ya hemos anunciado, los detalles son, en este caso, demasiado técnicos para entrar a fondo en ellos; sin llegar al fondo de la cuestión, daremos algunas pistas más en un capítulo posterior. Por el momento, y para ayudar un poco a la intuición, vamos a recordar dos cosas:
        \begin{itemize}
          \item Bajo ciertas condiciones, se puede convertir una distribución relacionada con la binomial en una normal estándar mediante tipificación.
          \item La suma de los cuadrados de varias normales estándar independientes da como resultado una variable de tipo $\chi^2$, con tanto s grados de libertad como normales independientes sumamos.
        \end{itemize}
        Con esas ideas en la cabeza, vamos a presentar el estadístico que usaremos para los datos del {\em Barómetro}:
        \[X^2=\dfrac{(o_{11}-e_{11})^2}{e_{11}}+\dfrac{(o_{12}-e_{12})^2}{e_{12}}+\dfrac{(o_{21}-e_{21})^2}{e_{21}}+\dfrac{(o_{22}-e_{22})^2}{e_{22}}.\]
        Como puede verse, hay un término por cada una de las cuatro celdas de la tabla de contingencia.  Y cada uno de esos términos es de la forma:
        \[\dfrac{(o_{ij}-e_{ij})^2}{e_{ij}}\]
        Para entender algo mejor este término, vamos a llamar $X_{12}$ a una variable aleatoria, que representa el valor de la posición $(1,2)$ (primera fila, segunda columna) de la tabla de contingencia. Naturalmente podríamos hacer lo mismo con las otras celdas de la tabla, y tendríamos cuatro variables $X_{ij}$ para $i,j=1,2$. Pero vamos a centrarnos en $X_{12}$ para fijar ideas. La variable $X_{12}$ toma un valor distinto en cada muestra de la población española. Si otras personas hubieran contestado a la encuesta para elaborar el {\em Barómetro} del CIS, obtendríamos números distintos. El valor que hemos llamado $o_{12}$ es el valor concreto de $X_{12}$ en una muestra concreta (la que se usó en el {\em Barómetro}). ¿Qué tipo de variable es $X_{12}$? Es decir, está claro que es discreta, pero ¿cuál es su distribución?  Podríamos verla como una variable de tipo binomial, donde {\em éxito} se define como {\em caer en la casilla (1,2) de la tabla}, y {\em fracaso} se define como {\em caer en cualquiera de las otras casillas}. La probabilidad de éxito, {\bf suponiendo que la hipótesis nula es correcta}, sería $p_{12}=\dfrac{e_{12}}{n}$. ¿Cuál sería la media $\mu(X_{12})$? Conviene recordar que otro nombre para la media es {\em valor esperado}. Así que no debería sorprendernos que el valor esperado de $X_{12}$ sea $e_{12}$.

        Por lo tanto, si estuviéramos tipificando la variable $X_{12}$, esperaríamos ver algo como:
        \[\dfrac{(o_{12}-e_{12})^2}{\sigma(X_{12})}.\]
        El numerador del segundo término del estadísitico, el que corresponde a $X_{12}$, parece el cuadrado de la tipificación de esta variable. Como si, en efecto, estuviéramos tipificando y elevando al cuadrado. Pero el problema es que el denominador de ese término del estadístico es $e_{12}$, mientras que, pensando en una binomial, nosotros esperaríamos
        \[\sigma^2(X_{12}=\left(\sqrt{n p_{12} q_{12}}\right)^2=e_{12} q_{12}.\]
        {\em Si hubiera sido una tipificación,} ¿habríamos podido decir que el Estadístico es la suma de cuatro normales estándar y por lo tanto que es una $\chi^2_4$? ¡No! Porque se necesitan normales {\em independientes}. Y está bastante claro que las cuatro variables $X_{ij}$ no pueden ser independientes: sus sumas tienen que ser iguales a los valores marginales de la tabla.  Aún así, lo esencial de la idea es correcto: sumamos algo parecido (¡pero no igual!) a los cuadrados de la tipificación de unas binomiales, que {\em no son independientes}. Y el resultado es, en efecto, una distribución $\chi^2$, pero esa falta de independencia se traduce en que obtenemos menos grados de libertad de los que esperábamos. Concretamente:\\[3mm]
        \fbox{\colorbox{Gris025}{\begin{minipage}{14cm}
        \begin{center}
        \vspace{2mm}
        {\bf Estadístico $\chi^2$ para una tabla de contingencia $2\times 2$}
        \end{center}
        Dada una tabla de contingencia $2\times 2$, con valores esperados $e_{ij}$ y valores observados $o_{ij}$ (para $i,j=1,2$), definimos el estadístico:
        \[X^2=\dfrac{(o_{11}-e_{11})^2}{e_{11}}+\dfrac{(o_{12}-e_{12})^2}{e_{12}}+\dfrac{(o_{21}-e_{21})^2}{e_{21}}+\dfrac{(o_{22}-e_{22})^2}{e_{22}}.\]
        Entonces, {\sf mientras ninguno de los valores $e_{ij}$ sea menor de $5$}, el estadístico $X^2$ sigue una distribución $\chi^2_3$, con {\sf tres grados de libertad}.
        \end{minipage}}}\\[3mm]
        Llamamos la atención del lector sobre el hecho de que son tres grados de libertad, y que la razón para esto es la falta de independencia entre las variables que caracterizan al problema. Para justificar esto, con algo de rigor, necesitaríamos más detalles técnicos, y hablar de la distribución multinomial; dejamos esos detalles para otro momento. Lo que sí podemos hacer es justificar informalmente ese único grado de libertad. En general, un grado de libertad significa que sólo podemos elegir uno de los valores que describen el problema. En nuestro caso, volvamos a la tabla de contingencia inicial, vacía salvo por los valores marginales:
        \begin{center}
        \begin{tabular}{|c|c|c|c|}
          \hline
          % after \\: \hline or \cline{col1-col2} \cline{col3-col4} ...
           & Hombres & Mujeres & Total \\
           \hline
          Creyentes & ?? & ?? & 1864 \\
          \hline
          No creyentes & ?? & ?? & 588 \\
          \hline
          Total & 1205 & 1247 & 2452 \\
          \hline
        \end{tabular}
        \end{center}
        Si escribimos uno cualquiera (y sólo uno) de los valores que faltan, enseguida nos daremos cuenta de que los tres valores restantes han quedado automáticamente determinados por esa primera elección. Es decir, que dados los valores marginales, sólo podemos elegir un número de la tabla. Eso indica que sólo hay un grado de libertad en este problema.


    \item La información sobre la distribución del estadístico nos permite contestar a la pregunta que habíamos dejado pendiente:  ¿es el número de hombres no creyentes que refleja el {\em Barómetro} significativamente más alto de lo esperado? Más concretamente, la pregunta que vamos a responder es: ¿se alejan los valores observados significativamente de los esperados? Hacemos las cuentas de este ejemplo, calculando el valor del estadístico:
        \[X^2=\dfrac{(o_{11}-e_{11})^2}{e_{11}}+\dfrac{(o_{12}-e_{12})^2}{e_{12}}+\dfrac{(o_{21}-e_{21})^2}{e_{21}}+\dfrac{(o_{22}-e_{22})^2}{e_{22}}=\]
        \[=\dfrac{(849-916)^2}{916}+\dfrac{(1015-948)^2}{948}+\dfrac{(356-289)^2}{289}+\dfrac{(232-299)^2}{299}
            \approx 40.18
        \]
        (Téngase en cuenta que los valores $e_{ij}$ que aparecen antes son aproximados; para esta cuenta hemos usado valores más precisos). Y como sabemos que el estadístico se comporta como $\chi^2$, usamos las herramientas habituales (R, Calc) para obtener el p-valor: $2.26\cdot 10^{-10}$. Este p-valor tan pequeño nos lleva a rechazar la hipótesis nula: tenemos razones para creer que la distribución de las creencias religiosas y el género están relacionados.

    \item En \textattachfile{TablasContingencia-BarometroCIS.R}{\textcolor{blue}{este fichero de comandos R}} se pueden ver \Rlogo{las instrucciones} correspondientes a este ejemplo. La herramienta fundamental es la función {\tt chisq.test}. Esta \link{http://fernandosansegundo.wordpress.com/2013/05/13/tablas-de-contingencia-y-contraste-de-independencia-test-chi-cuadrado/}{entrada del blog} también se corresponde con la discusión de este apartado. Siguiendo con la conexión entre las tablas de contingencia $2\times 2$ y el contraste de dos proporciones binomiales, que hemos mencionado antes, es interesante comparar los resultados de {\tt chisq.test} con los de {\tt prop.test}.

    \item Cuando se usa la función {\tt chisq.test} de R para el ejemplo del {\em Barómetro}, se obtiene un valor del estadístico igual $40.2253$ y un correspondiente p-valor igual a $2.263\cdot 10^{-10}$. Es decir, que los resultados son significativos; podemos rechazar la hipótesis nula, y decir que los datos apoyan la hipótesis alternativa de que las creencias religiosas están relacionadas con el género.


\subsection{El caso general}

    \item La generalización de lo anterior corresponde al caso en el que queremos contrastar la posible relación entre dos variables categóricas $A_1$ y $A_2$, con $n_1$ y $n_2$ niveles, respectivamente. Al considerar todas las combinaciones posibles de cada nivel de $A_1$ con cada uno de los niveles de $A_2$, obtendríamos entonces, para una muestra con $n$ observaciones,  una tabla de contingencia $n_1\times n_2$, con $n_1$ filas y $n_2$ columnas, como esta:

        \begin{table}[h]
        \begin{center}
        \begin{tabular}{cccccc}
           &
           &
          \multicolumn{3}{c}{\bf {\rule{0mm}{0.5cm}Variable $A_2$}}&
          \\[3mm]
          \cline{3-5}
           &
           &
          \multicolumn{1}{|c}{$b_1$}&
          $\cdots$ &
          \multicolumn{1}{c|}{$b_{n_2}$} &
          Total
          \\
          \cline{2-6}
          \multicolumn{1}{c}{\multirow{3}{*}[-1em]{\bf \begin{tabular}{cc}Variable\\$A_2$\end{tabular}}}&
          \multicolumn{1}{|c}{$a_{1}$ }&
          \multicolumn{1}{|c}{$o_{11}$} &
          $\cdots$ &
          \multicolumn{1}{c|}{$o_{1n_2}$} &
          $o_{1\bullet}$
          \\
           &
          \multicolumn{1}{|c}{$\vdots$}&
          \multicolumn{1}{|c}{ }&
           $\ddots$ &
          \multicolumn{1}{c|}{ }&
            $\vdots$
          \\
          &
          \multicolumn{1}{|c}{$a_{n_1}$}&
          \multicolumn{1}{|c}{$o_{n_11}$}&
          $\cdots$ &
          \multicolumn{1}{c|}{$o_{1n_2}$}&
          $o_{1\bullet}$
          \\
          \cline{2-6}
           &
          Total&
          \multicolumn{1}{|c}{$o_{\bullet 1}$}&
          $\cdots$ &
          \multicolumn{1}{c|}{$o_{\bullet n_2}$}&
          $o_{\bullet\bullet}$=n
        \end{tabular}
        \label{tabla:tablaContingenciaGeneral}\caption{Tabla de contingencia general}
        \end{center}
        \end{table}
        Para escribir los valores marginales de la tabla hemos utilizado una notación similar a la que usamos para el ANOVA. Así, por ejemplo, $o_{\bullet 1}$ representa la suma de todos los elementos d ela primera columna de la tabla, y $o_{2\bullet}$ es la suma de la segunda fila.

    \item Además, naturalmente, esta tabla va acompañada por la correspondiente tabla de valores esperados, $e_{ij}$, calculados de esta manera:
        \[e_{ij}=\dfrac{o_{i\bullet}\cdot o_{\bullet j}}{o_{\bullet\bullet}}.\]
        Es la misma receta que hemos usado en el caso de tablas $2\times 2$: primero se calcula la proporción que predice el valor marginal por columnas, que es:
        \[\dfrac{o_{\bullet j}}{o_{\bullet\bullet}},\]
        y se multiplica por el valor marginal por filas $o_{i\bullet}$ para obtener el valor esperado.

    \item La hipótesis que queremos contrastar, en el caso general, es en realidad la misma que en el caso $2\times 2$:
        \[H_0=\left\{\mbox{Los valores esperados $e_{ij}$ describen correctamente la distribución de probabilidad.}\right\}\]
        Y ya tenemos todos los ingredientes necesarios para enunciar el principal resultado de esta sección:\\[3mm]
        \fbox{\colorbox{Gris025}{\begin{minipage}{14cm}
        \begin{center}
        \vspace{2mm}
        {\bf Estadístico $\chi^2$ para una tabla de contingencia $n_1\times n_2$}
        \end{center}
        Dada una tabla de contingencia $n_1\times n_2$, como la Tabla \ref{tabla:tablaContingenciaGeneral} (página \pageref{tabla:tablaContingenciaGeneral}), con valores observados $o_{ij}$, y valores esperados $e_{ij}$ (para $i,j=1,2$), definimos el estadístico:
        \[X^2=\sum_{i=1}^{n_1}\sum_{j=1}^{n_2}\left(\dfrac{(o_{ij}-e_{ij})^2}{e_{ij}}\right).\]
        Es decir, sumamos un término para cada casilla de la tabla. Entonces, {\sf mientras ninguno de los valores $e_{ij}$ sea menor de $5$}, el estadístico $X^2$ sigue una distribución $\chi^2_{n-1}$, con
        \[k=(n_1-1)(n_2-1)\]
        grados de libertad.
        \end{minipage}}}\\[3mm]
        Obsérvese que en el caso $2\times 2$ (es decir, $n_1=n_2=1$), el número de grados de libertad es $k=(2-1)\cdot(2-1)=1$. Para una tabla $3\times 4$ se tiene $k=(3-1)\cdot(4-1)=6$ grados de libertad. Este número de grados de libertad puede justificarse, informalmente al menos, con el mismo tipo de razonamiento que empleamos en el caso $2\times 2$. Por ejemplo, en esa tabla $3\times 4$, si escribimos los valores de las dos primeras filas y las tres primeras columnas (o en general, seis valores cualesquiera), los restantes seis valores se obtienen usando los valores marginales, con lo que en realidad tenemos sólo seis grados de libertad.









%        Quedándonos muy en la superficie, vamos a recordar que las distribuciones muestrales que entendemos bien son las de las variables normales. Lo que vamos a hacer es enredar un poco con las fórmulas hasta conseguir que aparezcan las normales por algún sitio; cosa que ya hemos hecho antes, por otra parte. La ventaja con la que jugamos es que muchos tipos de variables se comportan en el límite como variables normales. Y a eso vamos a apelar, también en este caso.

%        Volvamos al ejemplo de creyentes/no creyentes y empecemos representando los valores marginales mediante una notación parecida a la que usamos en el capítulo \ref{cap:IntroduccionANOVA} para el ANOVA:
%        \[e_{1\bullet}=e_{11}+e_{12}=1864,\qquad e_{2\bullet}=e_{21}+e_{22}=588\]
%        Es decir que $e_{1\bullet}$ y $e_{1\bullet}$ son las sumas por filas de la tabla de valores esperados, y en este ejemplo representan el total de creyentes y no creyentes, respectivamente, sin tener en cuenta el género
%        Naturalmente, en la tabla de valores observados podemos definir $o_{1\bullet}$ y $o_{2\bullet}$ de la misma forma. Y debería estar claro que
%        \[o_{1\bullet}=e_{1\bullet},\quad\mbox{ y también }\quad o_{2\bullet}=e_{2\bullet}.\]

%        Para entender lo que sigue, vamos a llamar $X_{12}$ a una variable aleatoria, que representa el valor de la posición $(1,2)$ (primera fila, segunda columna) de la tabla de contingencia. Naturalmente podríamos hacer lo mismo con las otras celdas de la tabla, y tendríamos cuatro variables $X_{ij}$ para $i,j=1,2$. Pero vamos a centrarnos en $X_{12}$ para fijar ideas. La variable $X_{12}$ toma un valor distinto en cada muestra de la población española. Si otras personas hubieran contestado a la encuesta para elaborar el {\em Barómetro} del CIS, obtendríamos números distintos. El valor que hemos llamado $o_{12}$ es el valor concreto de $X_{12}$ en una muestra concreta (la que se usó en el {\em Barómetro}). ¿Qué tipo de variable es $X_{12}$? Es decir, está claro que es discreta, pero ¿cuál es su distribución?  Podríamos verla como una variable de tipo binomial, donde {\em éxito} se define como {\em caer en la casilla (1,2) de la tabla}, y {\em fracaso} se define como {\em caer en cualquiera de las otras casillas}. La probabilidad de éxito, {\bf suponiendo que la hipótesis nula es correcta}, sería $p_{12}=\dfrac{e_{12}}{n}$. ¿Cuál sería la media $\mu(X_{12})$? Conviene recordar que otro nombre para la media es {\em valor esperado}. Así que no debería sorprendernos que el valor esperado de $X_{12}$ sea $e_{12}$. Podemos verlo también usando la formula de la esperanza para una binomial: $n\cdot p_{12}=e_{12}$. ¿Cuál sería entonces la normal estándar que usaríamos para aproximar esta binomial? Esta:
%        \[\dfrac{o_{12}-e_{12}}{\sqrt{n\cdot p_{12}\cdot q_{12}}}\]
%        Podemos hacer lo mismo
%


        %Podríamos seguir adelante pensando en las variables $X_{12}, X_{12}, X_{21}, X_{22}$ como binomiales, para después aproximarlas por normales, y el resultado sería correcto. Pero además de correcto queremos que sea sencillo. Así que, para simplificar, vamos a intercalar un paso más: usando lo que aprendimos en el Capítulo \ref{cap:DistribucionesRelacionadasBinomial} vamos a aproximar cada una de esas binomiales por una distribución de Poisson, que luego aproximaremos por la correspondiente normal.


        %¿Cuál es el estadístico más evidente? Está claro que la hipótesis nula se verá en un aprieto cuando las diferencias
        %\[e_1-o_1,\ldots,e_6-o_6\]
        %entre lo que observamos y lo que predice $H_0$ sean grandes. Y, como ya tenemos experiencia, no gastamos ni un segundo en explicarle al lector porque, en realidad, nos interesan las diferencias al cuadrado:
        %\[(e_1-o_1)^2,\ldots,(e_6-o_6)^2.\]
        %El siguiente paso en la construcción del estadístico es, no obstante, más delicado.





\end{itemize}












\section{¿Cómo podemos detectar un dado cargado? El test $\chi^2$ de homogeneidad.}

\begin{itemize}
    \item En \textattachfile{Cap13-dado5000.csv}{\textcolor{blue}{este fichero}} están almacenados los resultados de 5000 lanzamientos de un dado. La tabla de frecuencias observadas correspondiente a esos 5000 lanzamientos es esta:
    \begin{center}
    \begin{tabular}{|l|c|c|c|c|c|c|}
      \hline
      % after \\: \hline or \cline{col1-col2} \cline{col3-col4} ...
      Resultado & 1 & 2 & 3 & 4 & 5 & 6 \\
      \hline
      Frecuencia & 811 & 805 & 869 & 927 & 772 & 816\\
      \hline
    \end{tabular}
    \end{center}
    %Vamos a llamar a estos seis números $o_1,o_2,\ldots,o_6$ (la o es por {\em observados}), que es la notación habitual en este tipo de discusiones.
    ¿No hay demasiados cuatros? ¿Es un dado cargado? ¿Cómo podríamos averiguarlo?

    Bueno, tenemos en la cabeza, desde luego un modelo {\sf teórico} de lo que esperamos , que corresponde con nuestra asignación de probabilidad $1/6$ para cada uno de los posibles resultados. Es, de hecho, la idea misma de un dado no cargado: es un dado que, al lanzarlo muchas veces, produce una tabla de frecuencias cada vez más parecida a la tabla ideal. ¿Y cuál es esa tabla ideal para 5000 lanzamientos de un dado no cargado?:
    \begin{center}
    \begin{tabular}{|l|c|c|c|c|c|c|}
      \hline
      % after \\: \hline or \cline{col1-col2} \cline{col3-col4} ...
      Resultado & 1 & 2 & 3 & 4 & 5 & 6 \\
      \hline
      Frecuencia \rule{0cm}{0.6cm}& $\dfrac{5000}{6}$ & $\dfrac{5000}{6}$ & $\dfrac{5000}{6}$ & $\dfrac{5000}{6}$ & $\dfrac{5000}{6}$ & $\dfrac{5000}{6}$\\[2mm]
      \hline
    \end{tabular}
    \end{center}
    donde $\dfrac{5000}{6}\approx 833$.
    %Llamaremos, a los seis números de esta segunda tabla, (que son todos iguales en este ejemplo) $e_1, e_2, \ldots, e_6$. Esa notación es, de nuevo, la que se usa habitualmente (la e es de {\em esperado}, o {\em  expected} en inglés)

    \item La tabla de frecuencias que abre este capítulo no es, desde luego, ideal: hay demasiados cuatros y pocos cincos. ¿Pero son esas diferencias con el ideal suficientemente grandes para considerarlas {\sf significativas}?
        La palabra {\em ``significativas''}, a estas alturas del curso, debería ponernos en guardia. Claro, es que esta situación tiene todos los ingredientes de un contraste de hipótesis. Hay una hipótesis nula, que podemos describir así:
        \[H_0=\{\mbox{ el dado no está cargado.} \}\]
        o también
        \[H_0=\left\{\mbox{ la probabilidad de cada uno de los valores $1,2,\ldots,6$ es $\dfrac{1}{6}$} \right\}\]
        Y al obtener unos valores muestrales, distintos de los que predice la hipótesis nula, nos preguntamos si esos valores son tan distintos de los esperados como para que, a alguien que cree en la hipótesis nula, le resulte muy difícil aceptar que son fruto del azar.

    \item Naturalmente, si vamos a hacer un contrate de hipótesis, necesitamos un estadístico, y además, información sobre la distribución muestral de ese estadístico cuando $H_0$ es cierta.
        %¿Cuál es el estadístico más evidente? Está claro que la hipótesis nula se verá en un aprieto cuando las diferencias
        %\[e_1-o_1,\ldots,e_6-o_6\]
        %entre lo que observamos y lo que predice $H_0$ sean grandes. Y, como ya tenemos experiencia, no gastamos ni un segundo en explicarle al lector porque, en realidad, nos interesan las diferencias al cuadrado:
        %\[(e_1-o_1)^2,\ldots,(e_6-o_6)^2.\]
        %El siguiente paso en la construcción del estadístico es, no obstante, más delicado.
        Para entender lo que vamos a hacer hay que tener muy presente esa idea, de que necesitamos entender la distribución muestral del estadístico. Y como las distribuciones muestrales que entendemos bien son las de las variables normales, lo que vamos a hacer es enredar un poco con las fórmulas hasta conseguir que aparezcan las normales por algún sitio; cosa que ya hemos hecho antes, por otra parte. La ventaja con la que jugamos es que muchos tipos de variables se comportan en el límite como variables normales. Y a eso vamos a apelar, también en este caso.

    \item Además, para simplificar aún más, y dado que en este ejemplo hay  ``demasiados cuatros'', vamos a darle al cuatro un papel estelar frente a los demás números. Es decir, que vamos a clasificar los resultados del lanzamiento del dado sólo en dos clases: ``cuatro'' y ``no cuatro''.  Y como este capítulo ha empezado con este ambiente de tahures y dados cargados, no queremos que el lector piense que esta elección del cuatro frente a los demás valores esconde alguna fullería. No se trata de eso. Si pudiéramos trabajar con los seis números individualmente y, a la vez, mantener las matemáticas necesarias dentro de lo elemental, lo haríamos. Pero no nos es posible, así que vamos a usar este truco para poder mantener la discusión dentro de un grado de dificultad aceptable. Y, desde luego, señalaremos cuál es el punto en el que las cosas se vuelven mucho más complicadas de argumentar, cuando se consideran los seis valores del dado por separado.

    \item Para hacer esto, rehacemos la tabla inicial, la de los valores observados, agrupando en una columna todo lo que no son cuatros:
            \begin{center}
            \begin{tabular}{|l|c|c|}
              \hline
              % after \\: \hline or \cline{col1-col2} \cline{col3-col4} ...
              Resultado & 4 & no es 4 \\
              \hline
              Frecuencia & 927 & 4073\\
              \hline
            \end{tabular}
            \end{center}
            Y llamaremos ahora $o_1=927$, $o_2=4073$ (la $o$ es de observados). Obsérvese que:
            \[n=5000=o_1+o_2\]
            La tabla ideal o teórica, por otra parte, se convierte en:
            \begin{center}
            \begin{tabular}{|l|c|c|c|c|c|c|}
              \hline
              % after \\: \hline or \cline{col1-col2} \cline{col3-col4} ...
              Resultado & 4 & no es 4 \\
              \hline
              Frecuencia \rule{0cm}{0.6cm}& $\dfrac{5000}{6}\approx 833$ & $5000\cdot\dfrac{5}{6}\approx 4167$\\[2mm]
              \hline
            \end{tabular}
            \end{center}
            Y llamaremos a estos números:
            \[e_1=833, e_2=4167.\]
            Obsérvese que, de nuevo:
            \[n=5000=e_1+e_2,\]
            Para terminar de introducir toda la notación necesaria vamos a llamar $p_1$ a la probabilidad de obtener un cuatro si $H_0$ es cierta, que naturalmente es $1/6$, y llamaremos $q_1=1-p_1=5/6$ a la probabilidad de obtener un ``no 4''. Se cumple que:
            \[e_1=n\, p,\quad e_2=n\,q,\]
            y en seguida vamos a usar la relación:
            \[o_1-n\cdot p=-(o_2-n\cdot q),\]
            que se deduce de las anteriores. De hecho, la razón por la que hemos decidido tratar al cuatro frente a todos los demás valores es que, al reducir el problema a sólo dos posibles resultados, se cumplen todas estas relaciones que nos van a facilitar mucho las cosas. Si hubiéramos trabajado con los seis resultados posibles del dado en pie de igualdad, al llegar a este punto el número de relaciones aumentaría mucho, y las matemáticas se volverían muy complicadas.


    \item Para entender lo que haremos vamos a dejar por un momento el dado y a pensar en un ejemplo todavía más simple: una moneda de la que queremos saber si está trucada. La discusión es muy parecida a la del dado, pero en este caso, al lanzar la moneda $n$ veces, sólo hay dos valores observados $o_1$ (número de caras), $o_2$ (número de cruces) y dos valores esperados $e_1,e_2$. La hipótesis nula es
        \[H_0=\left\{\mbox{ la probabilidad de obtener cara es $\dfrac{1}{2}$} \right\}\]
        Para llegar a una variable normal, vamos a pensar en la cantidad $o_1$, el número de caras. Podemos verla como uno de los valores posibles de la variable aleatoria:
        \[O_1=\{\mbox{número de caras obtenidas al lanzar $n$ veces la moneda.}\}\]


        Para llegar hasta ahí, vamos a pensar en la cantidad $o_1$. Podemos verla como uno de los valores posibles de la variable aleatoria:
        \[O_1=\{\mbox{número de unos obtenidos al lanzar 5000 veces el dado.}\}\]
        Y, {\em si suponemos que la hipótesis nula es cierta} (como hacemos siempre en los contrastes), entonces $O_1$ es una variable aleatoria binomial de tipo $B(n,p)$ donde $n=5000$ y $p=\dfrac{1}{6}$. ¿Por qué es bueno habernos dado cuenta de esto? Porque estamos buscando variables normales, y sabemos (Capítulo ) que una binomial $B(n,p)$ con $n$ grande (y $n\cdot p$ no demasiado pequeño) se aproximará mucho a la normal $N(\mu,\sigma)$, siendo $\mu=n\,p$ y $\sigma^2=n\,p\,q$.


        El lector estará pensando, tal vez, ``y ahora haremos la media de las diferencias...''. Sí, pero ¿en qué sentido hacemos ``la media''? Para que el problema resulte más evidente. Supongamos que la proporción de no creyentes en la población (en conjunto, hombres y mujeres) fuera realmente muy baja, de un cinco por ciento. Entonces los números  $e_{11}$ y $o_{11}$ (ambos números se refieren a hombres creyentes) serían mucho más grandes que los correspondientes $e_{21}$ y $o_{21}$ (ambos se refieren a hombres no creyentes, de los que habría muchos menos). Por lo tanto la diferencia al cuadrado $(e_{11}-o{11})^2$ podría ser mucho, mucho más grande que $(e_{21}-o{21})^2$, hasta el punto de hacer inapreciable el valor de esta segunda diferencia en la ``media''. Y sin embargo, puede que ese sea precisamente el valor interesante. Los no creyentes pueden ser pocos, y {\em a la vez}, podría ocurrir que el número de mujeres no creyentes fuera el doble que el de hombres no creyentes, por ejemplo. Pero una ``media'' hecha sin criterio eliminaría esa información.


\end{itemize}


    \chapter{Regresión logística y métodos relacionados}\label{cap:RegresionLogistica}
    % !Mode:: "Tex:UTF-8"

\section{Regresión Logística}

\begin{itemize}

    \item 
    
\end{itemize}








%\appendix{Apéndices}






%##############################################################################################################################
%##############################################################################################################################
%##############################################################################################################################
%##############################################################################################################################
%##############################################################################################################################
%##############################################################################################################################
%\newpage
%\bibliography{BibliografiaCalculo}\addcontentsline{toc}{section}{Bibliografía}


\end{document}
